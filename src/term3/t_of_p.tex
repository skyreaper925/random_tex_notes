%! Author = user
%! Date = 18.01.2024

\documentclass[a4paper, 14pt]{article}
%\documentclass[draft]{article}

\usepackage[T2A]{fontenc}
\usepackage[utf8]{inputenc}
\usepackage[english, russian]{babel}
\usepackage[top = 2cm, bottom = 2cm, left = 2cm, right = 2cm]{geometry}
\usepackage{indentfirst}
\usepackage{xcolor}
\usepackage{hyperref}
\usepackage{gensymb}
\usepackage{pgfplots}
\usepackage{amsmath, amsfonts, amsthm, mathtools}
\usepackage{amssymb}
\usepackage{physics, multirow, float}
\usepackage{wrapfig, tabularx}
\usepackage{icomma} % Clever comma: 0,2 - number while 0, 2 - two numbers
\usepackage{tikz, standalone}
\usepackage{fancyhdr,fancybox}
\usepackage{booktabs}
\usepackage{listings}
\usepackage{lstmisc}
\usepackage{stmaryrd}
\usepackage{lastpage}

%\полуторный интервал
\onehalfspacing

\hypersetup
{   colorlinks = false,
    linkcolor = blue,
    pdftitle = {analysis},
    pdfauTheoremor = {Володин Максим},
    allcolors = [RGB]{010 090 200}
}

%\gravarphicspaTheorem{{images/}}
%\DeclareGravarphicsExtensions{.pdf,.png,.jpg}

\restylefloat{table}
\usetikzlibrary{external}

\mathtoolsset{showonlyrefs = true} % Numbers will appear only where \eqref{} in Theoreme text LINKED
\pagestyle{fancy}

\fancyhf{}
\fancyhead[L]{Теория вероятностей}
\fancyhead[R]{Конспект билетов}
\fancyfoot[L]{}
\fancyfoot[R]{\thepage /\pageref{LastPage}}

\pgfplotsset{compat=1.18}

\begin{document}

{\huge
    \begin{center}
    {\textbf{Конспект билетов}}
        \\
        Теория вероятностей
    \end{center}
}
    \tableofcontents \newpage
    
    \section{Дискретное вероятностное пространство, классическая вероятность, геометрическая вероятность}
    
    \textbf{Опр} \textit{Элементарные события (исходы) и и х пространство}
    
    \textbf{Опр} \textit{Дискретное вероятностное пространство}
    
    \textbf{Опр} \textit{Классическая вероятностная модель}
    
    \textbf{Опр} \textit{Модель геометрической вероятности}
    
    \section{Колмогоровское определение вероятностного пространства.
    Свойства вероятности}
    
    \subsection{Колмогоровское определение вероятностного пространства}
    
    \textbf{Опр} \textit{Событие и вероятностная мера (вероятность)}
    
    \subsection{Свойства вероятности}
    
    Вероятность обладает 7 свойствами; доказывать некоторые из них лучше, опираясь на рисунок
    
    \textbf{Опр} \textit{Вероятностное пространство}
    
    \section{Независимость событий, условная вероятность, формула полной вероятности, формула Байеса}
    
    \textbf{Опр} \textit{Независимые события}
    
    \textbf{Опр} \textit{Условная вероятность}
    
    \textbf{Опр} \textit{Разбиение}

    \textbf{Theorem} \textit{Формула полной вероятности}
    
    \textbf{Theorem} \textit{Формула Байеса}
    
    \section{Схема испытаний Бернулли: два определения и их эквивалентность}
    
    \textbf{Опр} \textit{Попарно независимые события}
    
    \textbf{Опр} \textit{Независимые в совокупности события}
    
    \textbf{Опр} \textit{Схема испытаний Бернулли}
    
    \section{Распределения в $\mathbb{R}$, функция распределения и её свойства.
    Теорема о построении вероятностной меры на ($\mathbb{R}, \mathcal{B}(\mathbb{R}$)) по функции распределения}
    
    \subsection{Распределения в $\mathbb{R}$, функция распределения и её свойства}
    
    \textbf{Опр} \textit{Борелевская сигма-алгебра $\mathcal{B}(\mathbb{R}$}
    
    \textbf{Опр} \textit{Распределение вероятностей}
    
    \textbf{Опр} \textit{Функция распределения}
    
    \textbf{Свойства} \textit{Функции распределения}
    
    Первое свойство тривиально.
    Во втором надо понимать связь пределов и параметров.
    С третьим я не согласен
    
    \subsection{Теорема о построении вероятностной меры на ($\mathbb{R}, \mathcal{B}(\mathbb{R}$)) по функции
    распределения}
    
    \textbf{Theorem}
    
    \section{Дискретные и абсолютно непрерывные распределения в $\mathbb{R}$.
    Плотность.
    Связь плотности и функции распределения.
    Примеры}
    
    \subsection{Дискретные и абсолютно непрерывные распределения в $\mathbb{R}$}
    
    \textbf{Опр} \textit{Дискретные распределения вероятности}
    
    \subsection{Плотность}
    
    \textbf{Опр} \textit{Абсолютно непрерывное распределение, его плотность}
    
    \subsection{Связь плотности и функции распределения}
    
    Плотность и функция распределения связаны формулой Ньютона--Лейбница
    
    \subsection{Примеры}
    
    By the text
    
    \section{Распределения в $\mathbb{R}^n$, функция распределения и её свойства.
    Теорема о построении вероятностной меры на ($\mathbb{R}^n, \mathcal{B}(\mathbb{R}^n$) по функции распределения
        (б/д)}
    
    \subsection{Распределения в $\mathbb{R}^n$, функция распределения и её свойства}
    
    By the text
    
    Многомерная функция распределения обладает 3 свойствами.
    Первое доказывается вводом функции одной переменной (как с частными производными), а затем совместно с остальным
    свойствами доказывается аналогично одномерному случаю
    
    \subsection{Теорема о построении вероятностной меры на ($\mathbb{R}^n, \mathcal{B}(\mathbb{R}^n$) по функции
    распределения (б/д)}
    
    By the text
    
    \section{Топологическое пространство.
    Индуцированная топология.
    Карта и атлас на топологическом пространстве.
    Общие (абстрактные) определения многообразия и гладкого многобразия.
    Классы гладкости $C^k$ отображений из одного гладкого многообразия в другое}
    
    \subsection{Топологическое пространство}
    
    \textbf{Опр} \textit{Топологическое пространство}
    
    \textbf{Опр} \textit{Топология}
    
    \textbf{Опр} \textit{Открытое в топологическом пространстве множество}
    
    \textbf{Опр} \textit{Семейство всех открытых подмножеств метрического пространства с метрикой}
    
    \textbf{Опр} \textit{Окрестность точки}
    
    \textbf{Опр} \textit{Внутренность множества}
    
    \textbf{Опр} \textit{Замыкание множества}
    
    \subsection{Индуцированная топология}
    
    \textbf{Опр} \textit{Индуцированная топология}
    
    \textbf{Лемма 1}
    
    Доказывается по определению, с привлечением старых множеств, породивших новую топологию
    
    \textbf{Опр} \textit{Хаусдорфово топологическое пространство}
    
    \textbf{Опр} \textit{Предел по топологическому пространству}
    
    Заметим, что у хаусдорфова пространства не может быть двух различных пределов; иначе может
    
    \textbf{Лемма 2}
    
    Доказывается выбором специальных окрестностей, которые не пересекаются
    
    \textbf{Опр} \textit{База топологического пространства}
    
    \textbf{Лемма 3}
    
    Возьмём пересечения открытых шаров с рациональными радиусами и координатами
    
    \textbf{Опр} \textit{Непрерывное в точке отображение}
    
    \textbf{Опр} \textit{Секвенциально непрерывное в точке отображение}
    
    \textbf{Опр} \textit{(Секвенциально) непрерывное отображение}
    
    \textbf{Опр} \textit{Гомеоморфизм}
    
    \textbf{Лемма 4}
    
    Докажем от частного к общему с помощью непрерывности и открытости объединения открытых
    
    \textbf{Опр} \textit{Компактное топологическое пространство}
    
    \textbf{Опр} \textit{Секвенциально компактное топологическое пространство}
    
    \textbf{Опр} \textit{Секвенциально компактное множество}
    
    \textbf{Лемма 5}
    
    Возьмём открытое покрытие множества, перейдём к прообразам, выберем там открытое покрытие и конечное подпокрытие
    
    \textbf{Опр} \textit{Гомеоморфные множества}
    
    \textbf{Опр} \textit{Топологический инвариант}
    
    \textbf{Опр} \textit{Линейно-связное топологическое пространство}
    
    Компактность и линейная связность являются топологическими инвариантами, поскольку они сохраняются при любом
    непрерывном отображении
    
    \subsection{Карта и атлас на топологическом пространстве}
    
    \textbf{Опр} \textit{$n$-мерная карта на топологическом пространстве}
    
    \textbf{Опр} \textit{Гомеоморфизм карты, район действия карты, область параметров карты}
    
    \textbf{Опр} \textit{Атлас на топологическом пространстве}
    
    \subsection{Общие (абстрактные) определения многообразия и гладкого многобразия}
    
    \textbf{Опр} \textit{$n$-мерное абстрактное многообразие}
    
    \textbf{Опр} \textit{Замена координат, отображение перехода, отображение склейки}
    
    \subsection{Классы гладкости $C^k$ отображений из одного гладкого многообразия в другое}
    
    \textbf{Опр} \textit{Гладкий диффеоморфизм}
    
    \textbf{Опр} \textit{Гладкий атлас}
    
    Атлас на многообразии, состоящий из одной карты, считается гладким
    
    \textbf{Опр} \textit{Эквивалентные гладкие атласы}
    
    \textbf{Опр} \textit{Гладкая структура, определяемая атласом}
    
    Гладкая структура --- отношение эквивалентности
    
    \textbf{Опр} \textit{Гладкое $n$-мерное многообразие}
    
    \textbf{Опр} \textit{Карта на гладком многообразии}
    
    \textbf{Опр} \textit{Локальная система координат карты}
    
    \textbf{Опр} \textit{Класс $C^k$-гладких отображений}
    
    \textbf{Опр} \textit{Координатное представление отображения}
    
    \textbf{Опр} \textit{Диффеоморфные гладкие многообразия}
    
    \section{Теорема о гладком атласе на гладком подмногообразии пространства $\maTheorembb{R}^N$.
    Достаточное условие гладкости подмногообразия пространства $\maTheorembb{R}^N$ в терминах карты}
    
    \subsection{Теорема о гладком атласе на гладком подмногообразии пространства $\maTheorembb{R}^N$.}
    
    \textbf{Theorem} \textit{О гладком атласе на гладком подмногообразии}
    
    \begin{enumerate}
        \item По лемме, $\forall P$ найдётся карта на топологическом пространстве $M$, порождённая каноническим
        диффеоморфизмом, район действия которой содержит точку $P$.
        Семейство всех таких карт составляет атлас; покажем, что он гладкий
        \item Фиксируем $\forall P$, вводим новые обозначения и рассматриваем отображения замены координат
        \item Они будут состоять из суперпозиции гладких диффеоморфизмов, что докажет и диффеоморфность замены координат
    \end{enumerate}
    
    \subsection{Достаточное условие гладкости подмногообразия пространства $\maTheorembb{R}^N$ в терминах карты}
    
    \textbf{Опр} \textit{Порождённая каноническим диффеоморфизмом карта}
    
    \textbf{Опр} \textit{Порождённая каноническим диффеоморфизмом в некоторой окрестности точки карта.}
    
    \textbf{Theorem} \textit{Достаточное условие гладкости подмногообразия в терминах карты.}
    
    \begin{enumerate}
        \item Считаем $V$ открытым подмножеством согласно лемме и воспользуемся теоремой о ранге матрицы
        \item От исходного отображения перейдём к $f(x)$ с новыми обозначениями и запишем Матрицу Якоби отображения
        \item Увидим, что в $x_0$ матрица невырождена, что позволяет использовать теорему об обратном отображении
        \item $M$ будет задано простой системой уравнений, то есть имеет вид подпространства или полуподпространства
        \item Теперь докажем, что $M$ подмногообразие.
        В случае внутренней точки рассматриваем сужения и пересечения, вводя новые обозначения.
        \item Осталось показать, что параметры находились в линейной части пространства.
        В случае внутренней точки доказываем сначала прямое, а потом и обратное включения с помощью шаманства
        \item Случай граничной точки следует заменой подпространства на полуподпространство
        \item Согласно определению отображения $f$, справедливо равенство, которое и завершает доказательство
    \end{enumerate}
    
    \section{Касательный вектор к абстрактному гладкому многообразию как оператор дифференцирования.
    Теорема о структуре множества $T_P (M)$ касательных векторов.
    Изменение координат касательного вектора при замене локальной системы координат}
    
    \subsection{Касательный вектор к абстрактному гладкому многообразию как оператор дифференцирования}
    
    \textbf{Опр} \textit{Производная функции по вектору в точке}
    
    \textbf{Опр} \textit{Касательный вектор}
    
    Оператор обладает свойством локальности
    
    \textbf{Обозначение} \textit{Множество всех касательных векторов}
    
    \textbf{Опр} \textit{Соответствующие абстрактный и обычный касательный векторы}
    
    \textbf{Опр} \textit{Координаты касательного вектора}
    
    Касательный вектор является линейным оператором.
    Это следует из свойства линейности и правила Лейбница для производной функции одной переменной
    
    \subsection{Теорема о структуре множества $T_P (M)$ касательных векторов.}
    
    \textbf{Theorem} \textit{О структуре множества касательных векторов.}
    
    \begin{enumerate}
        \item Фиксируем произвольную ЛСК в окрестности точки и введём новые обозначения.
        \item Перейдём к равенствам для любой функции $f$.
        В итоге получим линейное пространство.
        \item Покажем, что коэффициенты разложения вектора по системе векторов определены однозначно.
        Это следует из существования соответствующего геометрического касательного вектора
        \item Итого, операторный набор составляет базис в $T_P (M)$
    \end{enumerate}
    
    \textbf{Опр} \textit{Производная функции по геометрическому касательному вектору}
    
    \textbf{Опр} \textit{Изоморфизмом, изоморфные линейные пространства}
    
    \subsection{Изменение координат касательного вектора при замене локальной системы координат}
    
    \textbf{Лемма}
    
    Доказывается записью вектора в двух базисах и с помощью теоремы о дифференцировании сложной функции
    
    \section{Край многообразия.
    Теорема о независимости краевой точки карты от карты}
    
    \subsection{Край многообразия}
    
    \textbf{Опр} \textit{Край допустимой области параметров}
    
    Край, вообще говоря, не совпадает с границей множества, потому как граничные точки могут не принадлежать множеству
    
    \textbf{Опр} \textit{Краевая точка карты}
    
    \subsection{Теорема о независимости краевой точки карты от карты.}
    
    \textbf{Theorem} \textit{О независимости краевой точки от карты.}
    
    \begin{enumerate}
        \item Докажем от противного: пусть краевая для одной карты и нет для другой
        \item $\exists$ две окрестности, операции с которым показывают, что $x_2$ внутренняя точка множества $V_2$.
        \item Сделаем замену координат, а потом и тождественное изображение.
        Получим невырожденность замены координат.
        \item Воспользуемся теоремой о неявной функции и получим окрестность точки $x_1$ в $V_1$ первой карты
        \item Таким образом, точка $x_1$ не лежит на границе области, а значит, $P$ не краевая точка карты, противоречие
    \end{enumerate}
    
    \textbf{Опр} \textit{Край гладкого многообразия}
    
    \section{Ориентация гладкого многообразия.
    Существование ровно двух ориентаций линейно-связного ориентируемого многообразия}
    
    \subsection{Ориентация гладкого многообразия}
    
    \textbf{Опр} \textit{Согласованные (по ориентации) карты}
    
    \textbf{Опр} \textit{Ориентирующий атлас}
    
    \textbf{Опр} \textit{(Не)ориентируемое многообразие}
    
    \textbf{Опр} \textit{Согласованные атласы}
    
    \textbf{Опр} \textit{Ориентация многообразия}
    
    \textbf{Опр} \textit{Ориентированное многообразие}
    
    \textbf{Опр} \textit{Карты, соответствующие ориентации (согласованные с ориентацией) гладкого многообразия}
    
    \subsection{Существование ровно двух ориентаций линейно-связного ориентируемого многообразия.}
    
    \textbf{Theorem} \textit{О двух ориентациях многообразия.}
    
    \begin{enumerate}
        \item Возьмём ориентирующий атлас и сделаем из него атлас с противоположной ориентацией.
        Заметим, что допустимая область параметров таковой остаётся
        \item Таким образом, существуют по крайней мере две различные ориентации многообразия -- это класс всех
        атласов, согласованных с первым или со вторым
        \item Покажем, что третьей ориентации многообразия не существует.
        Фиксируем ориентирующий атлас.
        \item Выберем $\forall P$ и проанализируем знак якобиана замены координат
        \item Знак якобиана непрерывно зависит от точки, притом не может обращаться в ноль как якобиан диффеоморфизма
        \item Итак, в зависимости от знака якобиана, кандидат согласован либо с первым, либо со вторым атласом
    \end{enumerate}
    
    \section{Ориентация гладкого $(N - 1)$-мерного подмногообразия пространства $\maTheorembb{R}^N$.
    Теорема о непрерывной нормали}
    
    \subsection{Ориентация гладкого $(N - 1)$-мерного подмногообразия пространства $\maTheorembb{R}^N$}
    
    \textbf{Опр} \textit{Единичный вектор нормали к многообразию}
    
    \textbf{Опр} \textit{Согласованные единичный вектор нормали и карта}
    
    \textbf{Опр} \textit{Согласованный с ориентацией многообразия вектор нормали}
    
    \subsection{Теорема о непрерывной нормали.}
    
    \textbf{Theorem} \textit{О непрерывной нормали}
    
    \begin{enumerate}
        \item $\Rightarrow$: покажем непрерывность функции.
        Зафиксируем $\forall P_0$ и покажем, что частичный предел последовательности стремящихся точек единственен.
        \item Зафиксируем карту с $P_0$ и перейдём к пределу в скалярных произведениях и по модулю
        \item Условие согласованности нормали и карты $\Leftrightarrow \det > 0$.
        Перейдя к пределу, получим равенство пределов нормалей, притом функция нормали непрерывна в любой точке $M$
        \item $\Leftarrow$: перейдём к карте и возьмём правый базис (или изменим до правого)
        \item Пользуемся непрерывностью и линейной связностью для получения правого базиса в каждой точке
        \item Полученный атлас будет ориентирующим, так как все карты имеют правые тройки, поэтому они согласованы
    \end{enumerate}
    
    \section{Теорема о построении ориентирующего атласа для края многообразия на основе ориентирующего атласа
    исходного многообразия.
    Согласование ориентации гладкого многообразия и ориентации его края}
    
    \subsection{Теорема о построении ориентирующего атласа для края многообразия на основе ориентирующего атласа
    исходного многообразия}
    
    \textbf{Лемма 2}
    
    \begin{enumerate}
        \item Зафиксируем гладкий атлас и перейдём от обычной окрестности к открытому шару или полушару
        \item Из предыдущей леммы $\Rightarrow \exists$ диффеоморфизм между область параметров и (полу)пространством
        \item Введём новые обозначения, получим гладкий атлас, а затем и гладкий атлас из (полу)пространства
    \end{enumerate}
    
    \textbf{Лемма 3}
    
    \begin{enumerate}
        \item Пользуясь предыдущей леммой, получим гладкий атлас с нужной областью параметров
        \item У всех карт, ориентации которых не совпадают с ориентацией $M$, поменяем вторую координату (область
        параметров, заметим, не изменится)
    \end{enumerate}
    
    \textbf{Theorem} \textit{О построении ориентирующего атласа для края многообразия}
    
    \begin{enumerate}
        \item Сначала получим гладкий атлас на крае, а затем покажем, что $\forall$ две карты атласа согласованы
        \item От открытого образа перейдём к открытому прообразу и рассмотрим отображение замены координат.
        Покажем, что якобиан сужения на крае положителен (то есть атлас ориентирующий)
        \item Зафиксируем краевую точку и посчитаем частные производные по координатам.
        \item Перейдём к якобианам и получим, что атлас на краю действительно ориентирующий (исходный таковой по условию)
    \end{enumerate}
    
    \subsection{Согласование ориентации гладкого многообразия и ориентации его края}
    
    \textbf{Опр} \textit{Согласованная ориентация края и многообразия}
    
    \textbf{Опр} \textit{Перенос ориентации между многообразиями}
    
    \section{Тензорное поле на многообразии.
    Изменение компонент тензорного поля при замене локальной системы координат.
    Выражение тензорного поля через его компоненты с помощью операции тензорного произведения}
    
    \subsection{Тензорное поле на многообразии}
    
    \textbf{Опр} \textit{Тензор на пространстве}
    
    \textbf{Опр} \textit{Компоненты (координаты) тензора}
    
    \textbf{Опр} \textit{Тензорное произведение тензора}
    
    Оно не коммутативно
    
    \textbf{Опр} \textit{Векторное, ковекторное поле, скалярное поле}
    
    \textbf{Опр} \textit{Тензорное поле}
    
    \textbf{Опр} \textit{Компоненты тензорного поля}
    
    \subsection{Изменение компонент тензорного поля при замене локальной системы координат.}
    
    \textbf{Theorem} \textit{О законе изменения компонент тензорного поля при замене ЛСК}
    
    Перейдём к другим координатам по формуле, подставим это в старый тензор, вынесем константы и получим новую формулу
    
    \subsection{Выражение тензорного поля через его компоненты с помощью операции тензорного произведения}
    
    \textbf{Опр} \textit{Сумма двух тензорных полей (одинакового типа)}
    
    \textbf{Опр} \textit{Произведение тензорного и скалярного полей}
    
    \textbf{Опр} \textit{Тензорное произведение тензорных полей.}
    
    \textbf{Theorem}
    
    \section{Дифференциальные формы на гладком многообразии, их представление через внешнее произведение
    дифференциалов координатных функций.
    Внешний дифференциал дифференциальной формы, его независимость от локальной системы координат}
    
    \subsection{Дифференциальные формы на гладком многообразии, их представление через внешнее произведение
    дифференциалов координатных функций}
    
    \textbf{Опр} \textit{Внешняя форма}
    
    \textbf{Опр} \textit{Перестановка чисел, транспозиция}
    
    \textbf{Опр} \textit{Знак перестановки}
    
    \textbf{Опр} \textit{Перестановка набора элементов}
    
    \textbf{Опр} \textit{Альтернирование тензора}
    
    \textbf{Опр} \textit{Внешнее произведение внешних форм} \\
    
    \textbf{Опр} \textit{Дифференциальная форма}
    
    \textbf{Опр} \textit{Множество гладких дифференциальных форм}
    
    \textbf{Опр} \textit{Внешнее произведение дифференциальных форм}
    
    \textbf{Опр} \textit{Моном}
    
    \textbf{Опр} \textit{Внешний дифференциал}
    
    Из определения внешнего дифференциала и линейности дифференциала скалярного поля следует линейность внешнего
    дифференциала формы
    
    \subsection{Внешний дифференциал дифференциальной формы, его независимость от локальной системы координат}
    
    \textbf{Theorem} \textit{Об инвариантности внешнего дифференциала формы относительно ЛСК}
    
    \begin{enumerate}
        \item Сначала рассмотрим скалярное поле ($q = 0$), для которого утверждение верно из прошлых параграфов
        \item В случае $q = 1$ покажем, что мы доказываем и заменим компоненты тензорного поля при замене ЛСК.
        \item Сведём к доказываемому, преобразуем и проверим требуемое равенство
        \item Переобозначим индексы суммирования, воспользуемся кососимметричностью и получим требуемое
        \item Случай $q > 1$ рассматривается аналогично
    \end{enumerate}
    
    \section{Правило Лейбница для внешнего дифференциала внешнего произведения двух дифференциальных форм.}
    
    \textbf{Theorem} \textit{Правило Лейбница для внешнего дифференциала.}
    
    \begin{enumerate}
        \item Запишем общий вид дифференциальной формы.
        В силу линейности достаточно рассмотреть суммы из одного слагаемого.
        \item Используем определение внешнего дифференциала и правило Лейбница для скалярных полей
        \item Далее используем свойство дифференциальных форм по перестановкам и получаем требуемое
    \end{enumerate}
    
    \section{Перенос касательных векторов.
    Выражение для переноса базисного вектора касательного пространства через частные производные координатных функций
    отображения}
    
    \subsection{Перенос касательных векторов}
    
    \textbf{Опр} \textit{Прямой перенос}
    
    \subsection{Выражение для переноса базисного вектора касательного пространства через частные производные
    координатных функций отображения.}
    
    \textbf{Theorem} \textit{Выражение для прямого переноса базисного вектора касательного пространства через частные
    производные координатных функций отображения.}
    
    \begin{enumerate}
        \item Запишем переход к другой системе координат и преобразуем его
        \item Последовательно используем определение прямого переноса, требованием, чтобы функция зависела от
        аргумента производной при дифференцировании, производной сложной функции и ещё раз определением прямого переноса
    \end{enumerate}
    
    \section{Перенос дифференциальных форм при отображении многообразий.
    Выражение для переноса базисного вектора кокасательного пространства через частные производные координатных
    функций отображения}
    
    \subsection{Перенос дифференциальных форм при отображении многообразий}
    
    \textbf{Опр} \textit{Дифференциал отображения, касательное отображение}
    
    \textbf{Опр} \textit{Обратный перенос}
    
    \subsection{Выражение для переноса базисного вектора кокасательного пространства через частные производные
    координатных функций отображения.}
    
    \textbf{Theorem} \textit{Выражение для обратного переноса базисного вектора кокасательного пространства через частные
    производные координатных функций отображения}
    
    \section{Обратный перенос дифференциальных форм при суперпозиции отображений многообразий.
    Коммутативность операций внешнего дифференцирования и обратного переноса дифференциальной формы}
    
    \subsection{Обратный перенос дифференциальных форм при суперпозиции отображений многообразий}
    
    \textbf{Лемма 2}
    
    Первый пункт доказывается по конспекту, а второй лучше всего по лекции 9 (1:25:00)
    
    \subsection{Коммутативность операций внешнего дифференцирования и обратного переноса дифференциальной формы.}
    
    \textbf{Theorem} \textit{О коммутативности внешнего дифференцирования и отображения обратного переноса.}
    
    \begin{enumerate}
        \item Рекомендуется доказывать по лекции 9 (1:50:00)
        \item Сначала рассмотрим случай скалярной функции и воспользуемся определением дифференциала отображения и Л2
        \item В общем случае рассмотрим лишь моном (из-за линейности).
        Запишем обратный перенос формы, возьмём дифференциал и применим правило Лейбница (для последнего и остальных
        слагаемого)
        \item Второе слагаемое в этой формуле будет ноль, поэтому слагаемые будут \("\)выноситься\("\).
        \item После вынесения всего воспользуемся случаем скалярной функции и получим требуемое.
    \end{enumerate}
    
    \textbf{Опр} \textit{Гладкое подмногообразие, каноническая ЛСК}
    
    \section{Теорема о разбиении единицы на многообразии}
    
    \textbf{Опр} \textit{Компактное многообразие}
    
    \textbf{Лемма 1}
    
    \begin{enumerate}
        \item Введём обозначение допустимой области параметров и $x_0$.
        \item Возьмём окрестность поменьше и воспользуемся леммой о функции шапочки.
        Соорудим новую функцию и обозначение и получим требуемое
    \end{enumerate}
    
    \textbf{Theorem} \textit{О разбиении единицы на многообразии}
    
    \begin{enumerate}
        \item Для любой точки выберем индекс окрестности, которой она принадлежит
        \item В силу леммы найдутся окрестность поменьше и гладкая функция.
        \item Выделим из компакта конечное подпокрытие и напишем пару следствий из леммы.
        \item Введём гладкие функции $\gamma$, используя лишь оставшуюся нам часть от единицы.
        Посчитаем, как выглядит разность этих функций и единицы.
        \item В силу второго следствия сумма $\gamma_i = 1$, но пока это разбиение не подчинено покрытию.
        \item Введём новые множества $S_i$, затем определим $\rho_i$.
        Сумма таковых по всем картам и будет равна единице
    \end{enumerate}
    
    \section{Определение интеграла от дифференциальной формы и его корректность.
    Криволинейный и поверхностный интегралы второго рода}
    
    \subsection{Определение интеграла от дифференциальной формы и его корректность}
    
    \textbf{Опр} \textit{Носитель дифференциальной формы}
    
    \textbf{Опр} \textit{Финитная дифференциальная форма}
    
    \textbf{Опр} \textit{Интеграл от дифференциальной формы по допустимой области параметров}
    
    \textbf{Опр} \textit{Интеграл от дифференциальной формы с носителем в районе действия одной карты}
    
    \textbf{Лемма}
    
    Благодаря лемме, можно сделать вывод, что оба определения интеграла совпадают (достаточно взять стандартную ЛСК)
    
    \textbf{Опр} \textit{Интеграл от дифференциальной формы по гладкому многообразию.}
    
    \textbf{Theorem} \textit{Корректность определения интеграла}
    
    \begin{enumerate}
        \item Существование интеграла следует из разбиения единицы, поэтому осталось доказать независимость от атласа.
        \item Введём новую функцию $\rho_{ij}$ и требуемые суммы просуммируем в нужном порядке
        Используя линейности и выражение для суммы $\rho_{ij}$, получим равенство
    \end{enumerate}
    
    Благодаря лемме, можно сделать вывод, что определения 2 и 3 интеграла совпадают (достаточно взять одну карту)
    
    \subsection{Криволинейный и поверхностный интегралы второго рода}
    
    \textbf{Опр} \textit{Криволинейный интеграл второго рода}
    
    Для него можно получить формулу через интеграл для отрезка.
    Для этого достаточно ввести гладкую параметризацию и перейти к другим координатам
    
    \textbf{Опр} \textit{Поверхность, поверхностный интеграл второго рода}
    
    Для него можно получить формулу через кратный интеграл по \("\)допустимой области\("\).
    Аналогично криволинейному случаю меняем переменные, применяем теорему о координатном представлении обратного
    переноса дифференциальной формы и соединяем всё в один интеграл
    
    \section{Теорема Стокса.}
    
    \textbf{Theorem} \textit{Стокса. }
    
    \section{Условия независимости криволинейного интеграла от пути интегрирования.
    Связь условий точности и замкнутости дифференциальных форм}
    
    \subsection{Условия независимости криволинейного интеграла от пути интегрирования}
    
    \textbf{Опр} \textit{Точная форма, обобщённый потенциал}
    
    \textbf{Опр} \textit{Потенциальное ковекторное поле, скалярный потенциал.}
    
    \textbf{Theorem} \textit{Условия независимости криволинейного интеграла от пути интегрирования}
    
    \begin{enumerate}
        \item $3 \Rightarrow 2$: используя представление через компоненты, сведём к производной по времени и после
        интегрирования получим зависимость от концов, которые совпадают в случае замкнутости.
        \item $2 \Rightarrow 1$: тривиально следует после смены ориентации второй кривой.
        \item $1 \Rightarrow 3$: фиксируем точку и вводим скалярный потенциал.
        \item Введём новые кривые и сведём интеграл по кривой к разности этого потенциала.
        \item Теперь рассмотрим малое приращение $\delta$ вдоль одной координаты $i$.
        Интеграл всей формы обнулится, за исключением компоненты с $i$.
        \item Воспользуемся малостью $\delta$, потом вычислим производную и получим требуемое после цепочки равенств
    \end{enumerate}
    
    \subsection{Связь условий точности и замкнутости дифференциальных форм}
    
    \textbf{Опр} \textit{Замкнутая форма}
    
    \textbf{Лемма}
    
    Необходимость очевидна.
    Недостаточность докажем от противного, рассмотрев форму дифференциала частного и её интеграл по тригонометру
    
    \section{Теорема о цепном равенстве.
    Лемма Пуанкаре}
    
    \subsection{Теорема о цепном равенстве}
    
    \textbf{Лемма} \textit{О цепном равенстве.}
    
    \begin{enumerate}
        \item Рекомендуется доказывать по лекции 11 (2:30:00)
        \item Введём ЛСК, запишем два возможных вида дифференциальной формы $\beta$ и определим действие на них.
        В результате получим линейное отображение.
        \item Рассмотрим действие на формах с $dt$ и докажем для них цепное равенство.
        Аналогично для форм без $dt$
        \item Любая форма может быть представлена как сумма конечного числа слагаемых с и без $dt$, поэтому доказано
    \end{enumerate}
    
    \textbf{Опр} \textit{Стягиваемое в точку многообразие}
    
    \textbf{Опр} \textit{Выпуклое множество}
    
    \subsection{Лемма Пуанкаре.}
    
    \textbf{Theorem} \textit{Лемма Пуанкаре}
    
    \begin{enumerate}
        \item Из точности замкнутость следует всегда, поэтому доказываем лишь в обратную сторону.
        \item Воспользуемся определением стягиваемости и рассмотрим дифференциальную форму $\beta$
        \item Применим лемму о цепном равенстве, преобразуем и получим требуемое равенство, из которой следует точность
    \end{enumerate}
    
    \section{Риманова метрика.
    Выражение для индуцированной римановой метрики в полярной системе координат на плоскости и сферической системе
    координат в трехмерном пространстве}
    
    \subsection{Риманова метрика}
    
    \textbf{Опр} \textit{Риманова метрика или (ковариантный) метрический тензор}
    
    \textbf{Опр} \textit{Риманово многообразие}
    
    \textbf{Опр} \textit{Скалярное произведение}
    
    \textbf{Опр} \textit{Длина, угол}
    
    \textbf{Опр} \textit{Матрица Грама на Риманово многообразие}
    
    \textbf{Опр} \textit{Индуцированная метрика}
    
    \subsection{Выражение для индуцированной римановой метрики в полярной системе координат на плоскости и
    сферической системе координат в трехмерном пространстве}
    
    Найдём компоненты метрического тензора индуцированной метрики.
    В итоге, матрица Грама индуцированной метрики определяется через матрицу Якоби гомеоморфизма карты
    
    Теперь можно вычислить матрицы Грама некоторых часто используемых метрик
    
    \textbf{Опр} \textit{Первая квадратичная форма (гипер) поверхности}
    
    Первая квадратичная форма является значением тензора, то она не зависит от ЛСК на поверхности
    
    \section{Определение формы Риманова объёма и ее связь с дифференциальной формой (тензором Леви-Чивиты).
    Определение интеграла первого рода скалярной функции по гладкому многообразию.
    Поток векторного поля через двумерное ориентируемое подмногообразие пространства $\maTheorembb{R}^3$, выражение
    потока через интеграл от дифференциальной формы и интеграл первого рода}
    
    \subsection{Определение формы Риманова объёма и ее связь с дифференциальной формой (тензором Леви-Чивиты)}
    
    \textbf{Опр} \textit{$n$-мерный параллелепипед в пространстве $\maTheorembb{R}^N$, $n$-мерный объём}
    
    \textbf{Опр} \textit{Форма Риманова объема}
    
    \textbf{Опр} \textit{Тензор Леви-Чивиты}
    
    \textbf{Theorem} \textit{О тензоре Леви-Чивиты}
    
    Достаточно показать, что при замене ЛСК значение формы не меняется
    
    \textbf{Опр} \textit{Символ Леви-Чивиты}
    
    \subsection{Определение интеграла первого рода скалярной функции по гладкому многообразию.}
    
    \textbf{Опр} \textit{Интеграл первого рода}
    
    \textbf{Опр} \textit{Риманов объём, площадь поверхности}
    
    \textbf{Пример} \textit{Длина кривой}
    
    \textbf{Пример} \textit{Площадь поверхности}
    
    \subsection{Поток векторного поля через двумерное ориентируемое подмногообразие пространства $\maTheorembb{R}^3$,
        выражение потока через интеграл от дифференциальной формы и интеграл первого рода}
    
    \textbf{Опр} \textit{Поток векторного поля через поверхность.}
    
    \textbf{Theorem} \textit{О выражении потока векторного поля через интеграл от дифференциальной формы}
    
    \section{Дивергенция и ротор векторного поля в области трехмерного евклидова пространства.
    Геометрический смысл дивергенции и ротора векторного поля.
    Условия существования скалярного и векторного потенциалов векторного поля в области трехмерного евклидова
    пространства}
    
    \subsection{Дивергенция и ротор векторного поля в области трехмерного евклидова пространства}
    
    \textbf{Опр} \textit{Свёртка тензора}
    
    \textbf{Опр} \textit{Свёртка тензорного поля}
    
    \textbf{Опр} \textit{Свёртка тензоров (или тензорных полей)}
    
    \textbf{Опр} \textit{Операция опускания индекса}
    
    \textbf{Опр} \textit{Контравариантный метрический тензор}
    
    \textbf{Опр} \textit{Ковариантные компоненты векторного поля}
    
    \textbf{Опр} \textit{Криволинейным интегралом второго рода}
    
    \textbf{Опр} \textit{Смешанное и векторное произведение векторных полей}
    
    Данные определения соответствуют стандартным определениям этих понятий
    
    \textbf{Опр} \textit{Градиент}
    
    \textbf{Опр} \textit{Дивергенция}
    
    \textbf{Опр} \textit{Ротор (вихрь)}
    
    \textbf{Лемма 1}
    
    Используем правило Лейбница; в последнем пункте следует перейти к рассмотрению отдельной компоненты
    
    \textbf{Лемма 2}
    
    \begin{enumerate}
        \item В первом пункте перейдём к рассмотрению отдельной компоненты и получим смешанную частную производную.
        \item Изменим порядок дифференцирования и изменится знак.
        Получили, что компонента равна минус себе, что возможно лишь в нулевом случае
        \item Во втором пункте действуем в лоб и сводим к первому пункту
        \item В третьем в лоб
    \end{enumerate}
    
    \textbf{Опр} \textit{Оператор Гамильтона}
    
    \subsection{Геометрический смысл дивергенции и ротора векторного поля.}
    
    \textbf{Theorem} \textit{Геометрическое определение дивергенции.}
    
    \textbf{Theorem} \textit{Геометрическое определение ротора}
    
    \subsection{Условия существования скалярного и векторного потенциалов векторного поля в области трехмерного
    евклидова пространства}
    
    \textbf{Опр} \textit{Скалярный потенциал}
    
    \textbf{Опр} \textit{Бизвихревое поле.}
    
    \textbf{Theorem} \textit{О существовании скалярного потенциала}
    
    \textbf{Опр} \textit{Векторный потенциал}
    
    \textbf{Опр} \textit{Бездивергентное поле.}
    
    \textbf{Theorem} \textit{О существовании векторного потенциала}
    
    \section{Определение производной Ли тензорного поля через его обратный перенос фазовым потоком.
    Выражение компонент производной Ли тензорного поля по векторному полю через компоненты этих полей.
    Выражение производной Ли для тензорных полей типов (0,0), (1,0) и (0,1)}
    
    \subsection{Определение производной Ли тензорного поля через его обратный перенос фазовым потоком}
    
    \textbf{Опр} \textit{Производная семейства тензорных полей в точке}
    
    \textbf{Опр} \textit{Фазовый поток}
    
    \textbf{Опр} \textit{Производная Ли}
    
    \subsection{Выражение компонент производной Ли тензорного поля по векторному полю через компоненты этих полей}
    
    \textbf{Лемма}
    
    \begin{enumerate}
        \item Рекомендуется доказывать по лекции 14 (2:35:00)
        \item Выразим $y(x)$ через координатное представление потока и запишем обратный перенос тензорного поля
        фазового потока.
        \item Возьмём $\frac{d y_t (x_0)}{dt}$ и разложим $y_t (x_0)$ по формуле Тейлора полностью и по координатам.
        \item Выразим отсюда компоненты обратного переноса явно и с помощью обратных матриц и разложения по Тейлору.
        \item Подставим всё в обратный перенос и перемножим скобки с точностью до $o(t)$.
        Используем свойство символа Кронекера и получаем итоговую формулу
    \end{enumerate}
    
    \subsection{Выражение производной Ли для тензорных полей типов (0,0), (1,0) и (0,1)}
    
    \textbf{Пример} \textit{Производная Ли скалярного поля}
    
    \textbf{Опр} \textit{Первый интеграл}
    
    \textbf{Пример} \textit{Производная Ли векторного поля}
    
    \textbf{Пример} \textit{Производная Ли ковекторного поля}
    
    \section{Коммутативность производной Ли и внешнего дифференциала формы.}
    
    \textbf{Theorem} \textit{О коммутативности производной Ли и внешнего дифференцирования}
    
    \section{Правило Лейбница для внутреннего произведения векторного поля на внешнее произведение двух
    дифференциальных форм}
    
    \textbf{Опр} \textit{Внутреннее произведение.}
    
    \textbf{Theorem} \textit{Правило Лейбница для внутреннего умножения}
    
    \section{Магическое тождество Картана.}
    
    \textbf{Theorem} \textit{Тождество Картана (формула гомотопии)}
    
    Доказывается аналогично правилу Лейбница для внутреннего умножения

\end{document}
