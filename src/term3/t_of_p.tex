%! Author = user
%! Date = 18.01.2024

\documentclass[a4paper, 14pt]{article}
%\documentclass[draft]{article}

\usepackage[T2A]{fontenc}
\usepackage[utf8]{inputenc}
\usepackage[english, russian]{babel}
\usepackage[top = 2cm, bottom = 2cm, left = 2cm, right = 2cm]{geometry}
\usepackage{indentfirst}
\usepackage{xcolor}
\usepackage{hyperref}
\usepackage{gensymb}
\usepackage{pgfplots}
\usepackage{amsmath, amsfonts, amsthm, mathtools}
\usepackage{amssymb}
\usepackage{physics, multirow, float}
\usepackage{wrapfig, tabularx}
\usepackage{icomma} % Clever comma: 0,2 - number while 0, 2 - two numbers
\usepackage{tikz, standalone}
\usepackage{fancyhdr,fancybox}
\usepackage{booktabs}
\usepackage{listings}
\usepackage{lstmisc}
\usepackage{stmaryrd}
\usepackage{lastpage}

%\полуторный интервал
\onehalfspacing

\hypersetup
{   colorlinks = false,
    linkcolor = blue,
    pdftitle = {analysis},
    pdfauTheoremor = {Володин Максим},
    allcolors = [RGB]{010 090 200}
}

%\gravarphicspaTheorem{{images/}}
%\DeclareGravarphicsExtensions{.pdf,.png,.jpg}

\restylefloat{table}
\usetikzlibrary{external}

\mathtoolsset{showonlyrefs = true} % Numbers will appear only where \eqref{} in Theoreme text LINKED
\pagestyle{fancy}

\fancyhf{}
\fancyhead[L]{Теория вероятностей}
\fancyhead[R]{Конспект билетов}
\fancyfoot[L]{}
\fancyfoot[R]{\thepage /\pageref{LastPage}}

\pgfplotsset{compat=1.18}

\begin{document}

{\huge
    \begin{center}
    {\textbf{Конспект билетов}}
        \\
        Теория вероятностей
    \end{center}
}
    \tableofcontents \newpage
    
    \section{Дискретное вероятностное пространство, классическая вероятность, геометрическая вероятность}
    
    \textbf{Опр} \textit{Элементарные события (исходы) и и х пространство}
    
    \textbf{Опр} \textit{Дискретное вероятностное пространство}
    
    \textbf{Опр} \textit{Классическая вероятностная модель}
    
    \textbf{Опр} \textit{Модель геометрической вероятности}
    
    \section{Колмогоровское определение вероятностного пространства.
    Свойства вероятности}
    
    \subsection{Колмогоровское определение вероятностного пространства}
    
    \textbf{Опр} \textit{Событие и вероятностная мера (вероятность)}
    
    \subsection{Свойства вероятности}
    
    Вероятность обладает 7 свойствами; доказывать некоторые из них лучше, опираясь на рисунок
    
    \textbf{Опр} \textit{Вероятностное пространство}
    
    \section{Независимость событий, условная вероятность, формула полной вероятности, формула Байеса}
    
    \textbf{Опр} \textit{Независимые события}
    
    \textbf{Опр} \textit{Условная вероятность}
    
    \textbf{Опр} \textit{Разбиение}

    \textbf{Theorem} \textit{Формула полной вероятности}
    
    \textbf{Theorem} \textit{Формула Байеса}
    
    \section{Схема испытаний Бернулли: два определения и их эквивалентность}
    
    \textbf{Опр} \textit{Попарно независимые события}
    
    \textbf{Опр} \textit{Независимые в совокупности события}
    
    \textbf{Опр} \textit{Схема испытаний Бернулли}
    
    \section{Распределения в $\mathbb{R}$, функция распределения и её свойства.
    Теорема о построении вероятностной меры на ($\mathbb{R}, \mathcal{B}(\mathbb{R}$)) по функции распределения}
    
    \subsection{Распределения в $\mathbb{R}$, функция распределения и её свойства}
    
    \textbf{Опр} \textit{Борелевская сигма-алгебра $\mathcal{B}(\mathbb{R}$}
    
    \textbf{Опр} \textit{Распределение вероятностей}
    
    \textbf{Опр} \textit{Функция распределения}
    
    \textbf{Свойства} \textit{Функции распределения}
    
    Первое свойство тривиально.
    Во втором надо понимать связь пределов и параметров.
    С третьим я не согласен
    
    \subsection{Теорема о построении вероятностной меры на ($\mathbb{R}, \mathcal{B}(\mathbb{R}$)) по функции
    распределения}
    
    \textbf{Theorem}
    
    \section{Дискретные и абсолютно непрерывные распределения в $\mathbb{R}$.
    Плотность.
    Связь плотности и функции распределения.
    Примеры}
    
    \subsection{Дискретные и абсолютно непрерывные распределения в $\mathbb{R}$}
    
    \textbf{Опр} \textit{Дискретные распределения вероятности}
    
    \subsection{Плотность}
    
    \textbf{Опр} \textit{Абсолютно непрерывное распределение, его плотность}
    
    \subsection{Связь плотности и функции распределения}
    
    Плотность и функция распределения связаны формулой Ньютона--Лейбница
    
    \subsection{Примеры}
    
    By the text
    
    \section{Распределения в $\mathbb{R}^n$, функция распределения и её свойства.
    Теорема о построении вероятностной меры на ($\mathbb{R}^n, \mathcal{B}(\mathbb{R}^n$) по функции распределения
        (б/д)}
    
    \subsection{Распределения в $\mathbb{R}^n$, функция распределения и её свойства}
    
    By the text
    
    Многомерная функция распределения обладает 3 свойствами.
    Первое доказывается вводом функции одной переменной (как с частными производными), а затем совместно с остальным
    свойствами доказывается аналогично одномерному случаю
    
    \subsection{Теорема о построении вероятностной меры на ($\mathbb{R}^n, \mathcal{B}(\mathbb{R}^n$) по функции
    распределения (б/д)}
    
    By the text
    
    \section{Дискретные и абсолютно непрерывные распределения в $\mathbb{R}^n$.
    Плотность.
    Связь плотности и функции распределения.
    Примеры}
    
    By the text
    
    \section{Случайные величины и случайные векторы.
    Характеристики случайной величины (вектора): распределение вероятностей, функция распределения, плотность.
    Действия над случайными величинами (векторами)}
    
    \subsection{Случайные величины и случайные векторы}
    
    By the text
    
    \subsection{Характеристики случайной величины (вектора): распределение вероятностей, функция распределения, плотность}
    
    By the text
    
    \textbf{Утв}
    
    Для доказательства необходимости поместим борелевское множество на $i$ место и воспользуемся сохранением
    бореливости.
    Для достаточно распишем, что такое декартово произведение и воспользуемся свойством сигма-алгебры
    
    \subsection{Действия над случайными величинами (векторами)}
    
    \textbf{Theorem}
    
    Доказывается по определению случайного вектора
    
    \section{Теорема о плотности $\varphi(\xi)$.
    Маргинальные распределения.
    Вычисление маргинальной плотности}
    
    \subsection{Теорема о плотности $\varphi(\xi)$}
    
    \textbf{Theorem}
    
    \begin{enumerate}
        \item Перейдём к интегралу плотности вероятности и разделим $M$ на два множества
        \item Та часть, что не пересекается с областью значений, вклад в интеграл давать не будет.
        \item Сделаем замену переменных в соответствии с теоремой и проследим, что множество интегрирования верно
    \end{enumerate}
    
    \subsection{Маргинальные распределения}
    
    By the text
    
    \subsection{Вычисление маргинальной плотности}
    
    \textbf{Theorem} \textit{Вычисление маргинальной плотности}
    
    Расписываем плотность по определению и находим общие части с переменными, которые могут именоваться по-другому
    
    \section{Случайные величины и случайные векторы.
    Независимость и критерий независимости.
    Независимость функций от векторов}
    
    \subsection{Случайные величины и случайные векторы. Независимость и критерий независимости}
    
    By the text
    
    \textbf{Theorem} \textit{Критерий независимости}
    
    Необходимость докажем, используя определение декартова произведения
    Для доказательства достаточности распишем вторую разность
    
    By the text
    
    \subsection{Независимость функций от векторов}
    
    By the text
    
    \section{Независимость случайных величин. Формула свёртки и её обобщения для разности, произведения и частного}
    
    By the text
    
    \section{Математическое ожидание случайной величины: дискретные и абсолютно непрерывные величины.
    Примеры}
    
    By the text
    
    \section{Основные свойства математического ожидания.
    Математическое ожидание произведения независимых величин}
    
    By the text
    
    В 7 свойстве расписываем заведомо неотрицательную величину и получаем условие на детерминант, откуда получаем
    требуемое.
    В 8 лучше сразу записать произведение детерминантов
    
    \section{Теорема о замене переменных в интеграле Лебега (б/д).
    Подсчёт математического ожидания от функции от случайной величины.
    Примеры}
    
    By the text
    
    \section{Дисперсия, ковариация, корреляция и их свойства.
    Примеры}
    
    \textbf{Опр} \textit{Дисперсия}
    
    \textbf{Свойства} \textit{Дисперсии}
    
    Первое свойство доказывается в силу линейности матожидания (выносим, где надо, скаляры).
    Это и последнее свойство можно будет доказать с помощью ковариации
    
    By the text
    
    \textbf{Утв}
    
    Лучше доказывать по \href{https://ru.wikipedia.org/wiki/Корреляция}{Википедии}
    
    \section{Неравенство Коши – Буняковского.
    Неравенство Маркова.
    Неравенство Чебышёва.
Закон больших чисел в форме Чебышёва}
    
    By the text
    
    \section{Виды сходимостей и взаимосвязи между ними}
    
    By the text
    
    \textbf{Theorem} \textit{О связи видов сходимости}
    
    \begin{enumerate}
        \item С помощью трёх последовательностей
        \item Аналогично.
        \item Пользуемся ограниченностью случайной величины, определением непрерывности и в конце расписываем
        матожидание по линейности: $X$ ограничен из ограниченности индикатора, $Y$ --- из определения непрерывности,
        $Z$ --- из ограниченности функции на том промежутке
    \end{enumerate}
    
    \section{Виды сходимостей. Критерий сходимости п.н.
    Теорема Рисса и её следствие.
    Наследование сходимости при арифметических операциях}
    
    \subsection{Виды сходимостей. Критерий сходимости п.н.}
    
    \textbf{Theorem} \textit{Критерий сходимости п.н.}
    
    Рассмотрим определение несходимости и введём обозначение для кванторной записи.
    Смотрим объединение, затем переходим к всеобъемлющему обозначению и к пределу в силу свойства меры
    
    \subsection{Теорема Рисса и её следствие}
    
    \textbf{Theorem} \textit{Рисса.}
    
    \begin{enumerate}
        \item Записываем определения сходимости по вероятности и предела.
        \item Переходим к объединению по $k$ без $\varepsilon$, затем череда сравнений и неравенств предельный переход
        \item В конце пользуемся критерием сходимости п.н.
    \end{enumerate}
    
    By the text
    
    В доказательстве наследования для сходимости по распределению везде пользуемся следствием теоремы Рисса
    
    \section{Критерий слабой сходимости в терминах функции распределения (б/д).
    Центральная предельная теорема (б/д).
    Переформулировка в интегральном виде.
    Теорема Муавра-Лапласа: локальная и интегральная}
    
    By the text
    
    \subsection{Теорема Муавра-Лапласа: локальная и интегральная}
    
    Лучше всего доказывать по лекции 9 (49:15)
    
    \textbf{Theorem} \textit{Муавра-Лапласа локальная}
    
    \begin{enumerate}
        \item Воспользуемся формулой Стирлинга для всех факториалов
        \item Неравенство с $\varphi$ сведём к равномерной сходимости.
        \item Распишем вероятность суммы и перейдём к задаче сведения степеней к экспоненте.
        Для этого прологарифмируем, введём функцию, посчитаем её производные в разных точках.
        \item Воспользуемся остаточным членом в форме Лагранжа.
        \item Перейдём к равномерной сходимости и получим требуемое
    \end{enumerate}
    
    \textbf{Theorem} \textit{Муавра-Лапласа интегральная}
    
    Интегральная теорема следует из локальной.
    Надо лишь расписать сумму, вычленить мелкость разбиения и перейти к пределу, то есть интегралу
    
    \section{Закон больших чисел и усиленный закон больших чисел (б/д)}
    
    By the text
    
    \section{Предельная теорема Пуассона.}
    
    \textbf{Theorem} \textit{Предельная Пуассона}
    
    Доказывается расписыванием левой и правой частей по определению и сравнением в конце
    
    \section{Гауссовские векторы. Эквивалентность определений (доказательство в одну сторону).
    Теорема о том, что распределение гауссовского вектора однозначно задаётся ковариационной матрицей, вектором средних.
    Плотность гауссовского вектора.
    Независимость компонент}
    
    \subsection{Гауссовские векторы. Эквивалентность определений (доказательство в одну сторону).}
    
    \textbf{Theorem} \textit{О равносильных определениях гауссовского вектора}
    
    \begin{enumerate}
        \item $1 \Rightarrow 2$: записываем производящую функцию сдвига вектора.
        \item Перейдём к диагональной матрице, но не с помощью одного поворота, а двух, где второй хитро задан.
        \item Посчитаем характеристическую функцию нового вектора и получим состав этого нового вектора.
        \item Вернёмся к исходному вектору и получим требуемое.
    \end{enumerate}
    
    \subsection{Плотность гауссовского вектора.}
    
    \textbf{Theorem} \textit{О плотности}
    
    \begin{enumerate}
        \item Рассмотрим несодержательный вырожденный случай и перейдём к невырожденному.
        \item Введём матрицу $A$ и рассмотрим диффеоморфизм с ней связанный.
        \item Перейдём к обратному диффеоморфизму и вычислим плотность уже требуемого вектора
    \end{enumerate}
    
    \subsection{Теорема о том, что распределение гауссовского вектора однозначно задаётся ковариационной матрицей,
        вектором средних}
    
    \subsection{Независимость компонент}
    
    \textbf{Следствие}
    
    Действительно, сведём к характеристической функции и зафиксируем отсутствие недиагональных компонент
    
    \section{Случайные процессы.
    Дискретное и непрерывное время.
    Траектории случайного процесса.
    Примеры}
    
    By the text
    
    \section{Симметричное случайное блуждание на прямой.
    Траектории.
    Распределение: $P(S_n = k)$.
    Принцип отражения и вероятность возвращения в нуль}
    
    By the text
    
    \subsection{Принцип отражения и вероятность возвращения в нуль.}
    
    \begin{enumerate}
        \item Сведём задачу к точке $(1,1)$ и посчитаем для конкретного $n$.
        \item Просуммируем для всех $n$.
        Сумму можно посчитать явно и после разных подстановок получить требуемое
    \end{enumerate}
    
    \section{Распределение максимума случайного блуждания. Закон повторного логарифма (б/д)}
    
    By the text
    
    В следствии из распределения модуля используем факт: $P(S_n) = 1 - F_{S_n}(x-1), P(M_n) = 1 - F_{M_n}(x-1)$
    
    By the text
    
    \section{Ветвящийся процесс Гальтона–Ватсона.
    Производящая функция и её свойства.
    Вероятность вырождения и технология ее вычисления}
    
    \subsection{Определение формы Риманова объёма и ее связь с дифференциальной формой (тензором Леви-Чивиты)}
    
    \textbf{Опр} \textit{$n$-мерный параллелепипед в пространстве $\maTheorembb{R}^N$, $n$-мерный объём}
    
    \textbf{Опр} \textit{Форма Риманова объема}
    
    \textbf{Опр} \textit{Тензор Леви-Чивиты}
    
    \textbf{Theorem} \textit{О тензоре Леви-Чивиты}
    
    Достаточно показать, что при замене ЛСК значение формы не меняется
    
    \textbf{Опр} \textit{Символ Леви-Чивиты}
    
    \subsection{Определение интеграла первого рода скалярной функции по гладкому многообразию.}
    
    \textbf{Опр} \textit{Интеграл первого рода}
    
    \textbf{Опр} \textit{Риманов объём, площадь поверхности}
    
    \textbf{Пример} \textit{Длина кривой}
    
    \textbf{Пример} \textit{Площадь поверхности}
    
    \subsection{Поток векторного поля через двумерное ориентируемое подмногообразие пространства $\maTheorembb{R}^3$,
        выражение потока через интеграл от дифференциальной формы и интеграл первого рода}
    
    \textbf{Опр} \textit{Поток векторного поля через поверхность.}
    
    \textbf{Theorem} \textit{О выражении потока векторного поля через интеграл от дифференциальной формы}
    
    \section{Дивергенция и ротор векторного поля в области трехмерного евклидова пространства.
    Геометрический смысл дивергенции и ротора векторного поля.
    Условия существования скалярного и векторного потенциалов векторного поля в области трехмерного евклидова
    пространства}
    
    \subsection{Дивергенция и ротор векторного поля в области трехмерного евклидова пространства}
    
    \textbf{Опр} \textit{Свёртка тензора}
    
    \textbf{Опр} \textit{Свёртка тензорного поля}
    
    \textbf{Опр} \textit{Свёртка тензоров (или тензорных полей)}
    
    \textbf{Опр} \textit{Операция опускания индекса}
    
    \textbf{Опр} \textit{Контравариантный метрический тензор}
    
    \textbf{Опр} \textit{Ковариантные компоненты векторного поля}
    
    \textbf{Опр} \textit{Криволинейным интегралом второго рода}
    
    \textbf{Опр} \textit{Смешанное и векторное произведение векторных полей}
    
    Данные определения соответствуют стандартным определениям этих понятий
    
    \textbf{Опр} \textit{Градиент}
    
    \textbf{Опр} \textit{Дивергенция}
    
    \textbf{Опр} \textit{Ротор (вихрь)}
    
    \textbf{Лемма 1}
    
    Используем правило Лейбница; в последнем пункте следует перейти к рассмотрению отдельной компоненты
    
    \textbf{Лемма 2}
    
    \begin{enumerate}
        \item В первом пункте перейдём к рассмотрению отдельной компоненты и получим смешанную частную производную.
        \item Изменим порядок дифференцирования и изменится знак.
        Получили, что компонента равна минус себе, что возможно лишь в нулевом случае
        \item Во втором пункте действуем в лоб и сводим к первому пункту
        \item В третьем в лоб
    \end{enumerate}
    
    \textbf{Опр} \textit{Оператор Гамильтона}
    
    \subsection{Геометрический смысл дивергенции и ротора векторного поля.}
    
    \textbf{Theorem} \textit{Геометрическое определение дивергенции.}
    
    \textbf{Theorem} \textit{Геометрическое определение ротора}
    
    \subsection{Условия существования скалярного и векторного потенциалов векторного поля в области трехмерного
    евклидова пространства}
    
    \textbf{Опр} \textit{Скалярный потенциал}
    
    \textbf{Опр} \textit{Бизвихревое поле.}
    
    \textbf{Theorem} \textit{О существовании скалярного потенциала}
    
    \textbf{Опр} \textit{Векторный потенциал}
    
    \textbf{Опр} \textit{Бездивергентное поле.}
    
    \textbf{Theorem} \textit{О существовании векторного потенциала}
    
    \section{Определение производной Ли тензорного поля через его обратный перенос фазовым потоком.
    Выражение компонент производной Ли тензорного поля по векторному полю через компоненты этих полей.
    Выражение производной Ли для тензорных полей типов (0,0), (1,0) и (0,1)}
    
    \subsection{Определение производной Ли тензорного поля через его обратный перенос фазовым потоком}
    
    \textbf{Опр} \textit{Производная семейства тензорных полей в точке}
    
    \textbf{Опр} \textit{Фазовый поток}
    
    \textbf{Опр} \textit{Производная Ли}
    
    \subsection{Выражение компонент производной Ли тензорного поля по векторному полю через компоненты этих полей}
    
    \textbf{Лемма}
    
    \begin{enumerate}
        \item Рекомендуется доказывать по лекции 14 (2:35:00)
        \item Выразим $y(x)$ через координатное представление потока и запишем обратный перенос тензорного поля
        фазового потока.
        \item Возьмём $\frac{d y_t (x_0)}{dt}$ и разложим $y_t (x_0)$ по формуле Тейлора полностью и по координатам.
        \item Выразим отсюда компоненты обратного переноса явно и с помощью обратных матриц и разложения по Тейлору.
        \item Подставим всё в обратный перенос и перемножим скобки с точностью до $o(t)$.
        Используем свойство символа Кронекера и получаем итоговую формулу
    \end{enumerate}
    
    \subsection{Выражение производной Ли для тензорных полей типов (0,0), (1,0) и (0,1)}
    
    \textbf{Пример} \textit{Производная Ли скалярного поля}
    
    \textbf{Опр} \textit{Первый интеграл}
    
    \textbf{Пример} \textit{Производная Ли векторного поля}
    
    \textbf{Пример} \textit{Производная Ли ковекторного поля}
    
    \section{Коммутативность производной Ли и внешнего дифференциала формы.}
    
    \textbf{Theorem} \textit{О коммутативности производной Ли и внешнего дифференцирования}
    
    \section{Правило Лейбница для внутреннего произведения векторного поля на внешнее произведение двух
    дифференциальных форм}
    
    \textbf{Опр} \textit{Внутреннее произведение.}
    
    \textbf{Theorem} \textit{Правило Лейбница для внутреннего умножения}
    
    \section{Магическое тождество Картана.}
    
    \textbf{Theorem} \textit{Тождество Картана (формула гомотопии)}
    
    Доказывается аналогично правилу Лейбница для внутреннего умножения

\end{document}
