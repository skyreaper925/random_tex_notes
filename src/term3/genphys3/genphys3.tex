%! Author = user
%! Date = 23.12.2023

\documentclass[a4paper, 14pt]{article}
%\documentclass[draft]{article}

\usepackage[T2A]{fontenc}
\usepackage[utf8]{inputenc}
\usepackage[english, russian]{babel}
\usepackage[top = 2cm, bottom = 2cm, left = 2cm, right = 2cm]{geometry}
\usepackage{indentfirst}
\usepackage{xcolor}
\usepackage{hyperref}
\usepackage{gensymb}
\usepackage{pgfplots}
\usepackage{amsmath, amsfonts, amsthm, mathtools}
\usepackage{amssymb}
\usepackage{physics, multirow, float}
\usepackage{wrapfig, tabularx}
\usepackage{icomma} % Clever comma: 0,2 - number while 0, 2 - two numbers
\usepackage{tikz, standalone}
\usepackage{fancyhdr,fancybox}
\usepackage{lastpage}
\usepackage{booktabs}
\usepackage{listings}
\usepackage{lstmisc}
\usepackage{stmaryrd}

%\полуторный интервал
\onehalfspacing

\hypersetup
{   colorlinks = false,
    linkcolor = blue,
    pdftitle = {genphys3},
    pdfauthor = {Володин Максим},
    allcolors = [RGB]{010 090 200}
}

%\gravarphicspath{{images/}}
%\DeclareGravarphicsExtensions{.pdf,.png,.jpg}

\restylefloat{table}
\usetikzlibrary{external}

\mathtoolsset{showonlyrefs = true} % Numbers will appear only where \eqref{} in the text LINKED
\pagestyle{fancy}

\fancyhf{}
\fancyhead[L]{Общая физика. Электричество и магнетизм}
\fancyhead[R]{Конспект билетов}
\fancyfoot[L]{}
\fancyfoot[R]{\thepage /\pageref{LastPage}}

\pgfplotsset{compat=1.18}

\begin{document}
    
    \tableofcontents \newpage
    
    \section{Электрические заряды и электрическое поле.
    Закон сохранения заряда, элементарный заряд.
    Напряжённость электрического поля.
    Закон Кулона.
    Гауссова система единиц (СГС) и система СИ.
    Принцип суперпозиции.
    Электрическое поле диполя}
    
    \subsection{Электрические заряды и электрическое поле}
    
    \textbf{Опр} \textit{Электромагнитное поле} \textcolor{gray}{Создаваемое электрическими телами и действующее на ...}
    
    \textbf{Опр} \textit{Электрический заряд}
    
    Существуют лишь положительные и отрицательные заряды
    
    \subsection{Закон сохранения заряда, элементарный заряд}
    
    \textbf{Закон} \textit{Сохранения заряда}
    
    \textbf{Опр} \textit{Элементарный заряд} \textcolor{gray}{Заряд электрона с противоположным знаком}
    
    Любой заряд кратен элементарному заряду, равному $e = 1,6 \cdot 10^{-19} Kl = 4,8 \cdot 10^{-10}$
    
    \subsection{Закон Кулона}
    
    \textbf{Закон} \textit{Кулона}
    
    \subsection{Напряжённость электрического поля}
    
    \textbf{Опр} \textit{Напряжённость электрического поля в точке}
    
    \textbf{Опр} \textit{Силовая линия}
    
    \subsection{Гауссова система единиц (СГС) и система СИ}
    
    \textbf{Опр} \textit{Гауссова система единиц (СГС)}
    
    \textcolor{blue}{Система единиц измерения, в которой основными единицами являются единица длины сантиметр,
        единица массы грамм и единица времени секунда}
    
    Она широко использовалась до принятия Международной системы единиц (СИ).
    Другое название — абсолютная физическая система единиц
    
    \textbf{Опр} \textit{Система СИ}
    
    \textcolor{blue}{Система единиц, основанная на Международной системе величин, вместе с наименованиями и обозначениями, а также
    набором приставок и их наименованиями и обозначениями вместе с правилами их применения, принятая Генеральной
    конференцией по мерам и весам (CGPM)}
    
    В качестве основных физических величин в ISQ используются длина $l$, масса $m$, время $t$, электрический
    ток $I$, термодинамическая температура $T$, сила света $J$, количество вещества $N$
    
    \subsection{Принцип суперпозиции}
    
    \textbf{Принцип} \textit{Суперпозиции}
    
    \subsection{Электрическое поле диполя}
    
    \textbf{Опр} \textit{Элементарный (примитивный) диполь}
    
    \textbf{Опр} \textit{Плечо диполя}
    
    \textbf{Опр} \textit{Дипольный момент диполя}
    
    \textbf{Опр} \textit{Точечный диполь}
    
    Чтобы найти поле точечного диполя, надо
    
    \begin{enumerate}
        \item Рассмотреть частные случаи: поле на оси и на перпендикуляре к ней.
        \item Рассмотреть случай произвольной точки с помощью принципа суперпозиции.
        \item Ввести новые дипольные моменты и получить где надо скалярное произведение
    \end{enumerate}
    
    \section{Теорема Гаусса для электрического поля в вакууме в интегральной и дифференциальной формах.
    Её применение для нахождения электростатических полей}
    
    \subsection{Теорема Гаусса для электрического поля в вакууме в интегральной и дифференциальной формах}
    
    \textbf{Опр} \textit{Направленная площадь}
    
    \textbf{Опр} \textit{Поток вектора напряжённости через площадочку}
    
    \textbf{Опр} \textit{Поток вектора напряжённости через замкнутую поверхность}
    
    \textbf{Th} \textit{Теорема Гаусса в интегральной форме}
    
    \textbf{Опр} \textit{Дивергенция векторного поля}
    
    \textbf{Th} \textit{Теорема Гаусса в дифференциальной форме}
    
    И то, как это следует из интегральной формы.
    
    \begin{enumerate}
        \item Запишем определение телесного угла, перпендикулярной площадочки и создадим скалярное произведение.
        \item Посчитаем $d \Phi$ от точеченого заряда внутри, идя от скалярного произведения до телесного угла.
        \item Посчитаем полный поток от точеченого заряда в зависимости от того, внутри он или снаружи
        \item В силу аддитивности потока получаем требуемое выражение
    \end{enumerate}
    
    \subsection{Её применение для нахождения электростатических полей}
    
    С помощью теорема Гаусса становится просто находить поля
    
    \begin{itemize}
        \item Равномерно заряженной плоскости
        \item Равномерно заряженной нити
        \item Равномерно заряженного шара
        \item Нейтральной сферической полости внутри шара (с помощью умного нуля)
    \end{itemize}
    
    Также можно доказать, что поле шара вне его будет совпадать с полем точечного диполя (рассмотерть маленький сдвиг,
    который и будет плечом такого диполя).
    Поверхностное распределение заряда найдём из теоремы косинусов для полярного угла
    
    \section{Потенциальный характер электростатического поля.
    Теорема о циркуляции электростатического поля.
    Потенциал и разность потенциалов.
    Связь напряжённости поля с градиентом потенциала.
    Граничные условия для вектора $E$}
    
    \subsection{Потенциальный характер электростатического поля}
    
    Если посчитать работу сил поля при перемещении заряда по определению, то получим условие потенциальности поля.
    Совокупность потенциальных полей образует суммарное потенциальное поле
    
    \subsection{Потенциал и разность потенциалов}
    
    \textbf{Опр} \textit{Потенциал}
    
    \textcolor{blue}{Физическая величина, служащая скалярной энергетической характеристикой электростатического поля
    и для конкретной рассматриваемой точки равная потенциальной энергии пробного заряда, помещённого в данную
    точку, отнесённой к величине этого заряда.}
    
    \textbf{Опр} \textit{Разность потенциалов}
    
    \subsection{Связь напряжённости поля с градиентом потенциала}
    
    Получается, что электростатическое поле можно охарктеризовать потенциалом, поэтому оно и потенциально:
    $E = -grad \phi = \nabla \phi$
    
    Расписав дифференциал потенциала, можно получить проекции поля на координаты
    
    \subsection{Теорема о циркуляции электростатического поля}
    
    \textbf{Th} \textit{О циркуляции в интегральной форме}
    
    Это напрямую следует из потенциальности поля
    
    Если напишем физический смысл ротора и выразим оттуда $\int Edr$, то получим \\
    
    \textbf{Th} \textit{О циркуляции в дифференциальной форме}
    
    По сути, означает равенство перекрёстных производных
    
    \subsection{Граничные условия для вектора $E$}
    
    Для скачка нормальной компоненты применим теорему Гаусса, а для равенства тангенциальных -- теорему о циркуляции
    
    \section{Уравнения Пуассона и Лапласа.
    Проводники в электрическом поле.
    Граничные условия на поверхности проводника.
    Единственность решения электростатической задачи.
    Метод изображений.
    Изображение точечного заряда в проводящих плоскости и сфере}
    
    \subsection{Уравнения Пуассона и Лапласа}
    
    Если взять суперпозицию двух уравнений с $E$, содержащих $\div$ $\grad$, то получим \\
    
    \textbf{Утв} \textit{Уравнение Пуассона}
    
    Зная, что $\div \grad = \Updelta$, то можно записать уравнение в другой форме.
    
    В частном случае области пространства, свободной от зарядов, имеем \\
    
    \textbf{Утв} \textit{Уравнение Лапласа}
    
    \subsection{Проводники в электрическом поле}
    
    \textbf{Опр} \textit{Проводник}
    
    В состоянии равновесия проводники обладают двумя свойствами (отсутсвие токов и зарядов внутри (заряды только \ldots)
    
    \subsection{Граничные условия на поверхности проводника}
    
    Записав классические граничные условия, воспользуемся свойством поля внутри проводника и получим новые условия
    
    \subsection{Единственность решения электростатической задачи}
    
    \textbf{Опр} \textit{Электростатическая задача}
    
    Граничные условия этой задачи могут быть двух типов: Дирихле и Цеймана \\
    
    \textbf{Th} \textit{О единственности решения уравнения Лапласа}
    
    Доказывается от противного простыми рассуждениями об исследовании потенциала в терминах производных \\
    
    \textbf{Th} \textit{О единственности решения уравнения Пуассона}
    
    Доказывается от противного введением новой функции (разности кандидатов) и сведением задачи к предыдущей
    
    \subsection{Метод изображений}
    
    Можно показать, что поле внутри области не зависит от зарядов вне её (по теореме единственности).
    Поэтому в зависимости от ситуации можно менять проводящую поверхность на группу зарядов и наоборот
    
    \textbf{Опр} \textit{Изображения зарядов}
    
    \subsection{Изображение точечного заряда в проводящих плоскости и сфере}
    
    В случае проводящей плоскости имеем равноудалённый заряд другого знака
    
    В случае заземлённой сферы говорим некоторые наводящие соображения о положении заряда-изображения, рассматриваем
    подобие треугольников и записываем равенство нулю потенциала на сфере
    
    В случае изолированной заряженной сферы мы к заряду из предыдущей задачи добавляем заряд в центр сферы так, чтобы
    их суммарный заряд стал исходному.
    Потенциал сферы в таком случае $\varphi = \frac{q''}{R}$
    
    \section{Электрическое поле в веществе.
    Поляризация диэлектриков.
    Свободные и связанные заряды.
    Вектор поляризации и вектор электрической индукции.
    Поляризуемость частиц среды.
    Диэлектрическая проницаемость среды.
    Теорема Гаусса в диэлектриках.
    Граничные условия на границе двух диэлектриков}
    
    \subsection{Электрическое поле в веществе}
    
    При помещении вещества в электрическое поле происходит пространственное перераспределение заряда
    
    \textbf{Опр} \textit{Диэлектрик}
    
    \subsection{Поляризация диэлектриков}
    
    \textbf{Опр} \textit{Поляризация}
    
    \subsection{Свободные и связанные заряды}
    
    \textbf{Опр} \textit{Свободные заряды}
    
    Скорость звука можно запросто вывести из соответствующего уравнения механики
    
    \textbf{Опр} \textit{Связанные (поляризационные) заряды}
    
    \subsection{Вектор поляризации и вектор электрической индукции}
    
    \textbf{Опр} \textit{Вектор поляризации}
    
    С помощью этого вектора можно найти поверхностную плотность поляризационных зарядов.
    
    \begin{enumerate}
        \item Запишем объём косого параллелепипеда через скалярное произведение $(S, l)$.
        \item Найдём дипольный момент и вектор поляризации поверхностных зарядов с плотностью поверхностных $\sigma$.
        \item Запишем проекцию последнего на нормаль и после преобразований получим $\sigma$
    \end{enumerate}
    
    Также найдём объёмную плотность поляризационных зарядов.
    
    \begin{enumerate}
        \item Запишем вышедший через нормальную площадку и поверхность в целом из вещества заряд под действием
        внешнего поля
        \item Тогда внутри остался суммарный поляризационный заряд, равный интегралу
        \item Если воспользоваться теоремой Гаусса -- Остроградского, то можно получить выражения для дивергенции
        вектора поляризации
        \item Если поляризация однородная, то плотность поляризационных зарядов равна нулю
    \end{enumerate}
    
    \subsection{Теорема Гаусса в диэлектриках}
    
    Если учесть наличие поляризационных снарядов, то теорема Гаусса в диэлектриках примет немного иной вид в обеих
    формах.
    Удобнее всего будет её записать, используя новое обозначение
    
    \textbf{Опр} \textit{Вектор электрической индукции $D$}
    
    \subsection{Поляризуемость частиц среды}
    
    При слабых внешних полях смещение зарядов мало и пропорционально приложенному полю (как производная)
    
    \textbf{Опр} \textit{Поляризуемость среды}
    
    \subsection{Диэлектрическая проницаемость среды}
    
    Если подставить поляризуемость в выражение для $D$, то можно ввести новое обозначение
    
    \textbf{Опр} \textit{Диэлектрическая проницаемость среды}
    
    Переписав теорему Гаусса для $E$ увидим, что в силу $\varepsilon > 1$ поляризационные заряды ослабляют поле
    
    \subsection{Граничные условия на границе двух диэлектриков}
    
    Для скачка нормальной компоненты запишем теорему Гаусса и получим $4 \pi \sigma_{free}$
    
    Для тангенциальной компоненты запишем теорему о циркуляции электрического поля и вновь получим то же самое
    
    \section{Электрическая ёмкость.
    Конденсаторы.
    Вычисление ёмкостей плоского, сферического и цилиндрического конденсаторов.
    Энергия электрического поля и её локализация в пространстве.
    Объёмная плотность энергии.
    Взаимная энергия зарядов.
    Энергия в системе заряженных проводников}
    
    \subsection{Электрическая ёмкость}
    
    Проведя мысленный эксперимент, поймём, что отношение $\frac{\varphi}{q}$ характеризует проводник
    
    \textbf{Опр} \textit{Электрическая ёмкость}
    
    \textbf{Опр} \textit{Ёмкость пары проводников (взаимная ёмкость)}
    
    \subsection{Конденсаторы}
    
    \textbf{Опр} \textit{Конденсатор}
    
    Поле внутри конденсатор однородное (две бесконечные плоскости), краевых эффектов почти нет
    
    Из выражения для поля можно запросто найти разность потенциалов окладок
    
    \subsection{Вычисление ёмкостей плоского, сферического и цилиндрического конденсаторов}
    
    Ёмкость плоского конденсатора находим по принципу суперпозиции как ёмкость двух бесконечных плоскостей разных знаков
    
    \textbf{Опр} \textit{Сферический конденсатор}
    
    Его поле можно найти по определению, предварительно посчитав разность потенциалов обклодок
    
    \textbf{Опр} \textit{Цилиндрический конденсатор}
    
    Записав теорему Гаусса для $D$, получим поле между обкладками.
    Затем посчитав разность потенциалов по определению, находим ёмкость
    
    \subsection{Энергия электрического поля и её локализация в пространстве}
    
    \textbf{Опр} \textit{Энергия пары зарядов}
    
    \textcolor{blue}{Энергия, которую необходимо затратить, чтобы их сблизить (работа) в соотвествии с её
    определением через потенциал}
    
    \textbf{Утв} \textit{Энергия электрического поля локализовано в пространстве}
    
    Иными словами, оно зависит лишь от небольшой окрестности вокруг рассматриваемой точки.
    Ведь если речь идёт об удалённых объектах, меняющихся со временем, то значение энергии уже может быть некорректным
    
    \subsection{Энергия в системе заряженных проводников}
    
    \textbf{Утв} \textit{Энергия всей системы зарядов}
    
    В формуле могут быть изменения в случае непрерывного распределения заряда по объёму или площади поверхности
    
    Например, можно найти поле плоского конденсатора
    
    \subsection{Объёмная плотность энергии}
    
    Преобразовав выражение для энергии конденсатора, можно выделить в ней
    
    \textbf{Опр} \textit{Объёмная плотность энергии}
    
    В случае $D = \varepsilon E$ выражение можно соответствующе переписать
    
    В случае сложной связи $D$ и $E$ можно используя $D = 4 \pi \sigma$ и общего вида энергии $dU = \varphi \delta q$ получить выражение
    объёмной плотность энергии в случае
    
    В общем случае
    
    \begin{enumerate}
        \item Рассмотрим вариацию энергии зарядов с $\delta \rho$.
        \item Исключим $\delta \rho$, используя теорему Гаусса.
        \item Воспользуемся следствием из правила дифференцирования сложной функции
        \item Одно слагаемое через замкнутую бесконечно удалённую поверхность будет ноль, а второе и даст привичное
        нам искомое выражени
    \end{enumerate}
    
    \subsection{Взаимная энергия зарядов}
    
    \textbf{Опр} \textit{Взаимная энергия}
    
    Если записать выражение через объёмную плотность энергии, то получим величину всегда неотрицательную (в отличие
    от исходной).
    Это связано с тем, что последняя включает в себе две группы слагаемых --- собственную и взаимную энергии
    
    \section{Энергия электрического поля в веществе.
    Энергия диполя во внешнем поле (жёсткий и упругий диполи).
    Силы, действующие на диполь в неоднородном электрическом поле.
    Энергетический метод вычисления сил (МВП), вычисление сил при постоянных зарядах и при постоянных потенциалах}
    
    \subsection{Энергия электрического поля в веществе}
    
    $W = \delta w V$
    
    \subsection{Энергия диполя во внешнем поле (жёсткий и упругий диполи)}
    
    \textbf{Утв} \textit{}
    
    Найти энергию жёсткого диполя во внешнем поле можно по определению, вспомнив определение разности потенциалов и
    дипольного момента
    
    В случае упругого диполя $p = \beta E$, поэтому взяв интеграл по $dE$ (оно меняется в силу переменной длины
    диполя) получим искомое
    
    \subsection{Силы, действующие на диполь в неоднородном электрическом поле}
    
    В неоднородном случае, чтобы получить силу надо взять градиент от энергии
    
    \subsection{Энергетический метод вычисления сил (МВП), вычисление сил при постоянных зарядах и при постоянных
    потенциалах}
    
    Используя определение элементарной работы, получим выражение для силы.
    В частном случае электростатической силы получим $\delta A = - dW$
    
    \textbf{Утв} \textit{Сила при постоянном заряде и сила при постоянных потенциалах равны}
    
    \begin{enumerate}
        \item В случае $q = const$ запишем $W(x)$ и продииференцируем для поиска силы при $q = const$
        \item В случае $\Delta \varphi = const$ следует дополнительно учитывать работу батареи
        \item Продифференцировав выражение для $\delta A_{mech}$ получим искомую силу
        \item Видно, что силы равны и они диэлектрики втягиваются в область сильного поля (знак $+$)
    \end{enumerate}
    
    \section{Постоянный ток.
    Сила тока, объёмная и поверхностная плотности тока.
    Закон Ома в интегральной и локальной формах.
    Уравнение непрерывности для плотности заряда.
    Закон Джоуля–Ленца в интегральной и локальной формах.
    Токи в неограниченных средах}
    
    \subsection{Постоянный ток}
    
    \textbf{Опр} \textit{Электрический ток}
    \textcolor{gray}{Упорядоченное движение заряженных частиц (электронов и ионов)}
   
    \textbf{Опр} \textit{Постоянный ток}
    
    \subsection{Сила тока, объёмная и поверхностная плотности тока}
    
    \textbf{Опр} \textit{Сила тока}
    
    \textbf{Опр} \textit{Плотность тока}
    
    Известно как получить формулу для этой величины
    
    \textbf{Опр} \textit{Линейная плотность тока}
    
    Аналогично вводятся поверхностная и объёмная плотности тока
    
    \subsection{Уравнение непрерывности для плотности заряда}
    
    Запишем закон сохранения заряда для произвольной области пространства.
    Воспользовавшись определением дивергенции, получим
    
    \textbf{Утв} \textit{Уравнение непрерывности для плотности заряда}
    
    По-другому это уравнение называют ЗСЗ в дифференциальной форме
    
    В стационарном случае в область втекает столько же заряда, сколько и вытекает
    
    \subsection{Закон Ома в интегральной и локальной формах}
    
    \textbf{Закон} \textit{Ома}
    
    \textbf{Закон} \textit{Ома в дифференциальной форме}
    
    \subsection{Закон Джоуля–Ленца в интегральной и локальной формах}
    
    \textbf{Закон} \textit{Джоуля–Ленца в локальной форме}
    
    \begin{enumerate}
        \item Запишем скорость зарядов в среде.
        Среднее её значение будет иметь простой вид из-за свойств флуктуаций.
        \item Запишем объёмную плотность энергии (через работу) и, воспользовавшись выражениями для $j$, получим искомое
    \end{enumerate}
    
    При наличии объёмных токов подставим интегральные величины и получим
    
    \textbf{Закон} \textit{Джоуля–Ленца в интегральной форме}
    
    \subsection{Токи в неограниченных средах}
    
    Пользуясь теоремой Гаусса можно получить сопротивление неограниченной среды, по которой течёт ток между двумя
    электродами.
    Если расстояние между ними $\gg$ их размеров, то сопротивление превращается в сумму $\frac{\varepsilon_i}{4
    \pi \lambda_i C_i}$, то есть сопротивление зависит от геометрии кажого электрода, а не от их взаимного расположения
    
        \section{Постоянный ток в замкнутых электрических цепях.
        Электродвижущая сила.
        Закон Ома для участка цепи.
        Правила Кирхгофа.
        Работа и мощность постоянного тока}
    
     \subsection{Постоянный ток в замкнутых электрических цепях}
    
    В случае замкнутой цепи закон Ома выглядит привычно
    
    \textbf{Закон} \textit{Ома в интегральной форме}

         \subsection{Электродвижущая сила}
    
    \textbf{Опр} \textit{Электродвижущая сила}
    
    \textcolor{blue}{Формальное обозначение интгерала}

         \subsection{Закон Ома для участка цепи}
    
    Полагая ток во всей участках цепи одинаковым, из дифференциального закона Ома получим
    
    \textbf{Закон} \textit{Ома для участка цепи}
    
    \begin{enumerate}
        \item Выразим из дифференциальной формы циркуляцию вектора $E$.
        \item Введём полное сопртивление участка и ЭДС (может быть как положительным, так и отрицательным)
        \item Подставив всё в одно уравнение, получи требуемое
    \end{enumerate}

         \subsection{Правила Кирхгофа}
    
\textbf{Утв} \textit{Правила Кирхгофа}
    
    Они доказываются с помощью ЗСЗ и закона Ома для участка цепи
    
         \subsection{Работа и мощность постоянного тока}

    В прошлом билете было получено выражение для мощности тока в локальной и интегральной формах.
    Если учесть наличие ЭДС, можно получить ещё две формулы мощности.
    Домножив каждую на время, получим работу
    
    \section{Магнитное поле постоянного тока в вакууме.
    Вектор магнитной индукции.
    Сила Лоренца.
    Сила Ампера.
    Закон Био–Савара.
    Теорема о циркуляции магнитного поля в вакууме.
    Теорема Гаусса для магнитного поля.
    Магнитное поле прямого провода, соленоида, тороидальной катушки}
    
    \subsection{Магнитное поле постоянного тока в вакууме}
    
    \textbf{Опр} \textit{Магнитное поле}
    
    \subsection{Вектор магнитной индукции}
    
    \textbf{Опр} \textit{Вектор магнитной индукции}
    
    \textcolor{blue}{Вектор, определяющий силу, действующую на движущйися заряд и харкатеризующйи магнитное поле}
    
    \subsection{Сила Лоренца}
    
 \textbf{Опр} \textit{Сила Лоренца}
    
    \subsection{Сила Ампера}
    
    \textbf{Опр} \textit{Сила Ампера}
    
    \textbf{Закон} \textit{Ампера}
    
    Нетрудно показать эквивалентность разных выражений для сил в законе Ампера, а также, свзяь сил Лоренца и Ампера
    
    \subsection{Закон Био–Савара}
    
    \textbf{Закон} \textit{Био–Савара}
    
    Он экспериментальный и тоже формулируется для линейного и объёмного элемента тока
    
    \subsection{Теорема о циркуляции магнитного поля в вакууме}
    
    Если взять закон Био–Савара и произвести с ним преобразования по законам векторного анализа, то можно получить
    векторный потенциал и показать, что он биздевергентен
    
    По аналогии с $\Updelta \varphi$, можно получить
    
    \textbf{Th} \textit{О циркуляции магнитного поля в вакууме в дифференциальной форме}
    
    Воспользовавшиьс формулой Стокса и перейдся к интегралу по контуру, получим
    
    \textbf{Th} \textit{О циркуляции магнитного поля в вакууме в интегральной форме}
    
    \subsection{Теорема Гаусса для магнитного поля}
    
    \textbf{Th} \textit{Гаусса для магнитного поля в дифференциальной форме}
    
    \textbf{Th} \textit{Гаусса для магнитного поля в интегральной форме}
    
    \subsection{Магнитное поле прямого провода, соленоида, тороидальной катушки}
    
    Найти поле прямого провода можно с помощью закона Био-Савара.
    
    \begin{enumerate}
        \item Запишем этот закон и перейдём к скалярному выражению, раскрыв векторное проивзедение
        \item Выразив $dx$ через выражение для $\tg \alpha$ подставим его в выражение
        \item Сделав замену и после интегрирования получим требуемое полке
    \end{enumerate}
    
    Поле соленоида и тороидальнгой катушки можно найти с помощью теоремы о циркуляции, использовав, если необходимо,
    плотность намотки и линейную плотность тока
    
    \section{Магнитный момент тока.
    Точечный магнитный диполь.
    Сила и момент сил, действующие на виток с током в магнитном поле.
    Эквивалентность витка с током и магнитного диполя}
    
    \subsection{Магнитный момент тока}
    
    \textbf{Опр} \textit{Магнитный момент}
    
    Он направлен по нормали к плоскости витка
    
    \subsection{Точечный магнитный диполь}
    
    \textbf{Опр} \textit{Точечный магнитный диполь}
    
    \textcolor{blue}{То же, что и плоская замкнутая проводящая рамка площади $S$ по которой течёт ток $I$}
    
    \subsection{Сила и момент сил, действующие на виток с током в магнитном поле}
    
    В однородном магнитном поле на виток с током суммарная действующая сила равна нулю.
    В неоднородном поле силу легче все найти через потенциальную энергию.
    
    \begin{enumerate}
        \item Посчитаем работу поля по повороту витка
        \item Увидим: она зависит лишь от начальных и конечных состояний, то есть можно ввести потенциальную энергию.
        \item Продифференцируем жнергию и получим силу
        \item При условии отсутсвия токов проводимости в среде после преобразований векторного анализа можно дать её
        более простой вид
    \end{enumerate}
    
    С помощью магнитного момента можно найти момент сил, действующие на магнитный диполь
    
    \subsection{Эквивалентность витка с током и магнитного диполя}
    
    Возьмём векторный потенциал зарядов, движущихся в ограниченной области как сумму и преобразуем его.
    Затем, переходя к магнитному полю, увидим эквивалентность магнитного <диполя> и витка с током
    
    \section{Поверхностное натяжение.
    Коэффициент поверхностного натяжения, краевой угол.
    Смачивание и несмачивание.
    Формула Лапласа.
    Свободная энергия и внутренняя энергия поверхности}
    
    \subsection{Поверхностное натяжение}
    
    Все молекулы жидкости испытывают притяжение со стороны других молекул.
    Вблизи поверхности жидкости у молекулы в сфере молекулярного действия находится меньше молекул, к которым она
    притягивается, поэтому возникает сила, стремящаяся втянуть её с её поверхности внутрь жидкости
    
    \textbf{Опр} \textit{Поверхностное натяжение} \textcolor{gray}{Работа, необходимая для увеличения поверхности ...}
    
    \subsection{Коэффициент поверхностного натяжения, краевой угол}
    
    \begin{enumerate}
        \item Заметим, что в изотермическом процессе работа идёт на изменение свободной энергии: $F = F_V + F_s$, где
        первое слагаемое пропорционально объёму плёнки, а второе -- площади её поверхности
        \item Тогда коэффициент поверхностного натяжения $\sigma = \frac{F_s}{\Pi}$
        \item Другое выражение для $\sigma$ даётся через механическую работу силы $2f$ (двойка, потому как у плёнки
        есть две поверхности -- внешняя и внутренняя)
        \item Получим, что $\sigma$ есть сила, приходящаяся на единицу длины границы поверхности
    \end{enumerate}
    
    \textbf{Опр} \textit{Краевой угол смачивания} \textcolor{gray}{Угол между касаталеьной, проведённой к ...}
    
    Если рассмотреть участок на границе трёх сред (газа, плёнки и поверхности), то из равенства сил на этот участок
    получим выражение из определения коэффициента поверхностного натяжения получим выражение для краевого угла
    смачивания
    
    \subsection{Смачивание и несмачивание}
    
    Проанализируем выражение для данного угла $\frac{\sigma_{sg} - \sigma_{sl}}{\sigma_{gl}}$:
    \begin{itemize}
        \item $> 1:$ жидкость растекается по поверхности ТТ -- полное смачивание
        \item $<-1:$ жидкость принимает элипсообразную форму капли -- полное несмачивание
        \item $0 < \theta < \frac{\pi}{2}:$ частичное смачивание
        \item $\frac{\pi}{2} < \theta < \pi:$ частичное несмачивание
    \end{itemize}
    
    \subsection{Формула Лапласа}
    
    Если рассмотреть небольшую часть сферической поверхности жидкости и использую определения коэффициента
    поверхностного натяжения, радиуса кривизны, площади и воспользоваться малостью угла, то можно получить формулу
    Лапласа, дающей численное выражение разности давления жидкости и газа над поверхностью
    
    \subsection{Свободная энергия и внутренняя энергия поверхности}
    
    Запишем выражение для свободной энергии поверхности, воспользовавшись смыслом частных производных и коэффициента
    поверхностного натяжения.
    Получим выражение для поверхностной внутренней энергии
    
    \section{Зависимость давления насыщенного пара от кривизны поверхности жидкости.
    Роль зародышей в образовании фазы.
    Кипение}
    
    \subsection{Зависимость давления насыщенного пара от кривизны поверхности жидкости}
    
    \begin{enumerate}
        \item Рассмотрим хитрый сосуд с плоской частью и капилляром
        \item Используя формулы Торичелли, удельного объёма, Лапласа и барометрическую, получим выражение для
        логарифма отношения давлений через давление.
        \item Данная формула неявная, поэтому чтобы получить аналитичность в случае малой разности, разложим логарифм
        в ряд и получим более простую формулу $P = P_0 + \frac{\nu_l}{\nu_s - \nu_l}\sigma K$
    \end{enumerate}
    
    Рассмотрим толщу воды при $T \approx T_{lg}$
    
    \begin{enumerate}
        \item Пусть в толще жидкости образовался пузырёк.
        Тогда запишем его давление через новую формулу и через формулу Лапласа (условие равновесия).
        Равенство достигается в случае критического радиуса пузырька
        \item Если радиус пузыря меньше, то он схлопнется, а если больше -- продолжит расти
    \end{enumerate}
    
    \subsection{Роль зародышей в образовании фазы}
    
    \begin{enumerate}
        \item Аналогично выражению для радиуса пузырька, можно получить критический радиус капли в процессе
        конденсации.
        Он будет в $\frac{\nu_l}{\nu_s}$ больше
        \item И вновь, если радиус капли меньше критического, то она схлопнется, а если больше -- то начнётся её рост
        \item Подобные пузырьки и капли могут образовываться около песчинок, взвеси, трещинок и других неровностей
        поверхностей и среды.
    \end{enumerate}
    
    \subsection{Кипение}
    
    Если такие неровности достаточно большие, то начнётся кипение
    
    \textbf{Опр} \textit{Кипение} \textcolor{gray}{Фазовый переход «жидкость -- пар», происходящий с образованием ...}
    
    \section{Уравнение Ван-дер-Ваальса как модель неидеального газа.
    Изотермы газа Ван-дер-Ваальса.
    Критические параметры.
    Приведённое уравнение Ван-дер-Ваальса, закон соответственных состояний}
    
    \subsection{Уравнение Ван-дер-Ваальса как модель неидеального газа}
    
    Модель Ван-дер-Ваальса учитывает две особенности реального газа: это наличие объёма у молекул и их взаимное
    притяжение друг к другу.
    Отсюда следует необходимость введения двух новых параметров-констант $a$ и $b$
    
    \begin{enumerate}
        \item Учтём запрещённый объём для каждой молекулы введя $b: V^{'} = V - \nu b$
        \item Чтобы учесть притяжение между ними, рассмотрим нейтральную молекулу газа
        \item При сближении с другой нейтральной, они начинают ориентироваться разнонаправленно; между ними возникает
        сила ВдВ
        \item Выразим давление из промежуточного уравнения ВдВ.
        Конечное давление на стенки сосуда будет меньше давления в случае ИГ, потому как часть частиц притягивается и
        сталкивается между собой
        \item Оценим эту разница через череду пропорциональностей: $\Delta P \sim F \sim n \cdot n \sim n^2 \sim \frac{1}{V^2}$
        \item Чтобы записать равенство, введём $a: \Delta P = \frac{a \nu^2}{V^2}$
        \item Приведя промежуточное равенство с учётом поправок к нормальному виду, получим уравнение ВдВ
    \end{enumerate}
    
    \subsection{Изотермы газа Ван-дер-Ваальса}
    
    \begin{enumerate}
        \item Начнём изображать изотермы Ван-дер-Ваальса в координатах $P - V$: они имеют вид кубического трёхчлена
        \item Найти координаты точек экстремумов можно из уравнения $(\frac{\partial P}{\partial V})_T = 0$, а если
        приравнять эту производную к давлению, то можно получить уравнение кривой, соединяющей все такие точки --
        спинодали
        \item При увеличении температуры минимум и максимум сольются в одну точку перегиба, а после данной точки
        будут походить на изотермы ИГ (гипербола)
    \end{enumerate}
    
    \subsection{Критические параметры}
    
    Чтобы найти эту критическую точку и её параметры ($P, V, T$) мы имеем три уравнения:
    
    \begin{enumerate}
        \item первая производная равна нулю (минимум и максимум -- экстремум -- слились в ней)
        \item вторая производная равна нулю (точка перегиба)
        \item уравнение Ван-дер-Ваальса
    \end{enumerate}
    
    Другой способ получения параметров -- записать куб разности через формулу сокращённого умножения
    
    \subsection{Приведённое уравнение Ван-дер-Ваальса, закон соответственных состояний}
    
    Если ввести новые переменные -- отношения текущих параметров к критическим и подставить их в уравнение ВдВ, а
    затем подставить выражения для критических параметров и выполнить преобразования, то получим приведённое
    уравнение ВдВ.
    Из него следует
    
    \textbf{Закон} \textit{Соответственных состояний}
    
    \textcolor{blue}{Для различных веществ одинаковым значениям $\varphi$ и $\pi$ соотвествует лишь одно (и то же)
        значение $\tau$}
    
    \section{Метастабильные состояния: переохлаждённый пар, перегретая жидкость (на примере модели Ван-дер-Ваальса).
    Изотермы реального газа, правило Максвелла и правило рычага}
    
    \subsection{Метастабильные состояния: переохлаждённый пар, перегретая жидкость (на примере модели Ван-дер-Ваальса)}
    
    \begin{enumerate}
        \item Из уравнения ВдВ и вида изотерм такого газа можно сделать вывод, что одному значению давления могут
        соответствовать разные значения объёма
        \item То есть существуют термодинамически неустойчивые состояния (действительно, мы расширяем газ, а он греется)
        \item Таким образом, на изотермах ВдВ можно различить четыре вида состояний: стабильные, метастабильные,
        термодинамически неустойчивые, а также, устойчивую смесь жидкой и парообразной фазы (прямая бинодаль)
    \end{enumerate}
    
    \subsection{Изотермы реального газа, правило Максвелла и правило рычага}
    
    Изотермы реального газа имеют две фазы на участке бинодали и одну выше критической точки
    
    \textbf{Утв} \textit{Правило Максвелла}
    
    \textcolor{blue}{Кривая термодинамически нейстойчивого участка пересекает прямую устойчивой смеси так, чтобы
    полученные площади были равны}
    
    \begin{enumerate}
        \item Рассмотрим квазистатический цикл через эти точки.
        Из равенства Клаузиуса следует, что $\delta Q = 0$ (мы двигались по изотерме, поэтому $T = const$)
        \item Из возврата в ту же точку следует, что $dU = 0 \Rightarrow A = 0$.
        \item Так как в данном цикле мы проходим два круговых участка, притом один по часовой (работа положительна),
        а второй против (отрицательна), то суммарная работа есть ноль только в случае равенства площадей (работа есть
        ориентированная площадь под графиком)
    \end{enumerate}
    
    Найдём соотношение между количество жидкости и газа для их устойчивой смеси.
    Для этого запишем объём жидкой и газообразной части через значения плотности на концах участках и воспользуемся ЗСМ.
    Получим, что массы жидкости и пара соотносятся по правилу рычага
    
    \section{Внутренняя энергия и энтропия газа Ван-дер-Ваальса.
    Равновесное и неравновесное расширение газа Ван-дер-Ваальса в теплоизолированном сосуде}
    
    \subsection{Внутренняя энергия и энтропия газа Ван-дер-Ваальса}
    
    \begin{enumerate}
        \item Для начала рассмотрим внутреннюю энергию как $U(T, V)$.
        Запишем её полный дифференциал и воспользуемся соотношением Максвелла
        \item В итоге получим термическое уравнение состояния
        \item Теперь получим выражение для внутренней энергии газа ВдВ как $U(T, V)$ через её полный дифференциал
        \item Воспользуемся определениями $c_V$, только термическим уравнением состояния, выражением для
        $(\frac{\partial P}{\partial T})_V$, уравнением ВдВ
        \item Подставляя все выкладки, получим требуемое выражение.
        Заметим, что внутренняя энергия газа ВдВ определена с точностью до константы
        \item Найдём энтропию газа ВдВ как $S(T, V)$ вновь через её полный дифференциал
        \item Распишем каждый частичный дифференциал (частную производную), использую выкладки для внутренней энергии
        и I начало ТД
        \item Вновь получим функцию, определённую с точностью до константы
    \end{enumerate}
    
    \subsection{Равновесное и неравновесное расширение газа Ван-дер-Ваальса в теплоизолированном сосуде}
    
    \begin{enumerate}
        \item Рассмотрим свободное расширение газа в вакуум в неравновесном процессе и найдём изменение температуры
        такого газа.
        \item $\Delta Q = 0$ в силу теплоизолированности, а $A = 0$, потому что не над чем совершать работу, поэтому
        и $\Delta U = 0$ по I началу ТД
        \item $\Rightarrow U_i = c_V T_i - \frac{a}{V_i}$.
        Выразим из $U_1 = U_2$ разность температур и получим её отрицательность, то есть газ охладился.
%        \item Это можно объяснить совершением работы против сил притяжения молекул за счёт кинетической энергии молекул
    \end{enumerate}
    
    Теперь рассмотрим равновесное адиабатическое расширение газа ВдВ, записав изменение его энтропии.
    Получим псевдо ЭДТ
    
    \section{Течение идеальной жидкости.
    Уравнение Бернулли сжимаемой и несжимаемой жидкости.
    Изоэнтропическое истечение газа из отверстия}
    
    \subsection{Течение идеальной жидкости. Уравнение Бернулли сжимаемой и несжимаемой жидкости}
    
    Идеальная жидкость течёт без трения и теплопередач
    
    \begin{enumerate}
        \item Для каждого сечения $\rho vS = const$ в силу ЗСМ
        \item Записав работу по смещению как разность энергий, получим константную сумму (энтальпия)
        \item Перейдя к удельным величинам и раскрыв состав удельной энергии (кинетическая + потенциальная), получим
        уравнение Бернулли:
        \[ \frac{P}{\rho} + gh + \frac{v^2}{2} + u = const \]
        \item Уравнение для несжимаемой жидкости не будет иметь слагаемого с плотностью (она не изменяется)
    \end{enumerate}
    
    \subsection{Изоэнтропическое истечение газа из отверстия}
    
    \begin{enumerate}
        \item Найдём скорость истечения газа из отверстия в условиях малого перепада высот $\iota_i + \frac{v^2}{2} = const$
        \item Считая скорость внутри сосуда пренебрежимой, получим выражение через удельную энтальпию
        \item Вспомнив определение энтальпии и формулу Майера, получим более конкретную формулу; при желании из-под
        корня можно вынести скорость звука
        \item При адиабатическом истечении ИГ можно переписать отношение температур как отношение давлений и получить
        немного другую запись.
        Из неё видно, что скорость газа максимальна при расширении в вакуум $P_2 = 0$, притом $v > c_{sound}$
    \end{enumerate}
    
    \section{Эффект Джоуля—Томсона.
    Дифференциальный эффект Джоуля–Томсона для газа Ван-дер-Ваальса, температура инверсии}
    
    \subsection{Эффект Джоуля—Томсона}
    
    \textbf{Опр} \textit{Дроссель} \textcolor{gray}{Местное препятствие газовому потоку}
    
    \textbf{Опр} \textit{Дросселирование} \textcolor{gray}{Медленное протекание газа под действием постоянного ...}
    
    \textbf{Опр} \textit{ЭДТ} \textcolor{gray}{Изменение температуры газа при адиабатическом дросселировании}
    
    Существует положительный и отрицательный ЭДТ (напомним, разность давлений всегда отрицательна).
    Данный процесс называют изоэнтальпическим (его можно описать с помощью уравнения Бернулли) из-за его медленности.
    Для ИГ он не наблюдается, потому как из постоянства энтальпии следует постоянство температуры.
    Таким образом, ЭДТ позволяет определить степень неидеальности газа
    
    \subsection{Дифференциальный эффект Джоуля–Томсона для газа Ван-дер-Ваальса, температура инверсии}
    
    \textbf{Опр} \textit{Дифференциальный ЭДТ} \textcolor{gray}{ЭДТ при малых перепадах давления (т.е. ещё и изобарный)}
    
    \begin{enumerate}
        \item Представим энтальпию как $I(T, P)$ и выразим $(\frac{\partial T}{\partial P})_I$ через условие $dI = 0$
        \item Числитель выразим через стандартное определение энтальпии и с помощью соотношений Максвелла, а
        знаменатель -- по определению
        \item Далее выразим $(\frac{\partial V}{\partial T})_P$ через формулу $3 = -1$ и получим общий вид ЭДТ
        \item В случае газа ВдВ, подставим значения соответствующих частных производных
        \item Найдём температуру инверсии, приравняв $(\frac{\partial T}{\partial P})_I$ к нулю.
        При меньших температурах имеем охлаждение, а при больших -- нагрев газа при дросселировании
    \end{enumerate}
    
    \section{Распределение частиц идеального газа по проекциям и модулю скорости (распределение Максвелла).
    Наиболее вероятная, средняя и среднеквадратичная скорости.
    Распределение Максвеллапо энергиям}
    
    \subsection{Распределение частиц идеального газа по проекциям и модулю скорости (распределение Максвелла)}
    
    Число молекул в ИГ со средней плотностью $n$, обладающими скоростями в интервале $[v, v + dv]$, определяется
    распределением Максвелла
    
    Интегрированием по направлениям скорости сводится к замене третьего дифференциала, что приводит к распределению
    по абсолютной величине скорости
    
    \subsection{Наиболее вероятная, средняя и среднеквадратичная скорости}
    
    Найдём максимум формулы распределения путём дифференцирования и приравнивания к нулю.
    Получим наиболее вероятную скорость частицы
    
    Вопрос о среднем значении случайной знаковой величины бессмысленен -- это ноль.
    Если рассматривать лишь положительные значения, то благодаря специальной формуле, получим среднюю скорость.
    Аналогичным образом получается среднеквадратичная скорость
    
    \subsection{Распределение Максвелла по энергиям}
    
    Для получения распределения по энергиям, достаточно заменить переменную в распределении Максвелла по скоростям
    
    \section{Среднее число молекул, сталкивающихся в единицу времени с единичной площадкой.
    Средняя энергия молекул, вылетающих в вакуум через малое отверстие}
    
    \subsection{Среднее число молекул, сталкивающихся в единицу времени с единичной площадкой}
    
    \begin{enumerate}
        \item Рассмотрим столкновения газа с неподвижной стенкой и выделим группу молекул со скоростью $v$ плотностью
        $dn(v)$
        \item В соответствующий телесный угол летит лишь доля молекул $\frac{d\Omega}{4\pi}$, а за время $dt$ до
        поверхности долетят лишь молекулы в объёме $v_x Sdt = v \cos \theta Sdt$
        \item Последовательно суммируем по всем углам и по все скоростям, деля промежуточный результат на $Sdt$,
        чтобы получить поток частиц (число в единицу времени и единицу площади)
    \end{enumerate}
    
    \subsection{Средняя энергия молекул, вылетающих в вакуум через малое отверстией}
    
    Чтобы найти эту величину, надо посчитать полную уносимую энергию и разделить её на полный поток (отношение
    интегралов)
    
    \section{Распределение Больцмана в поле внешних сил.
    Барометрическая формула}
    
    \subsection{Распределение Больцмана в поле внешних сил}
    
    Поместим газ в потенциальное поле сил $n = n(z)$.
    Если мы захотим посчитать среднее число частиц $dN$ в объёме $dV$ со скоростями в $d^3 v$, то получим
    распределение Максвелла -- Больцмана, где нормировочная константа определяется из условия $\int dN = N$, с $n_0
    = n(v = 0)$
    
    \subsection{Барометрическая формула}
    
    \begin{enumerate}
        \item Рассмотрим цилиндрик газа в поле тяжести и запишем для него второй закон Ньютона
        \item Вспомним основную формулу МКТ, разделим переменные и проинтегрируем
        \item В конце перейдём от $n$ к $P$ и получим барометрическую формулу в произвольном потенциальном поле
    \end{enumerate}
    
    \section{Статистика классических идеальных систем.
    Микро- и макросостояния.
    Статистический вес.
    Распределение Гиббса для идеального газа (без вывода)}
    
    \subsection{Статистика классических идеальных систем}
    
    Существует два постулата статистического описания больших систем:
    
    \begin{enumerate}
        \item Все разрешённые микросостояния равновероятны
        \item Термодинамически равновесным является то микросостояние, которое реализуется наибольшим числом
        микросостояний, то есть является наиболее вероятным состоянием
    \end{enumerate}
    
    \subsection{Микро- и макросостояния}
    
    \textbf{Опр} \textit{Микросостояние} \textcolor{gray}{Состояние системы, определяемое одновременным заданием ...}
    
    \textbf{Опр} \textit{Макросостояние} \textcolor{gray}{Состояние системы, характеризуемое небольшим числом ...}
    
    Одно макросостояние мб реализовано большим числом микросостояний за счёт перестановки частиц не меняющей
    наблюдаемого состояния
    
    \subsection{Статистический вес}
    
    \textbf{Опр} \textit{Фазовое пространство} \textcolor{gray}{Пространство частиц, координат и импульсов системы}
    
    Если разбить всё фазовое пространство на ячейки такого объёма, что на каждую будет приходиться лишь одно
    микросостояние, то число состояний в объёме окажется равным $dG = \frac{d\Gamma}{\Gamma_0}$.
    Различные ячейки отвечают разным микросостояниям, но могут отвечать одному макросостоянию
    
    \textbf{Опр} \textit{Статистический вес} \textcolor{gray}{Число микросостояний, реализующих данное макросостояние}
    
    $G = \frac{N!}{N_1! \dots N_m!}$
    
    \subsection{Распределение Гиббса для идеального газа (без вывода)}
    
    Чтобы найти термодинамическое равновесное состояние, нужно найти условия максимума $G$.
    В этом нам поможет распределение Гиббса
    
    \textbf{Опр} \textit{Явление вырождения} \textcolor{gray}{Случай обладания системой одной и той же энергией в ...}
    
    \textbf{Опр} \textit{Кратность вырождения} \textcolor{gray}{Число состояний, обладающих одним и тем же ...}
    
    Выше предполагалось, что энергия частицы (подсистемы) пробегает дискретный набор значений, но результат можно
    обобщить и на непрерывный ряд значений.
    Для этого достаточно приписать частице значение энергии ячейке в фазовом пространстве с объёмом $d\Gamma = dr dp$.
    В этих условиях распределение Гиббса немного изменится
    
    \section{Статистические определения энтропии и температуры.
    Аддитивность энтропии.
    Закон возрастания энтропии.
    Третье начало термодинамики}
    
    \subsection{Статистические определения энтропии и температуры}
    
    \textbf{Утв} \textcolor{blue}{Энтропия определяется формулой Больцмана $S = klnG$}
    
    Пусть подсистема с энергией $E$ находится в состоянии, близком к равновесному (энтропия максимальна со значением $S$).
    Имеет место связь $S = S(E)$ (если энергия меняется, то меняется и энтропия)
    
    \textbf{Опр} \textit{Статистическая температура} \textcolor{gray}{$\frac{1}{T} = \frac{dS}{dE}$ в предположении ...}
    
    Отсюда можно выразить энтропию через сумму, а также, дифференциал энтропии
    
    \subsection{Аддитивность энтропии}
    
    \begin{enumerate}
        \item Разобьём систему на две подсистемы.
        Новые микросостояния будут меняться независимо
        \item Статвес системы равен произведению чисел способов, которым могут быть осуществлены состояния каждой
        новой подсистемы: $G = G_1 G_2$
        \item По свойствам логарифма, отсюда следует аддитивность энтропии, то есть энтропия системы есть сумма
        энтропий её подсистем
    \end{enumerate}
    
    \subsection{Закон возрастания энтропии}
    
    \textbf{Th} \textit{Закон возрастания энтропии}
    
    \textcolor{blue}{Среди всех направлений эволюции системы предпочтительным является то, при котором вероятность
    конечного состояния оказывается наибольшей}
    
    \textbf{Следствие 1} \textcolor{blue}{С наибольшей вероятностью энтропия замкнутой системы растёт (не убывает)}
    
    \textbf{Следствие 2} \textcolor{blue}{В состоянии термодинамического равновесия энтропия максимальна}
    
    \subsection{Третье начало термодинамики}
    
    \textbf{Закон} \textit{III начало ТД}
    \textcolor{blue}{
        \begin{enumerate}
            \item При приближении к абсолютному нулю, энтропия стремится к конечному значению $S_0$
            \item Все процессы при абсолютном нуле, переводящие систему из одного равновесного состояния
            в другое, происходят без изменения энтропии
        \end{enumerate}}
    
    Энтропию в нуле можно положить нулевой и использовать как предел.
    Третье начало (по-другому -- теорема Нернста) есть экспериментальный факт, верный для всех чистых кристаллических
    веществ, квантовых жидкостей и газов
    
    \section{Изменение энтропии при смешении газов, парадокс Гиббса}
    
    \begin{enumerate}
        \item Рассмотрим два разных смешения ИГ, для чего можно использовать привычную формулу для энтропии
        \item Запишем изменение энтропии в случае разных молекул (необратимый процесс смешения).
        Можно даже явно посчитать энтропию и показать её положительность
        \item В случае одинаковых газов ничего как будто и не произошло: $\Delta S = 0$
        \item А если газы будут совсем немного отличаться, то энтропия смешения будет ноль или не ноль?
        Парадокс
    \end{enumerate}
    
    Парадокс разрешим.
    В реальности данный предельный переход не выполним: либо газы различны, либо нет, поэтому и энтропия либо $>0$,
    либо равна $0$
    
    \section{Классическая теория теплоёмкостей.
    Закон равномерного распределения энергии теплового движения по степеням свободы.
    Теплоёмкость кристаллов (закон Дюлонга—Пти)}
    
    \subsection{Классическая теория теплоёмкостей}
    
    \begin{enumerate}
        \item Как было выяснено с помощью распределения Максвелла, средняя энергия поступательного движения молекул
        есть $\frac{3}{2} kT$
        \item В таком случае в силу наличия трёх поступательных СС получим энергию в $\frac{kT}{2}$ и теплоёмкость
        $\frac{k}{2}$ на одну СС.
        Молярная теплоёмкость будет равна $\frac{3}{2}R$
        \item Теперь рассмотрим молекулу как ТТ с тремя главными моментами инерции и запишем её полную энергию
        вращательного движения
        \item Из распределения Гиббса следует, что $\overline{w^2_i} = \frac{kT}{I_i}$, откуда полная вращательная
        энергия есть $\frac{3}{2} kT$
        \item В случае линейной молекулы ($I_3 = 0$) получим полную энергию $kT$
        \item Для колебательной СС из механики мы знаем, что средняя кинетическая энергия при колебаниях равна
        средней потенциальной, поэтому на каждую такую СС приходится энергия $\frac{kT}{2} + \frac{kT}{2} = kT$
        \item Итого, молекула из $N$ атомов имеет $3N$ СС, из которых по три поступательные и вращательные и $3N - 6$
        колебательных (в случае линейной молекулы $3N - 5$ из-за уменьшения количества колебательных СС)
    \end{enumerate}
    
    \subsection{Закон равномерного распределения энергии теплового движения по степеням свободы}
    
    \textbf{Закон} \textit{РРЭСС}
    
    \textcolor{blue}{Если макроскопическая система подчиняется законам классической механики, то на каждое слагаемое
    в энергии, квадратично зависящее от координат / скоростей молекул, приходится энергия $\frac{kT}{2}$ и
    теплоёмкость $\frac{k}{2}$}
    
    \subsection{Теплоёмкость кристаллов (закон Дюлонга — Пти)}
    
    \textbf{Закон} \textit{Дюлонга -- Пти}
    
    \textcolor{blue}{ТТ (кристалл) представляет собой совокупность атомов, находящихся в окрестности своего п.р.
    В нём эти атомы могут совершать колебания в трёх направлениях, так что каждый обладает средней энергией в $3kT$.
    Отсюда молярная теплоёмкость кристаллов $c_V = 3R$}
    
    \section{Зависимость теплоёмкости $c_V$ газов от температуры.
    Возбуждение и замораживание степеней свободы, характеристические температуры}
    
    \subsection{Зависимость теплоёмкости $c_V$ газов от температуры}
    
    График зависимости выглядит плавно-ступенчато, с разрывами ближе к нулю и последней ступенькой в точке
    $(10^3 K; \frac{7}{2}R)$
    
    \subsection{Возбуждение и замораживание степеней свободы, характеристические температуры}
    
    Поступательные СС появляются $T = 0 K$ (при всех положительных температурах).
    Вращательные СС появляются при $T \sim 100 K$.
    Колебательные СС появляются при $T \sim 1000 K$
    
    Характеристические температуры берутся из квантовой механики и связаны с энергией молекулы, жёсткостью связи и
    приведённой массой
    
    \section{Флуктуации в термодинамических системах.
    Влияние флуктуаций на чувствительность измерительных приборов (на примере пружинных весов)}
    
    \subsection{Флуктуации в термодинамических системах}
    
    \textbf{Опр} \textit{Флуктуация} \textcolor{gray}{Случайное отклонение физической величины от её среднего значения}
    
    \textbf{Опр} \textit{Среднеквадратичная флуктуация} \textcolor{gray}{Среднее квадрата флуктуации}
    
    \textbf{Опр} \textit{Дисперсия} \textcolor{gray}{Корень среднеквадратичного отклонения}
    
    \textbf{Опр} \textit{Относительная среднеквадратичная флуктуация} \textcolor{gray}{Отношение дисперсии к среднему}
    
    Во многих случаях флуктуации физической величины имеют гауссово распределение
    
    \subsection{Влияние флуктуаций на чувствительность измерительных приборов (на примере пружинных весов)}
    
    Рассмотрим потенциальную энергию пружинных весов.
    С одной стороны, она обусловлена энергией пружинки, с другой -- ТРРЭСС, что позволяет нам оценить флуктуацию
    их показаний и минимальную массу для измерения
    
    \section{Зависимость флуктуаций от числа частиц, составляющих систему.
    Флуктуация числа частиц в выделенном объёме}
    
    \subsection{Зависимость флуктуаций от числа частиц, составляющих систему}
    
    \begin{enumerate}
        \item Рассмотрим (экстенсивную) термодинамическую величину.
        Чтобы найти её флуктуацию, распишем среднее квадрата её флуктуации для каждой подсистемы, используя
        статистическую независимость
        \item Получим знакомый результат: квадрат среднего минус среднее в квадрате, то есть среднеквадратичное
        отклонение
        \item Умножим в конце на число элементарных подсистем $N$ и получим искомую флуктуацию
        \item В случае интенсивной величины рассуждения будут похожи, однако среднее (равно как и переход от частей
        к целому) будет считаться по другому.
        Также нам предстоит сразу считать среднеквадратичное отклонение
    \end{enumerate}
    
    \subsection{Флуктуация числа частиц в выделенном объёме}
    
    \begin{enumerate}
        \item Рассмотрим участок сосуда малого объёма $\nu$, содержащего $n$ частиц и посчитаем энтропию такой системы
        \item Запишем условие равновесия такой системы (концентрация внутри участка и вне его одинакова)
        \item Тогда можно приближённо вычислить изменение энтропии, разложив её по формуле Тейлора
        \item Отсюда следует, что искомая флуктуации числа частиц есть $\sqrt{n}$
    \end{enumerate}
    
    \section{Связь вероятности флуктуации и энтропии системы.
    Флуктуации температуры в заданном объёме.
    Флуктуация объёма в изотермическом и адиабатическом процессах}
    
    \subsection{Связь вероятности флуктуации и энтропии системы}
    
    Вероятность пропорциональна $\exp \left(\frac{\Delta P \Delta V - \Delta T \Delta S}{2kT}\right)$
    
    \subsection{Флуктуации температуры в заданном объёме}
    
    \begin{enumerate}
        \item Пусть система отделена от внешней среды жёсткой ($V = const$) теплопроводящей оболочкой.
        Запишем среднеквадратичное отклонение энергии системы (результат выводится через статсуммы)
        \item Пусть в результате флуктуаций в подсистему поступило количество теплоты $\delta Q$.
        Тогда $\delta T = \frac{\delta Q}{c_V}$, откуда, с учётом определения $c_V$, найдём и среднеквадратичное
        отклонение температуры
    \end{enumerate}
    
    \subsection{Флуктуация объёма в изотермическом и адиабатическом процессах}
    
    \begin{enumerate}
        \item Введём силу, действующую на газ под поршнем в изотермическом процессе с одной стороны как $S \Delta P$
        (с использованием частной производной), а с другой подобно пружинке, чью жёсткость выразим через равенство
        потенциальной и температурной энергий
        \item Выразим среднеквадратичное отклонение объёма как $S^2 x^2$
        \item Воспользуемся постоянством температуры, выразим нужную частную производную
        \item Подставляем и находим нужную среднеквадратичную флуктуацию
        \item В адиабатическом процессе отличие будет лишь в виде частной производной (вместо $PV = const$ теперь $PV^\gamma = const$)
    \end{enumerate}
    
    \section{Столкновения. Эффективное газокинетическое сечение.
    Длина свободного пробега.
    Частота столкновений молекул между собой}
    
    \subsection{Столкновения. Эффективное газокинетическое сечение}
    
    \textbf{Опр} \textit{Эффективное газокинетическое сечение молекулы} \textcolor{gray}{Площадь поперечного ...}
    
    Молекулы провзаимодействуют, если центр какой-либо молекулы попадёт в этот цилиндр.
    Формулы немного отличается для случая, когда молекула движется в среде из молекулы другого размера
    
    \subsection{Длина свободного пробега}
    
    В цилиндре, закреплённом за каждой молекулой, находится по $\sigma \lambda n$ частиц.
    Хотя бы одно столкновение произойдёт, если эта величина станет равна единице.
%    (в случае среды с тяжёлыми частицами)
    При желании можно найти время свободного пробега (поделить длину на скорость) и относительно уточнить данную формулу
    
    \subsection{Частота столкновений молекул между собой}
    
    Найдём частоту столкновений $f_i$ \underline{одной} молекулы с другими.
    Если в единице объёма находится $n$ молекул, то всего будет произведено $\frac{1}{2}n \frac{1}{\tau}$ столкновений.
    Данное число можно выразить как через микро-, так и через макропараметры
    
    \section{Диффузия: закон Фика, коэффициент диффузии. Дифференциальное уравнение одномерной диффузии.
    Коэффициент диффузии в газах}
    
    \subsection{Диффузия: закон Фика, коэффициент диффузии. Дифференциальное уравнение одномерной диффузии}
    
    \textbf{Опр} \textit{Средняя скорость течения газа} \textcolor{gray}{$\overline{u} = \frac{1}{N} \sum_i v_i$}
    
    Суммирование производится по всем молекулам в единице объёма
    
    \textbf{Опр} \textit{Плотность потока} \textcolor{gray}{$\overline{j} = n\overline{u}$}
    
    \textbf{Опр} \textit{Диффузия} \textcolor{gray}{Неравновесный процесс пространственного перераспределения ...}
    
    \textbf{Опр} \textit{Относительная концентрация компонентов} \textcolor{gray}{$c_i = \frac{n_i}{n}, n = \sum_i n_i$}
    
    \textbf{Закон} \textit{Фика}
    
    В обычном случае применяется коэффициент диффузии, но если $n = const$, то имеем взаимную диффузию и
    соответствующий коэффициент.
    Также существуют поправки в случае ненулевой скорости течения газов: $j_1 + j_2 = n \overline{u}$.
    В трёхмерном случае вводится градиент концентрации
    
    Вышесказанное позволяет записать дифференциальное уравнение (одномерной диффузии)
    
    \textbf{Опр} \textit{Самодиффузия} \textcolor{gray}{Диффузия частиц в среде из частиц того же сорта ...}
    
    Этот процесс можно изучать только если часть частиц как-то помечена
    
    \subsection{Коэффициент диффузии в газах}
    
    Найдём значение коэффициента диффузии через рассмотрение переноса молекул вдоль оси.
    Сравнивая полученное выражение с законом Фика, получим $D = \frac{1}{3} \overline{v}\lambda$; при
    желании, его можно расписать по-подробнее
    
    \section{Теплопроводность: закон Фурье, коэффициент теплопроводности. Дифференциальное уравнение одномерной
    теплопроводности.
    Коэффициент теплопроводности в газах}
    
    \subsection{Теплопроводность: закон Фурье, коэффициент теплопроводности}
    
    \textbf{Опр} \textit{Теплопроводность} \textcolor{gray}{Неравновесный процесс, вид передачи тепла от более ...}
    
    \textbf{Опр} \textit{Плотность потока тепла $q$} \textcolor{gray}{Количество тепловой энергии, пересекающей ...}
    
    Поток тепла направлен в сторону убывания температуры (это обуславливает знак минус)
    
    \textbf{Закон} \textit{Фурье}
    
    В нём используется коэффициент теплопроводности.
    В трёхмерном случае вводится градиент температуры
    
    Вышесказанное позволяет записать дифференциальное уравнение (одномерной теплопроводности)
    
    Иногда используют коэффициент температуропроводности $a = \frac{\varkappa}{c_V}$, где $c_V$ -- теплоёмкость
    вещества на единицу объёма
    
    \subsection{Коэффициент теплопроводности в газах}
    
    \begin{enumerate}
        \item Найдём значение коэффициента теплопроводности через рассмотрение переноса тепла вдоль оси в условии
        перемещения газа как целого ($N_{up} = N_{down}$)
        \item Для этого надо записать энергию одной молекулы в фиксированной точке (используя $c^1_V$)
        \item Получим непонятное выражение для потока и перейдём к градиенту температур
        \item Сравнивая полученное выражение с законом Фурье, получим $\varkappa = \frac{1}{3} n \overline{v} \lambda
        c_V = n c^m_V D$; при желании, его можно расписать по-подробнее
    \end{enumerate}
    
    \section{Вязкость: закон Ньютона, коэффициенты динамической и кинематической вязкости.
    Коэффициент вязкости в газах}
    
    \subsection{Вязкость: закон Ньютона, коэффициенты динамической и кинематической вязкости}
    
    \textbf{Опр} \textit{Вязкость} \textcolor{gray}{Перенос тангенциальной компоненты импульса в направлении ...}
    
    \textbf{Закон} \textit{вязкого трения Ньютона}
    
    В нём используется коэффициент вязкости или динамическая вязкость.
    В трёхмерном случае вводится градиент скорости
    
    Иногда используют кинематическую вязкость $\nu = \frac{\eta}{\rho}$, где $c_V$ -- теплоёмкость вещества на
    единицу объёма
    
    \subsection{Коэффициент вязкости в газах}
    
    \begin{enumerate}
        \item Зафиксируем среднюю скорость упорядоченного движения молекул в условии равенства потоков молекул
        \item Запишем результирующий импульс, приобретаемый верхним слоем за время $\tau$
        \item Делим этот импульс на $\tau$ и сравниваем полученное выражение с законом вязкого трения Ньютона,
        получим $\eta = \frac{1}{3} mn \overline{v} \lambda c_V = \rho D \Rightarrow \nu = \frac{\eta}{\rho} = D$
    \end{enumerate}
    
    \section{Диффузия как процесс случайных блужданий. Закон смещения частицы при диффузии (закон
    Эйнштейна—Смолуховского).
    Скорость передачи тепла при теплопроводности}
    
    \subsection{Диффузия как процесс случайных блужданий. Закон смещения частицы при диффузии (закон
    Эйнштейна—Смолуховского)}
    
    \textbf{Опр} \textit{Подвижность} \textcolor{gray}{$B = \frac{\overline{v}}{F_{fr}}$}
    
    \textbf{Опр} \textit{Формула Стокса} \textcolor{gray}{$B = \frac{1}{6 \pi R \eta}$}
    
    \textbf{Закон} \textit{Эйнштейна—Смолуховского}
    
    \textcolor{blue}{Маленькая частица, движущаяся в среде, испытывает действие двух типов сил: силы торможения за
    счёт вязкого трения и флуктуационной силы со стороны молекул среды, чьё среднее действие есть ноль}
    
    \begin{enumerate}
        \item Посчитаем полное смещение частицы после $N$ шажков.
        Наивный подход даст нулевой результат
        \item Однако если посмотреть квадрат смещения, то сумма шажков разобьётся на сумму квадратичных и
        перекрёстных слагаемых, притом сумма последних равна нулю в силу их независимости
        \item Получили среднеквадратичное отклонение.
        Заменим число частиц на $\frac{t}{\tau}$ и пристальным взглядом увидим удвоенный коэффициент диффузии
        (удвоенность следует из интегрирования ослабляющегося потока частиц)
        \item Полученная формула и есть ЗЭС, обобщаемый на случай больших размерностей
    \end{enumerate}
    
    Данный закон показывает, что броуновское движение частиц аналогично процессу диффузии
    
    \subsection{Скорость передачи тепла при теплопроводности}
    
    \begin{enumerate}
        \item Рассмотрим процесс передачи тепла в одномерном случае и запишем соответствующие уравнения
        \item После пары преобразований, получим дифференциальное уравнение теплопроводности, в котором можно
        использовать коэффициент температуропроводности
        \item Запишем скорость передачи тепла через производную и элементарные приращения
        \item Получим, что $l \sim \chi t$, то есть оценку на скорость теплопередачи
    \end{enumerate}
    
    \section{Подвижность макрочастицы.
    Броуновское движение.
    Связь подвижности частицы и коэффициента диффузии облака частиц (соотношение Эйнштейна).
    Закон Эйнштейна—Смолуховского для броуновской частицы}
    
    \subsection{Подвижность макрочастицы}
    
    \textbf{Опр} \textit{Подвижность} \textcolor{gray}{$B = \frac{\overline{v}}{F_{fr}}$}
    
    Видно, что подвижность макрочастицы такая же, как и у макрочастицы
    
    \subsection{Броуновское движение}
    
    \textbf{Опр} \textit{Броуновская частица} \textcolor{gray}{Макроскопическая частица размера $\sim 10^{-6} m$}
    
    \textbf{Опр} \textit{Броуновское движение} \textcolor{gray}{Беспорядочное движение броуновских частиц в жидкосте ...}
    
    \begin{enumerate}
        \item Рассмотрим движение частицы в среде с помощью 2ЗН, используя запись силы через подвижность
        \item В случае нулевой внешней силы (частица предоставлена сама себе) получим уравнение, проинтегрировав
        которое можно получить формулу скорости такой частицы.
        Такова роль трения в движении частицы
        \item Роль диффузионной силы обуславливает среднеквадратичную скорость, которую можно запросто найти из условия
        равновесия частицы со средой
    \end{enumerate}
    
    \subsection{Связь подвижности частицы и коэффициента диффузии облака частиц (соотношение Эйнштейна)}
    
    Теперь опишем движение частицы в координатном пространстве.
    Для этого мы уже не можем пользоваться результатами для микрочастиц.
    Хорошей оценкой является $D \sim \frac{l^2}{\tau} \sim u^2 \tau \sim \dots \sim kTB$, однако она неточна
    
    Для получения численного коэффициента воспользуемся методом, предложенным Эйнштейном
    \begin{enumerate}
        \item Запишем равенство потоков частиц в поле силы тяжести, обусловленных диффузией и наличие гравитационной
        силы
        \item По ходу дела нам предстоит воспользоваться распределением Больцмана и определение подвижности
        \item Приравниваем потоки и получаем точное значение коэффициента диффузии
    \end{enumerate}
    
    \subsection{Закон Эйнштейна—Смолуховского для броуновской частицы}
    
    Для броуновской частицы данный закон будет иметь аналогичный вид, что и для макрочастицы (с учётом размерности
    пространства)
    
    \section{Явления переноса в разреженных газах: эффузия (эффект Кнудсена), зависимость коэффициента
    теплопроводности газа от давления}
    
    \textbf{Опр} \textit{Эффузия} \textcolor{gray}{Медленное течение газа через малые отверстия}
    
    \textbf{Опр} \textit{Число Кнудсена} \textcolor{gray}{$Kn = \frac{\lambda}{L}, L$ -- характерный размер сосуда}
    
    В случае $Kn \gg 1$ имеем свободное молекулярное течение
    
    Если аз находится в сосуде, в стенке которого есть малое отверстие площадью $S$, то в состоянии равновесия
    молекулярные потоки с обоих сторон равны ($\frac{1}{4} n_i \overline{v_i}$).
    Тогда можно записать равенство, именуемое эффектом Кнудсена, в чьём общем случае учитываются разные молекулы газа
    с обеих сторон
    
    Так как $\varkappa \sim \sqrt{\frac{T}{m}}$, то в силу только что полученного соотношения при течении
    разреженных газов имеем $\varkappa \sim \frac{P}{\sqrt{m}}$
    
    \section{Течение разреженного газа по прямолинейной трубе.
    Формула Кнудсена}
    
    \subsection{Течение разреженного газа по прямолинейной трубе}
    
    Когда разреженный газ течёт по прямолинейной трубе, то его $\lambda \sim L$, а так как у стенок трубки находятся
    тоже молекулы, то для описания течения больше подходит процесс диффузии: $N^* = N f\left(\frac{a}{L}\right) \sim
    NC\frac{a}{L}$, притом указанное линейное приближение даёт небольшую погрешность
    
    В случае если имеется поток с обоих сторон, то можно записать полное число частиц в трубе, и использовав ранее
    известные формулы, преобразовать её.
    Полученное выражение будет сильно отличаться от формулы Пузейля гидродинамического течения.
    При желании можно найти константу $A$ для трубы
    
    \subsection{Формула Кнудсена}
    
    В случае течения разреженного газа по трубе, по-новому приблизим $D$ и запишем приращения $\frac{dn}{dx}$ вместо
    дифференциалов.
    В итоге получим формулу Кнудсена

\end{document}
