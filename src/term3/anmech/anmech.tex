%! Author = user
%! Date = 12.01.2024

\documentclass[a4paper, 14pt]{article}

%\hypersetup
%{   colorlinks,
%    pdftitle={analysis_themeslinalg summary},
%    pdfauthor={Володин Максим},
%    allcolors=[RGB]{010 090 200}
%}

\usepackage[T2A]{fontenc}
\usepackage[utf8]{inputenc}
\usepackage[english, russian]{babel}
\usepackage[top = 2cm, bottom = 2cm, left = 2cm, right = 2cm]{geometry}
\usepackage{indentfirst}
\usepackage{xcolor}
\usepackage{hyperref}
\usepackage{gensymb}
\usepackage{pgfplots}
\usepackage{amsmath, amsfonts, amsthm, mathtools}
\usepackage{amssymb}
\usepackage{physics, multirow, float}
\usepackage{wrapfig, tabularx}
\usepackage{icomma} % Clever comma: 0,2 - number while 0, 2 - two numbers
\usepackage{tikz, standalone}
\usepackage{fancyhdr,fancybox}
\usepackage{booktabs}
\usepackage{listings}
\usepackage{lstmisc}
\usepackage{stmaryrd}
\usepackage{graphicx}

%\полуторный интервал
\onehalfspacing

\graphicspath{{images/}}
\DeclareGraphicsExtensions{.pdf,.png,.jpg}

\restylefloat{table}
\usetikzlibrary{external}

\mathtoolsset{showonlyrefs = true} % Numbers will appear only where \eqref{} in the text LINKED
\pagestyle{fancy}

\fancyhf{}
\fancyhead[R]{Конспект билетов}
\fancyfoot[R]{\thepage /\pageref{LastPage}}
\fancyhead[L]{Аналитическая механика}

\pgfplotsset{compat=1.18}

\begin{document}

{\huge
    \begin{center}
    {\textbf{Конспект билетов}}
        \\
        Аналитическая механика
    \end{center}
}
    \tableofcontents \newpage
    
    \section{Кинематика точки. Траектория, скорость и ускорение точки.
    Скорость и ускорение точки в полярных координатах}
    
    \subsection{Кинематика точки}
    
    \textbf{Опр} \textit{Кинематика точки. Траектория, скорость и ускорение точки}
    
    \textcolor{blue}{Раздел механики, изучающий математическое описание (средствами геометрии, алгебры,
        математического анализа…) движения материальной точки без рассмотрения причин движения (массы, сил и т. д.)}
    
    \textbf{Опр} \textit{Траектория}
    
    \textbf{Опр} \textit{Скорость}
    
    \textbf{Опр} \textit{Ускорение}
    
    \subsection{Скорость и ускорение точки в полярных координатах}
    
    \textbf{Опр} \textit{Радиальная ось}
    
    \textbf{Опр} \textit{Трансверсальная ось}
    
    Для того чтобы получить скорость и ускорение в полярных координатах, достаточно выразить $x$ и $y$ в терминах $r,
    \varphi$, продифференцировать нужное число раз и вычленить базисные векторы
    
    \section{Кинематика точки. Естественный трёхгранник.
    Теорема Гюйгенса о разложении ускорения точки на тангенциальное и нормальное}
    
    \subsection{Кинематика точки. Естественный трёхгранник}
    
    \textbf{Опр} \textit{Естественный способ задания движения}
    
    \textbf{Опр} \textit{Естественный трёхгранник}
    
    \subsection{Теорема Гюйгенса о разложении ускорения точки на тангенциальное и нормальное}
    
    Запишем две формулы из дифференциальной геометрии и продифференцируем $r$ и $v$ с их учётом.
    Получим две компоненты ускорения: тангенциальное и нормальное
    
    \textbf{Theorem} \textit{Гюйгенса о разложении ускорения}
    
    \section{Криволинейные координаты точки.
    Коэффициенты Ламе.
    Скорость и ускорение точки в криволинейных координатах.
    Скорость точки в цилиндрических и сферических координатах}
    
    \subsection{Криволинейные координаты точки}
    
    \textbf{Опр} \textit{Криволинейные координаты}
    
    \textbf{Опр} \textit{Первая координатная линия}
    
    \textbf{Опр} \textit{Первая координатная ось}
    
    Аналогично определяются и последующие координатные линии и оси
    
    \subsection{Коэффициенты Ламе}
    
    \textbf{Опр} \textit{Единичный вектор координатной оси}
    
    \textbf{Опр} \textit{Коэффициент Ламе}
    
    \textbf{Опр} \textit{Ортогональные криволинейные координаты}
    
    \subsection{Скорость и ускорение точки в криволинейных координатах}
    
    Скорость находится по определению.
    Ускорение смотреть в конспекте Холостовой с 8 страницы
    
    \subsection{Скорость точки в цилиндрических и сферических координатах}
    
    \textbf{Опр} \textit{Цилиндрическая система координат}
    
    \textbf{Опр} \textit{Сферическая система координат}
    
    Скорость точки в этих координатах находится с помощью коэффициентов Ламе
    
    \section{Угловая скорость и угловое ускорение твёрдого тела.
    Скорости и ускорения точек твёрдого тела в общем случае его движения (формулы Эйлера и Ривальса)}
    
    \subsection{Угловая скорость и угловое ускорение твёрдого тела}
    
    \textbf{Опр} \textit{Поступательно движущаяся и связанная системы координат}
    
    \textbf{Опр} \textit{Углы Эйлера}
    
    \textcolor{blue}{Углы, описывающие поворот абсолютно твердого тела в трёхмерном евклидовом пространстве}
    
    \textbf{Опр} \textit{Линия узлов}
    
    \textcolor{blue}{Пересечение координатных плоскостей начальной и конечной СК}
    
    \textbf{Опр} \textit{Угол прецессии, нутации, собтвенного вращения}
    
    Переход от одной системы координат к другой посредством вращений на углы, можно задать с помощью матриц поворота
    
    \textcolor{blue}{Матрица поворота (или матрица направляющих косинусов)}
    
    \textbf{Опр} \textit{Ортогональная матрица, которая используется для выполнения собственного ортогонального
    преобразования в евклидовом пространстве.
    При умножении любого вектора на матрицу поворота длина вектора сохраняется.
    Определитель матрицы поворота равен единице}
    
    Матрицы поворота вокруг различных осей выглядят по-разному
    
    \textbf{Опр} \textit{Угловая скорость}
    
    \textcolor{blue}{Физическая величина, характеризующая быстроту и направление вращения материальной точки или
    абсолютно твёрдого тела относительно оси}
    
    \textbf{Опр} \textit{Угловое ускорение}
    
    \subsection{Скорости и ускорения точек твёрдого тела в общем случае его движения (формулы Эйлера и Ривальса)}
    
    \textbf{Theorem} \textit{Формула Эйлера}
    
    \textbf{Формула} \textit{Ривальса} \textcolor{gray}{Получается формальным дифференцированием формулы Эйлера}
    
    \section{Плоское движение твёрдого тела.
    Мгновенный центр скоростей.
    Мгновенный центр ускорений}
    
    \subsection{Плоское движение твёрдого тела}
    
    \textbf{Опр} \textit{Плоское движение}
    
    \subsection{Мгновенный центр скоростей}
    
    \textbf{Theorem} \textit{О мгновенном центре скоростей}
    
    \textbf{Опр} \textit{Мгновенный центр скоростей}
    
    \subsection{Мгновенный центр ускорений}
    
    \textbf{Theorem} \textit{О мгновенном центре ускорений}
    
    \textbf{Опр} \textit{Мгновенный центр ускорений}
    
    Мгновенный центр ускорений можно построить за два шага
    
    \section{Кинематические инварианты.
    Кинематический винт.
    Мгновенная винтовая ось}
    
    \subsection{Кинематические инварианты}
    
    \textbf{Опр} \textit{Инвариант}
    
    \textcolor{blue}{Величина, остающаяся неизменной при преобразованиях}
    
    \textbf{Опр} \textit{Первый кинематический инвариант}
    
    \textbf{Опр} \textit{Второй кинематический инвариант}
    
    Отсюда следует, что величины проекции скоростей двух точек поступательного движущегося тела на прямую, их
    соединяющуюся, равны
    
    \subsection{Кинематический винт}
    
    \textbf{Опр} \textit{Кинематический винт}
    
    \textbf{Опр} \textit{Параметр винта}
    
    \subsection{Мгновенная винтовая ось}
    
    Если расписать итоговую скорость точки по координатам, то можно получить
    
    \textbf{Опр} \textit{Мгновенная винтовая ось}
    
    \textbf{Опр} \textit{Правый и левый винт}
    
    \section{Алгебра кватернионов}
    
    \textbf{Опр} \textit{Кватернион, кватернионные единицы}
    
    \textbf{Свойства} \textit{Кватернионного сложения}
    
    \textbf{Опр} \textit{Скалярная и векторные части кватерниона}
    
    \textbf{Свойства} \textit{Кватернионного умножения единиц}
    
    \textbf{Свойства} \textit{Кватернионного умножения}
    
    \textbf{Опр} \textit{Сопряжённый кватернион}
    
    \textbf{Опр} \textit{Норма кватерниона, нормированный кватернион}
    
    \textbf{Опр} \textit{Обратный кватернион}
    
    \textbf{Форма} \textit{Тригонометрическая записи кватерниона}
    
    Результат умножения кватернионов в таком случае получается из свойств тригонометрии
    
    \textbf{Аналог} \textit{Формулы Муавра}
    
    \section{Кватернионный способ задания ориентации твёрдого тела.
    Теорема Эйлера о конечном повороте}
    
    \subsection{Кватернионный способ задания ориентации твёрдого тела}
    
    \textbf{Опр} \textit{Неподвижный и связанный базисы}
    
    \textbf{Theorem}
    
    \subsection{Теорема Эйлера о конечном повороте}
    
    \textbf{Theorem} \textit{Эйлера о конечном повороте}
    
    Если воспользоваться предыдущей теоремой, то видно, что при повороте положение $e$ сохраняется, а $j$ поворачивается
    
    \section{Формулы сложения поворотов твёрдого тела в кватернионах.
    Параметры Родрига-Гамильтона.
    Теорема Эйлера о конечном повороте твёрдого тела с неподвижной точкой}
    
    \subsection{Формулы сложения поворотов твёрдого тела в кватернионах}
    
    Можно показать, что результирующий кватернион после $N$ поворотов будет записан в обратном порядке в одном базисе
    
    \subsection{Параметры Родрига-Гамильтона}
    
    \textbf{Опр} \textit{Параметры Родрига-Гамильтона}
    
    Если записать преобразованные от смены базисные единицы и подставить в новый кватернион, то он будет выражен в
    исходном базисе через параметры Родрига-Гамильтона.
    Порядок записи кватернионов в результирующем повороте будет уже прямой
    
    \subsection{Теорема Эйлера о конечном повороте твёрдого тела с неподвижной точкой}
    
    \textbf{Theorem} \textit{Эйлера о конечном повороте твёрдого тела с неподвижной точкой}
    
    \section{Кинематические уравнения вращательного движения твёрдого тела в кватернионах (уравнения Пуассона).
    Прецессионное движение твёрдого тела}
    
    \subsection{Кинематические уравнения вращательного движения твёрдого тела в кватернионах (уравнения Пуассона)}
    
    \textbf{Опр} \textit{Угловая скорость} \textcolor{gray}{Через предел}
    
    \textbf{Уравнение} \textit{Пуассона}
    
    Можно показать, что два определения угловой скорости эквивалентны.
    В конце мы придём к уравнению Эйлера (то есть верному утверждению), а значит мы были правы
    
    \subsection{Прецессионное движение твёрдого тела}
    
    Рассмотрим вращение оси тела вокруг неподвижной вращающейся оси и решим уравнение Пуассона для этого случая
    
    \section{Кинематика сложного движения точки.
    Вычисление скоростей и ускорений в сложном движении}
    
    \subsection{Кинематика сложного движения точки}
    
    \textbf{Опр} \textit{Относительное, переносное и абсолютное движение}
    
    Можно посчитать относительные и абсолютные производные радиус-вектора и получить их связь
    
    \subsection{Вычисление скоростей и ускорений в сложном движении}
    
    \textbf{Опр} \textit{Относительные, переносные и абсолютные скорость и ускорение}
    
    \textbf{Theorem} \textit{О сложении скоростей}
    
    \textbf{Theorem} \textit{О сложении ускорений или теорема Кориолиса}
    
    \textbf{Опр} \textit{Кориолисово ускорение}
    
    \section{Кинематика сложного движения тела.
    Сложение мгновенных вращений твёрдого тела вокруг пересекающихся осей.
    Кинематические уравнения Эйлера}
    
    \subsection{Кинематика сложного движения тела. Сложение мгновенных вращений твёрдого тела вокруг пересекающихся
    осей}
    
    \textbf{Сложение} \textit{Поступательных движений}
    
    \textbf{Сложение} \textit{Вращательных движений}
    
    \subsection{Кинематические уравнения Эйлера}
    
    Если тело участвует одновременно в трёх вращениях, то записав суммарное вращение в проекциях на связанные оси, имеем
    
    \textbf{Уравнения} \textit{Эйлера кинематические}
    
    \section{Кинематика сложного движения тела.
    Сложение мгновенных вращений твёрдого тела вокруг параллельных осей.
    Пара вращений}
    
    \subsection{Кинематика сложного движения тела. Сложение мгновенных вращений твёрдого тела вокруг параллельных осей}
    
    \textbf{Сложение} \textit{Сонаправленных вращений}
    
    \textbf{Сложение} \textit{Разнонаправленных вращений}
    
    \subsection{Пара вращений}
    
    \textbf{Пара} \textit{Вращений}
    
    \textbf{Опр} \textit{Момент и плечо пары вращений}
    
    Поступательное движение можно заменить на пару вращений бесчисленным множеством способов
    
    \section{Сложное движение твёрдого тела.
    Общий случай сложения движений}
    
    \subsection{Сложное движение твёрдого тела}
    
    \textbf{Лемма}
    
    \subsection{Общий случай сложения движений}
    
    В общем случае движения приведём все поступательные и вращательные к единой точке приложения по алгоритму
    
    \textbf{Алгоритм} \textit{Приведения к простому движению}
    
    \section{Момент силы относительно точки и оси, главный вектор и главный момент сил системы.
    Элементарная работа сил системы.
    Работа сил, приложенных к твёрдому телу.
    Силовое поле.
    Силовая функция.
    Потенциал}
    
    \subsection{Момент силы относительно точки и оси, главный вектор и главный момент сил системы}
    
    \textbf{Опр} \textit{Сила}
    
    \textcolor{blue}{Мера воздействия тел друг на друга, причина ускорения точки}
    
    \textbf{Аксиома} \textit{Инерции}
    
    \textbf{Опр} \textit{Инертность, масса}
    
    \textbf{Закон} \textit{Динамики основной}
    
    \textbf{Аксиома} \textit{Взаимодействия материальных точек}
    
    \textbf{Аксиома} \textit{Независимости действия сил (принцип суперпозиции)}
    
    \textbf{Опр} \textit{Главный вектор всех сил системы}
    
    \textbf{Опр} \textit{Момент силы относительно точки}
    
    \textbf{Опр} \textit{Момент силы относительно оси}
    
    Можно показать корректность этого определения
    
    \textbf{Опр} \textit{Главный момент сил системы}
    
    \subsection{Элементарная работа сил системы}
    
    \textbf{Опр} \textit{Элементарная работа}
    
    Можно получить выражение для полной работы
    
    \subsection{Работа сил, приложенных к твёрдому телу}
    
    В общем случае работа внутренних сил ненулевая.
    Запишем суммарную работу всех сил системы
    
    \subsection{Силовое поле}
    
    \textbf{Опр} \textit{Силовое поле}
    
    \textcolor{blue}{Векторное поле в пространстве, в каждой точке которого на точку действует определённая
    по величине и направлению сила (вектор силы)}
    
    \subsection{Силовая функция}
    
    \textbf{Опр} \textit{Силовая функция}
    
    \textbf{Опр} \textit{Потенциальное поле}
    
    \textbf{Опр} \textit{Потенциальная сила}
    
    \textbf{Опр} \textit{(Не)стационарное поле}
    
    \subsection{Потенциал}
    
    \textbf{Опр} \textit{Потенциал}
    
    \textcolor{blue}{Скалярная величина, характеризующая силовое поле}
    
    \textbf{Утв}
    
    \textbf{Опр} \textit{Потенциальная энергия}
    
    \textcolor{blue}{Скалярная физическая величина, представляющая собой часть полной механической
    энергии системы, находящейся в поле консервативных сил}
    
    \section{Количество движения.
    Центр масс.
    Теорема об изменении количества движения системы.
    Теорема о движении центра масс}
    
    \subsection{Количество движения}
    
    \textbf{Опр} \textit{Количество движения (импульс)}
    
    \subsection{Центр масс}
    
    \textbf{Опр} \textit{Центр масс системы}
    
    \subsection{Теорема об изменении количества движения системы}
    
    \textbf{Theorem} \textit{Об изменении количества движения системы}
    
    \subsection{Теорема о движении центра масс}
    
    \textbf{Theorem} \textit{О движении центра масс}
    
    \section{Главный момент количества движения (кинетический момент) системы относительно заданного центра.
    Кинетический момент системы для ее движения относительно центра масс.
    Теорема Кенига о вычислении кинетического момента}
    
    \subsection{Главный момент количества движения (кинетический момент) системы относительно заданного центра}
    
    \textbf{Опр} \textit{Момент импульса (кинетический момент) точки}
    
    \subsection{Кинетический момент системы для ее движения относительно центра масс}
    
    \textbf{Опр} \textit{Кинетический момент (главный момент количества движения) системы}
    
    \textbf{Опр} \textit{Кинетический момент системы относительно точки}
    
    Можно показать корректность этого определения
    
    Покажем связь главных моментов двух точек в общем и частном случаях
    
    \subsection{Теорема Кенига о вычислении кинетического момента}
    
    \textbf{Опр} \textit{Кёнигова система координат}
    
    Найдём выражения для скорости и кинетического момента точки и системы
    
    Если под движением системы относительно центра масс понимать движение в Кёниговой системе координат, то верна
    
    \textbf{Theorem} \textit{Кёнига о кинетическом моменте}
    
    \section{Теорема об изменении кинетического момента системы}
    
    Посчитаем производную кинетического момент относительно точки
    
    \textbf{Theorem} \textit{Об изменении кинетического момента системы}
    
    Также рассмотрим частные случаи теоремы
    
    \section{Кинетическая энергия системы.
    Теорема Кенига о вычислении кинетической энергии.
    Теорема об изменении кинетической энергии системы.
    Закон сохранения полной механической энергии системы}
    
    \subsection{Кинетическая энергия системы}
    
    \textbf{Опр} \textit{Кинетическая энергия системы}
    
    Запишем, как она преобразуется при смене системы координат
    
    \subsection{Теорема Кенига о вычислении кинетической энергии}
    
    В частном случае прошлых выкладок получаем
    
    \textbf{Theorem} \textit{Кенига о вычислении кинетической энергии}
    
    \subsection{Теорема об изменении кинетической энергии системы}
    
    \textbf{Theorem} \textit{Об изменении кинетической энергии системы}
    
    \subsection{Закон сохранения полной механической энергии системы}
    
    \textbf{Закон} \textit{Сохранения полной механической энергии системы}
    
    \section{Евклидовы и унитарные пространства.
    Матрица Грама и её свойства.
    Неравенство Коши -- Буняковского -- Шварца, неравенство треугольника.
    Метрика.
    Выражение скалярного произведения в координатах}
    
    \textbf{Опр} \textit{Евклидово (унитарное) пространство} \textcolor{gray}{Пространство над полем с фиксированным
    скалярным произведением}
    
    \textbf{Опр} \textit{Норма (длина) вектора} \textcolor{gray}{$\abs{x} =  \beta (x, x)$}
    
    Норма неотрицательна и нулевая в случае нулевого вектора
    
    \textbf{Опр} \textit{Ортогональные векторы} \textcolor{gray}{$\beta (x, y) = 0$}
    
    \subsection{Матрица Грама и её свойства}
    
    \textbf{Опр} \textit{Матрица Грама} \textcolor{gray}{Матрица система векторов: $g_{ij} = (a_i, a_j)$}
    
    \textbf{Утв} \textcolor{blue}{Определитель матрицы Грама положителен при л.н.з системе и ноль иначе}
    
    \begin{enumerate}
        \item На л.н.з. векторах матрица Грама есть матрица п.о. билинейной симметрической формы, поэтому её
        детерминант положителен.
        \item В случае л.з. системы составим её нетривиальную л.к., домножим на векторы $a_j$ и повторим так $\forall j \in \overline{1, n}$
        \item Тогда если составить из строчек матричное уравнение, то получим $\Gamma x = 0$, что в силу $x \neq 0$
        означает вырожденность $\Gamma$
    \end{enumerate}
    
    \subsection{Неравенство Коши -- Буняковского -- Шварца, неравенство треугольника}
    
    \textbf{Следствие} \textit{Неравенство Коши -- Буняковского -- Шварца}
    
    \textcolor{blue}{$\forall a, b \in \mathscr{E} \abs{a} \abs{b} \geq \abs{(a, b)}$}
    
    Достаточно воспользоваться предыдущей теоремой и раскрыть определитель, сняв в конце квадраты
    
    \textbf{Следствие} \textit{Неравенство треугольника}
    
    \textcolor{blue}{$\forall a, b \in \mathscr{E} \abs{a} \abs{b} \geq \abs{a + b}$}
    
    Достаточно расписать $\abs{a+b}^2$ и воспользоваться предыдущим неравенством
    
    \subsection{Метрика}
    
    На $\mathscr{E}$ введём метрику как $\rho (a, b) = \abs{b - a}$.
    Заметим, что в таком случае выполняются все 4 аксиомы метрики (неотрицательность, ноль при нуле, симметричность и
    неравенство треугольника)
    
    \subsection{Выражение скалярного произведения в координатах}
    
    \textbf{Опр} \textit{Скалярное произведение} \textcolor{gray}{Билинейная (эрмитова) симметричная положительно ...}
    
    \textbf{Th} \textit{Скалярное произведение}
    
    \textcolor{blue}{$(a, b) = x^T \Gamma \overline{y}$}
    
    Это верно, потому как в случае базисных векторов матрица Грамма совпадает с матрицей билинейной формы скалярного
    произведения
    
    \section{Ортогональные системы векторов и подпространств.
    Существование ортонормированных базисов (ОНБ).
    Изоморфизм евклидовых пространств.
    Ортогональные и унитарные матрицы.
    Переход от ОНБ к ОНБ}
    
    \subsection{Ортогональные системы векторов и подпространств}
    
    \textbf{Опр} \textit{Ортогональная, ортонормированная система} \textcolor{gray}{Векторы системе попарно ...}
    
    \textbf{Утв} \textit{Теорема Пифагора}
    
    \textcolor{blue}{$\abs{a_1 + ... + a_n}^2 = \abs{a_1}^2 + \dots \abs{a_n}^2$}
    
    Раскрываем по линейности и ортогональности
    
    \textbf{Утв} \textcolor{blue}{Система ортогональная $\Leftrightarrow$ матрица Грама ортогональная}
    
    Следует из определения матрицы Грама.
    Аналогично в ортонормированном случае матрица Грама единичная
    
    \textbf{Следствие 1} \textcolor{blue}{Ортогональная система ненулевых векторов л.н.з.}
    
    Потому как соответсвующая матрица Грама невырождена
    
    \textbf{Следствие 2} \textcolor{blue}{При ортогональном базисе матрица формы скалярного произведения имеет
    диагональный вид, а при ОНБ -- канонический}
    
    \subsection{Существование ортонормированных базисов (ОНБ)}
    
    \textbf{Следствие 3} \textcolor{blue}{В конечномерном евклидовом пространстве существует ОНБ}
    
    Потому как существуют ортогональные системы.
    Если мы их запишем в виде матрицы формы, то, так как мы их умеем приводить к каноническому виду, мы получим ОНБ
    
    \subsection{Изоморфизм евклидовых пространств}
    
    \textbf{Опр} \textit{Изоморфизм евклидовых пространств} \textcolor{gray}{Изоморфизм линейных пространств и ...}
    
    \textbf{Утв} \textcolor{blue}{Отображение изоморфно $\Leftrightarrow$ оно переводит ОНБ в ОНБ}
    
    $\Rightarrow:$ в силу сохранения скалярного произведения и соразмерности пространств (следствие изоморфности)
    $\Leftarrow:$ отображение переводит базис в базис, поэтому перед нами обычный изоморфизм линейных
    пространств.
    Применим отображение на двух произвольных векторах пространства.
    И получим, что сохраняется скалярное произведение, то есть перед нами изоморфизм линейных пространств по определению
    
    \textbf{Th} \textcolor{blue}{Два конечномерных евклидова пространства изоморфны $\Leftrightarrow$ они соразмерны}
    
    $\Rightarrow:$ в силу свойств изоморфизма линейных пространств
    $\Leftrightarrow:$ приведём базисных обеих пространств к ОНБ и построим отображение, переводящее базис в базис.
    По предыдущему утверждению, перед нами изоморфизм
    
    \subsection{Ортогональные и унитарные матрицы}
    
    \textbf{Опр} \textit{Ортогональная, унитарная матрицы} \textcolor{gray}{Над разными полями, множества пересекаются}
    
    \textbf{Утв} \textcolor{blue}{Матрицы $Q, R$ унитарны $\Rightarrow$ матрицы $Q^T, \overline{Q}, Q^*, QR, Q^{-1}$
        унитарны}
    
    Непосредственно проверяется определение
    
    \textbf{Утв} \textcolor{blue}{Детерминант унитарной матрицы единичен}
    
    Для доказательства достаточно расписать определитель в определении и воспользоваться свойствами определителя
    
    \textbf{Утв} \textcolor{blue}{Для комплекснозначных матриц $Q$ следующие условия эквивалентны
        \begin{enumerate}
            \item $Q$ унитарна
            \item $\exists Q^{-1},  Q^{-1} = Q^*$
            \item Столбцы $Q$ образуют ОНБ в унитарном пространстве столбцов
        \end{enumerate}               }
    
    \begin{itemize}
        \item $1 \Leftrightarrow 2:$ по определению
        \item $1 \Leftrightarrow 3:$ в силу определения унитарной матрицы возьмём скалярное произведение и получим,
        что каждый элемент результата есть $\delta_{ij}$, то есть перед нами ОНБ
        \item Столбцы $Q$ образуют ОНБ в унитарном пространстве столбцов
    \end{itemize}
    
    \subsection{Переход от ОНБ к ОНБ}
    
    \textbf{Следствие} \textit{Переход от ОНБ к ОНБ}
    
    \textcolor{blue}{Базисы ОНБ $\Leftrightarrow$ матрица перехода между ними ортогональная (унитарная)}
    
    $\Rightarrow:$ потому что произведение матриц единично $\Leftrightarrow$ матрицы единичны
    $\Leftarrow:$ по определению матрицы перехода
    
    \section{Ортогональное дополнение подпространства.
    Ортогональная проекция.
    Алгоритм ортогонализации Грама-Шмидта}
    
    \subsection{Ортогональное дополнение подпространства}
    
    \textbf{Опр} \textit{Ортогональное дополнение} \textcolor{gray}{Множество всех векторов, ортогональных ...}
    
    Пространство образует со своим ортогональным дополнением прямую сумму
    
    \textbf{Th} \textcolor{blue}{Сумма подпространства и его ортогонального дополнения есть всё евклидово пространство}
    
    \begin{enumerate}
        \item Достаточно научиться представлять любой вектор пространства в виде суммы $U$ и $U^T$
        \item Выберем ортогональный базис в $U$ и запишем его линейную комбинацию + вектор $c \in  U^T$
        \item Теперь надо подобрать такие коэффициенты, чтобы $c \bot U$
        \item Заменим условие на эквивалентные, вспомним про ортонормированность базиса и выразим коэффициенты
    \end{enumerate}
    
    \textbf{Следствие 1} \textcolor{blue}{$\dim U = k, \dim \mathscr{E} = n \rightarrow \dim U^T = n - k$}
    
    \textbf{Следствие 2} \textcolor{blue}{$(U^T)^T = U$}
    
    Потому как одно пространство вложено в другое и у них, по предыдущему следствию, равны размерности
    
    \textbf{Следствие 2} \textcolor{blue}{В конечномерном случае данную ортогональную систему из ненулевых векторов
    можно дополнить до ОНБ}
    
    Достаточно дополнить векторами из ортогонального дополнения
    
    \subsection{Ортогональная проекция}
    
    \textbf{Опр} \textit{Ортогональная проекция} \textcolor{gray}{Проекция на подпространство вдоль (параллельно) $U^T$}
    
    \textbf{Утв} \textit{Формула проекции}
    
    \textcolor{blue}{${pr}_U \overline{a} = \sum_i \frac{(a_i, b_i)}{(b_i, b_i)} b_i$}
    
    Следствие последней теоремы
    
    \textbf{Утв} \textit{Ортогональное дополнение в координатах}
    
    \textcolor{blue}{Ортогональное дополнение есть пространство решений уравнения $(A_1, \dots, A_n)^* x = 0$}
    
    $x \in A^T \Leftrightarrow x \bot A \Leftrightarrow A_i \bot x \Leftrightarrow A_i^* x = 0$ и перейдём к матричной
    записи.
    Решение полученного уравнения и есть ортогональное дополнение
    
    \subsection{Алгоритм ортогонализации Грама-Шмидта}
    
    \textbf{Утв} \textcolor{blue}{Существует метод найти ортогональный базис в заданном подпространстве}
    
    \begin{itemize}
        \item Рассмотрим линейную оболочку подпространства.
        Если $a_1 = 0$, то выкинем его из линейной оболочки
        \item Если $a_1 \neq 0$, то оставим его таким, какой он есть: $b_1 = a_1$
        \item Если все $a_k$ до текущего уже ортогонализованы, то $b_{k+1} = a_{k+1} - {pr}_{<b_1, \dots, b_k>} a_{k+1}$
    \end{itemize}
    
    При необходимости, полученную систему можно нормировать для получения ОНБ
    
    \section{Описание линейных функций на евклидовом (унитарном) пространстве}
    
    \section{Преобразование, сопряжённое данному.
    Его линейность, существование и единственность, его матрица в ОНБ.
    Теорема Фредгольма}
    
    \subsection{Преобразование, сопряжённое данному}
    
    \textbf{Опр} \textit{Сопряжённое преобразование} \textcolor{gray}{$\varphi^*: (\varphi(a), b) = (a, \varphi(b))$}
    
    \subsection{Его линейность, существование и единственность, его матрица в ОНБ}

%    Линейность следует из линейности
    
    \textbf{Утв} \textcolor{blue}{$\psi = \varphi^* \Leftrightarrow B = A^*$}
    
    Достаточно расписать результат формы на паре векторов, определение сопряжённого преобразование и взглянуть на
    матрицы
    
    \textbf{Следствие 1} \textcolor{blue}{$\varphi^*$ единственно}
    
    Потому как у каждой матрицы есть единственная сопряжённо-транспонированная
    
    \textbf{Следствие 1} \textcolor{blue}{Для сопряжённых преобразований справедливо 4 свойства}
    
    Первые три следуют из аналогичных свойств для матриц, а последнее из свойств комплексного сопряжения
    
    \textbf{Th} \textcolor{blue}{$U$ инвариантно относительно $\varphi \Leftrightarrow U^\bot$ инвариантно
    относительно $\varphi^*$}
    
    Достаточно вспомнить определения инвариантности, ортогонального дополнения и сопряжённого образования
    
    \subsection{Теорема Фредгольма}
    
    \textbf{Th} \textit{Фредгольма}
    
    \textcolor{blue}{$\ker \varphi^* = (\Im \varphi)^\bot$}
    
    \begin{enumerate}
        \item Докажем вложенность ядра в чужой образ и равенство размерностей.
        Это будет означать равенство
        \item Равенство размерностей доказывается по прошлым утверждениям
        \item Чтобы доказать вложенность рассмотрим произвольный вектор ядра, воспользуемся определениями
        ортогонального дополнения, образа и сопряжённого преобразования
    \end{enumerate}
    
    \section{Самосопряжённые линейные преобразования.
    Свойства самосопряжённых преобразований, существование ОНБ из собственных векторов}
    
    \subsection{Самосопряжённые линейные преобразования}
    
    \textbf{Опр} \textit{Сопряжённое линейное преобразование} \textcolor{gray}{$\varphi^* = \varphi$}
    
    В таком случае $(\varphi (a), b) = (a, \varphi(b))$
    
    \subsection{Свойства самосопряжённых преобразований, существование ОНБ из собственных векторов}
    
    \textbf{Th} \textcolor{blue}{$\varphi$ самосопряжено $\Leftrightarrow A = A^*$}
    
    Аналогично доказательству для сопряжённых преобразований
    
    \textbf{Th} \textcolor{blue}{У самосопряжённого преобразования все характеристические числа действительны}
    
    В $\mathbb{C}$ достаточно расписать определение самосопряжённого преобразования, собственного числа и прийти к
    равенству $\lambda = \overline{\lambda}$, что означает действительность
    
    Так как в $\mathbb{C}$ доказано, что характеристическое уравнение имеет лишь действительные корни.
    А симметрические вещественные матрицы являются частным случаем эрмитовых, поэтому теорема доказана и в $\mathbb{R}$
    
    \textbf{Утв} \textcolor{blue}{У самосопряжённого преобразования различные корневые подпространства перпендикулярны}
    
    Достаточно рассмотреть два вектора из разных корневых подпространств, расписать определение самосопряжённого
    преобразования, собственного числа и прийти к единственному случаю $(a_i, a_j) = 0$
    
    \textbf{Th} \textit{Основная теорема о самосопряжённых преобразованиях}
    
    \textcolor{blue}{Для самосопряжённого преобразования сущетсвует ОНБ из собственных векторов}
    
    \begin{enumerate}
        \item Пусть $\dim \mathscr{E} = n$.
        В случае $n = 1$ очевидно
        \item Ортогональное дополнение первого вектора ОНБ инвариантно относительно $\varphi*$, как и относительно $\varphi$ в
        силу самосопряжённости
        \item Поэтому мы получили ортонормированный базис на сужении размерности $n - 1$ и их объединение будет ОНБ на
        подпространстве соответствующей размерности
    \end{enumerate}
    
    \section{Ортогональные и унитарные преобразования, их свойства.
    Канонический вид унитарного и ортогонального преобразования.
    Нормальные преобразования унитарных пространств}
    
    \subsection{Ортогональные и унитарные преобразования, их свойства}
    
    \textbf{Опр} \textit{Ортогональное (униатрное) преобразование} \textcolor{gray}{$(\varphi (a), \varphi (b)) = (a, b)$}
    
    \textbf{Утв} \textcolor{blue}{$\varphi$ ортогонально (матрица перехода между ОНБ) $\Leftrightarrow \varphi$
        изоморфизм евклидовых (унитарных) пространств}
    
    $\Rightarrow:$ в силу биективности (ОНБ переходит в ортонормированную систему из $n$ векторов, то есть в ОНБ,
    потому что скалярное произведение сохранено)
    $\Leftarrow:$ достаточно расписать скалярное произведение двух произвольных векторов и воспользоваться
    изоморфностью (идея как при изоморфизме линейных пространств).
    Получим сохранение скалярного произведения и ортогональность $\varphi$ по определению
    
    \textbf{Следствие 1} \textcolor{blue}{Ортогональное преобразование переводит ОНБ в ОНБ}
    
    \textbf{Следствие 2} \textcolor{blue}{Преобразование ортогонально $\Leftrightarrow$ его матрица ортогональна}
    
    Потому как ортогональная матрица -- матрица перехода между ОНБ
    
    \textbf{Следствие 3} \textcolor{blue}{Преобразование ортогонально $\Leftrightarrow \varphi$ обратимо и
    матрица $\varphi^{-1} = \varphi^*$}
    
    Достаточно расписать определение унитарного преобразования
    
    \textbf{Утв} \textit{Групповые свойства}
    
    \textcolor{blue}{Для ортогональных преобразований их композиция и обратное тоже ортогональное}
    
    Достаточно привести к определению
    
    \textbf{Утв} \textcolor{blue}{Характеристические числа ортогональных преобразований по модулю равны единице}
    
    Достаточно расписать определение и вспомнить про комплексное сопряжение
    
    \subsection{Канонический вид унитарного и ортогонального преобразования}
    
    \textbf{Th} \textit{Канонический вид унитарного преобразования}
    
    \textcolor{blue}{Для унитарного преобразования сущетсвует ОНБ из собственных векторов}
    
    \begin{enumerate}
        \item Пусть $\dim \mathscr{E} = n$.
        В случае $n = 1$ очевидно
        \item Ортогональное дополнение первого вектора ОНБ инвариантно относительно $\varphi^*$, как и относительно $\varphi^{-1}$ в
        силу ортогональности.
        При изучении инвариантных подпространств мы выяснили, что это эквивалентно инвариантности и относительно $\varphi$
        \item Поэтому мы получили ортонормированный базис на сужении размерности $n - 1$ и их объединение будет ОНБ на
        подпространстве соответствующей размерности
    \end{enumerate}
    
    \section{Полярное разложение линейного преобразования в евклидовом пространстве, его существование}
    
    \textbf{Лемма} \textit{О главных направлениях}
    
    \textcolor{blue}{Для линейного преобразования $\varphi$ существует
    ОНБ $e_1, \dots, e_n: \varphi(e_1), \dots, \varphi(e_n)$ образуют ортогональную систему}
    
    Рассмотрим оператор $\varphi^* \varphi$ (проверяется, что он СС) и ОНБ из его собственных векторов (по теореме).
    Далее, пользуясь СС-ю получаем, что $(\varphi(e_i), \varphi(e_j)) = \dots = \lambda_i \delta_{ij}$
    
    \textbf{Th} \textcolor{blue}{Для линейного преобразования $\varphi \exists$ самосопряжённое
    преобразвоание $\psi$ и ортогональное (унитарное) $\theta$}
    
    \begin{enumerate}
        \item Рассмотрим ортогональную систему из леммы, притом $\abs{\varphi(e_i)} = \sqrt{\lambda_i}$.
        При необходимости, переупорядочим её
        \item Отнормируем систему и дополним её до ОНБ, убрав, при необходимости, нулевые векторы.
        Получим ОНБ $f_1, \dots, f_n$
        \item Теперь определим $\psi = \varphi \theta^{-1}$ и убедимся, что $\psi(f_i) = \dots = \sqrt{\lambda_i} f_i$
        \item Итого, нежные отображения подобраны
    \end{enumerate}
    
    Совсем необязательно, что данные преобразования коммутируют (перестановочны).
    Однако можно применить теорему к $\varphi^*$ и взять сопряжение с обеих сторон.
    Тогда мы как раз получим другой порядок
    
    \section{Квадратичные (эрмитовы) формы в евклидовых (унитарных) пространствах.
    Присоединенный оператор.
    Существование ОНБ, в котором квадратичная (эрмитова) форма имеет диагональный вид.
    Применение к классификации кривых второго порядка.
    Одновременное приведение пары квадратичных форм к диагональному виду}
    
    \subsection{Квадратичные (эрмитовы) формы в евклидовых (унитарных) пространствах}
    
    \textbf{Опр} \textit{Квадратичная форма в евклидовом пространстве} \textcolor{gray}{$\beta_\varphi (a, b) = (a, \varphi(b))$}
    
    \textbf{Утв} \textcolor{blue}{В случае ОНБ $B = \overline{A}$}
    
    Пользуемся результатом действия билинейной формы на паре векторов и сравниваем записи.
    
    В случае произвольного базиса $B = \Gamma \overline{A}$
    
    \textbf{Следствие 1} \textcolor{blue}{Задана биекция между линейными преобразованиями и билинейными формами}
    
    \textbf{Следствие 2} \textcolor{blue}{Задана биекция между множество самосопряжённых операторов и квадратичных форм}
    
    Потому как и тем, и другим соответствует симметричная матрица
    
    Итого, изучения биленейных форм можно свести к изучению операторов (и наоборот), а изучение квадратичных -- к
    самосопряжённым операторам
    
    \subsection{Существование ОНБ, в котором квадратичная (эрмитова) форма имеет диагональный вид}
    
    \textbf{Th} \textit{Приведение к главным осям}
    
    \textcolor{blue}{Существует ОНБ, в котором матрица квадратичной формы над ЕП имеет диагональный вид}
    
    Следует из того, что для самосопряжённого оператора существует ОНБ, в котором его матрица диагональна.
    Она отличается от требуемой не более, чем сопряжением
    
    \subsection{Применение к классификации кривых второго порядка}
    
    \textbf{Лемма} \textcolor{blue}{$\exists$ ПДСК, в которой кривая второго порядка задаётся уравнением без
    перекрёстных членов. Аналогично для поверхностей}
    
    Для предъявления такой ПДСК достаточно привести квадратичную форму к главным осям
    
    \subsection{Одновременное приведение пары квадратичных форм к диагональному виду}
    
    \textbf{Th} \textit{О паре форм}
    
    \textcolor{blue}{Если в векторном пространстве (без евклидовой / унитарной структуры) заданы две симметрические
    квадратичные формы, причём первая п.о. то существует базис, в котором первая имеет канониыеский вид, а вторая --
    диагональный}
    
    Достаточно объявить п.о. форму скалярным произведением.
    Тогда будет существовать базис, в котором вторая форма диагональна
    
    \addcontentsline{toc}{section}{Сопряжённое пространство} \part*{Сопряжённое пространство}
    
    \section{Линейные функции.
    Сопряжённое пространство, его размерность.
    Биортогональный базис.
    Замена биортогональных базисов.
    Канонический изоморфизм пространства и дважды сопряжённого к нему}
    
    \subsection{Линейные функции}
    
    \textbf{Опр} \textit{Линейная функция} \textcolor{gray}{Отображение, удовлетворяющая двум аксиомам}
    
    \subsection{Сопряжённое пространство, его размерность}
    
    \textbf{Опр} \textit{Сопряжённое (двойственное) пространство} \textcolor{gray}{Пространство ...}
    
    Элементы сопряжённого пространства -- линейные функционалы (функции), поэтому такие пространства также называют
    пространством линейных функций.
    Обозначаются как $V^*$
    
    \textbf{Утв} \textcolor{blue}{$\dim V^* = \dim V$}
    
    Следует из $\dim \mathbb{R} = \dim \mathbb{C} = 1$ и отождествления с матрицами размерности $nm$
    
    Применению линейной функции к вектору, удовлетворяющему четырём аксиомам, соответствует билинейная (
    полуторалинейная) форма
    
    \subsection{Биортогональный базис}
    
    \textbf{Опр} \textit{Взаимный / биортогональный / двойственный базис} \textcolor{gray}{$<e_i, e^j> = \delta_i^j$}
    
    \textbf{Утв} \textcolor{blue}{К данном базису существует и единственен взаимный}
    
    Любому элементу взаимного базиса соответствует строчная единица.
    Строчные единицы образуют базис в $V$, поэтому и элементы взаимного базиса образуют базис в $V^*$.
    Базис единственен по построению (в силу инъективности линейных функций)
    
    \textbf{Утв} \textcolor{blue}{Двойственный базис является базисом в $V^*$}
    
    В силу равенства размерностей пространств достаточно доказать л.н.з. $f_1, \dots, f_n$.
    Это делается от противного с применением $e_j \forall j$ на линейной комбинации
    
    \textbf{Утв} \textcolor{blue}{Если при фиксированном $a \in V <a, l> = 0 \forall l \in V^*$, то $a = 0$}
    
    От противного включим $a$ в какой-то базис
    
    \textbf{Утв} \textcolor{blue}{$<a, l> = x^i \overline{y_i}$}
    
    Следует из подстановки разложений по базисам и определения $\delta_i^j$
    
    \textbf{Следствие} \textcolor{blue}{$<a, e^i> = x^i$}
    
    \subsection{Замена биортогональных базисов}
    
    \textbf{Утв} \textcolor{blue}{Если $e^{'} = eS, e^{'*} = e^* C$, то $C = (S^{-1})^*$}
    
    Тензорно запишем $e^{'}$ как строки матрицы на векторы-столбцы и введём $R = C^T$, чтобы аналогично сделать с $e^{'*}$.
    Затем раскроем по условию биортогональности и вернёмся к матричной записи
    
    \subsection{Канонический изоморфизм пространства и дважды сопряжённого к нему}
    
    \textbf{Опр} \textit{Канонический изоморфизм} \textcolor{gray}{Не меняется при замене базиса}
    
    \textbf{Опр} \textit{Дважды сопряжённое пространство} \textcolor{gray}{Отображение, сопостовляющее
    вектору $a \in V$ отображение $\overleftarrow{a}: V^* \rightarrow \mathbb{R} (\mathbb{C})$ по
    правилу $<l, \overleftarrow{a}> = \overline{<a, l>}$ есть инъевтиный гомоморфизм (вложение) $V \rightarrow V^{**}$}
    
    \textbf{Th} \textit{Канонический изоморфизм между $V$ и $V^{**}$}
    
    \textcolor{blue}{Между линейным пространством и дважды сопряжённым к нему сущесвтует канонический изоморфизм}
    
    Для доказательства достаточно проверить линейность по обеим аргументам и тривиальность ядра (всё по
    определению).
    По критерию изоморфности в силу инъективности (тривиальность ядра) имеем изоморфизм
    
    \section{Аннулятор подпространства, соответствие между подпространствами V и V*.
    Сопряжённое преобразование, его свойства}
    
    \subsection{Аннулятор подпространства, соответствие между подпространствами $V$ и $V^*$}
    
    \textbf{Опр} \textit{Биортогональные множества} \textcolor{gray}{$\forall a \in U \forall l \in W <a, l> = 0$}
    
    \textbf{Утв} \textit{Признак биортогональности}
    
    \textcolor{blue}{$U \bot W \Leftrightarrow a_i \bot l_j$}
    
    $\Rightarrow:$ очевидно в силу вложенности
    $\Leftarrow:$ из разложения по базису и линейности
    
    \textbf{Опр} \textit{Аннулятор / биортогональное дополнение} \textcolor{gray}{Множество $W$ линейных функций}
    
    Обозначается как $U^\bot$

%    \textbf{Утв} \textcolor{blue}{Если $\forall U \subset V U^\bot \in V^*$}
    
    \textbf{Опр} \textit{Нуль-пространство} \textcolor{gray}{Обратное к аннулятору: множество $U$ векторов}

%    \textbf{Утв} \textcolor{blue}{Если $\forall W \subset V^* W^\bot \in V$}
    
    \textbf{Th} \textcolor{blue}{$(U^\bot)^\bot = U$ и $\dim U + \dim U^\bot = n$}
    
    \begin{enumerate}
        \item Выберем базис $e_1, \dots, e_k$ в $U$ и дополним его до базиса во всём пространстве векторами $e_{k+1},
        \dots, e_n$
        \item Далее рассмотрим линейную функцию, записанную в своём базисе и перейдём к системе, задающей $\bot$
        \item Получим, что тогда каждый коэффициент $\lambda_i = 0, i \in \overline{1, k}$, что говорит о структуре $
        U^\bot$
        \item Аналогичную операцию произведём в $V^*$ и докажем первый факт
        \item Собрав информацию о размерностях, получим второй факт
    \end{enumerate}
    
    \subsection{Сопряжённое преобразование, его свойства}
    
    \textbf{Опр} \textit{Сопряжённое преобразование} \textcolor{gray}{Отображение уже из пространства функций}
    
    \textbf{Утв} \textcolor{blue}{Сопряжённое преобразование лежит в пространстве функций}
    
    Проверяется линейность (4 аксиомы) с использованием определения
    
    \textbf{Утв} \textcolor{blue}{Сопряжённое преобразование соотвествует матрица $A^*$}
    
    Надо разложить в матричный вид равенства из определения сопряжённого пространства и сравнить их.
    Получив искомую структуру матрицы
    
    \textbf{Следствие 1} \textcolor{blue}{Верны 4 равенства}
    
    Введём взаимные базисы и перейдём к матрицам.
    Доказательства очевидны случаю евклидова пространства
    
    \textbf{Th} \textcolor{blue}{$U$ инвариантно относительно $\varphi \Leftrightarrow U^\bot$ инвариантно относительно $\varphi^*$}
    
    $\Rightarrow:$ возьмём $f \in U^\bot$ и распишем его применение по определению
    $\Leftarrow:$ следует из $\Rightarrow$, $(U^\bot)^\bot = U$ и $(\varphi^*)^* = \varphi$
    
    \textbf{Th} \textit{Фредгольма}
    
    \textcolor{blue}{$\ker \varphi^* = (\Im \varphi)^\bot$}
    
    \begin{enumerate}
        \item Аналогично случаю в ЕП: докажем вложенность ядра в чужой образ и равенство размерностей.
        Это будет означать требуемое равенство
        \item Равенство размерностей доказывается по прошлым утверждениям
        \item Чтобы доказать вложенность рассмотрим произвольный вектор ядра, воспользуемся определениями
        ортогонального дополнения, образа и сопряжённого преобразования
    \end{enumerate}
    
    \addcontentsline{toc}{section}{Тензоры} \part*{Тензоры}
    
    \section{Полилинейные отображения.
    Определение тензора типа $(p,q)$ на линейном пространстве $V$.
    Пространство $T^p_q (V)$ тензоров типа $(p,q)$.
    Тензорный базис в $T^p_q (V)$.
    Изменение компонент тензора при замене базиса}
    
    \subsection{Полилинейные отображения}


\end{document}
