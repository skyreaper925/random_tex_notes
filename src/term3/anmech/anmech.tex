%! Author = user
%! Date = 12.01.2024

\documentclass[a4paper, 14pt]{article}

%\hypersetup
%{   colorlinks,
%    pdftitle={analysis_themeslinalg summary},
%    pdfauthor={Володин Максим},
%    allcolors=[RGB]{010 090 200}
%}

\usepackage[T2A]{fontenc}
\usepackage[utf8]{inputenc}
\usepackage[english, russian]{babel}
\usepackage[top = 2cm, bottom = 2cm, left = 2cm, right = 2cm]{geometry}
\usepackage{indentfirst}
\usepackage{xcolor}
\usepackage{hyperref}
\usepackage{gensymb}
\usepackage{pgfplots}
\usepackage{amsmath, amsfonts, amsthm, mathtools}
\usepackage{amssymb}
\usepackage{physics, multirow, float}
\usepackage{wrapfig, tabularx}
\usepackage{icomma} % Clever comma: 0,2 - number while 0, 2 - two numbers
\usepackage{tikz, standalone}
\usepackage{fancyhdr,fancybox}
\usepackage{booktabs}
\usepackage{listings}
\usepackage{lstmisc}
\usepackage{stmaryrd}
\usepackage{graphicx}
\usepackage{lastpage}

%\полуторный интервал
\onehalfspacing

\graphicspath{{images/}}
\DeclareGraphicsExtensions{.pdf,.png,.jpg}

\restylefloat{table}
\usetikzlibrary{external}

\mathtoolsset{showonlyrefs = true} % Numbers will appear only where \eqref{} in the text LINKED
\pagestyle{fancy}

\fancyhf{}
\fancyhead[R]{Конспект билетов}
\fancyfoot[R]{\thepage /\pageref{LastPage}}
\fancyhead[L]{Аналитическая механика}

\pgfplotsset{compat=1.18}

\begin{document}

{\huge
    \begin{center}
    {\textbf{Конспект билетов}}
        \\
        Аналитическая механика
    \end{center}
}
    \tableofcontents \newpage
    
    \section{Кинематика точки. Траектория, скорость и ускорение точки.
    Скорость и ускорение точки в полярных координатах}
    
    \subsection{Кинематика точки}
    
    \textbf{Опр} \textit{Кинематика точки. Траектория, скорость и ускорение точки}
    
    \textcolor{blue}{Раздел механики, изучающий математическое описание (средствами геометрии, алгебры,
        математического анализа…) движения материальной точки без рассмотрения причин движения (массы, сил и т. д.)}
    
    \textbf{Опр} \textit{Траектория}
    
    \textbf{Опр} \textit{Скорость}
    
    \textbf{Опр} \textit{Ускорение}
    
    \subsection{Скорость и ускорение точки в полярных координатах}
    
    \textbf{Опр} \textit{Радиальная ось}
    
    \textbf{Опр} \textit{Трансверсальная ось}
    
    Для того чтобы получить скорость и ускорение в полярных координатах, достаточно выразить $x$ и $y$ в терминах $r,
    \varphi$, продифференцировать нужное число раз и вычленить базисные векторы
    
    \section{Кинематика точки. Естественный трёхгранник.
    Теорема Гюйгенса о разложении ускорения точки на тангенциальное и нормальное}
    
    \subsection{Кинематика точки. Естественный трёхгранник}
    
    \textbf{Опр} \textit{Естественный способ задания движения}
    
    \textbf{Опр} \textit{Естественный трёхгранник}
    
    \subsection{Теорема Гюйгенса о разложении ускорения точки на тангенциальное и нормальное}
    
    Запишем две формулы из дифференциальной геометрии и продифференцируем $r$ и $v$ с их учётом.
    Получим две компоненты ускорения: тангенциальное и нормальное
    
    \textbf{Theorem} \textit{Гюйгенса о разложении ускорения}
    
    \section{Криволинейные координаты точки.
    Коэффициенты Ламе.
    Скорость и ускорение точки в криволинейных координатах.
    Скорость точки в цилиндрических и сферических координатах}
    
    \subsection{Криволинейные координаты точки}
    
    \textbf{Опр} \textit{Криволинейные координаты}
    
    \textbf{Опр} \textit{Первая координатная линия}
    
    \textbf{Опр} \textit{Первая координатная ось}
    
    Аналогично определяются и последующие координатные линии и оси
    
    \subsection{Коэффициенты Ламе}
    
    \textbf{Опр} \textit{Единичный вектор координатной оси}
    
    \textbf{Опр} \textit{Коэффициент Ламе}
    
    \textbf{Опр} \textit{Ортогональные криволинейные координаты}
    
    \subsection{Скорость и ускорение точки в криволинейных координатах}
    
    Скорость находится по определению.
    Ускорение смотреть в конспекте Холостовой с 8 страницы
    
    \subsection{Скорость точки в цилиндрических и сферических координатах}
    
    \textbf{Опр} \textit{Цилиндрическая система координат}
    
    \textbf{Опр} \textit{Сферическая система координат}
    
    Скорость точки в этих координатах находится с помощью коэффициентов Ламе
    
    \section{Угловая скорость и угловое ускорение твёрдого тела.
    Скорости и ускорения точек твёрдого тела в общем случае его движения (формулы Эйлера и Ривальса)}
    
    \subsection{Угловая скорость и угловое ускорение твёрдого тела}
    
    \textbf{Опр} \textit{Поступательно движущаяся и связанная системы координат}
    
    \textbf{Опр} \textit{Углы Эйлера}
    
    \textcolor{blue}{Углы, описывающие поворот абсолютно твердого тела в трёхмерном евклидовом пространстве}
    
    \textbf{Опр} \textit{Линия узлов}
    
    \textcolor{blue}{Пересечение координатных плоскостей начальной и конечной СК}
    
    \textbf{Опр} \textit{Угол прецессии, нутации, собтвенного вращения}
    
    Переход от одной системы координат к другой посредством вращений на углы, можно задать с помощью матриц поворота
    
    \textcolor{blue}{Матрица поворота (или матрица направляющих косинусов)}
    
    \textbf{Опр} \textit{Ортогональная матрица, которая используется для выполнения собственного ортогонального
    преобразования в евклидовом пространстве.
    При умножении любого вектора на матрицу поворота длина вектора сохраняется.
    Определитель матрицы поворота равен единице}
    
    Матрицы поворота вокруг различных осей выглядят по-разному
    
    \textbf{Опр} \textit{Угловая скорость}
    
    \textcolor{blue}{Физическая величина, характеризующая быстроту и направление вращения материальной точки или
    абсолютно твёрдого тела относительно оси}
    
    \textbf{Опр} \textit{Угловое ускорение}
    
    \subsection{Скорости и ускорения точек твёрдого тела в общем случае его движения (формулы Эйлера и Ривальса)}
    
    \textbf{Theorem} \textit{Формула Эйлера}
    
    \textbf{Формула} \textit{Ривальса} \textcolor{gray}{Получается формальным дифференцированием формулы Эйлера}
    
    \section{Плоское движение твёрдого тела.
    Мгновенный центр скоростей.
    Мгновенный центр ускорений}
    
    \subsection{Плоское движение твёрдого тела}
    
    \textbf{Опр} \textit{Плоское движение}
    
    \subsection{Мгновенный центр скоростей}
    
    \textbf{Theorem} \textit{О мгновенном центре скоростей}
    
    \textbf{Опр} \textit{Мгновенный центр скоростей}
    
    \subsection{Мгновенный центр ускорений}
    
    \textbf{Theorem} \textit{О мгновенном центре ускорений}
    
    \textbf{Опр} \textit{Мгновенный центр ускорений}
    
    Мгновенный центр ускорений можно построить за два шага
    
    \section{Кинематические инварианты.
    Кинематический винт.
    Мгновенная винтовая ось}
    
    \subsection{Кинематические инварианты}
    
    \textbf{Опр} \textit{Инвариант}
    
    \textcolor{blue}{Величина, остающаяся неизменной при преобразованиях}
    
    \textbf{Опр} \textit{Первый кинематический инвариант}
    
    \textbf{Опр} \textit{Второй кинематический инвариант}
    
    Отсюда следует, что величины проекции скоростей двух точек поступательного движущегося тела на прямую, их
    соединяющуюся, равны
    
    \subsection{Кинематический винт}
    
    \textbf{Опр} \textit{Кинематический винт}
    
    \textbf{Опр} \textit{Параметр винта}
    
    \subsection{Мгновенная винтовая ось}
    
    Если расписать итоговую скорость точки по координатам, то можно получить
    
    \textbf{Опр} \textit{Мгновенная винтовая ось}
    
    \textbf{Опр} \textit{Правый и левый винт}
    
    \section{Алгебра кватернионов}
    
    \textbf{Опр} \textit{Кватернион, кватернионные единицы}
    
    \textbf{Свойства} \textit{Кватернионного сложения}
    
    \textbf{Опр} \textit{Скалярная и векторные части кватерниона}
    
    \textbf{Свойства} \textit{Кватернионного умножения единиц}
    
    \textbf{Свойства} \textit{Кватернионного умножения}
    
    \textbf{Опр} \textit{Сопряжённый кватернион}
    
    \textbf{Опр} \textit{Норма кватерниона, нормированный кватернион}
    
    \textbf{Опр} \textit{Обратный кватернион}
    
    \textbf{Форма} \textit{Тригонометрическая записи кватерниона}
    
    Результат умножения кватернионов в таком случае получается из свойств тригонометрии
    
    \textbf{Аналог} \textit{Формулы Муавра}
    
    \section{Кватернионный способ задания ориентации твёрдого тела.
    Теорема Эйлера о конечном повороте}
    
    \subsection{Кватернионный способ задания ориентации твёрдого тела}
    
    \textbf{Опр} \textit{Неподвижный и связанный базисы}
    
    \textbf{Theorem}
    
    \subsection{Теорема Эйлера о конечном повороте}
    
    \textbf{Theorem} \textit{Эйлера о конечном повороте}
    
    Если воспользоваться предыдущей теоремой, то видно, что при повороте положение $e$ сохраняется, а $j$ поворачивается
    
    \section{Формулы сложения поворотов твёрдого тела в кватернионах.
    Параметры Родрига-Гамильтона.
    Теорема Эйлера о конечном повороте твёрдого тела с неподвижной точкой}
    
    \subsection{Формулы сложения поворотов твёрдого тела в кватернионах}
    
    Можно показать, что результирующий кватернион после $N$ поворотов будет записан в обратном порядке в одном базисе
    
    \subsection{Параметры Родрига-Гамильтона}
    
    \textbf{Опр} \textit{Параметры Родрига-Гамильтона}
    
    Если записать преобразованные от смены базисные единицы и подставить в новый кватернион, то он будет выражен в
    исходном базисе через параметры Родрига-Гамильтона.
    Порядок записи кватернионов в результирующем повороте будет уже прямой
    
    \subsection{Теорема Эйлера о конечном повороте твёрдого тела с неподвижной точкой}
    
    \textbf{Theorem} \textit{Эйлера о конечном повороте твёрдого тела с неподвижной точкой}
    
    \section{Кинематические уравнения вращательного движения твёрдого тела в кватернионах (уравнения Пуассона).
    Прецессионное движение твёрдого тела}
    
    \subsection{Кинематические уравнения вращательного движения твёрдого тела в кватернионах (уравнения Пуассона)}
    
    \textbf{Опр} \textit{Угловая скорость} \textcolor{gray}{Через предел}
    
    \textbf{Уравнение} \textit{Пуассона}
    
    Можно показать, что два определения угловой скорости эквивалентны.
    В конце мы придём к уравнению Эйлера (то есть верному утверждению), а значит мы были правы
    
    \subsection{Прецессионное движение твёрдого тела}
    
    Рассмотрим вращение оси тела вокруг неподвижной вращающейся оси и решим уравнение Пуассона для этого случая
    
    \section{Кинематика сложного движения точки.
    Вычисление скоростей и ускорений в сложном движении}
    
    \subsection{Кинематика сложного движения точки}
    
    \textbf{Опр} \textit{Относительное, переносное и абсолютное движение}
    
    Можно посчитать относительные и абсолютные производные радиус-вектора и получить их связь
    
    \subsection{Вычисление скоростей и ускорений в сложном движении}
    
    \textbf{Опр} \textit{Относительные, переносные и абсолютные скорость и ускорение}
    
    \textbf{Theorem} \textit{О сложении скоростей}
    
    \textbf{Theorem} \textit{О сложении ускорений или теорема Кориолиса}
    
    \textbf{Опр} \textit{Кориолисово ускорение}
    
    \section{Кинематика сложного движения тела.
    Сложение мгновенных вращений твёрдого тела вокруг пересекающихся осей.
    Кинематические уравнения Эйлера}
    
    \subsection{Кинематика сложного движения тела. Сложение мгновенных вращений твёрдого тела вокруг пересекающихся
    осей}
    
    \textbf{Сложение} \textit{Поступательных движений}
    
    \textbf{Сложение} \textit{Вращательных движений}
    
    \subsection{Кинематические уравнения Эйлера}
    
    Если тело участвует одновременно в трёх вращениях, то записав суммарное вращение в проекциях на связанные оси, имеем
    
    \textbf{Уравнения} \textit{Эйлера кинематические}
    
    \section{Кинематика сложного движения тела.
    Сложение мгновенных вращений твёрдого тела вокруг параллельных осей.
    Пара вращений}
    
    \subsection{Кинематика сложного движения тела. Сложение мгновенных вращений твёрдого тела вокруг параллельных осей}
    
    \textbf{Сложение} \textit{Сонаправленных вращений}
    
    \textbf{Сложение} \textit{Разнонаправленных вращений}
    
    \subsection{Пара вращений}
    
    \textbf{Пара} \textit{Вращений}
    
    \textbf{Опр} \textit{Момент и плечо пары вращений}
    
    Поступательное движение можно заменить на пару вращений бесчисленным множеством способов
    
    \section{Сложное движение твёрдого тела.
    Общий случай сложения движений}
    
    \subsection{Сложное движение твёрдого тела}
    
    \textbf{Лемма}
    
    \subsection{Общий случай сложения движений}
    
    В общем случае движения приведём все поступательные и вращательные к единой точке приложения по алгоритму
    
    \textbf{Алгоритм} \textit{Приведения к простому движению}
    
    \section{Момент силы относительно точки и оси, главный вектор и главный момент сил системы.
    Элементарная работа сил системы.
    Работа сил, приложенных к твёрдому телу.
    Силовое поле.
    Силовая функция.
    Потенциал}
    
    \subsection{Момент силы относительно точки и оси, главный вектор и главный момент сил системы}
    
    \textbf{Опр} \textit{Сила}
    
    \textcolor{blue}{Мера воздействия тел друг на друга, причина ускорения точки}
    
    \textbf{Аксиома} \textit{Инерции}
    
    \textbf{Опр} \textit{Инертность, масса}
    
    \textbf{Закон} \textit{Динамики основной}
    
    \textbf{Аксиома} \textit{Взаимодействия материальных точек}
    
    \textbf{Аксиома} \textit{Независимости действия сил (принцип суперпозиции)}
    
    \textbf{Опр} \textit{Главный вектор всех сил системы}
    
    \textbf{Опр} \textit{Момент силы относительно точки}
    
    \textbf{Опр} \textit{Момент силы относительно оси}
    
    Можно показать корректность этого определения
    
    \textbf{Опр} \textit{Главный момент сил системы}
    
    \subsection{Элементарная работа сил системы}
    
    \textbf{Опр} \textit{Элементарная работа}
    
    Можно получить выражение для полной работы
    
    \subsection{Работа сил, приложенных к твёрдому телу}
    
    В общем случае работа внутренних сил ненулевая.
    Запишем суммарную работу всех сил системы
    
    \subsection{Силовое поле}
    
    \textbf{Опр} \textit{Силовое поле}
    
    \textcolor{blue}{Векторное поле в пространстве, в каждой точке которого на точку действует определённая
    по величине и направлению сила (вектор силы)}
    
    \subsection{Силовая функция}
    
    \textbf{Опр} \textit{Силовая функция}
    
    \textbf{Опр} \textit{Потенциальное поле}
    
    \textbf{Опр} \textit{Потенциальная сила}
    
    \textbf{Опр} \textit{(Не)стационарное поле}
    
    \subsection{Потенциал}
    
    \textbf{Опр} \textit{Потенциал}
    
    \textcolor{blue}{Скалярная величина, характеризующая силовое поле}
    
    \textbf{Утв}
    
    \textbf{Опр} \textit{Потенциальная энергия}
    
    \textcolor{blue}{Скалярная физическая величина, представляющая собой часть полной механической
    энергии системы, находящейся в поле консервативных сил}
    
    \section{Количество движения.
    Центр масс.
    Теорема об изменении количества движения системы.
    Теорема о движении центра масс}
    
    \subsection{Количество движения}
    
    \textbf{Опр} \textit{Количество движения (импульс)}
    
    \subsection{Центр масс}
    
    \textbf{Опр} \textit{Центр масс системы}
    
    \subsection{Теорема об изменении количества движения системы}
    
    \textbf{Theorem} \textit{Об изменении количества движения системы}
    
    \subsection{Теорема о движении центра масс}
    
    \textbf{Theorem} \textit{О движении центра масс}
    
    \section{Главный момент количества движения (кинетический момент) системы относительно заданного центра.
    Кинетический момент системы для ее движения относительно центра масс.
    Теорема Кенига о вычислении кинетического момента}
    
    \subsection{Главный момент количества движения (кинетический момент) системы относительно заданного центра}
    
    \textbf{Опр} \textit{Момент импульса (кинетический момент) точки}
    
    \subsection{Кинетический момент системы для ее движения относительно центра масс}
    
    \textbf{Опр} \textit{Кинетический момент (главный момент количества движения) системы}
    
    \textbf{Опр} \textit{Кинетический момент системы относительно точки}
    
    Можно показать корректность этого определения
    
    Покажем связь главных моментов двух точек в общем и частном случаях
    
    \subsection{Теорема Кенига о вычислении кинетического момента}
    
    \textbf{Опр} \textit{Кёнигова система координат}
    
    Найдём выражения для скорости и кинетического момента точки и системы
    
    Если под движением системы относительно центра масс понимать движение в Кёниговой системе координат, то верна
    
    \textbf{Theorem} \textit{Кёнига о кинетическом моменте}
    
    \section{Теорема об изменении кинетического момента системы}
    
    Посчитаем производную кинетического момент относительно точки
    
    \textbf{Theorem} \textit{Об изменении кинетического момента системы}
    
    Также рассмотрим частные случаи теоремы
    
    \section{Кинетическая энергия системы.
    Теорема Кенига о вычислении кинетической энергии.
    Теорема об изменении кинетической энергии системы.
    Закон сохранения полной механической энергии системы}
    
    \subsection{Кинетическая энергия системы}
    
    \textbf{Опр} \textit{Кинетическая энергия системы}
    
    Запишем, как она преобразуется при смене системы координат
    
    \subsection{Теорема Кенига о вычислении кинетической энергии}
    
    В частном случае прошлых выкладок получаем
    
    \textbf{Theorem} \textit{Кенига о вычислении кинетической энергии}
    
    \subsection{Теорема об изменении кинетической энергии системы}
    
    \textbf{Theorem} \textit{Об изменении кинетической энергии системы}
    
    \subsection{Закон сохранения полной механической энергии системы}
    
    \textbf{Закон} \textit{Сохранения полной механической энергии системы}
    
    \section{Основные теоремы динамики в неинерциальной системе отсчёта.
    Переносная и кориолисова силы инерции.
    Основные теоремы динамики для движения относительно центра масс}
    
    \subsection{Основные теоремы динамики в неинерциальной системе отсчёта. Переносная и кориолисова силы инерции}
    
    Выразим относительное ускорение в неИСО
    
    \textbf{Опр} \textit{Переносная и кориолисова силы инерции}
    
    \textbf{Закон} \textit{Основной динамики в неИСО}
    
    \textbf{Theorem} \textit{Об изменении количества движения}
    
    \textbf{Опр} \textit{Главный вектор внешних сил и сил инерции}
    
    \textbf{Theorem} \textit{О движении центра масс}
    
    Для неподвижной точки в неИСО справедлива
    
    \textbf{Theorem} \textit{Об изменении кинетического момента}
    
    \textbf{Theorem} \textit{Об изменении кинетической энергии}
    
    \subsection{Основные теоремы динамики для движения относительно центра масс}
    
    Все теоремы далее записаны в Кёниговой системе координат
    
    \textbf{Theorem} \textit{Об изменении количества движения}
    
    \textbf{Theorem} \textit{Об изменении кинетического момента}
    
    \textbf{Theorem} \textit{Об изменении кинетической энергии}
    
    \section{Движение материальной точки в центральном поле.
    Интеграл площадей; второй закон Кеплера.
    Уравнение Бине}
    
    \subsection{Движение материальной точки в центральном поле}
    
    \textbf{Опр} \textit{Центральное поле} \textcolor{gray}{Сила должна удовлетворять условию}
    
    \subsection{Интеграл площадей; второй закон Кеплера}
    
    \textbf{Опр} \textit{Интеграл площадей}
    
    \textbf{Опр} \textit{Радиальная и трансверсальная скорости}
    
    \textbf{Форма} \textit{Полярная интеграла площадей}
    
    \textbf{Опр} \textit{Секториальная скорость точки}
    
    \textbf{Закон} \textit{Кеплера II}
    
    \subsection{Уравнение Бине}
    
    Если переписать основной закон динамики в центральном поле, то получим
    
    \textbf{Уравнение} \textit{Бине}
    
    \section{Движение точки в поле всемирного тяготения: уравнение орбиты, законы Кеплера.
    Интеграл площадей, интеграл энергии, интеграл Лапласа.
    Задача двух тел}
    
    \subsection{Движение точки в поле всемирного тяготения: уравнение орбиты, законы Кеплера}
    
    \textbf{Опр} \textit{Поле всемирного тяготения}
    
    Если записать уравнение Бине для силы всемирного тяготения, то в конечном итоге получим
    
    \textbf{Уравнение} \textit{Орбиты}
    
    \textbf{Опр} \textit{Параметр и эксцентриситет орбиты}
    
    В зависимости от эксцентриситета, орбита будет являться той или иной конической поверхностью
    
    \textbf{Закон} \textit{Кеплера I}
    
    \textbf{Закон} \textit{Кеплера III}
    
    \subsection{Интеграл площадей, интеграл энергии, интеграл Лапласа}
    
    Интеграл площадей был рассмотрен в предыдущем билете
    
    \textbf{Интеграл} \textit{Энергии}
    
    \textbf{Интеграл} \textit{Лапласа}
    
    \textbf{Опр} \textit{Вектор Лапласа, истинная аномалия}
    
    \subsection{Задача двух тел}
    
    Рассмотрев движение двух тел в центральном поле друг друга, получим
    
    \textbf{Уравнение} \textit{Относительного движения точек}
    
    Эта система замкнута и её центр масс движется равномерно, поэтому можно найти закон изменения его радиус-вектора
    
    \section{Момент инерции системы относительно оси.
    Матрица тензора инерции.
    Эллипсоид инерции.
    Главные оси и главные моменты инерции}
    
    \subsection{Момент инерции системы относительно оси}
    
    \textbf{Опр} \textit{Момент инерции системы относительно оси}
    
    \textbf{Опр} \textit{Осевые и центробежные моменты инерции}
    
    \subsection{Матрица тензора инерции}
    
    \textbf{Опр} \textit{Тензор инерции системы для точки}
    
    \subsection{Эллипсоид инерции}
    
    Найдём момент инерции тела относительно оси, заданной направляющими косинусами
    
    \textbf{Уравнение} \textit{Эллипсоида инерции системы}
    
    \subsection{Главные оси и главные моменты инерции}
    
    \textbf{Опр 1} \textit{Главные оси инерции}
    
    \textbf{Опр} \textit{Главные моменты инерции}
    
    \textbf{Уравнение} \textit{Эллипсоида инерции в главных осях}
    
    \textbf{Опр 2} \textit{Главные оси инерции}
    
    Можно показать эквивалентность двух определений
    
    \textbf{Утв} \textit{Угол поворота осей для перехода к главным}
    
    \section{Моменты инерции относительно параллельных осей; теорема Гюйгенса – Штейнера.
    Преобразование матрицы тензора инерции при параллельном переносе осей.
    Свойства осевых моментов инерции}
    
    \subsection{Моменты инерции относительно параллельных осей; теорема Гюйгенса – Штейнера}
    
    \textbf{Theorem} \textit{Гюйгенса – Штейнера}
    
    \subsection{Преобразование матрицы тензора инерции при параллельном переносе осей}
    
    \textbf{Theorem} \textit{Гюйгенса – Штейнера для тензора инерции}
    
    \subsection{Свойства осевых моментов инерции}
    
    \textbf{Theorem} \textit{Неравенства треугольников для осевых моментов инерции}
    
    Равенство достигается только в случае плоского распределения масс
    
    \section{Кинетический момент и кинетическая энергия твёрдого тела, вращающегося вокруг неподвижной точки или
    вокруг неподвижной оси.
    Кинетический момент и кинетическая энергия твёрдого тела при его произвольном движении}
    
    \subsection{Кинетический момент и кинетическая энергия твёрдого тела, вращающегося вокруг неподвижной точки или
    вокруг неподвижной оси}
    
    \textbf{Утв} \textit{Кинетический момент тела, вращающегося вокруг неподвижной точки}
    
    Можно рассмотреть два частных случая и получить
    
    \textbf{Утв} \textit{Кинетический момент тела, вращающегося вокруг неподвижной оси}
    
    \textbf{Утв} \textit{Кинетическая энергия твёрдого тела, вращающегося вокруг неподвижной точки}
    
    Можно рассмотреть два частных случая и получить
    
    \textbf{Утв} \textit{Кинетическая энергия тела с неподвижной осью}
    
    \subsection{Кинетический момент и кинетическая энергия твёрдого тела при его произвольном движении}
    
    \textbf{Утв} \textit{Кинетический момент и кинетическая энергия твёрдого тела при его произвольном движении}
    
    \section{Дифференциальное уравнение вращения твёрдого тела вокруг неподвижной оси.
    Дифференциальные уравнения движения свободного твёрдого тела.
    Уравнения плоского движения твёрдого тела}
    
    \subsection{Дифференциальное уравнение вращения твёрдого тела вокруг неподвижной оси}
    
    Если запишем производную кинетического момента, то получим
    
    \textbf{Уравнение} \textit{Дифференциальное вращения твёрдого тела вокруг неподвижной оси}
    
    \subsection{Дифференциальные уравнения движения свободного твёрдого тела}
    
    \textbf{Уравнение} \textit{Эйлера динамические}
    
    Для свободного тела можно найти два интеграла движения
    
    \subsection{Уравнения плоского движения твёрдого тела}
    
    В случае плоского движения направим $Oz$ по вектору вращения и получим
    
    \textbf{Уравнение} \textit{Плоского движения твёрдого тела}
    
    \section{Дифференциальные уравнения движения твёрдого тела вокруг неподвижной точки. Динамические уравнения Эйлера.
    Случай Эйлера движения твёрдого тела вокруг неподвижной точки: первые интегралы уравнений движения; стационарные
    вращения}
    
    \subsection{Дифференциальные уравнения движения твёрдого тела вокруг неподвижной точки. Динамические уравнения
    Эйлера}
    
    Эти уравнения были подробно получены в прошлом билете
    
    \subsection{Случай Эйлера движения твёрдого тела вокруг неподвижной точки: первые интегралы уравнений движения;
    стационарные вращения}
    
    \textbf{Опр} \textit{Первый интеграл}
    
    Первый два интеграла были также получены в прошлом билете
    
    \textbf{Опр} \textit{Стационарное вращение}
    
    \textbf{Случай} \textit{Эйлера}
    
    \textcolor{blue}{Ось вращения неподвижная в теле и пространственный модуль $\omega$ постоянен}
    
    \textbf{Случай} \textit{Асимметрии}
    
    \textbf{Случай} \textit{Динамической симметрии}
    
    \textbf{Случай} \textit{Сферической симметрии}
    
    Итак, в случае Эйлера вращение может происходить только вокруг главных осей инерции
    
    \section{Случай Эйлера движения твёрдого тела вокруг неподвижной точки: регулярная прецессия в случае
    динамической симметрии тела; геометрическая интерпретация Пуансо}
    
    Найдём пару интегральчиков движения, из которых получим
    
    \textbf{Утв} \textit{Угловая скорость прецессии}
    
    \textbf{Утв} \textit{Угловая скорость собственного вращения}
    
    Таким образом, динамически симметричное тело в случае Эйлера совершает регулярную прецессию (вращение вокруг
    вращающейся оси).
    Найдём у этого явления геометрическую интерпретацию (Пуансо)
    
    \textbf{Опр} \textit{Плоскость Пуансо}
    
    Получим четыре новых факта
    
    \begin{enumerate}
        \item Пропорциональность скорости вращения и радиус-вектора постоянна
        \item Вектор кинетического момента перпендикулярен плоскости Пуансо
        \item Плоскость Пуансо неподвижна в абсолютном пространстве
        \item Скорость точки касания отсутствует (то есть качение без проскальзывания)
    \end{enumerate}
    
    \section{Вынужденная регулярная прецессия динамически симметричного твёрдого тела с неподвижной точкой.
    Основная формула гироскопии.
    О приближенной теории гироскопов}
    
    \subsection{Вынужденная регулярная прецессия динамически симметричного твёрдого тела с неподвижной точкой}
    
    \textbf{Опр} \textit{Гироскоп}
    
    Мы уже рассмотрели в предыдущем билете случай нулевого внешнего момента.
    Теперь рассмотрим случай ненулевого
    
    \subsection{Основная формула гироскопии}
    
    \textbf{Формула} \textit{Основная гироскопии}
    
    Внешний момент направлен по линии узлов и сохраняет своё абсолютное значение
    
    \subsection{О приближенной теории гироскопов}
    
    В случае, когда тело вращается сильно быстрее, чем поворачивается, верна
    
    \textbf{Формула} \textit{Приближенная гироскопии}
    
    \section{Уравнения движения тяжёлого твёрдого тела с неподвижной точкой.
    Первые интегралы.
    Случаи Эйлера, Лагранжа, Ковалевской интегрируемости уравнений движения}
    
    \subsection{Уравнения движения тяжёлого твёрдого тела с неподвижной точкой}
    
    Рассмотрим движение твёрдого тела в неподвижной и связной системах координат и получим
    
    \textbf{Уравнения} \textit{Пуассона}
    
    Если дополнительно запишем динамические уравнения Эйлера, то вкупе с Пуассоновыми, получим
    
    \textbf{Уравнения} \textit{Эйлера--Пуассона}
    
    \subsection{Первые интегралы}
    
    \textbf{Опр} \textit{Геометрический (тривиальный) интеграл}
    
    Помимо него, всегда существуют ещё два: проекция кинетического момента на $Oz$ и механическая энергия
    
    \subsection{Случаи Эйлера, Лагранжа, Ковалевской интегрируемости уравнений движения}
    
    Из теории дифференциальных уравнений известно, что для того, чтобы наша система из шести уравнений была решена
    хотя бы в квадратурах, необходимо, чтобы существовал дополнительный, независимый от них первый интеграл.
    Случаев такой интегрируемости не так много
    
    \begin{enumerate}
        \item Случай Эйлера
        \item Случай Лагранжа
        \item Случай Ковалевской
        \item Некоторые частные решения с определённым классом начальных условий
    \end{enumerate}
    
    \section{Случай Лагранжа движения твёрдого тела с неподвижной точкой.
    Регулярная прецессия в случае Лагранжа.
    Общий случай исследования движения; геометрическая интерпретация}
    
    \subsection{Случай Лагранжа движения твёрдого тела с неподвижной точкой}
    
    Сам случай Лагранжа упоминался в прошлом билете
    
    \subsection{Регулярная прецессия в случае Лагранжа}
    
    Еси запишем момент внешних сил в случае точной гироскопии, получим разное количество прецессий
    
    \subsection{Общий случай исследования движения; геометрическая интерпретация}
    
    В общем случае запишем первые интегралы, рассмотрим область возможных движений и получим 1 или 2 корня на $[-1; 1]$
    
    Рассмотрим геометрическую интерпретацию с помощью единичной сферы
    
    \textbf{Опр} \textit{Апекс}
    
    \section{Несвободные системы.
    Связи и их классификация.
    Возможные, действительные и виртуальные перемещения точек системы.
    Число степеней свободы системы}
    
    \subsection{Несвободные системы}
    
    \textbf{Опр} \textit{(Не)свободные системы}
    
    \subsection{Связи и их классификация}
    
    \textbf{Опр} \textit{Связь}
    
    \textbf{Опр} \textit{Удерживающая (двусторонняя неосвобождающая) связь}
    
    \textbf{Опр} \textit{Неудерживающая (односторонняя освобождающая) связь}
    
    \textbf{Опр} \textit{Геометрическая (конечная голономная) связь}
    
    \textbf{Опр} \textit{Дифференциальная ((не)интегрируемая или (неголономная) геометрическая) связь}
    
    У последних связей есть чёткое аналитическое представление, а в случае примера --- конька на льду --- можно
    показать её неинтегрируемость, используя понятие л.н.з,
    
    \textbf{Опр} \textit{Стационарная геометрическая или дифференциальная связь}
    
    \textbf{Опр} \textit{(Не)голономная система}
    
    \textbf{Опр} \textit{Склерономная и реономная система}
    
    \subsection{Возможные, действительные и виртуальные перемещения точек системы}
    
    \textbf{Опр} \textit{Возможное перемещение}
    
    \textbf{Опр} \textit{Действительное перемещение}
    
    \textbf{Опр} \textit{Виртуальное перемещение}
    
    \subsection{Число степеней свободы системы}
    
    \textbf{Опр} \textit{Число степеней свободы системы}
    
    \textbf{Опр} \textit{Число независимых обобщённых координат}
    
    У этих координат есть три свойства и их число совпадает со степенями свободы для голономных систем
    
    \section{Идеальные связи.
    Общее уравнение динамики (принцип Даламбера-Лагранжа)}
    
    \subsection{Идеальные связи}
    
    \textbf{Опр} \textit{Идеальные связи}
    
    Рассмотрим четыре системы и найдём работы
    
    \subsection{Общее уравнение динамики (принцип Даламбера-Лагранжа)}
    
    \textbf{Уравнения} \textit{Общее динамики (принцип Даламбера-Лагранжа)}
    
    \textbf{Опр} \textit{Принцип}
    
    \textbf{Опр} \textit{Вариационный принцип (дифференциальный или интегральный)}
    
    Принцип Даламбера -- Лагранжа является дифференциальным и обладает двумя свойствами
    
    \section{Элементарная работа сил системы в обобщённых координатах.
    Обобщённые силы и их вычисление.
    Случай потенциального поля сил}
    
    \subsection{Элементарная работа сил системы в обобщённых координатах}
    
    \textbf{Опр} \textit{Обобщённые скорости}
    
    \subsection{Обобщённые силы и их вычисление}
    
    \textbf{Опр} \textit{Обобщённая сила}
    
    В общем случае силы вычисляются как частные производные
    
    \subsection{Случай потенциального поля сил}
    
    В случае потенциального поля используется потенциальная энергия (полностью или частично)
    
    \section{Общее уравнение динамики в обобщённых координатах.
    Уравнения Лагранжа второго рода}
    
    \subsection{Общее уравнение динамики в обобщённых координатах}
    
    \textbf{Уравнения} \textit{Общее динамики в обобщённых координатах}
    
    \subsection{Уравнения Лагранжа второго рода}
    
    \textbf{Уравнения} \textit{Лагранжа второго рода}
    
    Они обладают свойством ковариантности
    
    \section{Уравнения Лагранжа второго рода в случае потенциальных сил.
    Функция Лагранжа.
    Циклические координаты и первые интегралы.
    Об уравнениях Лагранжа второго рода в неинерциальных системах отсчёта}
    
    \subsection{Уравнения Лагранжа второго рода в случае потенциальных сил}
    
    В случае потенциальных сил уравнения Лагранжа принимают более простой вид
    
    \subsection{Функция Лагранжа}
    
    \textbf{Опр} \textit{Функция Лагранжа (Лагранжиан)}
    
    \subsection{Циклические координаты и первые интегралы}
    
    \textbf{Опр} \textit{Циклические (безразличная) координата}
    
    \textbf{Опр} \textit{Первый интеграл, соответствующий циклической координате}
    
    \textbf{Опр} \textit{Позиционная координата}
    
    \subsection{Об уравнениях Лагранжа второго рода в неинерциальных системах отсчёта}
    
    В неинерциальных системах отсчёта уравнения Лагранжа можно записать двумя способами.
    Если использовать одни и те же обобщённые координаты, то в итоге получим одни и те же уравнения движения
    
    \section{Анализ выражения кинетической энергии системы как функции обобщённых скоростей.
    Разрешимость уравнений Лагранжа относительно обобщённых ускорений}
    
    \subsection{Анализ выражения кинетической энергии системы как функции обобщённых скоростей}
    
    Проанализируем кинетическую энергию (результат не будет зависеть от голономности системы)
    
    \textbf{Утв} \textit{Получили квадратичную функцию с тремя слагаемыми}
    
    В случае склерономной системы результат будет проще
    
    \textbf{Утв} \textit{В склерономной системе кинетическая энергия -- положительно определённая квадратичная форма}
    
    \textbf{Следствие} \textit{Часть кинетической энергии -- невырожденная квадратичная форма}
    
    \subsection{Разрешимость уравнений Лагранжа относительно обобщённых ускорений}
    
    \textbf{Утв} \textit{Система уравнений Лагранжа относительно обобщённых ускорений разрешима единственным образом}
    
    А при некоторых ограничениях решение единственно даже при произвольных начальных данных
    
    \section{Теорема об изменении полной механической энергии голономной системы.
    Интеграл энергии, консервативные системы.
    Обобщённый интеграл энергии (интеграл Якоби-Пенлеве), обобщенно-консервативные системы}
    
    \subsection{Теорема об изменении полной механической энергии голономной системы}
    
    \textbf{Theorem} \textit{Об изменении полной механической энергии голономной системы}
    
    У теоремы есть два частных случая
    
    \subsection{Интеграл энергии, консервативные системы}
    
    \textbf{Опр} \textit{Консервативная система}
    
    \textbf{Опр} \textit{Интеграл энергии}
    
    \subsection{Обобщённый интеграл энергии (интеграл Якоби-Пенлеве), обобщенно-консервативные cистемы}

    \textbf{Опр} \textit{Обобщённо-консервативная система}
    
    \textbf{Опр} \textit{Обобщённый интеграл энергии (интеграл Якоби-Пенлеве)}
    
    \section{Гироскопические силы.
    Диссипативные силы, функция Релея.
    Обобщённый потенциал.
    Натуральные и ненатуральные системы}
    
    \subsection{Гироскопические силы}
    
    \textbf{Опр} \textit{Гироскопическая сила}
    
    \textbf{Критерий} \textit{Гироскопичности}
    
    \textbf{Условие} \textit{Гироскопичности}
    
    \subsection{Диссипативные силы, функция Релея}
    
    \textbf{Опр} \textit{Непотенциальная диссипативная сила}
    
    \textbf{Опр} \textit{(Не)полная диссипация}
    
    \textbf{Опр} \textit{Диссипативная система}
    
    \textbf{Опр} \textit{Диссипативная функция Релея}
    
    \subsection{Обобщённый потенциал}
    
    \textbf{Опр} \textit{Обобщённый потенциал}
    
    Распишем его поподробнее и получим первый интегральчик
    
    \subsection{Натуральные и ненатуральные системы}
    
    \textbf{Опр} \textit{(Не)натуральные системы}
    
\end{document}
