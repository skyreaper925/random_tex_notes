%! Author = user
%! Date = 12.01.2024

\documentclass[a4paper, 14pt]{article}

%\hypersetup
%{   colorlinks,
%    pdftitle={analysis_themeslinalg summary},
%    pdfauthor={Володин Максим},
%    allcolors=[RGB]{010 090 200}
%}

\usepackage[T2A]{fontenc}
\usepackage[utf8]{inputenc}
\usepackage[english, russian]{babel}
\usepackage[top = 2cm, bottom = 2cm, left = 2cm, right = 2cm]{geometry}
\usepackage{indentfirst}
\usepackage{xcolor}
\usepackage{hyperref}
\usepackage{gensymb}
\usepackage{pgfplots}
\usepackage{amsmath, amsfonts, amsthm, mathtools}
\usepackage{amssymb}
\usepackage{physics, multirow, float}
\usepackage{wrapfig, tabularx}
\usepackage{icomma} % Clever comma: 0,2 - number while 0, 2 - two numbers
\usepackage{tikz, standalone}
\usepackage{fancyhdr,fancybox}
\usepackage{booktabs}
\usepackage{listings}
\usepackage{lstmisc}
\usepackage{stmaryrd}
\usepackage{graphicx}

%\полуторный интервал
\onehalfspacing

\graphicspath{{images/}}
\DeclareGraphicsExtensions{.pdf,.png,.jpg}

\restylefloat{table}
\usetikzlibrary{external}

\mathtoolsset{showonlyrefs = true} % Numbers will appear only where \eqref{} in the text LINKED
\pagestyle{fancy}

\fancyhf{}
\fancyhead[R]{Конспект билетов}
\fancyfoot[R]{\thepage /\pageref{LastPage}}
\fancyhead[L]{Аналитическая механика}

\pgfplotsset{compat=1.18}

\begin{document}

{\huge
    \begin{center}
    {\textbf{Конспект билетов}}
        \\
        Аналитическая механика
    \end{center}
}
    \tableofcontents \newpage
    
    \section{Кинематика точки. Траектория, скорость и ускорение точки.
    Скорость и ускорение точки в полярных координатах}
    
    \subsection{Кинематика точки}
    
    \textbf{Опр} \textit{Кинематика точки. Траектория, скорость и ускорение точки}
    
    \textcolor{blue}{Раздел механики, изучающий математическое описание (средствами геометрии, алгебры,
        математического анализа…) движения материальной точки без рассмотрения причин движения (массы, сил и т. д.)}
    
    \textbf{Опр} \textit{Траектория}
    
    \textbf{Опр} \textit{Скорость}
    
    \textbf{Опр} \textit{Ускорение}
    
    \subsection{Скорость и ускорение точки в полярных координатах}
    
    \textbf{Опр} \textit{Радиальная ось}
    
    \textbf{Опр} \textit{Трансверсальная ось}
    
    Для того чтобы получить скорость и ускорение в полярных координатах, достаточно выразить $x$ и $y$ в терминах $r,
    \varphi$, продифференцировать нужное число раз и вычленить базисные векторы
    
    \section{Кинематика точки. Естественный трёхгранник.
    Теорема Гюйгенса о разложении ускорения точки на тангенциальное и нормальное}
    
    \subsection{Кинематика точки. Естественный трёхгранник}
    
    \textbf{Опр} \textit{Естественный способ задания движения}
    
    \textbf{Опр} \textit{Естественный трёхгранник}
    
    \subsection{Теорема Гюйгенса о разложении ускорения точки на тангенциальное и нормальное}
    
    Запишем две формулы из дифференциальной геометрии и продифференцируем $r$ и $v$ с их учётом.
    Получим две компоненты ускорения: тангенциальное и нормальное
    
    \textbf{Theorem} \textit{Гюйгенса о разложении ускорения}
    
    \section{Криволинейные координаты точки.
    Коэффициенты Ламе.
    Скорость и ускорение точки в криволинейных координатах.
    Скорость точки в цилиндрических и сферических координатах}
    
    \subsection{Криволинейные координаты точки}
    
    \textbf{Опр} \textit{Криволинейные координаты}
    
    \textbf{Опр} \textit{Первая координатная линия}
    
    \textbf{Опр} \textit{Первая координатная ось}
    
    Аналогично определяются и последующие координатные линии и оси
    
    \subsection{Коэффициенты Ламе}
    
    \textbf{Опр} \textit{Единичный вектор координатной оси}
    
    \textbf{Опр} \textit{Коэффициент Ламе}
    
    \textbf{Опр} \textit{Ортогональные криволинейные координаты}
    
    \subsection{Скорость и ускорение точки в криволинейных координатах}
    
    Скорость находится по определению.
    Ускорение смотреть в конспекте Холостовой с 8 страницы
    
    \subsection{Скорость точки в цилиндрических и сферических координатах}
    
    \textbf{Опр} \textit{Цилиндрическая система координат}
    
    \textbf{Опр} \textit{Сферическая система координат}
    
    Скорость точки в этих координатах находится с помощью коэффициентов Ламе
    
    \section{Угловая скорость и угловое ускорение твёрдого тела.
    Скорости и ускорения точек твёрдого тела в общем случае его движения (формулы Эйлера и Ривальса)}
    
    \subsection{Угловая скорость и угловое ускорение твёрдого тела}
    
    \textbf{Опр} \textit{Поступательно движущаяся и связанная системы координат}
    
    \textbf{Опр} \textit{Углы Эйлера}
    
    \textcolor{blue}{Углы, описывающие поворот абсолютно твердого тела в трёхмерном евклидовом пространстве}
    
    \textbf{Опр} \textit{Линия узлов}
    
    \textcolor{blue}{Пересечение координатных плоскостей начальной и конечной СК}
    
    \textbf{Опр} \textit{Угол прецессии, нутации, собтвенного вращения}
    
    Переход от одной системы координат к другой посредством вращений на углы, можно задать с помощью матриц поворота
    
    \textcolor{blue}{Матрица поворота (или матрица направляющих косинусов)}
    
    \textbf{Опр} \textit{Ортогональная матрица, которая используется для выполнения собственного ортогонального
    преобразования в евклидовом пространстве.
    При умножении любого вектора на матрицу поворота длина вектора сохраняется.
    Определитель матрицы поворота равен единице}
    
    Матрицы поворота вокруг различных осей выглядят по-разному
    
    \textbf{Опр} \textit{Угловая скорость}
    
    \textcolor{blue}{Физическая величина, характеризующая быстроту и направление вращения материальной точки или
    абсолютно твёрдого тела относительно оси}
    
    \textbf{Опр} \textit{Угловое ускорение}
    
    \subsection{Скорости и ускорения точек твёрдого тела в общем случае его движения (формулы Эйлера и Ривальса)}
    
    \textbf{Theorem} \textit{Формула Эйлера}
    
    \textbf{Формула} \textit{Ривальса} \textcolor{gray}{Получается формальным дифференцированием формулы Эйлера}
    
    \addcontentsline{toc}{section}{Линейные отображения} \part*{Линейные отображения}
    
    \section{Линейные отображения и линейные преобразования векторных пространств (линейные операторы).
    Операции над линейными отображениями, линейное пространство линейных отображений.
    Алгебра линейных операторов.
    Изоморфизмы}
    
    \textbf{Опр} \textit{Линейное отображение} \textcolor{gray}{Отображение, удовлетворяющее двум аксиомам}

Отсюда следуют конечная линейность, отображение нулевого и противоположного вектора

Множество всех линейных отображений обозначается как $L(V, W)$.
В случае $W = V$ линейное отображение называют линейным преобразованием (оператором)

\textbf{Опр} \textit{Линейная функция (функционал)} \textcolor{gray}{Случай $\dim W = 1 (W = \mathbb{K})$}

\textbf{Утв} \textcolor{blue}{Под действием линейного отображения л.з. система остаётся л.з.}

Достаточно записать нетривиальную линейную комбинацию и взять её образ, используя уже известные аксиомы

\textbf{Утв} \textcolor{blue}{Ранг системы под действием линейного отображения не возрастает}

Это следует из определения ранга и противного к предыдущему утверждению.
В силу равенства ранга и размерности в конечномерном случае, получаем аналогичное неравенство для размерностей

\textbf{Утв} \textit{Образ подпространства}

\textcolor{blue}{Образ линейной оболочки есть линеная оболочка образов}

Действительно, если записать определение линейной оболочки (множество всех линейных комбинаций) и подействовать
отображением, то получится требуемое.
В частном случае, если взять базис (его линейная оболочка есть всё пространство), то образ пространства есть линейная
оболочка образов базисных векторов

\textbf{Опр} \textit{Линейное вложение} \textcolor{gray}{Инъективное линейное отображение}

\textbf{Утв} \textcolor{blue}{В случае линейного вложения л.н.з. система остаётся л.н.з.}

Действительно, если записать л.к. образов и \("\)вынести $\varphi$ за скобки\("\), то в силу инъективности
получим, л.к. исходных векторов.
В силу её линейной независимости, эта л.к. тривиальна, как и л.к. образов.
В частном случае, если взять базис, то получим равенства рангов $U$ и $\varphi (U)$, как и размерностей

\textbf{Th} \textcolor{blue}{Если взять базис $e_i$ в $V$ и произвольные векторы $c_i$ в $W$, то
    $\exists ! \varphi: \varphi(e_i) = c_i$. Дополнительно, $\varphi$ инъективно $\Leftrightarrow c_i$ л.н.з.}

\begin{enumerate}
    \item Для начала докажем единственность.
    Зафиксируем произвольный вектор $a$ пространства, разложим его по базису и рассмотрим $\varphi (a)$, имеющего
    единственные коэффициенты.
    В силу произвольности $a$ теорема справедлива
    \item Для доказательства существования, достаточно взять два произвольных вектора из пространства,
    подействовать на них отображением (с учётом $\varphi(e_i) = c_i$), затем проверить аксиомы линейного отображения
    \item $\Rightarrow$: следует из предыдущего утверждения
    \item $\Leftarrow$: от противного, с использованием определения инъективности, разложения $a - b$ по базису и
    $\varphi(e_i) = c_i$
\end{enumerate}

\subsection{Операции над линейными отображениями, линейное пространство линейных отображений}

\textbf{Опр} \textit{Сумма отображений} \textcolor{gray}{Такое отображение, что ...}

\textbf{Опр} \textit{Произведение отображения на скаляр} \textcolor{gray}{Такое отображение, что ...}

В комплексном случае скаляр заменяется на комплексно-сопряжённый.

Нетрудно проверить, что оба нововведённых отображения линейны.
Также проверкой доказывается ассоциативность, дистрибутивность и линейность в случае композиции отображений

\subsection{Алгебра линейных операторов}

На множестве $L(V, V)$ определены операции сложения, умножения на скаляр и умножения, поэтому $L(V, V)$ имеет
структуру ассоциативной алгебры (непустое множество (носитель) с заданным на нём набором операций и
отношений-сигнатурой).
Ассоциативная потому как заданы операции ассоциативного умножения, то есть $\forall k, l \in \mathbb{F}$ и
$\forall a, b, c \in A$ справедливо

\begin{enumerate}
    \item $a(b + c) = ab + ac$
    \item $(a + b)c = ac + bc$
    \item $(k+l)a = ka + la$
    \item $k(a + b) = ka + kb$
    \item $k(la) = (kl)a$
    \item $k(ab) = (ka)b = a(kb)$
    \item $1a = a$, где 1 -- единица кольца $\mathbb{K}$
\end{enumerate}

\textbf{Опр} \textit{Аннулирующий многочлен для оператора} \textcolor{gray}{$P(\varphi) = 0$}

\textbf{Опр} \textit{Минимальный многочлен} \textcolor{gray}{Аннулирующий многочлен с минимальной степенью}

\textbf{Утв} \textcolor{blue}{Пусть $\mu$ -- минимальный многочлен оператора $\varphi$, а $P \in \mathbb{F}$ --
произвольный.
Тогда $P$ аннулирует $\varphi \Leftrightarrow P \vdots \mu$ в кольце многочленов над $\mathbb{F}$}

\begin{enumerate}
    \item Разделим $P$ на $\mu$ с остатком и подставим в полученное равенство $\varphi$
    \item Воспользуемся условием и получим $P(\varphi) = 0 \Leftrightarrow r(\varphi) = 0$
    \item В таком случае остаток должен быть аннулирующим для $\varphi$, то есть его степень меньше степени
    минимального многочлена, поэтому $w$ не возникает только в случае $r \equiv 0$
    \item Таким образом, эквивалентность доказана
\end{enumerate}

Отсюда следует, что минимальный многочлен единственен с точностью до умножения на константу

\subsection{Изоморфизмы}

\textbf{Опр} \textit{Изоморфизм} \textcolor{gray}{Линейное биективное отображение}

\textbf{Опр} \textit{Изоморфные векторные пространства} \textcolor{gray}{Между ними существует изоморфизм}

\textbf{Утв} \textcolor{blue}{Обратный к изоморфизму изоморфизм}

\begin{enumerate}
    \item Биективность следует из тождеств для обратных функций
    \item Далее берутся векторы из образа и на них проверяются аксиомы линейного отображения
    \item Итого, обратный к изоморфизму изоморфизм по определению
\end{enumerate}

\textbf{Th} \textit{Классификация конечномерных векторных пространств}

\textcolor{blue}{Пространства изоморфны $\Leftrightarrow$ их размерности совпадают}

\begin{enumerate}
    \item $\Rightarrow:$ из изоморфности следует инъективность, а для инъективных отображений равенство доказано ранее
    \item $\Leftarrow:$ построим изоморфизм между элементами каждого пространствами и их координатными столбцами в
    них по фиксированному базису.
    Ранее было доказано, что такое разложение единственно.
    Достаточно обратить какое-то из отображений (по предыдущему утверждению оно тоже будет изоморфизмом).
    Итого, мы получили композицию изоморфизмов, то есть изоморфизм
\end{enumerate}

\textbf{Th} \textcolor{blue}{Если конечномерные пространства $U, V: \dim U = \dim V; e_i$ -- базис в $U$,
    $\varphi \in L(U,V)$. Тогда следующие условия эквивалентны:
    \begin{enumerate}
        \item $\varphi$ -- изоморфизм
        \item $\varphi$ инъективен
        \item $\varphi$ сюръективен
        \item $\varphi (e_i)$ есть базис в $V$
    \end{enumerate}         }

\begin{itemize}
    \item $1 \Rightarrow 2:$ по определению
    \item $2 \Rightarrow 3:$ из инъективности следует л.н.з. образа, поэтому $\dim (\varphi(U)) = \dim U \Rightarrow \varphi(U) \cong V
    \Rightarrow \varphi$ сюръективно
    \item $3 \Rightarrow 4:$ это следует из свойства линейной оболочки образов базисных векторов, связи
    размерности и ранга, определения ранга и базиса
    \item $4 \Rightarrow 1:$ по критерию инъективности, в силу л.н.з. $\varphi (e_i), \varphi$ будет инъективно,
    а из свойства линейной оболочки следует сюръективность $\varphi$
\end{itemize}
    
    \section{Матрица линейного отображения.
    Координатная запись линейного отображения.
    Связь операций над матрицами и над линейными отображениями.
    Изменение матрицы линейного отображения (преобразования) при замене базиса}
    
    \subsection{Матрица линейного отображения}
    
    \textbf{Опр} \textit{Матрица линейного отображения} \textcolor{gray}{Как и у матрицы перехода: склейка столбцов
    векторов $\varphi(e_i)$}
    
    Таким образом, существует биекция между $L(V, W)$ и $Mat(m, n)$, как и изоморфизм (проверяется).
    Отсюда следует, что размерность линейных операторов есть $mn$
    
    \subsection{Координатная запись линейного отображения}
    
    \textbf{Th} \textcolor{blue}{Если $\varphi \leftrightarrow A; a = eX, \varphi(a) = fY$, то $y = Ax$}
    
    Для доказательства достаточно записать определение координатного столбца, применить к ней $\varphi$ и в силу
    коммутируемости, поменять местами строки и столбцы, чтобы увидеть запись матричного умножения, что доказывает
    равенство
    
    \textbf{Следствие} \textcolor{blue}{Если дано неизвестное в плане линейности отображение $\varphi$, такое, что под
    его действием $y = Ax$, то оно линейное}
    
    В силу наличия биекции между матрицами и линейными отображениями, найдём такое $\phi$.
    В таком случае по Th., они будут иметь одинаковую координатную запись $y = Ax$, то есть равны и $\varphi$ линейно
    
    \textbf{Th} \textcolor{blue}{Если линейное преобразование $\varphi$ таково, что $\varphi(e_i) = e_i^{'}$, то
        $A: A \leftrightarrow \varphi$ есть матрица перехода между базисами}
    
    Следует из определений (матрица линейного преобразования будет удовлетворять определению матрицы перехода)
    
    \subsection{Связь операций над матрицами и над линейными отображениями}
    
    \textbf{Утв} \textcolor{blue}{Композиции линейных отображений соответствует произведение соответствующих матриц}
    
    Доказывается по определению (подстановкой)
    
    \textbf{Следствие} \textcolor{blue}{Обратному отображению соотвествует обратная матрица}
    
    Следует из предыдущего утверждения и того, что тождественному отображению соответствует единичная матрица
    
    \textbf{Следствие} \textcolor{blue}{$P(\varphi) \leftrightarrow P(A)$}
    
    \textbf{Следствие} \textcolor{blue}{$\varphi \leftrightarrow A$ задаёт изоморфизм алгебр линейных преобразований
    и квадратных матриц}
    
    То есть изоморфизм группы биективных линейных преобразований и группы невырожденных матриц
    
    \subsection{Изменение матрицы линейного отображения (преобразования) при замене базиса}
    
    \textbf{Th} \textcolor{blue}{Если $L(V, W); S: e^{'} = eS; y = Ax, y^{'} = Ax^{'}; f^{'} = fR$, то
        $A^{'} = R^{-1}AS$ (матрица линейного отображения в другом базисе)}
    
    Доказывается путём подстановок и комбинаций равенств
    
    В частном случае $L(V, V) A^{'} = S^{-1}AS$ \\
    
    \textbf{Следствие} \textcolor{blue}{Ранг матрицы линейного отображения не зависит от выбора базисов в
    пространствах}
    
    Потому что мы домножаем слева и справа на невырожденные матрицы
    
    \section{Ядро и образ, их описание в терминах матрицы линейного отображения.
    Критерий инъективности.
    Связь между размерностями ядра и образа}
    
    \subsection{Ядро и образ, их описание в терминах матрицы линейного отображения}
    
    \textbf{Опр} \textit{Образ линейного отображения} \textcolor{gray}{Множество всех векторов $V$ под
    действием $\varphi \in L(V, W)$}
    
    \textbf{Th} \textit{Координатное описание образа}
    
    \textcolor{blue}{Если $\varphi \in L(V, W)$, а $b \in W: b =
    fY$, то $b \in \Im \varphi \Leftrightarrow Y \in <a_{.1}, \dots, a_{.n}>$}
    
    Это следует из записи образа через линейную оболочку действия $\varphi$ на базисные векторы, определения матрицы
    линейного отображения и того факта, что $b = fY$ есть л.к. столбцов $Y$
    
    Отсюда также следует, что размерность образа равна рангу матрицы линейного отображения
    
    \textbf{Утв} \textcolor{blue}{В случае $\varphi \in L(V, W)$ прообразы образов в подмножестве $W$ являются
    подмножеством $V$}
    
    \textbf{Опр} \textit{Ядро линейного отображения} \textcolor{gray}{Множество всех векторов $V$, которые зануляются
    под действием $\varphi \in L(V, W)$}
    
    То есть ядро есть подмножество $V$.
    Также ядро можно охарактеризовать как полный прообраз нулевого пространства
    
    Если ядро пусто, то оператор невырожден
    
    \textbf{Th} \textit{Координатное описание ядра}
    
    \textcolor{blue}{Если $\varphi \in L(V, W)$, а $a \in V: a = eX$, то $a \in Ker \varphi \Leftrightarrow AX = 0$}
    
    В обе стороны по определению ядра
    
    Другими словами, в терминах координатных столбцов ядро задается как общее решение однородной СЛУ $Ax = 0$
    
    \textbf{Следствие} \textcolor{blue}{$\dim Ker \varphi = n - rg A$}
    
    \subsection{Критерий инъективности}
    
    \textbf{Th} \textit{Критерий инъективности}
    
    \textcolor{blue}{Если $\varphi \in L(V, W)$, инъективно
        $\Leftrightarrow Ker \varphi = 0$}
    
    $\Rightarrow:$ пользуемся $\varphi(0) = 0$
    
    $\Leftarrow:$ от противного с использованием определения инъективности
    
    \subsection{Связь между размерностями ядра и образа}
    
    \textbf{Th} \textcolor{blue}{В конечномерных пространствах $\dim Ker \varphi + \dim \Im \varphi = n$}
    
    Следует из $\dim \Im \varphi = rg A$ и $\dim Ker \varphi = n - rg A$
    
    \section{Аффинные преобразования, их свойства.
    Аффинная группа}
    
    \subsection{Аффинные преобразования, их свойства}
    
    \textbf{Опр} \textit{Аффинно-линейное преобразование}
    
    \textbf{Опр} \textit{Дифференциал отображения} \textcolor{gray}{Обозначение элемента $\varphi \in L(V, V)$}
    
    \textbf{Опр} \textit{Аффинное преобразование} \textcolor{gray}{Биективное преобразование}
    
    \textbf{Утв} \textcolor{blue}{Преобразование аффинно $\Leftrightarrow$ его дифференциал биективен}
    
    Для доказательства достаточно воспользоваться определением при одной фиксированной точке в нём
    
    \textbf{Утв} \textcolor{blue}{Композиция линейных и аффинных преобразований линейна и аффинна соответственно, а
    их дифференциал есть произведение дифференциалов}
    
    Следует из определения и того, что композиция биективных отображений биективна
    
    \textbf{Утв} \textcolor{blue}{Обратное к аффинному отображению отображение аффинно}
    
    Следует из определения
    
    \subsection{Аффинная группа}
    
    Аффинные преобразования образуют группу относительно композиции
    
    \textbf{Утв} \textcolor{blue}{$Y = AX + C$}
    
    Достаточно взять в определении аффинно-линейного преобразования точку $M = 0$
    
    Отсюда следует, что любое аффинное преобразование задаётся параллельным переносом и поворотом вокруг неподвижной
    точки, то есть линейное преобразование однозначно задаётся нулевой точкой и базисом
    
    \textbf{Th} \textcolor{blue}{Линейное преобразование $f$ аффинно $\Leftrightarrow$ переводит неколлинеарные
    точки в неколлинеарные}
    
    Построим ДСК на наших трёх точках, подействуем на них преобразованием и получим новую ДСК. $f$ однозначно задано
    этой ДСК.
    Поэтому $f$ аффинно $\Leftrightarrow$ $f$ биективно $\Leftrightarrow$ неколлинеарная система (л.н.з)
    переходит в неколлинеарную (в частности, система три точки)
    
    \textbf{Th} \textit{Связь аффинного преобразования с заменой координат}
    
    \textcolor{blue}{При аффинном преобразовании координатный столбец вектора не меняется}
    
    Достаточно воспользоваться координатной запись вектора, а потом к концевым точкам применить аффинное преобразование
    
    \textbf{Th} \textcolor{blue}{При аффинном преобразовании
        \begin{enumerate}
            \item прямая переходит в прямую
            \item параллельные прямые переходят в параллельные
            \item отношения длин отрезков сохраняются
            \item центральная симметрия сохраняется
        \end{enumerate}           }
    
    \begin{enumerate}
        \item достаточно параметризовать прямую и применить определение к концевым точкам
        \item аналогичным образом в силу линейности (из определения) сохраняются отношения длин отрезков (в случае с
        отрезками между прямыми они не схлопываются в точку)
        \item аналогично
        \item при центральной симметрии для любых двух симметричных точек центр есть середина соответствующего
        отрезка, а так как отношения сохраняются, то получаем сохранение определения
    \end{enumerate}
    
    \textbf{Th} \textit{Изменение площадей}
    
    \textcolor{blue}{При аффинном преобразовании, чей дифференциал имеет матрицу $A$, площадь фигуры умножается
    на $\abs{\det A}$}
    
    Покажем на примере параллелограмма.
    Достаточно расписать определение ориентированной площади, применить преобразование и взять модуль (настоящая
    площадь неотрицательна)
    
    \textbf{Th} \textcolor{blue}{При аффинном преобразовании порядок алгебраической кривой не меняется}
    
    При аффинном преобразовании координаты не меняются, то не поменяется и многочлен, задающий кривую (скалярное
    произведение коэффициентов на переменные), как и его порядок
    
    \addcontentsline{toc}{section}{Структура линейного преобразования} \part*{Структура линейного преобразования}
    
    \section{Инвариантные подпространства.
    Ограничение оператора на инвариантное подпространство.
    Фактороператор}
    
    \subsection{Инвариантные подпространства}
    
    \textbf{Опр} \textit{Инвариантное подпространство} \textcolor{gray}{Образ лежит в нём же}
    
    \textbf{Утв} \textcolor{blue}{Сумма и пересечение инвариантных подпространств инвариантно}
    
    Доказывается поэлементной проверкой определения
    
    \textbf{Утв} \textcolor{blue}{В случае коммутирующих преобразований ядро и образ одного инвариантно относительно
    другого}
    
    Доказывается по определению
    
    \textbf{Следствие} \textcolor{blue}{Ядро и образ многочлена $f(\varphi)$ инвариантны относительно $\varphi \in L(
    V, V)$}
    
    \textbf{Утв} \textcolor{blue}{$U$ инвариантнно относительно $\varphi \Leftrightarrow U$ инвариантнно
    относительно $\varphi - \lambda, \lambda \in \mathbb{F}$}
    
    Доказывается проверкой в одну сторону и путём взятия другой $\lambda$ в обратную
    
    Таким образом, в случае $\Im (\varphi - \lambda) \subset U \Rightarrow U$ инвариантно $\varphi$
    
    \subsection{Ограничение оператора на инвариантное подпространство}
    
    \textbf{Утв} \textcolor{blue}{Если $U$ инвариантно относительно $\varphi$ -- изоморфизма, то $U$ инвариантно
    относительно $\varphi^{-1}$}
    
    Достаточно рассмотреть сужение $\varphi$ на $U$.
    В силу инъективности это будет изоморфизм.
    Тогда обращаем его и получаем требуемое
    
    \textbf{Утв} \textcolor{blue}{Если $U_k$ инвариантно относительно линейное оболочки первых $k$ векторов
        $\Leftrightarrow a_{ij} = 0, i \in \overline{k+1, n}, j \in \overline{1, k}$, то есть матрица имеет
        блочно-диагональный вид, где второй квадрант есть сужение $\varphi$ на $U_k$}
    
    Достаточно воспользоваться определением матрицы линейного преобразования и вспомнить, что у нас базис не меняется
    
    \textbf{Утв} \textcolor{blue}{Если $\varphi \in L(V, V)$, а $P(\varphi), \deg P = k$ вырожден, то существует не
    более чем $k$-мерное инвариантное подпространство $V$ относительно $\varphi$}
    
    \begin{enumerate}
        \item Возьмём произвольный элемент ядра $a$ и покажем, что $U = <a, \varphi(a), \dots, \varphi^{k-1}(a)>$
        инвариантно относительно $\varphi$
        \item В силу индуктивности $\varphi^j$, достаточно доказать лишь что $\varphi^k (a) \in U$
        \item Подставляем $\varphi(a)$ в многочлен и получаем, что $\varphi^k (a)$ линейно выражается через остальные
        члены, что доказывает инвариантность и утверждение
    \end{enumerate}
    
    \subsection{Фактороператор}
    
    \textbf{Опр} \textit{Фактороператор} \textcolor{gray}{Линейный оператор, определённый формулой $\overline{\uppsi
    } (v + U) = \varphi (v) + U, \forall v \in V$}
    
    \section{Собственные векторы и собственные значения.
    Собственные подпространства.
    Характеристический многочлен и его инвариантность.
    Определитель и след преобразования}
    
    \subsection{Собственные векторы и собственные значения}

\textbf{Опр} \textit{Собственное значение} \textcolor{gray}{Существует $a \in V:$}

\textbf{Опр} \textit{Собственный значение} \textcolor{gray}{Ненулевой вектор $a$ преобразования ...}

\textbf{Утв} \textcolor{blue}{Ненулевой вектор $a$ собственный для $\varphi \Leftrightarrow <a>$ инвариантна
относительно $\varphi$}

В силу эквивалентности инвариантности наличию собственного значения

\textbf{Утв} \textcolor{blue}{Ненулевой вектор $a$ собственный для $\varphi$ с собственным
значением $\lambda \Leftrightarrow a \in \ker (\varphi - \lambda)$}

Достаточно вспомнить определение ядра

\subsection{Собственные подпространства}

\textbf{Опр} \textit{Собственное подпространство} \textcolor{gray}{Ядро $\ker (\varphi - \lambda)$, содержащее ...}

\textbf{Утв} \textcolor{blue}{Сумма подпространств $V_{\lambda_i}$ прямая}

\begin{enumerate}
    \item От противного: возьмём $a_1 \in V_{\lambda_1} \cap \sum_i V_{\lambda_i}$, то есть $a_1 = \sum_i a_i$
    \item Применим к этому равенству преобразование $\sqcap_2^k (\varphi - \lambda_k)$
    \item Справа у нас получится ноль, а слева -- нет, $w$
\end{enumerate}

\subsection{Характеристический многочлен и его инвариантность}

\textbf{Опр} \textit{Характеристический многочлен} \textcolor{gray}{Функция от константы. Не забыть про обозначение}

\textbf{Опр} \textit{Характеристическое уравнение} \textcolor{gray}{Равенства многочлена нулю}

\textbf{Опр} \textit{Характеристические числа} \textcolor{gray}{Корни арактерестического многочлена}

Характеристический многочлен можно записать и с учётом алгебраической кратности его корней

\textbf{Утв} \textcolor{blue}{Характерестический многочлен имеет
вид $(-1)^n \lambda^n + (-1)^{n-1} tr A + \dots + \abs{A}$}

Достаточно знать, что определитель есть функция от всех элементов матрицы, затем просто расписать коэффициенты
перед требуемыми степенями

Отсюда, в соответствии с теоремой Виета, сумма всех характеристических чисел равна следу, а произведение есть $\det A$

\textbf{Th} \textcolor{blue}{$\lambda$ -- собственное значение $\Leftrightarrow \lambda$ характеристическое число}

\begin{enumerate}
    \item $\lambda$ -- собственное значение $\Leftrightarrow \ker (\varphi - \lambda) \neq O$
    \item $\Leftrightarrow$ соответствующая СЛУ имеет нетривиальное решение
    \item $\Leftrightarrow$ соответствующая квадратная матрица вырождена
    \item $\Leftrightarrow$ соответствующий определитель равен нулю
    \item $\Leftrightarrow \lambda$ характеристическое число
\end{enumerate}

Стоит учесть, что данное утверждения верно лишь в $\mathbb{C}$.
В $\mathbb{R}$ собственные значения есть только вещественные характеристические числа

\textbf{Th} \textcolor{blue}{
    \begin{enumerate}
        \item В $\mathbb{C}$ у $\varphi~\exists$ одномерное инвариантное подпространство
        \item В $\mathbb{R}$ у $\varphi~\exists$ одномерное инвариантное подпространство в случае нечётного $n$
        \item У $\varphi~\exists$ ненулевое инвариантное подпространство размерности не выше 2
    \end{enumerate}         }

\begin{enumerate}
    \item По основной теореме алгебры у любого многочлена есть по крайней мере один комплексный корень
    \item Из анализа известно, что в таком случае у многочлена есть по крайней мере один вещественный корень
    \item Если у многочлена есть вещественный корень, то у него есть и одномерное инвариантное подпространство.
    Иначе рассмотрим комплексный корень.
    Из анализа известно, что его сопряжённый тоже будет корнем характеристического многочлена.
    Тогда многочлен $P$ с этими коэффициентами будет вещественен, а $\det P(A) = 0$ в силу наличия соответствущих
    собственных значений.
    В таком случае ранее было доказано, что у $\varphi~\exists$ двумерное инвариантное подпространство
\end{enumerate}

\subsection{Определитель и след преобразования}

\textbf{Утв} \textcolor{blue}{Если матрица оператора верхнетреугольна, то характеристические числа
характеристического многочлена совпадают с диагональными элементами}

Верно в силу того, что определитель верхнетреугольной матрицы равен произведению диагональных элементов

\textbf{Th} \textit{Инвариантность характеристического многочлена}

\textcolor{blue}{Характеристический многочлен не зависит от выбора базиса}

Достаточно записать характеристическое уравнение в двух базисах, перейти от одного к другому с помощью матрицы
перехода и преобразовать выражение

\textbf{Следствие} \textcolor{blue}{Определитель, след, набор характеристических чисел матрицы оператора не
зависят от выбора базиса}

Все вышеперечисленные термины выражаются через коэффициенты характеристического многочлена
    
    \section{Линейная независимость собственных подпространств, отвечающих различным собственным значениям.
    Алгебраическая и геометрическая кратность собственного значения.
    Критерий диагонализируемости преобразования}
    
        \subsection{Линейная независимость собственных подпространств, отвечающих различным собственным значениям}

    \textbf{Th} \textcolor{blue}{$\lambda$ -- собственное значение $\Leftrightarrow \lambda$ характеристическое число}

    \begin{enumerate}
        \item $\lambda$ -- собственное значение $\Leftrightarrow \ker (\varphi - \lambda) \neq O$
        \item $\Leftrightarrow$ соответствующая СЛУ имеет нетривиальное решение
        \item $\Leftrightarrow$ соответствующая квадратная матрица вырождена
        \item $\Leftrightarrow$ соответствующий определитель равен нулю
        \item $\Leftrightarrow \lambda$ характеристическое число
    \end{enumerate}

    \textbf{Th} \textcolor{blue}{Собственные векторы различных собственных значений л.н.з.}

    \begin{enumerate}
        \item Доказывается по индукции.
        База очевидна
        \item Докажем переход.
        Для этого рассмотрим $k+1$ собственный вектор, из которых $k$ заведомо л.н.з
        \item Применим к их л.к. $\varphi$.
        Из неё вычтем правильную л.к. первых $k$ векторов (чтобы обнулить $\alpha_{k+1}$)
        \item Итого, $k$ коэффициентов нули, а значит, и $k+1$-й тоже, то есть система осталась л.н.з.
    \end{enumerate}

    \subsection{Алгебраическая и геометрическая кратность собственного значения}

    \textbf{Опр} \textit{Геометрическая кратность} \textcolor{gray}{Размерность собственного подпространства}

    \textbf{Th} \textcolor{blue}{Геометрическая кратность не превосходит алгебраическую}

    \begin{enumerate}
        \item Рассмотрим собственное пространство размерности $s$ и произвольный базис в нём.
        Дополним его до базиса во всём пространстве
        \item Запишем матрицу линейного оператора.
        Она будет иметь блочно-диагональный вид
        \item Вычислим характеристический многочлен этой матрицы и непосредственно убедимся в доказанном утверждении (
        потому как в оставшемся многочлене собственное значение может быть корнем; в противном случае достигается
        равенство)
    \end{enumerate}

     \subsection{Критерий диагонализируемости преобразования}

    \textbf{Опр} \textit{Диагонализируемое преобразование} \textcolor{gray}{Существует базис, в котором матрица имеет
    диагональный вид}

    \textbf{Th} \textit{Первый критерий диагонализируемости}

    \textcolor{blue}{Если $\varphi \in L(V, V)$ имеет попарно различные собтсвенные значения $\lambda_i$
        кратнойстей $s_i$, то следующие условия эквивалентны:
    \begin{enumerate}
        \item $\varphi$ диагонализируем
        \item В пространстве существует базис из собственных векторов
        \item $\dim V_{\lambda_i} = s_i$
        \item $V = \oplus_i V_{\lambda_i}$
    \end{enumerate}             }

\begin{itemize}
    \item $1 \Leftrightarrow 2:$ в силу того, что матрица $A$ есть склейка применения $\varphi$ на базисные векторы
    \item $2 \Rightarrow 3:$ пользуемся необходимыми условиями и суммируем $t_i \leq \dim V_{\lambda_i} \leq s_i$ по $i$
    \item $3 \Rightarrow 4:$ так как собственные пространства разных собственных значений не пересекаются, то они
    разлагаются в прямую сумму.
    Сумма их размерностей будет $\sum_i s_i = n$, то есть всего пространства
    \item $4 \Rightarrow 2:$ достаточно выбрать базис в каждом подпространстве и объединить.
    Ранее доказывалось, что он будет базисом во всём пространстве (второй критерий прямой суммы)
\end{itemize}

     \textbf{Следствие} \textit{Достаточно условие диагонализируемости}

    \textcolor{blue}{Если характеристический многочлен имеет $n$ различных корней из поля, то $\varphi$
        диагонализируем}

    Действительно, в таком случае у каждого собственного подпространство размерность единица и они располагаются на
    главной диагонали

%    \textbf{Th} \textit{Второй критерий диагонализируемости}
%
%    \textcolor{blue}{Если $\varphi \in L(V, V)$ имеет попарно различные собтсвенные значения $\lambda_i$
%        кратнойстей $s_i$, то следующие условия эквивалентны:
%    \begin{enumerate}
%        \item $\varphi$ диагонализируем
%        \item $V_{\lambda_i} = V^{\lambda_i}$
%        \item $V = \oplus_i V_{\lambda_i}$
%    \end{enumerate}             }
%
%\begin{itemize}
%    \item $1 \Leftrightarrow 2:$ в силу того, что матрица $A$ есть склейка применения $\varphi$ на базисные векторы
%    \item $2 \Rightarrow 3:$ пользуемся необходимыми условиями и суммируем $t_i \leq \dim V_{\lambda_i} \leq s_i$ по $i$
%    \item $3 \Rightarrow 4:$ так как сосбтвенные пространства разных собственных значений не пересекаются, то они
%    разлагаются в прямую сумму. Сумма их размерностей будет $\sum_i s_i = n$, то есть всего прсотранства
%    \item $4 \Rightarrow 2:$ достаточно выбрать базис в каждом подпростарнстве и объединить. Ранее доказывалось, что
%    он будет базисом во всём пространстве (второй критерий прямой суммы)
%\end{itemize}
    
    \section{Инвариантные подпространства малой размерности в вещественном случае}
    
    \textbf{Th} \textcolor{blue}{
        \begin{enumerate}
            \item В $\mathbb{C}$ у $\varphi~\exists$ одномерное инвариантное подпространство
            \item В $\mathbb{R}$ у $\varphi~\exists$ одномерное инвариантное подпространство в случае нечётного $n$
            \item У $\varphi~\exists$ ненулевое инвариантное подпространство размерности не выше 2
        \end{enumerate}             }
    
    \begin{enumerate}
        \item По основной теореме алгебры у любого многочлена есть по крайней мере один комплексный корень
        \item Из анализа известно, что в таком случае у многочлена есть по крайней мере один вещественный корень
        \item Если у многочлена есть вещественный корень, то у него есть и одномерное инвариантное подпространство.
        Иначе рассмотрим комплексный корень.
        Из анализа известно, что его сопряжённый тоже будет корнем характеристического многочлена.
        Тогда многочлен $P$ с этими коэффициентами будет вещественен, а $\det P(A) = 0$ в силу наличия соответствущих
        собственных значений, то есть наш многочлен вырожден.
        В таком случае ранее было доказано, что у $\varphi~\exists$ двумерное (на самом деле, $\leq 2$) инвариантное
        подпространство
    \end{enumerate}
    
    \section{Треугольный вид матрицы преобразования.
    Теорема Гамильтона-Кэли}
    
    \subsection{Треугольный вид матрицы преобразования}
    
    \textbf{Лемма} \textcolor{blue}{$\exists~(n-1)$-мерное инвариантное подпространтво}
    
    \begin{enumerate}
        \item Возьмём произвольное $\lambda_0$ и сделаем выводы по размерности ядра и образа для $\varphi - \lambda_0$
        \item Выясним существования $U: \dim U = n - 1$ -- не более чем надмножество $\Im \varphi - \lambda_0$.
        Для его построения возьмём базис в образе и дополним его до базиса во всём $V$
        \item В конце возьмём базис для $U$ (первые $n-1$ вектор), задаваемый требуемое подпространство
    \end{enumerate}
    
    \textbf{Лемма} \textit{О треугольном виде}
    
    \textcolor{blue}{$\exists$ базис, в котором матрица $\varphi \in L(V, V)$ верхнетреугольна с
    заданным порядком расстановки характеристических числе по диагонали}
    
    \begin{enumerate}
        \item Возьмём произвольные $\lambda_0$ и инвариантное $n - 1$ подпространство
        \item Далее получим вид для $\varphi: \varphi = \lambda_0 e_n + \sum_1^{n-1} \mu_i f_i$
        \item Сделаем вывод о матрице оператора, характеристическом многочлене сужения
        \item Затем применяем спуск индукции, чтобы поместить на главную диагональ нужные базисные векторы
    \end{enumerate}
    
    \subsection{Теорема Гамильтона-Кэли}
    
    \textbf{Th} \textit{Гамильтона-Кэли}
    
    \textcolor{blue}{Характеристический многочлен является аннулирующим для матрицы оператора}
    
    Это следует из того, что $s_i \geq m_i$, то есть характеристический многочлен содержит в себе минимальный (то
    есть аннулирующий)
    
    \section{Корневые подпространства, их размерность.
    Разложение пространства в прямую сумму корневых.
    Жорданова нормальная форма, её существование и единственность.
    Минимальный многочлен, критерий диагонализируемости оператора в терминах минимального многочлена}
    
    \subsection{Корневые подпространства, их размерность}

\textbf{Опр} \textit{Корневое подпространство} \textcolor{gray}{Первое стабильное ядро}

Притом корневое пространство равно и все стабильным ядрам большей размерности

\textbf{Лемма} \textcolor{blue}{$\dim \ker (\varphi - \lambda_i)^{s_i} \geq s_i$}

\begin{enumerate}
    \item Запишем матрицу в верхнетреугольном виде, притом расположим $\lambda_i$ в первых $s_i$ диагональных
    клетках
    \item Применим к матрице преобразование $\varphi - \lambda_i$.
    Получим нильпотентный левый верхний блок
    \item Возведём матрицу в нужную степень с учётом перемножения блочных матриц и получим левый верхний блок нулей
    \item Тогда получим, что первые $s_i$ векторов принадлежат соответствующему ядру.
    А так как такими же могут быть и последующие векторы, то возможно строгое неравенство
\end{enumerate}

\textbf{Следствие} \textcolor{blue}{$\dim V^{\lambda_i} \geq s_i$}

\subsection{Разложение пространства в прямую сумму корневых}

\textbf{Лемма} \textcolor{blue}{Сумма любых степеней ядер $\varphi - \lambda_i$ прямая}

\begin{enumerate}
    \item От противного: пусть $\exists a_1 = V^{\lambda_1} \cap \sum_2^k V^{\lambda_i}$
    \item Тогда этот вектор можно разложить по этим подпространствам
    \item Применим к обеим частям равенства $\psi = \prod_2^k (\varphi - \lambda_i)$
    \item Тогда справа получим ноль, а слева -- нет, $w$
\end{enumerate}

В частности, сумма корневых пространств прямая

\textbf{Th} \textcolor{blue}{
    \begin{enumerate}
        \item $V = \oplus_i V^{\lambda_i}$
        \item $\dim V^\lambda_i = s_i$
        \item $m_i \leq s_i$, то есть стабилизация ядер наступает не позже $s_i$ шага
    \end{enumerate}             }

\begin{enumerate}
    \item В силу $\dim V^\lambda_i \geq s_i$ сложим неравенства по всем $i$.
    Тогда $\sum_i s_i \geq n$, однако у нас $\dim V = n$, поэтому $V = \oplus_i V^{\lambda_1}$
    \item Предыдущий пункт возможен лишь когда во всех неравенствах выполнено равенство
    \item $\dim \ker (\varphi - \lambda_i)^{s_i} \geq s_i = \dim V^\lambda_i$, однако $\ker (\varphi -
    \lambda_i) \subset V^\lambda_i$.
    В противном случае не выполнена формула суммы размерностей ядра и образа
\end{enumerate}

\subsection{Жорданова нормальная форма, её существование и единственность}

\textbf{Опр} \textit{Жорданова клетка} \textcolor{gray}{Верхнетреугольная матрица, в которой на главной диагонали ...}

\textbf{Опр} \textit{Жорданова матрица, ЖНФ} \textcolor{gray}{Блочно-диагональная матрица, каждый блок которой ...}

\textbf{Опр} \textit{Жорданов базис} \textcolor{gray}{Базис, в котором оператор имеет ЖНФ}

\textbf{Опр} \textit{Жорданова цепочка, присоединённый вектор}

\begin{enumerate}
    \item Рассмотрим жорданову клетку запишем её действие в строчном виде
    \item Рассмотрим новый оператор $\psi = \varphi - \lambda_0$.
    Под его действием векторы сваливаются в ядра меньшего по степени оператора
    \item Полученная последовательность называется жордановой цепочкой
    \item Вектор над данным называется присоединённым.
    Ясно, что он может быть и не единственен
\end{enumerate}

\textbf{Th} \textit{Существование ЖНФ}

\textcolor{blue}{Существует базис, в котором матрица оператора жорданова}

\begin{enumerate}
    \item Требуется доказать, что существует базис, являющийся объединением жордановых цепочек, то есть так надо
    сделать в каждом корневом подпространстве
    \item Рассмотрим нильпотентный оператор $\psi: V^\lambda_i \rightarrow V^\lambda_i$, являющийся ограничением
    оператора $\varphi - \lambda_i$ на $V^\lambda_i$
    \item Заметим, что если $a \in \ker \psi^t$, то $a \in \ker \psi^{t-1}$ (непосредственно проверяется)
    \item Теперь рассмотрим вложенную цепочку ядер и определим к последнему, $V^\lambda_i$ прямое (нулевое дополнение
    ) $W_{m_i + 1}$ до следующего ядра (которое будет являться самим $V^\lambda_i$)
    \item Рассмотрим ядро $\psi (W_{t+1})$, для которого выполнено три условия
    \item Из них мы можем определить $W_t$ как прямое дополнение $\ker \psi^{t-1}$ до $\ker \psi^t$ (дополним до базиса)
    \item Таким образом, применяя оператор $\psi (W_{t+1})$ мы спускаемся вниз по цепочке и, дополнив до базиса,
    продолжаем спуск
    \item Жорданов базис есть объединение базисов $W_i$ каждое из которых лежит в $\ker \psi^i$, то есть базисных
    разных $W_i$ пересекаются тривиально (к том же мы пользовались л.н.з. дополнением)
\end{enumerate}

\textbf{Th} \textit{Единственность ЖНФ}

\textcolor{blue}{Для данного базиса ЖНФ единственна с точностью до перестановки жордановых клеток}

\begin{enumerate}
    \item Требуется доказать, что $\forall \lambda_i$ количество жордановых цепочек данной длины в жордановом базисе
    определено однозначно
    \item Рассмотрим все цепочки, соответствующие фиксированному $\lambda_i$.
    Их линейная оболочка находится $\in V^\lambda_i$, $\dim <B_\lambda_i> \leq s_i$
    \item Так как всего в (жордановом) базисе у нас $n$ векторов, а  $<B_\lambda_i>$ и составляют этот базис, то в
    неравенствах выше достигается равенство
    \item Рассмотрим нильпотентный оператор $\psi: V^\lambda_i \rightarrow V^\lambda_i$, являющийся ограничением
    оператора $\varphi - \lambda_i$ на $V^\lambda_i$
    \item Его образ состоит из всех не верхних векторов в цепочках, образ его образа состоит из всех векторов, кроме \ldots
    \item Введём обозначения для количества жордановых цепочек длины $d$, отвечающих $\lambda_i$ за $c_i^d$
    \item Составим уравнения на их суммы, чья система легко решается и однозначно выражается через характеристики
    оператора $\varphi$, то есть инвариантно
\end{enumerate}

Из указанного рассмотрения нетрудно заметить, что степень $\lambda - \lambda_i$ характеристического многочлена $s_i$
равна длине всех цепочек, отвечающих $\forall \lambda_i$, а степень $m_i$ минимального многочлена равна длине
максимальной (иначе оператор обнулится не полностью, не по всем цепочкам-базисным векторам)

\subsection{Минимальный многочлен, критерий диагонализируемости оператора в терминах минимального многочлена}

\textbf{Опр} \textit{Минимальный многочлен} \textcolor{gray}{Многочлен, обгуляющий оператор, минимальной степени}

\textbf{Утв} \textcolor{blue}{Минимальный многочлен является аннулирующим}

\begin{enumerate}
    \item Хотя бы один аннулирующий многочлен существует, ведь если взять степени оператора, которые будет больше
    размерности пространства, то эта система будет л.з., а значит, у соответствующего многочлена будут ненулевые
    коэффициенты
    \item Теперь возьмём произвольный вектор $a \in V = \oplus_i V^{\lambda_i}$
    \item Так как ядра каждого одночлена содержатся в ядре минимального, то каждый член разложения из ядра
    минимального многочлена.
    Тогда и $a$ тоже
    \item Итого, в силу произвольности $a$, минимальный многочлен аннулирующий
\end{enumerate}

\textbf{Утв} \textcolor{blue}{Минимальный многочлен есть $\prod_i (\lambda - \lambda_i)^{m_i}$}

\begin{enumerate}
    \item От противного: пусть хотя бы одна степень тут меньше, то есть БОО $m_1^{'} < m_1$
    \item Тогда $\dim \ker (\varphi - \lambda_1)^{m_1^{'}} < s_1$ по лемме
    \item Если мы сложим все ядра такого вида то получи строгое неравенство.
    То есть существуют вектор пространства $a: \mu_\varphi (\varphi(a)) \neq 0$, то есть новый многочлен не минимальный
\end{enumerate}

\textbf{Th} \textit{Второй критерий диагонализируемости}

\textcolor{blue}{Если $\varphi \in L(V, V)$ имеет попарно различные собтсвенные значения $\lambda_i$
    кратнойстей $s_i$, то следующие условия эквивалентны:
    \begin{enumerate}
        \item $\varphi$ диагонализируем
        \item $V_{\lambda_i} = V^{\lambda_i}$
        \item $\mu_\varphi$ раскладывается на различные линейные множители
    \end{enumerate}
                }

\begin{itemize}
    \item $1 \Leftrightarrow 2:$ в силу того, что $V_\lambda_i \subseteq V^\lambda_i$ и $V = \oplus_i V_{\lambda
    _i}$, достигается равенство множеств
    \item $2 \Rightarrow 3:$ из 2 следует, что $m_i = 1 \forall i$, поэтому все $n$ множителей различны
\end{itemize}
    
    \addcontentsline{toc}{section}{Билинейные формы} \part*{Билинейные формы}
    
    \section{Билинейные (полуторалинейные) формы (функции).
    Координатная запись билинейной формы.
    Матрица билинейной формы и её изменение при замене базиса}
    
    \subsection{Билинейные (полуторалинейные) формы (функции)}
    
    \textbf{Опр} \textit{Биленейная форма, полуторалинейное отображение} \textcolor{gray}{Не забыть сопрячь на втором
    аргументе}
    
    \textbf{Опр} \textit{Матрица билинейной формы} \textcolor{gray}{Её элемент есть результат применения формы на
    пару базисных векторов}
    
    \subsection{Координатная запись билинейной формы}
    
    \textbf{Th} \textit{Координатная запись}
    
    \textcolor{blue}{$\beta (x, y) = x^T B \overline{y}$}
    
    Достаточно представить векторы в координатной записи и раскрыть по билинейности
    
    \textbf{Утв} \textcolor{blue}{Если произвольно отображения удовлетворяет
    равенству $\beta (x, y) = x^T B \overline{y}$, то это билинейная форма}
    
    Так как запись верна для всех векторов, то она верна и для базисных, к чему мы её и применим.
    Далее проверим аксиомы и убедимся, что перед нами билинейная форма
    
    \subsection{Матрица билинейной формы и её изменение при замене базиса}
    
    \textbf{Th} \textit{Изменение матрицы при замене базиса}
    
    \textcolor{blue}{$\beta^{'} = S^T B \overline{S}$}
    
    Достаточно вставить матрицу перехода на нужные места
    
    \section{Симметричные билинейные (полуторалинейные) формы.
    Взаимно-однозначное соответствие с квадратичными (эрмитовыми) формами}
    
    \subsection{Симметричные билинейные (полуторалинейные) формы}
    
    \textbf{Опр} \textit{(Эрмитова) симметричная форма} \textcolor{gray}{$\beta (x, y) = \overline{\beta (y, x)}$}
    
    \textbf{Th} \textcolor{blue}{Билинейная форма симметрична $\Leftrightarrow B = B^*, B^* = \overline{B^T}$}
    
    $\Rightarrow:$ по определению билинейной формы
    $\Leftarrow:$ надо воспользоваться тем, что результат билинейной формы есть число (матрица 1 на 1), а затем
    подогнать под определение, используя транспонирования и сопряжение
    
    \subsection{Взаимно-однозначное соответствие с квадратичными (эрмитовыми) формами}
    
    \textbf{Опр} \textit{Квадратичная (эрмитова) форма} \textcolor{gray}{$q(x) = \beta (x, x)$, порождённая билинейной}
    
    Учтём, что теперь $\beta (a, a) = \overline{\beta (a, a)}, q(a) =  x^T B \overline{x}$
    
    \textbf{Th} \textcolor{blue}{(Эрмитова) квадратичная форма порождается ровно одной билинейной $\Leftrightarrow B
    = B^*, B^* = \overline{B^T}$}
    
    \begin{enumerate}
        \item Требуется доказать, что значение билинейной формы однозначно восстанавливается по квадратичной
        \item В $\mathbb{R}$ достаточно рассмотреть $q(x + y)$
        \item В $\mathbb{C}$ достаточно рассмотреть $q(x + y)$ и $iq(x + iy)$, не забыв, где надо, про
        комплексное сопряжение и что $i^2 = -1$
    \end{enumerate}
    
    \section{Ядро билинейной функции.
    Ортогональное дополнение подпространства.
    Ограничение билинейной функции на подпространство.
    Критерий невырожденности подпространства.
    Существование нормального вида билинейной симметричной формы над полями $\mathbb{R}$ и $\mathbb{C}$}
    
    \subsection{Ядро билинейной функции}
    
    \section{Алгоритмы приведения квадратичной формы к нормальному виду (метод Лагранжа и сдвоенных элементарных
    преобразований матрицы)}
    
    \textbf{Опр} \textit{Диагональный вид формы} \textcolor{gray}{Матрица формы в данном базисе диагональна}
    
    \textbf{Опр} \textit{Канонический диагональный вид формы} \textcolor{gray}{Каждый элемент диагонали $\in \{-1;0;1\}$}
    
    Заметим, от диагонального вида легко перейти к каноническому (путём линейной замены)
    
    \textbf{Th} \textcolor{blue}{Существует базис, в котором квадратичная форма имеет диагональный вид}
    
    \begin{itemize}
        \item Допустим $b_{11} \neq 0, b_{m1} \neq 0$.
        Тогда применим сдвоенное элементарное преобразование: вычтем из $m$-й строки и столбца 1-ю строку и столбец,
        домноженные на соответствующий коэффициент ($b_{1m} \neq \overline{b_{m1}}$).
        Далее продолжим с матрицей меньшей размерности
        \item Если хотя бы один диагональный элемент не ноль, то поменяем местами базисные векторы и продолжим
        как в простом случае
        \item Если все диагональные элементы ноль, то прибавим к нему столбец и строку с ненулевым элементом $\lambda$
        той же
        линии (получится в результате сдвоенности $2\lambda$). Затем продолжим как в простом случае
    \end{itemize}
    
    \section{Закон инерции квадратичной (эрмитовой) формы.
    Положительный и отрицательный индексы инерции, их геометрическая характеризация.
    Критерий Сильвестра}
    
    \subsection{Закон инерции квадратичной (эрмитовой) формы}
    
    \textbf{Опр} \textit{Положительно (полу)определённая форма} \textcolor{gray}{Положительна (неотрицательна) на ...}
    
    \textbf{Опр} \textit{Отрицательно (полу)определённая форма} \textcolor{gray}{Аналогично}
    
    \textbf{Th} \textcolor{blue}{Квадратичная форма определена положительно $\Leftrightarrow d_i > 0 \forall i$}
    
    $\Rightarrow:$ $d_i = q(e_i)$ по определению положительной определённости
    $\Leftarrow:$ в силу $q(a) = \sum_i d_i \abs{x_i}^2$
    
    Отсюда следует, что определитель матрицы положительно определённой формы положителен
    
    Аналогичные критерии есть и у отрицательно- и полуопределённых форм
    
    \textbf{Опр} \textit{Положительный индекс инерции} \textcolor{gray}{Наибольшее число, для которого $\exists U$ ...}
    
    \textbf{Th} \textit{Об индексах инерции}
    
    \textcolor{blue}{Индексы инерции $p, q$ равны количеству соответствующих по знаку чисел среди $d_i$}
    
    \begin{enumerate}
        \item Докажем для положительного индекса инерции
        \item БОО положительные можно считать $p^{'}$ -- количество положительных $d_i$ -- элементов первыми.
        Тогда ограничение на этом подпространстве и будет отвечать определению положительного индекса инерции
        \item Пусть $p > p^{'}$.
        Тогда рассмотрим подпространство соответствующей размерности
        \item По формуле включений-исключений придём к тому, что пересечение пространства и отрицательно
        полуопределённого подпространства ненулевое.
        Тогда существует ненулевой вектор, на котором форма не определена однозначно, $w$
    \end{enumerate}
    
    \textbf{Следствие} \textcolor{blue}{Ранг квадратичной формы равен $p + q$}
    
    \textbf{Опр} \textit{Сигнатура квадратичной формы} \textcolor{gray}{($p, q, n - p - q$)}
    
    \subsection{Критерий Сильвестра}
    
    \textbf{Th} \textit{Критерий Сильвестра}
    
    \textcolor{blue}{Квадратичная форма определена положительна $\Leftrightarrow M_i > 0 \forall i$}
    
    \begin{enumerate}
        \item $\Rightarrow:$ достаточно рассмотреть сужение на каждое подпространство и вспомнить про следствие из
        закона инерции
        \item $\Leftrightarrow:$ от противного.
        Положим $m$ -- минимальное число первых базисных векторов элементов, на которых форма определена не
        положительно и рассмотрим сужение на них
        \item Тогда из $p \geq m - 1$ (в силу сужения на $m - 1$ подпространстве) и $p < m$, иначе $m$ определено
        неверно.
        Получаем равенство $p = m - 1$
        \item Перейдём к диагональному виду и рассмотрим определитель сужения на $m$-ное пространство.
        Он будет неположителен, $w$
    \end{enumerate}
    
    \section{Кососимметричные билинейные функции, приведение их к нормальному виду}
    
    \textbf{Опр} \textit{Кососимметрическая билинейная функция} \textcolor{gray}{$\beta (x, y) = - \overline{\beta (y, x)}$}
    
    Из определения видно, что $\beta (x, x) = 0$
    
    \textbf{Th} \textcolor{blue}{Билинейная форма кососимметрическая $\Leftrightarrow \overline{B^T}= -B$}
    
    $\Rightarrow:$ по определению кососимметрической формы
    $\Leftarrow:$ надо воспользоваться тем, что результат билинейной формы есть число (матрица 1 на 1), а затем
    подогнать под определение, используя транспонирования и сопряжение
    
    \textbf{Следствие} \textcolor{blue}{Билинейная форма в нечётномерное векторном пространстве вырождена}
    \textcolor{gray}{Потому как тождественна $O$}
    
    \textbf{Th} \textit{Канонический вид}
    
    \textcolor{blue}{Существует базис, в котором квадратичная форма имеет диагональный вид}
    
    \begin{itemize}
        \item Допустим $b_{11} = 0$ и соответствующие строка и столбец нулевые, то спускаем по размерности вниз
        \item Если хотя бы один не первый элемент строки не ноль, то поменяем местами вторую и эту строки.
        Затем путём элементарных преобразований сделаем +1 (на столбце получим схожую картину) и спустимся вниз
        \item Если и после первого есть ненулевой элемент, то его можно сдвоенными ЭПС сделать нулевым
    \end{itemize}
    
    \addcontentsline{toc}{section}{Пространства со скалярным произведением} \part*{Пространства со скалярным произведением}
    
    \section{Евклидовы и унитарные пространства.
    Матрица Грама и её свойства.
    Неравенство Коши -- Буняковского -- Шварца, неравенство треугольника.
    Метрика.
    Выражение скалярного произведения в координатах}
    
    \subsection{Евклидовы и унитарные пространства}
    
    \textbf{Опр} \textit{Евклидово (унитарное) пространство} \textcolor{gray}{Пространство над полем с фиксированным
    скалярным произведением}
    
    \textbf{Опр} \textit{Норма (длина) вектора} \textcolor{gray}{$\abs{x} =  \beta (x, x)$}
    
    Норма неотрицательна и нулевая в случае нулевого вектора
    
    \textbf{Опр} \textit{Ортогональные векторы} \textcolor{gray}{$\beta (x, y) = 0$}
    
    \subsection{Матрица Грама и её свойства}
    
    \textbf{Опр} \textit{Матрица Грама} \textcolor{gray}{Матрица система векторов: $g_{ij} = (a_i, a_j)$}
    
    \textbf{Утв} \textcolor{blue}{Определитель матрицы Грама положителен при л.н.з системе и ноль иначе}
    
    \begin{enumerate}
        \item На л.н.з. векторах матрица Грама есть матрица п.о. билинейной симметрической формы, поэтому её
        детерминант положителен.
        \item В случае л.з. системы составим её нетривиальную л.к., домножим на векторы $a_j$ и повторим так $\forall j \in \overline{1, n}$
        \item Тогда если составить из строчек матричное уравнение, то получим $\Gamma x = 0$, что в силу $x \neq 0$
        означает вырожденность $\Gamma$
    \end{enumerate}
    
    \subsection{Неравенство Коши -- Буняковского -- Шварца, неравенство треугольника}
    
    \textbf{Следствие} \textit{Неравенство Коши -- Буняковского -- Шварца}
    
    \textcolor{blue}{$\forall a, b \in \mathscr{E} \abs{a} \abs{b} \geq \abs{(a, b)}$}
    
    Достаточно воспользоваться предыдущей теоремой и раскрыть определитель, сняв в конце квадраты
    
    \textbf{Следствие} \textit{Неравенство треугольника}
    
    \textcolor{blue}{$\forall a, b \in \mathscr{E} \abs{a} \abs{b} \geq \abs{a + b}$}
    
    Достаточно расписать $\abs{a+b}^2$ и воспользоваться предыдущим неравенством
    
    \subsection{Метрика}
    
    На $\mathscr{E}$ введём метрику как $\rho (a, b) = \abs{b - a}$.
    Заметим, что в таком случае выполняются все 4 аксиомы метрики (неотрицательность, ноль при нуле, симметричность и
    неравенство треугольника)
    
    \subsection{Выражение скалярного произведения в координатах}
    
    \textbf{Опр} \textit{Скалярное произведение} \textcolor{gray}{Билинейная (эрмитова) симметричная положительно ...}
    
    \textbf{Th} \textit{Скалярное произведение}
    
    \textcolor{blue}{$(a, b) = x^T \Gamma \overline{y}$}
    
    Это верно, потому как в случае базисных векторов матрица Грамма совпадает с матрицей билинейной формы скалярного
    произведения
    
    \section{Ортогональные системы векторов и подпространств.
    Существование ортонормированных базисов (ОНБ).
    Изоморфизм евклидовых пространств.
    Ортогональные и унитарные матрицы.
    Переход от ОНБ к ОНБ}
    
    \subsection{Ортогональные системы векторов и подпространств}
    
    \textbf{Опр} \textit{Ортогональная, ортонормированная система} \textcolor{gray}{Векторы системе попарно ...}
    
    \textbf{Утв} \textit{Теорема Пифагора}
    
    \textcolor{blue}{$\abs{a_1 + ... + a_n}^2 = \abs{a_1}^2 + \dots \abs{a_n}^2$}
    
    Раскрываем по линейности и ортогональности
    
    \textbf{Утв} \textcolor{blue}{Система ортогональная $\Leftrightarrow$ матрица Грама ортогональная}
    
    Следует из определения матрицы Грама.
    Аналогично в ортонормированном случае матрица Грама единичная
    
    \textbf{Следствие 1} \textcolor{blue}{Ортогональная система ненулевых векторов л.н.з.}
    
    Потому как соответсвующая матрица Грама невырождена
    
    \textbf{Следствие 2} \textcolor{blue}{При ортогональном базисе матрица формы скалярного произведения имеет
    диагональный вид, а при ОНБ -- канонический}
    
    \subsection{Существование ортонормированных базисов (ОНБ)}
    
    \textbf{Следствие 3} \textcolor{blue}{В конечномерном евклидовом пространстве существует ОНБ}
    
    Потому как существуют ортогональные системы.
    Если мы их запишем в виде матрицы формы, то, так как мы их умеем приводить к каноническому виду, мы получим ОНБ
    
    \subsection{Изоморфизм евклидовых пространств}
    
    \textbf{Опр} \textit{Изоморфизм евклидовых пространств} \textcolor{gray}{Изоморфизм линейных пространств и ...}
    
    \textbf{Утв} \textcolor{blue}{Отображение изоморфно $\Leftrightarrow$ оно переводит ОНБ в ОНБ}
    
    $\Rightarrow:$ в силу сохранения скалярного произведения и соразмерности пространств (следствие изоморфности)
    $\Leftarrow:$ отображение переводит базис в базис, поэтому перед нами обычный изоморфизм линейных
    пространств.
    Применим отображение на двух произвольных векторах пространства.
    И получим, что сохраняется скалярное произведение, то есть перед нами изоморфизм линейных пространств по определению
    
    \textbf{Th} \textcolor{blue}{Два конечномерных евклидова пространства изоморфны $\Leftrightarrow$ они соразмерны}
    
    $\Rightarrow:$ в силу свойств изоморфизма линейных пространств
    $\Leftrightarrow:$ приведём базисных обеих пространств к ОНБ и построим отображение, переводящее базис в базис.
    По предыдущему утверждению, перед нами изоморфизм
    
    \subsection{Ортогональные и унитарные матрицы}
    
    \textbf{Опр} \textit{Ортогональная, унитарная матрицы} \textcolor{gray}{Над разными полями, множества пересекаются}
    
    \textbf{Утв} \textcolor{blue}{Матрицы $Q, R$ унитарны $\Rightarrow$ матрицы $Q^T, \overline{Q}, Q^*, QR, Q^{-1}$
        унитарны}
    
    Непосредственно проверяется определение
    
    \textbf{Утв} \textcolor{blue}{Детерминант унитарной матрицы единичен}
    
    Для доказательства достаточно расписать определитель в определении и воспользоваться свойствами определителя
    
    \textbf{Утв} \textcolor{blue}{Для комплекснозначных матриц $Q$ следующие условия эквивалентны
        \begin{enumerate}
            \item $Q$ унитарна
            \item $\exists Q^{-1},  Q^{-1} = Q^*$
            \item Столбцы $Q$ образуют ОНБ в унитарном пространстве столбцов
        \end{enumerate}               }
    
    \begin{itemize}
        \item $1 \Leftrightarrow 2:$ по определению
        \item $1 \Leftrightarrow 3:$ в силу определения унитарной матрицы возьмём скалярное произведение и получим,
        что каждый элемент результата есть $\delta_{ij}$, то есть перед нами ОНБ
        \item Столбцы $Q$ образуют ОНБ в унитарном пространстве столбцов
    \end{itemize}
    
    \subsection{Переход от ОНБ к ОНБ}
    
    \textbf{Следствие} \textit{Переход от ОНБ к ОНБ}
    
    \textcolor{blue}{Базисы ОНБ $\Leftrightarrow$ матрица перехода между ними ортогональная (унитарная)}
    
    $\Rightarrow:$ потому что произведение матриц единично $\Leftrightarrow$ матрицы единичны
    $\Leftarrow:$ по определению матрицы перехода
    
    \section{Ортогональное дополнение подпространства.
    Ортогональная проекция.
    Алгоритм ортогонализации Грама-Шмидта}
    
    \subsection{Ортогональное дополнение подпространства}
    
    \textbf{Опр} \textit{Ортогональное дополнение} \textcolor{gray}{Множество всех векторов, ортогональных ...}
    
    Пространство образует со своим ортогональным дополнением прямую сумму
    
    \textbf{Th} \textcolor{blue}{Сумма подпространства и его ортогонального дополнения есть всё евклидово пространство}
    
    \begin{enumerate}
        \item Достаточно научиться представлять любой вектор пространства в виде суммы $U$ и $U^T$
        \item Выберем ортогональный базис в $U$ и запишем его линейную комбинацию + вектор $c \in  U^T$
        \item Теперь надо подобрать такие коэффициенты, чтобы $c \bot U$
        \item Заменим условие на эквивалентные, вспомним про ортонормированность базиса и выразим коэффициенты
    \end{enumerate}
    
    \textbf{Следствие 1} \textcolor{blue}{$\dim U = k, \dim \mathscr{E} = n \rightarrow \dim U^T = n - k$}
    
    \textbf{Следствие 2} \textcolor{blue}{$(U^T)^T = U$}
    
    Потому как одно пространство вложено в другое и у них, по предыдущему следствию, равны размерности
    
    \textbf{Следствие 2} \textcolor{blue}{В конечномерном случае данную ортогональную систему из ненулевых векторов
    можно дополнить до ОНБ}
    
    Достаточно дополнить векторами из ортогонального дополнения
    
    \subsection{Ортогональная проекция}
    
    \textbf{Опр} \textit{Ортогональная проекция} \textcolor{gray}{Проекция на подпространство вдоль (параллельно) $U^T$}
    
    \textbf{Утв} \textit{Формула проекции}
    
    \textcolor{blue}{${pr}_U \overline{a} = \sum_i \frac{(a_i, b_i)}{(b_i, b_i)} b_i$}
    
    Следствие последней теоремы
    
    \textbf{Утв} \textit{Ортогональное дополнение в координатах}
    
    \textcolor{blue}{Ортогональное дополнение есть пространство решений уравнения $(A_1, \dots, A_n)^* x = 0$}
    
    $x \in A^T \Leftrightarrow x \bot A \Leftrightarrow A_i \bot x \Leftrightarrow A_i^* x = 0$ и перейдём к матричной
    записи.
    Решение полученного уравнения и есть ортогональное дополнение
    
    \subsection{Алгоритм ортогонализации Грама-Шмидта}
    
    \textbf{Утв} \textcolor{blue}{Существует метод найти ортогональный базис в заданном подпространстве}
    
    \begin{itemize}
        \item Рассмотрим линейную оболочку подпространства.
        Если $a_1 = 0$, то выкинем его из линейной оболочки
        \item Если $a_1 \neq 0$, то оставим его таким, какой он есть: $b_1 = a_1$
        \item Если все $a_k$ до текущего уже ортогонализованы, то $b_{k+1} = a_{k+1} - {pr}_{<b_1, \dots, b_k>} a_{k+1}$
    \end{itemize}
    
    При необходимости, полученную систему можно нормировать для получения ОНБ
    
    \section{Описание линейных функций на евклидовом (унитарном) пространстве}
    
    \section{Преобразование, сопряжённое данному.
    Его линейность, существование и единственность, его матрица в ОНБ.
    Теорема Фредгольма}
    
    \subsection{Преобразование, сопряжённое данному}
    
    \textbf{Опр} \textit{Сопряжённое преобразование} \textcolor{gray}{$\varphi^*: (\varphi(a), b) = (a, \varphi(b))$}
    
    \subsection{Его линейность, существование и единственность, его матрица в ОНБ}

%    Линейность следует из линейности
    
    \textbf{Утв} \textcolor{blue}{$\psi = \varphi^* \Leftrightarrow B = A^*$}
    
    Достаточно расписать результат формы на паре векторов, определение сопряжённого преобразование и взглянуть на
    матрицы
    
    \textbf{Следствие 1} \textcolor{blue}{$\varphi^*$ единственно}
    
    Потому как у каждой матрицы есть единственная сопряжённо-транспонированная
    
    \textbf{Следствие 1} \textcolor{blue}{Для сопряжённых преобразований справедливо 4 свойства}
    
    Первые три следуют из аналогичных свойств для матриц, а последнее из свойств комплексного сопряжения
    
    \textbf{Th} \textcolor{blue}{$U$ инвариантно относительно $\varphi \Leftrightarrow U^\bot$ инвариантно
    относительно $\varphi^*$}
    
    Достаточно вспомнить определения инвариантности, ортогонального дополнения и сопряжённого образования
    
    \subsection{Теорема Фредгольма}
    
    \textbf{Th} \textit{Фредгольма}
    
    \textcolor{blue}{$\ker \varphi^* = (\Im \varphi)^\bot$}
    
    \begin{enumerate}
        \item Докажем вложенность ядра в чужой образ и равенство размерностей.
        Это будет означать равенство
        \item Равенство размерностей доказывается по прошлым утверждениям
        \item Чтобы доказать вложенность рассмотрим произвольный вектор ядра, воспользуемся определениями
        ортогонального дополнения, образа и сопряжённого преобразования
    \end{enumerate}
    
    \section{Самосопряжённые линейные преобразования.
    Свойства самосопряжённых преобразований, существование ОНБ из собственных векторов}
    
    \subsection{Самосопряжённые линейные преобразования}
    
    \textbf{Опр} \textit{Сопряжённое линейное преобразование} \textcolor{gray}{$\varphi^* = \varphi$}
    
    В таком случае $(\varphi (a), b) = (a, \varphi(b))$
    
    \subsection{Свойства самосопряжённых преобразований, существование ОНБ из собственных векторов}
    
    \textbf{Th} \textcolor{blue}{$\varphi$ самосопряжено $\Leftrightarrow A = A^*$}
    
    Аналогично доказательству для сопряжённых преобразований
    
    \textbf{Th} \textcolor{blue}{У самосопряжённого преобразования все характеристические числа действительны}
    
    В $\mathbb{C}$ достаточно расписать определение самосопряжённого преобразования, собственного числа и прийти к
    равенству $\lambda = \overline{\lambda}$, что означает действительность
    
    Так как в $\mathbb{C}$ доказано, что характеристическое уравнение имеет лишь действительные корни.
    А симметрические вещественные матрицы являются частным случаем эрмитовых, поэтому теорема доказана и в $\mathbb{R}$
    
    \textbf{Утв} \textcolor{blue}{У самосопряжённого преобразования различные корневые подпространства перпендикулярны}
    
    Достаточно рассмотреть два вектора из разных корневых подпространств, расписать определение самосопряжённого
    преобразования, собственного числа и прийти к единственному случаю $(a_i, a_j) = 0$
    
    \textbf{Th} \textit{Основная теорема о самосопряжённых преобразованиях}
    
    \textcolor{blue}{Для самосопряжённого преобразования сущетсвует ОНБ из собственных векторов}
    
    \begin{enumerate}
        \item Пусть $\dim \mathscr{E} = n$.
        В случае $n = 1$ очевидно
        \item Ортогональное дополнение первого вектора ОНБ инвариантно относительно $\varphi*$, как и относительно $\varphi$ в
        силу самосопряжённости
        \item Поэтому мы получили ортонормированный базис на сужении размерности $n - 1$ и их объединение будет ОНБ на
        подпространстве соответствующей размерности
    \end{enumerate}
    
    \section{Ортогональные и унитарные преобразования, их свойства.
    Канонический вид унитарного и ортогонального преобразования.
    Нормальные преобразования унитарных пространств}
    
    \subsection{Ортогональные и унитарные преобразования, их свойства}
    
    \textbf{Опр} \textit{Ортогональное (униатрное) преобразование} \textcolor{gray}{$(\varphi (a), \varphi (b)) = (a, b)$}
    
    \textbf{Утв} \textcolor{blue}{$\varphi$ ортогонально (матрица перехода между ОНБ) $\Leftrightarrow \varphi$
        изоморфизм евклидовых (унитарных) пространств}
    
    $\Rightarrow:$ в силу биективности (ОНБ переходит в ортонормированную систему из $n$ векторов, то есть в ОНБ,
    потому что скалярное произведение сохранено)
    $\Leftarrow:$ достаточно расписать скалярное произведение двух произвольных векторов и воспользоваться
    изоморфностью (идея как при изоморфизме линейных пространств).
    Получим сохранение скалярного произведения и ортогональность $\varphi$ по определению
    
    \textbf{Следствие 1} \textcolor{blue}{Ортогональное преобразование переводит ОНБ в ОНБ}
    
    \textbf{Следствие 2} \textcolor{blue}{Преобразование ортогонально $\Leftrightarrow$ его матрица ортогональна}
    
    Потому как ортогональная матрица -- матрица перехода между ОНБ
    
    \textbf{Следствие 3} \textcolor{blue}{Преобразование ортогонально $\Leftrightarrow \varphi$ обратимо и
    матрица $\varphi^{-1} = \varphi^*$}
    
    Достаточно расписать определение унитарного преобразования
    
    \textbf{Утв} \textit{Групповые свойства}
    
    \textcolor{blue}{Для ортогональных преобразований их композиция и обратное тоже ортогональное}
    
    Достаточно привести к определению
    
    \textbf{Утв} \textcolor{blue}{Характеристические числа ортогональных преобразований по модулю равны единице}
    
    Достаточно расписать определение и вспомнить про комплексное сопряжение
    
    \subsection{Канонический вид унитарного и ортогонального преобразования}
    
    \textbf{Th} \textit{Канонический вид унитарного преобразования}
    
    \textcolor{blue}{Для унитарного преобразования сущетсвует ОНБ из собственных векторов}
    
    \begin{enumerate}
        \item Пусть $\dim \mathscr{E} = n$.
        В случае $n = 1$ очевидно
        \item Ортогональное дополнение первого вектора ОНБ инвариантно относительно $\varphi^*$, как и относительно $\varphi^{-1}$ в
        силу ортогональности.
        При изучении инвариантных подпространств мы выяснили, что это эквивалентно инвариантности и относительно $\varphi$
        \item Поэтому мы получили ортонормированный базис на сужении размерности $n - 1$ и их объединение будет ОНБ на
        подпространстве соответствующей размерности
    \end{enumerate}
    
    \section{Полярное разложение линейного преобразования в евклидовом пространстве, его существование}
    
    \textbf{Лемма} \textit{О главных направлениях}
    
    \textcolor{blue}{Для линейного преобразования $\varphi$ существует
    ОНБ $e_1, \dots, e_n: \varphi(e_1), \dots, \varphi(e_n)$ образуют ортогональную систему}
    
    Рассмотрим оператор $\varphi^* \varphi$ (проверяется, что он СС) и ОНБ из его собственных векторов (по теореме).
    Далее, пользуясь СС-ю получаем, что $(\varphi(e_i), \varphi(e_j)) = \dots = \lambda_i \delta_{ij}$
    
    \textbf{Th} \textcolor{blue}{Для линейного преобразования $\varphi \exists$ самосопряжённое
    преобразвоание $\psi$ и ортогональное (унитарное) $\theta$}
    
    \begin{enumerate}
        \item Рассмотрим ортогональную систему из леммы, притом $\abs{\varphi(e_i)} = \sqrt{\lambda_i}$.
        При необходимости, переупорядочим её
        \item Отнормируем систему и дополним её до ОНБ, убрав, при необходимости, нулевые векторы.
        Получим ОНБ $f_1, \dots, f_n$
        \item Теперь определим $\psi = \varphi \theta^{-1}$ и убедимся, что $\psi(f_i) = \dots = \sqrt{\lambda_i} f_i$
        \item Итого, нежные отображения подобраны
    \end{enumerate}
    
    Совсем необязательно, что данные преобразования коммутируют (перестановочны).
    Однако можно применить теорему к $\varphi^*$ и взять сопряжение с обеих сторон.
    Тогда мы как раз получим другой порядок
    
    \section{Квадратичные (эрмитовы) формы в евклидовых (унитарных) пространствах.
    Присоединенный оператор.
    Существование ОНБ, в котором квадратичная (эрмитова) форма имеет диагональный вид.
    Применение к классификации кривых второго порядка.
    Одновременное приведение пары квадратичных форм к диагональному виду}
    
    \subsection{Квадратичные (эрмитовы) формы в евклидовых (унитарных) пространствах}
    
    \textbf{Опр} \textit{Квадратичная форма в евклидовом пространстве} \textcolor{gray}{$\beta_\varphi (a, b) = (a, \varphi(b))$}
    
    \textbf{Утв} \textcolor{blue}{В случае ОНБ $B = \overline{A}$}
    
    Пользуемся результатом действия билинейной формы на паре векторов и сравниваем записи.
    
    В случае произвольного базиса $B = \Gamma \overline{A}$
    
    \textbf{Следствие 1} \textcolor{blue}{Задана биекция между линейными преобразованиями и билинейными формами}
    
    \textbf{Следствие 2} \textcolor{blue}{Задана биекция между множество самосопряжённых операторов и квадратичных форм}
    
    Потому как и тем, и другим соответствует симметричная матрица
    
    Итого, изучения биленейных форм можно свести к изучению операторов (и наоборот), а изучение квадратичных -- к
    самосопряжённым операторам
    
    \subsection{Существование ОНБ, в котором квадратичная (эрмитова) форма имеет диагональный вид}
    
    \textbf{Th} \textit{Приведение к главным осям}
    
    \textcolor{blue}{Существует ОНБ, в котором матрица квадратичной формы над ЕП имеет диагональный вид}
    
    Следует из того, что для самосопряжённого оператора существует ОНБ, в котором его матрица диагональна.
    Она отличается от требуемой не более, чем сопряжением
    
    \subsection{Применение к классификации кривых второго порядка}
    
    \textbf{Лемма} \textcolor{blue}{$\exists$ ПДСК, в которой кривая второго порядка задаётся уравнением без
    перекрёстных членов. Аналогично для поверхностей}
    
    Для предъявления такой ПДСК достаточно привести квадратичную форму к главным осям
    
    \subsection{Одновременное приведение пары квадратичных форм к диагональному виду}
    
    \textbf{Th} \textit{О паре форм}
    
    \textcolor{blue}{Если в векторном пространстве (без евклидовой / унитарной структуры) заданы две симметрические
    квадратичные формы, причём первая п.о. то существует базис, в котором первая имеет канониыеский вид, а вторая --
    диагональный}
    
    Достаточно объявить п.о. форму скалярным произведением.
    Тогда будет существовать базис, в котором вторая форма диагональна
    
    \addcontentsline{toc}{section}{Сопряжённое пространство} \part*{Сопряжённое пространство}
    
    \section{Линейные функции.
    Сопряжённое пространство, его размерность.
    Биортогональный базис.
    Замена биортогональных базисов.
    Канонический изоморфизм пространства и дважды сопряжённого к нему}
    
    \subsection{Линейные функции}
    
    \textbf{Опр} \textit{Линейная функция} \textcolor{gray}{Отображение, удовлетворяющая двум аксиомам}
    
    \subsection{Сопряжённое пространство, его размерность}
    
    \textbf{Опр} \textit{Сопряжённое (двойственное) пространство} \textcolor{gray}{Пространство ...}
    
    Элементы сопряжённого пространства -- линейные функционалы (функции), поэтому такие пространства также называют
    пространством линейных функций.
    Обозначаются как $V^*$
    
    \textbf{Утв} \textcolor{blue}{$\dim V^* = \dim V$}
    
    Следует из $\dim \mathbb{R} = \dim \mathbb{C} = 1$ и отождествления с матрицами размерности $nm$
    
    Применению линейной функции к вектору, удовлетворяющему четырём аксиомам, соответствует билинейная (
    полуторалинейная) форма
    
    \subsection{Биортогональный базис}
    
    \textbf{Опр} \textit{Взаимный / биортогональный / двойственный базис} \textcolor{gray}{$<e_i, e^j> = \delta_i^j$}
    
    \textbf{Утв} \textcolor{blue}{К данном базису существует и единственен взаимный}
    
    Любому элементу взаимного базиса соответствует строчная единица.
    Строчные единицы образуют базис в $V$, поэтому и элементы взаимного базиса образуют базис в $V^*$.
    Базис единственен по построению (в силу инъективности линейных функций)
    
    \textbf{Утв} \textcolor{blue}{Двойственный базис является базисом в $V^*$}
    
    В силу равенства размерностей пространств достаточно доказать л.н.з. $f_1, \dots, f_n$.
    Это делается от противного с применением $e_j \forall j$ на линейной комбинации
    
    \textbf{Утв} \textcolor{blue}{Если при фиксированном $a \in V <a, l> = 0 \forall l \in V^*$, то $a = 0$}
    
    От противного включим $a$ в какой-то базис
    
    \textbf{Утв} \textcolor{blue}{$<a, l> = x^i \overline{y_i}$}
    
    Следует из подстановки разложений по базисам и определения $\delta_i^j$
    
    \textbf{Следствие} \textcolor{blue}{$<a, e^i> = x^i$}
    
    \subsection{Замена биортогональных базисов}
    
    \textbf{Утв} \textcolor{blue}{Если $e^{'} = eS, e^{'*} = e^* C$, то $C = (S^{-1})^*$}
    
    Тензорно запишем $e^{'}$ как строки матрицы на векторы-столбцы и введём $R = C^T$, чтобы аналогично сделать с $e^{'*}$.
    Затем раскроем по условию биортогональности и вернёмся к матричной записи
    
    \subsection{Канонический изоморфизм пространства и дважды сопряжённого к нему}
    
    \textbf{Опр} \textit{Канонический изоморфизм} \textcolor{gray}{Не меняется при замене базиса}
    
    \textbf{Опр} \textit{Дважды сопряжённое пространство} \textcolor{gray}{Отображение, сопостовляющее
    вектору $a \in V$ отображение $\overleftarrow{a}: V^* \rightarrow \mathbb{R} (\mathbb{C})$ по
    правилу $<l, \overleftarrow{a}> = \overline{<a, l>}$ есть инъевтиный гомоморфизм (вложение) $V \rightarrow V^{**}$}
    
    \textbf{Th} \textit{Канонический изоморфизм между $V$ и $V^{**}$}
    
    \textcolor{blue}{Между линейным пространством и дважды сопряжённым к нему сущесвтует канонический изоморфизм}
    
    Для доказательства достаточно проверить линейность по обеим аргументам и тривиальность ядра (всё по
    определению).
    По критерию изоморфности в силу инъективности (тривиальность ядра) имеем изоморфизм
    
    \section{Аннулятор подпространства, соответствие между подпространствами V и V*.
    Сопряжённое преобразование, его свойства}
    
    \subsection{Аннулятор подпространства, соответствие между подпространствами $V$ и $V^*$}
    
    \textbf{Опр} \textit{Биортогональные множества} \textcolor{gray}{$\forall a \in U \forall l \in W <a, l> = 0$}
    
    \textbf{Утв} \textit{Признак биортогональности}
    
    \textcolor{blue}{$U \bot W \Leftrightarrow a_i \bot l_j$}
    
    $\Rightarrow:$ очевидно в силу вложенности
    $\Leftarrow:$ из разложения по базису и линейности
    
    \textbf{Опр} \textit{Аннулятор / биортогональное дополнение} \textcolor{gray}{Множество $W$ линейных функций}
    
    Обозначается как $U^\bot$

%    \textbf{Утв} \textcolor{blue}{Если $\forall U \subset V U^\bot \in V^*$}
    
    \textbf{Опр} \textit{Нуль-пространство} \textcolor{gray}{Обратное к аннулятору: множество $U$ векторов}

%    \textbf{Утв} \textcolor{blue}{Если $\forall W \subset V^* W^\bot \in V$}
    
    \textbf{Th} \textcolor{blue}{$(U^\bot)^\bot = U$ и $\dim U + \dim U^\bot = n$}
    
    \begin{enumerate}
        \item Выберем базис $e_1, \dots, e_k$ в $U$ и дополним его до базиса во всём пространстве векторами $e_{k+1},
        \dots, e_n$
        \item Далее рассмотрим линейную функцию, записанную в своём базисе и перейдём к системе, задающей $\bot$
        \item Получим, что тогда каждый коэффициент $\lambda_i = 0, i \in \overline{1, k}$, что говорит о структуре $
        U^\bot$
        \item Аналогичную операцию произведём в $V^*$ и докажем первый факт
        \item Собрав информацию о размерностях, получим второй факт
    \end{enumerate}
    
    \subsection{Сопряжённое преобразование, его свойства}
    
    \textbf{Опр} \textit{Сопряжённое преобразование} \textcolor{gray}{Отображение уже из пространства функций}
    
    \textbf{Утв} \textcolor{blue}{Сопряжённое преобразование лежит в пространстве функций}
    
    Проверяется линейность (4 аксиомы) с использованием определения
    
    \textbf{Утв} \textcolor{blue}{Сопряжённое преобразование соотвествует матрица $A^*$}
    
    Надо разложить в матричный вид равенства из определения сопряжённого пространства и сравнить их.
    Получив искомую структуру матрицы
    
    \textbf{Следствие 1} \textcolor{blue}{Верны 4 равенства}
    
    Введём взаимные базисы и перейдём к матрицам.
    Доказательства очевидны случаю евклидова пространства
    
    \textbf{Th} \textcolor{blue}{$U$ инвариантно относительно $\varphi \Leftrightarrow U^\bot$ инвариантно относительно $\varphi^*$}
    
    $\Rightarrow:$ возьмём $f \in U^\bot$ и распишем его применение по определению
    $\Leftarrow:$ следует из $\Rightarrow$, $(U^\bot)^\bot = U$ и $(\varphi^*)^* = \varphi$
    
    \textbf{Th} \textit{Фредгольма}
    
    \textcolor{blue}{$\ker \varphi^* = (\Im \varphi)^\bot$}
    
    \begin{enumerate}
        \item Аналогично случаю в ЕП: докажем вложенность ядра в чужой образ и равенство размерностей.
        Это будет означать требуемое равенство
        \item Равенство размерностей доказывается по прошлым утверждениям
        \item Чтобы доказать вложенность рассмотрим произвольный вектор ядра, воспользуемся определениями
        ортогонального дополнения, образа и сопряжённого преобразования
    \end{enumerate}
    
    \addcontentsline{toc}{section}{Тензоры} \part*{Тензоры}
    
    \section{Полилинейные отображения.
    Определение тензора типа $(p,q)$ на линейном пространстве $V$.
    Пространство $T^p_q (V)$ тензоров типа $(p,q)$.
    Тензорный базис в $T^p_q (V)$.
    Изменение компонент тензора при замене базиса}
    
    \subsection{Полилинейные отображения}


\end{document}
