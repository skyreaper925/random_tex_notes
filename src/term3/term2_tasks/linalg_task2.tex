\addcontentsline{toc}{section}{Второе задание}

\addcontentsline{toc}{subsection}{Положительная билейная форма} \subsection*{Положительная билинейная форма}

Для начала, давайте вспомним определение положительно определенной квадратичной формы.
Квадратичная форма $Q$ над полем вещественных чисел $\mathbb{R}$ называется положительно определенной, если для
любого ненулевого вектора $x$ из $\mathbb{R}^n$ значение $Q(x)$ положительно, то есть $Q(x) > 0$.

Теперь рассмотрим матрицу $A$, соответствующую данной квадратичной форме.
Пусть $a_{ij}$ -- элементы этой матрицы.
Тогда квадратичная форма $Q(x)$ может быть записана в виде $Q(x) = x^T A x$.
Заметим, что матрица $A$ симметрична, так как $a_{ij} = a_{ji}$.

Предположим, что максимальный по модулю элемент матрицы $A$ отрицательный, то есть $|a_{ij}| > |a_{kl}|$ для всех $i,
j,k,l$.
Тогда рассмотрим вектор $x$, у которого все компоненты равны нулю, кроме $i$-ой и $j$-ой, которые равны
соответственно $1$ и $-1$.
Тогда $x^T A x = a_{ij} - a_{ji} < 0$, так как $a_{ij}$ отрицательный.
Но это противоречит тому, что квадратичная форма $Q(x)$ положительно определена, так как мы нашли вектор $x$, для
которого $Q(x) < 0$.

Следовательно, максимальный по модулю элемент матрицы $A$ положителен

\addcontentsline{toc}{subsection}{След квадрата матрицы} \subsection*{След квадрата матрицы}

Для начала, давайте запишем квадратичную форму, соответствующую следу квадрата матрицы порядка $n$.
Пусть $A$ - матрица порядка $n$, тогда $Q(A) = \operatorname{tr}(A^2)$.

Заметим, что матрица $A^2$ является симметрической, так как $(A^2)^T = A^T A^T = A A = A^2$.
Следовательно, квадратичная форма $Q(A)$ является квадратичной формой на симметрических матрицах порядка $n$.

Теперь давайте найдем ранг этой квадратичной формы.
Для этого нам нужно найти количество независимых переменных, от которых зависит квадратичная форма.
Поскольку квадратичная форма зависит только от матрицы $A$, у которой $n^2$ элементов, то ранг квадратичной формы
равен $n^2$.

Чтобы найти сигнатуру квадратичной формы, нужно найти количество положительных, отрицательных и нулевых собственных
значений матрицы $A^2$.
Заметим, что собственные значения матрицы $A^2$ всегда неотрицательны, так как $\operatorname{tr}(A^2)$ является
суммой квадратов собственных значений матрицы $A$.
Следовательно, у нас нет отрицательных собственных значений, и сигнатура квадратичной формы равна $(n^2, 0)$.

\addcontentsline{toc}{subsection}{Угловые миноры} \subsection*{Угловые миноры}

Для данной квадратичной формы на трехмерном вещественном пространстве, у которой угловые миноры равны $0$, $0$ и
$\alpha > 0$, положительный индекс инерции может быть равен $1$, а отрицательный индекс инерции равен $0$.

Индексы инерции квадратичной формы определяются количеством положительных и отрицательных собственных значений
матрицы квадратичной формы.
Так как у данной квадратичной формы имеется только одно ненулевое собственное значение, а его знак положителен, то
положительный индекс инерции равен 1, а отрицательный индекс инерции равен 0.

Таким образом, индексы инерции для данной квадратичной формы имеют вид $(1,0)$.

\addcontentsline{toc}{subsection}{Положительная определённость} \subsection*{Положительная определённость}

Рассмотрим пример, который покажет, что из положительной определенности двух ограничений пространства, являющейся
прямой сумма двух своих подпространств, не следует положительная определённость квадратичной формы.

Пусть $V = \mathbb{R}^2$ и $V_1 = \operatorname{span}\{(1,0)\}$, $V_2 = \operatorname{span}\{(0,1)\}$ - подпространства в $V$.
Рассмотрим два ограничения пространства $V$:

\[ f_1(x,y) = x, \quad f_2(x,y) = y \]

Проверим, что оба ограничения положительно определены:

\begin{gather*}
    a(x,y) = x > 0, \quad \forall (x,y) \in V_1, \quad (x,y) \neq (0,0)\\
    f_2(x,y) = y > 0, \quad \forall (x,y) \in V_2, \quad (x,y) \neq (0,0)\\
\end{gather*}

Теперь рассмотрим квадратичную форму $Q(x,y) = x^2 + y^2$ на $V$.
Можно заметить, что $Q(x,y)$ не является положительно определенной, так как она принимает отрицательные значения на
векторах $(x,y)$, не лежащих в $V_1$ или $V_2$.
Например, на векторе $(1,1)$:

\[ Q(1,1) = 1^2 + 1^2 = 2 > 0 \]

Таким образом, мы показали, что из положительной определенности двух ограничений пространства, являющейся прямой
сумма двух своих подпространств, не следует положительная определённость квадратичной формы.

\addcontentsline{toc}{subsection}{Кососимметричный определитель} \subsection*{Кососимметричный определитель}

Для того, чтобы доказать, что определитель целочисленной кососимметрической матрицы является квадратом целого числа,
мы можем использовать следующий факт:

Если $A$ -- кососимметрическая матрица, то $\det(A)$ является квадратом определителя матрицы $B$, где $B = iA$ и $i$
-- мнимая единица.

Итак, пусть $A$ -- целочисленная кососимметрическая матрица.
Тогда $B=iA$ также является кососимметрической матрицей.
Кроме того, элементы $B$ являются комплексными числами с мнимой частью, равной целому числу.

Таким образом, определитель $B$ является квадратом модуля его определителя.
Модуль комплексного числа с мнимой частью, равной целому числу, всегда является целым числом.

Следовательно, $\det(B)$ является квадратом целого числа.
Но $\det(B) = \det(iA) = i^n \det(A)$, где $n$ - порядок матрицы $A$.

Так как $A$ - кососимметрическая матрица, то $\det(A)$ является мнимым числом.
Поэтому $i^n\det(A)$ является квадратом целого числа.

Таким образом, мы доказали, что определитель целочисленной кососимметрической матрицы является квадратом целого числа

\addcontentsline{toc}{subsection}{Идемпотенция} \subsection*{Идемпотенция}

Самосопряженное преобразование -- это линейное преобразование, которое равно своему сопряженному.
Идемпотентное преобразование -- это линейное преобразование, которое при повторном применении к вектору даёт тот же
самый вектор.

Пусть $A$ -- матрица линейного преобразования.
Тогда самосопряжённость означает, что $A = A^*$, где $A^*$ -- сопряженная матрица.
Идемпотентность означает, что $A^2 = A$.

Рассмотрим матрицу $A$ размера $n \times n$.
Тогда $A$ самосопряжена, если $A = A^*$, то есть $a_{ij} = \overline{a_{ji}}$ для всех $i, j$.
Идемпотентность означает, что $A^2 = A$, то есть $A$ является проектором на некоторое подпространство
$V \subseteq \mathbb{C}^n$.

Пусть $A$ -- самосопряженный идемпотентный оператор.
Тогда $A^2 = A$ и $A = A^*$.
Рассмотрим собственные значения $\lambda$ матрицы $A$.
Так как $A$ идемпотентна, то $\lambda^2 = \lambda$, то есть $\lambda = 0$ или $\lambda = 1$.
Также, так как $A$ самосопряжена, то существует ортонормированный базис из собственных векторов матрицы $A$.
Пусть $V$ - подпространство, порожденное собственными векторами, соответствующими собственному значению $1$.
Тогда $A$ является проектором на $V$.

Таким образом, все самосопряженные идемпотентные операторы являются проекторами на некоторое подпространство
$V \subseteq \mathbb{C}^n$.
Обратно, любой проектор на подпространство $V$ является самосопряженным идемпотентным оператором.

\addcontentsline{toc}{subsection}{Общий ОНБ} \subsection*{Общий ОНБ}

Пусть $A$ и $B$ -- два самосопряженных оператора в евклидовом пространстве $V$.
Предположим, что у них есть общий ортонормированный базис из собственных векторов.
Тогда для любого вектора $v$ из этого базиса выполняется:
\begin{gather*}
    A = \lambda_v v,\\
    Bv = \mu_v v,\\
\end{gather*}
где $\lambda_v$ и $\mu_v$ -- собственные значения операторов $A$ и $B$ соответственно.
Таким образом, операторы $A$ и $B$ диагонализуемы и коммутируют в этом базисе:
\[ ABv = BA v = A(\mu_v v) = \mu_v Av = \mu_v \lambda_v v = \lambda_v Bv. \]
Обратно, предположим, что операторы $A$ и $B$ коммутируют.
Тогда они диагонализуемы в одном и том же ортонормированном базисе из собственных векторов.
Действительно, если $v_1, v_2, \dots, v_n$ -- ортонормированный базис из собственных векторов оператора $A$, то
каждый вектор $v_i$ является также собственным вектором оператора $B$, так как
\[ ABv_i = BA v_i = A(\mu_i v_i) = \mu_i Av_i = \mu_i \lambda_i v_i = \lambda_i Bv_i. \]
Таким образом, операторы $A$ и $B$ имеют общий ортонормированный базис из собственных векторов.

\addcontentsline{toc}{subsection}{Преобразование линейно} \subsection*{Преобразование линейно}

Пусть $f: V \to V$ -- отображение, сохраняющее скалярное произведение.
Тогда для любых векторов $u,v \in V$ выполняется
\[ \langle f(u), f(v) \rangle = \langle u,v \rangle. \]
Заметим, что это равенство можно переписать в виде
\[ \langle f(u+v), f(u+v) \rangle = \langle u+v,u+v \rangle. \]
Раскрывая скобки, получаем
\[ \langle f(u),f(u) \rangle + 2\langle f(u),f(v) \rangle + \langle f(v),f(v) \rangle = \langle u,u \rangle + 2\langle u,v \rangle + \langle v,v \rangle. \]
Так как $f$ сохраняет скалярное произведение, то $\langle f(u),f(u) \rangle = \langle u,u \rangle$ и $\langle f(v),f(v) \rangle = \langle v,v \rangle$.
Поэтому
\[ 2\langle f(u),f(v) \rangle = 2\langle u,v \rangle. \]
Таким образом, для любых векторов $u,v \in V$ выполнено $\langle f(u),f(v) \rangle = \langle u,v \rangle$, что
означает, что $f$ сохраняет углы и длины векторов.
Поэтому $f$ является изометрией евклидова пространства $V$.

Для доказательства линейности отображения $f$ достаточно заметить, что из сохранения скалярного произведения следует
линейность векторного отображения $f$.
Действительно, для любых векторов $u,v \in V$ и чисел $\alpha, \beta \in \mathbb{R}$ имеем
\[ \langle f(\alpha u + \beta v),f(\alpha u + \beta v) \rangle = \langle \alpha u + \beta v, \alpha u + \beta v \rangle. \]
Раскрывая скобки, получаем
\[ \alpha^2 \langle f(u),f(u) \rangle + 2\alpha\beta \langle f(u),f(v) \rangle + \beta^2 \langle f(v),f(v) \rangle = \alpha^2 \langle u,u \rangle + 2\alpha\beta \langle u,v \rangle + \beta^2 \langle v,v \rangle. \]
Так как $f$ сохраняет скалярное произведение, то $\langle f(u),f(u) \rangle = \langle u,u \rangle$ и $\langle f(v),f(v) \rangle = \langle v,v \rangle$.
Поэтому
\[ 2\alpha\beta \langle f(u),f(v) \rangle = 2\alpha\beta \langle u,v \rangle. \]
Таким образом, для любых векторов $u,v \in V$ выполнено $\langle f(\alpha u + \beta v),f(\alpha u + \beta v) \rangle = \langle \alpha u + \beta v, \alpha u + \beta v \rangle$, что
означает, что $f$ сохраняет углы и длины векторов.
Поэтому $f$ является изометрией евклидова пространства $V$, а изометрия линейна.

\addcontentsline{toc}{subsection}{Чётный ранг} \subsection*{Чётный ранг}

Пусть $V$ -- евклидово пространство, на котором задан оператор линейного преобразования $\varphi$, такой что $\varphi(v) \perp v$ для
любого вектора $v \in V$.
Нам нужно доказать, что ранг $\varphi$ является четным числом.

Заметим, что $\varphi(v) \perp v$ означает, что $\varphi(v)$ лежит в ортогональном дополнении к пространству, порожденному вектором $v$.
Таким образом, $\varphi(v)$ ортогонально любому вектору, лежащему в этом пространстве.

Рассмотрим произвольный вектор $v \in V$ и его ортогональное дополнение $W = \{w \in V \mid w \perp v\}$.
Так как $\varphi(v) \in W$ для любого $v \in V$, то $\varphi(W) \subseteq W$.
Более того, $\varphi(W)$ также ортогонально любому вектору из $W$, так как $\varphi(w) \perp w$ для любого $w \in W$.

Рассмотрим теперь ограничение $\varphi$ на $W$.
Так как $\varphi(W) \subseteq W$, то ограничение $\varphi|_W$ является линейным оператором на $W$.
Кроме того, $\varphi(w) \perp w$ для любого $w \in W$, поэтому $\varphi|_W$ является самосопряженным оператором на $W$.

Так как $\varphi|_W$ самосопряжен, то существует ортонормированный базис $e_1, \dots, e_k$ в $W$, состоящий из
собственных векторов $\varphi|_W$, где $k$ -- ранг $\varphi|_W$.
Так как $\varphi(W) \subseteq W$, то любой собственный вектор $\varphi|_W$ также является собственным вектором $\varphi$.

Расширим базис $e_1, \dots, e_k$ до ортонормированного базиса $e_1, \dots, e_n$ в $V$.
Тогда матрица оператора $\varphi$ в этом базисе имеет вид:

\[ [\varphi] =
\begin{pmatrix}
    \lambda_1 I_k & 0 \\
    0             & 0
\end{pmatrix}, \]

где $\lambda_1, \dots, \lambda_k$ -- собственные значения $\varphi|_W$, $I_k$ -- единичная матрица размера $k \times
k$, а $0$ -- нулевая матрица размера $(n-k) \times k$ и $0$ размера $(n-k) \times (n-k)$.

Таким образом, ранг матрицы $[\varphi]$ равен $k$, что является четным числом.
Следовательно, ранг оператора $\varphi$ также является четным числом.