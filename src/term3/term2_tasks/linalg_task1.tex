\addcontentsline{toc}{subsection}{Плотность матричного множества} \subsection*{Плотность матричного множества}

Для начала, давайте определим, что значит \("\)множество диагонализируемых квадратных матриц плотно над полем
комплексных чисел\("\).
Это означает, что любая квадратная матрица $A$ размера $n \times n$ с элементами из поля комплексных чисел может
быть приближена произвольно близко другой диагональной матрицей $D$ размера $n \times n$ с элементами из того же
поля, то есть существует последовательность диагонализируемых матриц $A_k$ таких, что $\lim_{k \to \infty} A_k = D$.

Для доказательства этого факта можно воспользоваться теоремой о жордановой нормальной форме.
Эта теорема утверждает, что любая квадратная матрица $A$ размера $n \times n$ с элементами из поля комплексных
чисел подобна матрице вида $J = \mathrm{diag}(J_1, J_2, \ldots, J_k)$, где каждый блок $J_i$ имеет вид
\[
    J_i = \begin{pmatrix}
              \lambda_i & 1         & 0         & \cdots & 0         \\
              0         & \lambda_i & 1         & \cdots & 0         \\
              0         & 0         & \lambda_i & \cdots & 0         \\
              \vdots    & \vdots    & \vdots    & \ddots & \vdots    \\
              0         & 0         & 0         & \cdots & \lambda_i
    \end{pmatrix},
\]
где $\lambda_i$ -- собственное значение матрицы $A$, а размерность блока $J_i$ равна количеству жордановых клеток,
соответствующих этому собственному значению.

Теперь, если мы хотим приблизить матрицу $A$ диагональной матрицей $D$, мы можем заменить каждый блок $J_i$
матрицы $J$ на диагональную матрицу $\mathrm{diag}(\lambda_i, \lambda_i, \ldots, \lambda_i)$, что даст нам диагональную матрицу $D'$.
Однако, если мы теперь рассмотрим матрицу $A'$, которая получается из матрицы $A$ заменой блоков $J_i$ на
соответствующие диагональные матрицы, то матрицы $A'$ и $D'$ не обязательно будут подобны, и мы не можем
гарантировать, что последовательность $A_k$ сходится к $D$.

Однако, мы можем заменить каждый блок $J_i$ матрицы $J$ на матрицу вида
\[
    J_i' = \begin{pmatrix}
               \lambda_i & \epsilon_i & 0          & \cdots & 0         \\
               0         & \lambda_i  & \epsilon_i & \cdots & 0         \\
               0         & 0          & \lambda_i  & \cdots & 0         \\
               \vdots    & \vdots     & \vdots     & \ddots & \vdots    \\
               0         & 0          & 0          & \cdots & \lambda_i
    \end{pmatrix},
\]
где $\epsilon_i$ -- произвольно малое комплексное число, и затем рассмотреть матрицу $A''$, которая получается из
матрицы $A$ заменой блоков $J_i$ на соответствующие матрицы $J_i'$.
В этом случае матрицы $A''$ и $D'$ будут подобны, и мы можем гарантировать, что последовательность $A_k$ сходится
к $D$ при $\epsilon_i \to 0$ для всех $i = 1, \ldots, k$.

Таким образом, мы доказали, что множество диагонализируемых квадратных матриц плотно над полем комплексных чисел


\addcontentsline{toc}{subsection}{Многочлены} \subsection*{Многочлены}
\textit{Докажите, что характеристический многочлен линейного преобразования делится на характеристический
многочлен его ограничения на инвариантном подпространстве}

Это утверждение называется \("\)основной теоремой о блочном виде матрицы линейного оператора\("\).
Пусть $V$ -- векторное пространство над полем $F$, $L: V \rightarrow V$ -- линейный оператор, $U$ -- инвариантное
подпространство пространства $V$, т.е. $L(U) \subseteq U$.
Обозначим через $L_U$ ограничение оператора $L$ на подпространство $U$.

Тогда существуют такие базисы $e_1, \ldots, e_k$ в $U$ и $f_1, \ldots, f_m$ в $V$, что матрица оператора $L$ в
базисе $e_1, \ldots, e_k, f_1, \ldots, f_m$ имеет блочно-диагональную форму:
\[
    \begin{pmatrix}
        A_1 & 0   \\
        0   & A_2
    \end{pmatrix},
\]
где $A_1$ -- матрица оператора $L_U$ в базисе $e_1, \ldots, e_k$, а $A_2$ -- матрица оператора $L_{U^\perp}$ в
базисе $f_1, \ldots, f_m$.

Теперь заметим, что характеристический многочлен матрицы оператора $L$ равен произведению характеристических
многочленов матриц $A_1$ и $A_2$, т.е.
\[ \chi_L(t) = \chi_{L_U}(t) \cdot \chi_{L_{U^\perp}}(t). \]
Это следует из того, что характеристический многочлен матрицы блочно-диагональной формы равен произведению
характеристических многочленов диагональных блоков.

Таким образом, мы доказали, что характеристический многочлен линейного оператора $L$ делится на
характеристический многочлен его ограничения $L_U$ на инвариантном подпространстве $U$.

\addcontentsline{toc}{subsection}{Поворот на угол} \subsection*{Поворот на угол}

\textit{Найти подпространства трёхмерного геометрического пространства, инвариантные относительно поворота на
ненулевой угол $\alpha$ вокруг прямой $x = ta$}

Пусть $V$ -- трёхмерное геометрическое пространство, $L$ -- прямая в $V$ с направляющим вектором $a$.
Тогда оператор поворота $R_{\alpha}$ вокруг прямой $L$ на угол $\alpha$ можно выразить через матрицу поворота
вокруг оси $z$ в базисе, связанном с прямой $L$.
Для этого нужно выбрать ортонормированный базис $e_1, e_2, e_3$, где $e_3 = a$, и заменить матрицу поворота в
базисе $e_1, e_2, e_3$ на матрицу поворота в базисе $e_1', e_2', e_3$, где $e_1' = e_1$, $e_2' = e_2$,
$e_3' = R_{\alpha}e_3$.
Матрица перехода между этими базисами имеет вид:

\[
    S = \begin{pmatrix}
            \cos\alpha & -\sin\alpha & 0 \\
            \sin\alpha & \cos\alpha  & 0 \\
            0          & 0           & 1
    \end{pmatrix}.
\]

Теперь пусть $U$ -- подпространство $V$, инвариантное относительно поворота $R_{\alpha}$.
Тогда любой вектор $v \in U$ может быть представлен в виде $v = x + ty$, где $x \in L$, $y \in L^{\perp}$,
$t \in \mathbb{R}$.
Заметим, что $R_{\alpha}x = x$, так как $x$ лежит в прямой $L$, инвариантной относительно поворота.
Кроме того, $R_{\alpha}y \in L^{\perp}$, так как прямая $L^{\perp}$ ортогональна к направлению поворота.
Следовательно, $R_{\alpha}v = R_{\alpha}(x + ty) = x + tR_{\alpha}y \in U$.
Таким образом, $U$ содержит все векторы вида $x + ty$, где $x \in L$ и $y \in L^{\perp}$, и является инвариантным
относительно поворота $R_{\alpha}$.

Таким образом, мы получили, что любое подпространство $U$ в $V$, инвариантное относительно поворота на ненулевой
угол $\alpha$ вокруг прямой $L$, является прямой $L$ вместе с её ортогональным дополнением $L^{\perp}$.

\addcontentsline{toc}{subsection}{Подпространства} \subsection*{Подпространства}

Пусть дано линейное преобразование $T: V \rightarrow V$ над линейным пространством $V$ размерности $n$ с $n$
попарно различными собственными значениями $\lambda_1, \lambda_2, \dots, \lambda_n$.
Для любого собственного значения $\lambda_i$ рассмотрим его собственное подпространство $V_i = \{v \in V: T(v) = \lambda_i v\}$.
Так как собственные значения попарно различны, то собственные подпространства линейно независимы.
Кроме того, сумма всех собственных подпространств равна всему пространству $V$, так как каждый вектор $v \in V$
может быть разложен в сумму векторов из собственных подпространств, соответствующих попарно различным собственным
значениям.
Следовательно, мы получили разложение пространства $V$ в прямую сумму собственных подпространств:

\[V = V_1 \oplus V_2 \oplus \dots \oplus V_n.\]

Каждое из подпространств $V_i$ инвариантно относительно $T$, так как для любого вектора $v \in V_i$ выполнено $T(
v) = \lambda_i v \in V_i$.
Количество инвариантных подпространств, как мы видим, равно числу всех возможных комбинаций прямых сумм
собственных подпространств, то есть $2^n - 1$, так как каждое из $n$ собственных подпространств может входить или
не входить в данную прямую сумму.


\addcontentsline{toc}{subsection}{Инварианты} \subsection*{Инварианты}

Найти все инвариантные подпространства оператора, матрица которого в некотором базисе равна жордановой клетке

Для того чтобы найти все инвариантные подпространства оператора, матрица которого в некотором базисе равна
жордановой клетке, нужно сначала найти жорданов базис для этой матрицы.

Жорданов базис - это базис, в котором матрица оператора имеет жорданову форму.
Жорданова форма матрицы является блочно-диагональной матрицей, где каждый блок - это жорданова клетка.
Каждая жорданова клетка соответствует одному из собственных значений оператора и содержит на главной диагонали
это собственное значение, а также единицы на диагонали над главной диагональю.

Как только вы найдете жорданов базис для матрицы оператора, вы можете определить все инвариантные подпространства, их
размерности, а также базисы для каждого подпространства.

Для примера, давайте рассмотрим жорданову клетку размера $3 \times 3$ с собственным значением $\lambda$.
Его жорданов базис состоит из трех векторов: $v_1$, $v_2$ и $v_3$.
Жорданова клетка для этого собственного значения имеет следующий вид:

\[
    \begin{pmatrix}
        \lambda & 1       & 0       \\
        0       & \lambda & 1       \\
        0       & 0       & \lambda \\
    \end{pmatrix}
\]

Для определения инвариантных подпространств можно рассмотреть подпространства, порожденные комбинациями этих
векторов.
Возможны следующие случаи:
\begin{enumerate}
    \item Подпространство, порожденное одним вектором $v_i$ (размерность 1).
    Это подпространство инвариантно относительно оператора, так как $Av_i = \lambda v_i + w$, где $w$ - линейная
    комбинация $v_1$, $v_2$ и $v_3$ с коэффициентами, не равными нулю.
    Таким образом, $Av_i$ принадлежит тому же подпространству, что и $v_i$.
    \item Подпространство, порожденное двумя векторами $v_i$ и $v_j$ (размерность 2).
    Это подпространство инвариантно
    относительно оператора, если $v_i$ и $v_j$ являются собственными векторами для одного и того же собственного
    значения, то есть $\lambda_i = \lambda_j$.
    В этом случае $Av_i = \lambda_i v_i + w_1$ и $Av_j = \lambda_j v_j + w_2$, где $w_1$ и $w_2$ - линейные
    комбинации $v_1$, $v_2$ и $v_3$ с коэффициентами, не равными нулю.
    Тогда $A(v_i + v_j) = (\lambda_i + \lambda_j)(v_i + v_j) + (w_1 + w_2)$, что означает, что $v_i + v_j$ также
    является собственным вектором для собственного значения $\lambda_i + \lambda_j$.
    \item Подпространство, порожденное всеми тремя векторами $v_1$, $v_2$ и $v_3$ (размерность 3).
    Это подпространство
    инвариантно относительно оператора, так как $A(v_1 + v_2 + v_3) = \lambda_1 v_1 + \lambda_2 v_2 + \lambda_3 v_3 +
    w$, где $w$ -- линейная комбинация $v_1$, $v_2$ и $v_3$ с коэффициентами, не равными нулю.
    Таким образом, $v_1 + v_2 + v_3$ также является собственным вектором для собственного значения $\lambda_1 + \lambda_2 + \lambda_3$.
\end{enumerate}

Таким образом, мы нашли все три инвариантных подпространства для данной жордановой клетки.
Аналогично можно определить инвариантные подпространства для матрицы оператора, которая имеет жорданову форму в
некотором базисе.

\addcontentsline{toc}{subsection}{Перестановки} \subsection*{Перестановки}

Для того, чтобы доказать, что два перестановочных линейных преобразования комплексного пространства имеют общий
собственный вектор, мы можем воспользоваться следующим фактом:

Если два линейных преобразования перестановочные, то они коммутируют с любым многочленом от них.
В частности, они коммутируют с минимальным многочленом каждого из них.

Допустим, что у нас есть два перестановочных линейных преобразования A и B комплексного пространства V. Пусть λ -
собственное значение A, и v - соответствующий ему собственный вектор, то есть Av = λv.
Тогда, поскольку A и B перестановочные, мы можем записать:

ABv = BA v = B(λv) = λBv

Таким образом, Bv также является собственным вектором A, соответствующим собственному значению λ.
Если λ не является собственным значением B, то мы можем повторить этот процесс, используя минимальный многочлен B
вместо A. Таким образом, мы получим общий собственный вектор для $A$ и $B$.

Таким образом, мы доказали, что два перестановочных линейных преобразования комплексного пространства имеют общий
собственный вектор.

\addcontentsline{toc}{subsection}{Теорема Гамильтона Кэли}
\subsection*{Теорема Гамильтона Кэли}

Теорема Гамильтона-Кэли утверждает, что любая матрица $A$ удовлетворяет своему характеристическому уравнению:

\[ \det(\lambda I - A) = 0, \]

где $I$ - единичная матрица, а $\lambda$ - собственное значение матрицы $A$.

Для доказательства этой теоремы воспользуемся фактом, что множество диагонализируемых матриц плотно над полем
комплексных чисел.
Это означает, что любая матрица $A$ может быть приближена диагональной матрицей $D$ с любой
точностью.
То есть существует последовательность диагонализируемых матриц $A_n$, которые сходятся к матрице $A$
при $n \rightarrow \infty$.
При этом, каждая матрица $A_n$ имеет собственные значения $\lambda_{1,n}, \lambda_{2,n}, \dots, \lambda_{m,n}$ и
соответствующие им собственные векторы $v_{1,n}, v_{2,n}, \dots, v_{m,n}$.

Рассмотрим характеристическое уравнение для матрицы $A_n$:

\[ \det(\lambda I - A_n) = (\lambda - \lambda_{1,n})(\lambda - \lambda_{2,n}) \dots (\lambda - \lambda_{m,n}). \]

Так как матрица $A_n$ диагонализируема, то существует невырожденная матрица $P_n$, такая что:

\[ A_n = P_n D_n P_n^{-1}, \]

где $D_n$ -- диагональная матрица, элементы на диагонали которой равны собственным значениям матрицы $A_n$.
Подставим это выражение в характеристическое уравнение:

\[ \det(\lambda I - A_n) = \det(\lambda I - P_n D_n P_n^{-1}) = \det(P_n(\lambda I - D_n)P_n^{-1}) =
\det(\lambda I - D_n). \]

Таким образом, характеристическое уравнение для матрицы $A_n$ сводится к уравнению для диагональной матрицы $D_n$.
Поскольку последовательность матриц $A_n$ сходится к матрице $A$, то последовательность диагональных матриц $D_n$
также сходится к диагональной матрице $D$, элементы на диагонали которой равны собственным значениям матрицы $A$.
Таким образом, характеристическое уравнение для матрицы $A$ равно:

\[ \det(\lambda I - A) = \lim_{n \rightarrow \infty} \det(\lambda I - A_n) = \lim_{n \rightarrow \infty} (\lambda -
\lambda_{1,n})(\lambda - \lambda_{2,n}) \dots (\lambda - \lambda_{m,n}). \]

Так как каждая матрица $A_n$ имеет собственные значения, то и матрица $A$ также имеет собственные значения.
Следовательно, характеристическое уравнение для матрицы $A$ имеет вид:

\[ \det(\lambda I - A) = (\lambda - \lambda_1)(\lambda - \lambda_2) \dots (\lambda - \lambda_m), \]

где $\lambda_1, \lambda_2, \dots, \lambda_m$ -- собственные значения матрицы $A$.
Следовательно, теорема Гамильтона-Кэли доказана.


\subsection*{Оператор трёхкратного дифференцирования} \addcontentsline{toc}{subsection}{Оператор трёхкратного
дифференцирования}

Для нахождения жордановой нормальной формы (ЖНФ) оператора трехкратного дифференцирования в пространстве
вещественных многочленов степени не выше 9, нужно выполнить следующие шаги:

\begin{enumerate}

    \item Найти все собственные значения оператора.
    Для этого решим характеристическое уравнение:

    \[
        \det(\lambda I - A) = \det \begin{pmatrix}
                                       \lambda & 0       & 0       & 0       & 0       & 0       & 0       & 0       & 0       \\
                                       0       & \lambda & 0       & 0       & 0       & 0       & 0       & 0       & 0       \\
                                       0       & 0       & \lambda & 0       & 0       & 0       & 0       & 0       & 0       \\
                                       0       & 0       & 0       & \lambda & 0       & 0       & 0       & 0       & 0       \\
                                       0       & 0       & 0       & 0       & \lambda & 0       & 0       & 0       & 0       \\
                                       0       & 0       & 0       & 0       & 0       & \lambda & 0       & 0       & 0       \\
                                       0       & 0       & 0       & 0       & 0       & 0       & \lambda & 0       & 0       \\
                                       0       & 0       & 0       & 0       & 0       & 0       & 0       & \lambda & 0       \\
                                       0       & 0       & 0       & 0       & 0       & 0       & 0       & 0       & \lambda \\
        \end{pmatrix} = \lambda^9.
    \]

    Отсюда видно, что у оператора есть только одно собственное значение - ноль.
    \item Найти размерности жордановых клеток для каждого собственного значения.
    Так как у нас только одно
    собственное значение, то достаточно найти размерность жордановых клеток для нулевого собственного значения.
    Для этого нужно найти ядро оператора $(A - \lambda I)^k$, где $\lambda = 0$ и $k$ - порядок клетки.
    Начнем с $k=1$:

    \[ (A - \lambda I)^1 = A - \mathbf{0} = A. \]

    Чтобы найти ядро оператора $A$, нужно решить систему уравнений:

    \[ Ax = \mathbf{0}. \]

    Пусть $x = a_9 x^9 + a_8 x^8 + \dots + a_1 x + a_0$, тогда

    \[ Ax = a_9 \frac{d^3}{dx^3} x^9 + a_8 \frac{d^3}{dx^3} x^8 + \dots + a_1 \frac{d^3}{dx^3} x + a_0 \frac{d^3}{dx^3}
    (1) = a_9 \cdot 84\cdot70\cdot56\cdot x^6 + \dots + a_1 \cdot6\cdot2\cdot x = \mathbf{0}. \]

    Отсюда следует, что $a_9 = a_8 = \dots = a_1 = a_0 = 0$, так как каждый множитель в выражении для $Ax$ содержит
    производную третьего порядка, а значит не может быть равен нулю для всех $x$.
    Следовательно, размерность жордановой клетки для собственного значения ноль равна единице
    \item Найти жорданов базис.
    Жорданов базис строится на основе жордановых клеток.
    Для каждой жордановой клетки
    размерности $m$ строится матрица $J_m$, которая имеет вид:

    \[
        J_m = \begin{pmatrix}
                  \lambda & 1      & \cdots & \cdots & \cdots & \cdots & \cdots & \cdots & \cdots  \\
                  \cdots  & \cdots & \cdots & \cdots & \cdots & \cdots & \cdots & \cdots & \cdots  \\
                  \cdots  & \cdots & \cdots & \cdots & \cdots & \cdots & \cdots & \cdots & \cdots  \\
                  \cdots  & \cdots & \cdots & \cdots & \cdots & \cdots & \cdots & \cdots & 1       \\
                  \cdots  & \cdots & \cdots & \cdots & \cdots & \cdots & \cdots & \cdots & \lambda \\
        \end{pmatrix},
    \]

    где $\lambda$ - собственное значение.
    Жорданов базис составляется из жордановых клеток путем объединения столбцов матрицы $J_m$.
    В нашем случае жорданов базис будет состоять из единственного многочлена $x^8$.
    \item Найти минимальный многочлен оператора.
    Минимальный многочлен оператора - это многочлен наименьшей степени, который обнуляет оператор.
    Так как у нашего оператора только одно собственное значение - ноль, то минимальный многочлен должен быть
    делителем многочлена $\lambda^9$.
    Также известно, что минимальный многочлен должен иметь такие же неприводимые множители, как и
    характеристический многочлен.
    Следовательно, минимальный многочлен оператора равен $\lambda^k$, где $k$ -- порядок наибольшей жордановой
    клетки.
    В нашем случае $k=1$, поэтому минимальный многочлен равен $\lambda$.
\end{enumerate}