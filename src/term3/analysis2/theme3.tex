\subsection{Теорема о неявной функции для одного уравнения}

\textbf{Опр} \textit{Неявная скалярная функция}
\textcolor{gray}{$F(x, y) = 0$, но $y = y(x)$ неизвестен}

\textbf{Th} \textcolor{blue}{Если выполнены 4 условия ($F (x_0, y_0) = 0$ и непрерывна в окрестности, $F^{'}_y (x, y)
~\exists$ в окрестности, непрерывна в точке и ненулевая), то $\exists~\varphi(x): F(x^*, y) = 0$ имеет решение в виде $y^* = \varphi (x^*)$}

\begin{enumerate}
    \item Будем полагать $F^{'}_y (x, y) > 0$.
    Тогда $\exists \varepsilon_1: F^{'}_y (x, y) > 0~\forall (x, y) \in U_{\varepsilon_1}(x_0, y_0)$
    \item Перейдём от круглой окрестности к квадратной путём взятия $\delta = \frac{\varepsilon_1}{\sqrt{2}}$ или
    меньшего числа
    \item В силу $\varepsilon_1: F^{'}_y (x, y) > 0, F$ будет строго возрастать по $y$ на $[y_0 - \delta, y_0 + \delta]$.
    Тогда $F(x_0, y - \delta) < 0, F(x_0, y + \delta) > 0$
    \item В силу непрерывности $F$ в $U_{\varepsilon_1}(x_0, y_0)~\exists \gamma \in [0, \delta): \forall x \in U_{\gamma}(x_0)~F(x, y_0 - \delta) < 0, F(x, y_0 + \delta) > 0$
    \item Применяем теорему о промежуточном значении $\forall x \in U_{\gamma}(x_0)$ и получаем единственность
    решения в силу непрерывности $F$
    \item Вспоминая про п.3, вдобавок получаем непрерывность полученной функции в $x_0$
\end{enumerate}

\subsection{Теорема Лагранжа о среднем для вектор-функции нескольких переменных}

\textbf{Опр} \textit{Операторная норма матрицы}
\textcolor{gray}{Максимальная длина вектора, полученная умножением матрицы на единичный вектор} \\

Максимум существует в силу непрерывности $f(x) = Ax$ и компактности множества векторов единичной длины

Введённая норма удовлетворяет всем 4-м аксиомам нормы (неотрицательности, существование нуля, линейность и
неравенство треугольника).
Первые три очевидны, а последнее доказывается через неравенство треугольника для чисел (длин векторов) $Ax$ \\

\textbf{Лемма} \textit{Об операторной норме}
\textcolor{gray}{$\abs{Ax} \leq \|A\| \abs{x}$ и $\|AB\| \leq \|A\| \|B\|$ для совместных матриц}

\begin{enumerate}
    \item Рассмотрим тривиальный случай $x = 0$ и нормировку в общем случае
    \item Два раза используем уже доказанную часть леммы для любого вектора, доказывая нужное утверждение
\end{enumerate}

\textbf{Th} \textit{Лагранжа о среднем для вектор-функции}
\textcolor{gray}{По аналогии с обычной теоремой, но здесь имеем неравенство и матрицу Якоби}

\begin{enumerate}
    \item Рассмотрим функцию $f(t) = g(y + t(y^{'} - y))$ и продифференцируем её как \underline{сложную} функцию
    \item Запишем $\abs{g(y^{'}) - g(y)}$ в терминах функции $f(t)$ и применим обычную теорему Лагранжа о среднем
    \item Воспользуемся леммой об операторной норме матрицы и получим требуемое выражение
\end{enumerate}

\subsection{Принцип Банаха сжимающих отображений}

\textbf{Опр} \textit{Сжимающее отображение}
\textcolor{gray}{Расстояние между значениями фукнции меньше расстояния между аргументами}

\textbf{Th} \textit{Принцип Банаха сжимающий отображений}
\textcolor{gray}{Уравнение $y = g(y)$ имеет решение в виде предела последовательности, заданной формулой $y_{k+1} = g
(y_k)$}

\begin{enumerate}
    \item Применим определение сжимающего отображения для $\varrho(y_{k+1},y_k)~k$ раз (получим неравенство)
    \item Далее преобразуем неравенство, последовательно используя неравенство треугольника, определение сжимающего
    отображения, сумму геометрической прогрессии и отбрасывание вычитаемого
    \item $\exists N:$ полученное выражение $< \varepsilon $, что означает фундаментальность ($\Rightarrow$
    сходимость) последовательности в силу полноты пространства
    \item В силу непрерывности отображения (следует из сжимаемости) в пределе получим требуемое равенство
    \item Единственность доказывается от противного, используя определение сжимающего отображения
\end{enumerate}

\subsection{Теорема о неявной функции для системы уравнений}

Смотреть в рукописном конспекте

\subsection{Теорема об обратном отображении}

\textbf{Л1} \textit{Прообраз открытого образа непрерывной функции открыт}

\begin{enumerate}
    \item Выберем произвольную точку из прообраза и запишем для неё определение открытости \underline{образа}
    \item Применим определение непрерывности функции для основного множества, получим $\delta-$окрестность,
    которая под действием функции будет полностью переходить в $\varepsilon-$окрестность
    \item В силу произвольности точки утверждение доказана для всего множества значений
\end{enumerate}

\textbf{Опр} \textit{Класс $k$ раз непрерывно дифференцируемых отображений}

\textbf{Опр} \textit{Класс бесконечно дифференцируемых отображений}
\textcolor{gray}{Пересечение по натуральным числам}

\textbf{Опр} \textit{Гладкие и $C^k$ гладкие отображения} \textcolor{gray}{Элементы соответсвующих классов}

\textbf{Опр} \textit{Гладкие и $C^k$ гладкие диффеоморфизмы}
\textcolor{gray}{ВОО, причём оно и его обратное есть $C^k$ гладкие отображения}

\textbf{Л2} \textcolor{blue}{$C^1$ гладкий диффеоморфизм имеет одинаковую размерность прообраза и образа, а также его
матрица Якоби не вырождена} \\

Дифференцируем оба тождества в определении обратной функции, чтобы получить два нестрогих неравенство, дающих
вместе равенство и пользуемся
\begin{enumerate}
    \item тем, что ранг произведения матриц не превосходит ранга каждого сомножителя
    \item значением ранга единичной матрицы
\end{enumerate}

В конце вспоминаем критерий обратимости \\

\textbf{Опр} \textit{Якобиан}
\textcolor{gray}{Определитель матрицы Якоби функции с совпадающими размерностями векторов прообраза и образа} \\

Матрица Якоби обратного отображения равна обратной матрице к матрице Якоби исходного отображения, а значит, якобиан
обратного отображения равен обратной величине к якобиану исходного отображения \\

\textbf{Опр} \textit{Сужение (ограничение) отображения} \textcolor{gray}{Отображение на подмножестве}

\textbf{Опр} \textit{Окрестность точки в метрическом пространстве} \textcolor{gray}{Произвольное открытое множество,
    содержащее точку}

\textbf{Th.1} \textit{Об обратном отображении} \textcolor{blue}{$C^1$ гладкое отображение с неравным нулю Якобианом
локально обратимо (то есть её сужение является $C^1$ гладким диффеоморфизмом)}

\begin{enumerate}
    \item Обозначим $x_0 = g(y_0)$ и применим теорему о неявной функции к $F (x, y) = g(y) - x$ (можно показать, что
    все условия теоремы выполнены)
    \item Рассмотрим сужение на $U(y_0)$, полученном из теоремы о неявной функции, которое будет ВОО.
    Полученный $y = \varphi (x)$ является единственным решением уравнения $F (x, y) = \overline{0}$ относительно $y$, то есть мы получили обратное отображение
    \item В конце пользуемся Л1 и тем, что по теореме о неявной функции обратная функция будет непрерывно
    дифференцируема
\end{enumerate}

Заметим, что наше отображение может и не быть глобально обратимым (в примеры годится вектор-функция с периодическими
функциями) \\

\textbf{Th.2} \textcolor{blue}{$C^k$ гладкое отображение с неравным нулю Якобианом
является $C^k$ гладким диффеоморфизмом}

\begin{enumerate}
    \item Базовый случай доказан в предыдущей теореме.
    Иначе рассмотрим матрицу Якоби обратной функции поэлементно, ведь каждый её элемент будет выражаться как многочлен
    относительно элементов исходной матрицы, делённый на её определитель
    \item Используя невырожденность матрицы, получаем, что все элементы обратной являются $C^{k-1}$ гладкими
    функциями от $x$
    \item В таком случае сама обратная функция является $C^k$ гладкой
\end{enumerate}
Аналогичный результат можно получить и для $C^{\infty}$ гладких функций

\subsection{Теорема о расщеплении отображений}

\textbf{Опр} \textit{Отображения, меняющие и не меняющие $i-$ю координату}

\textbf{Th} \textit{О расщеплении отображений} \\
\textcolor{blue}{$C^k$ гладкое отображение с неравным нулю Якобианом представимо в виде суперпозиции $C^k$ гладких
диффеоморфизмов, каждый из которых меняет только одну координату, и линейного диффеоморфизма меняющего только порядок
координат}

\begin{enumerate}
    \item В силу невырожденности матрицы в первом столбце найдётся отличный от нуля элемент
    \item Рассмотрим отображение, которое меняет  только  первую  координату, и его матрицу Якоби.
    Она невырождена, поэтому в силу теоремы об обратном отображении в некоторой окрестности точки $x_0$ отображение $
    g_1$ является диффеоморфизмом, а согласно следующей теореме, является $C^k$ гладким диффеоморфизмом
    \item Тогда исходное отображение является суперпозицией отображения, которое меняет местами первую и $k-$ю
    компоненты (или является тождественным отображением в случае $k = 1$) и отображения, переводящего \("\)иксы\("\) в \("\)игреки\("\)
    \item Таким образом, наше отображение представимо в виде трёх $C^k$ гладких диффеоморфизмов, меняющий порядок
    координат, меняющий только первую и не меняющий первую координату
    \item Применяя аналогичные рассуждения к последнему диффеоморфизму, получим требуемое утверждение.
    Далее по индукции
\end{enumerate}

Результат этой теоремы будет использован далее при доказательстве теоремы о замене переменных в кратном интеграле

