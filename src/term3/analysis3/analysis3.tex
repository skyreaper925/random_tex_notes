%! Author = user
%! Date = 28.12.2023

\documentclass[a4paper, 14pt]{article}
%\documentclass[draft]{article}

\usepackage[T2A]{fontenc}
\usepackage[utf8]{inputenc}
\usepackage[english, russian]{babel}
\usepackage[top = 2cm, bottom = 2cm, left = 2cm, right = 2cm]{geometry}
\usepackage{indentfirst}
\usepackage{xcolor}
\usepackage{hyperref}
\usepackage{gensymb}
\usepackage{pgfplots}
\usepackage{amsmath, amsfonts, amsthm, mathtools}
\usepackage{amssymb}
\usepackage{physics, multirow, float}
\usepackage{wrapfig, tabularx}
\usepackage{icomma} % Clever comma: 0,2 - number while 0, 2 - two numbers
\usepackage{tikz, standalone}
\usepackage{fancyhdr,fancybox}
\usepackage{lastpage}
\usepackage{booktabs}
\usepackage{listings}
\usepackage{lstmisc}
\usepackage{stmaryrd}

%\полуторный интервал
\onehalfspacing

\hypersetup
{   colorlinks = false,
    linkcolor = blue,
    pdftitle = {analysis},
    pdfauthor = {Володин Максим},
    allcolors = [RGB]{010 090 200}
}

%\gravarphicspath{{images/}}
%\DeclareGravarphicsExtensions{.pdf,.png,.jpg}

\restylefloat{table}
\usetikzlibrary{external}

\mathtoolsset{showonlyrefs = true} % Numbers will appear only where \eqref{} in the text LINKED
\pagestyle{fancy}

\fancyhf{}
\fancyhead[L]{КИТП}
\fancyhead[R]{Конспект билетов}
\fancyfoot[L]{}
\fancyfoot[R]{\thepage /\pageref{LastPage}}


\pgfplotsset{compat=1.18}

\begin{document}
    
    \tableofcontents \newpage
    
    \section{Теорема о выражении меры множества через интеграл от меры сечений.
    Теорема Фубини}
    
    \subsection{Теорема о выражении меры множества через интеграл от меры сечений}
    
    \textbf{Л1}
    
    \textcolor{blue}{Пусть есть счётный набор конечно измеримых убывающих вложенных множеств $X_i$.
    Мера множества, являющегося счётным пересечением есть предел мер}
    
    \textbf{Л2}
    
    \textcolor{blue}{Если интеграл неотрицательной функции по множеству равен нулю, то сама функция равна нулю почти
    на всём множестве}
    
    \textbf{Th} \textit{О выражении меры множества через интеграл от меры сечений}
    
    \begin{enumerate}
        \item Для начала докажем для клетки.
        Её сечение будет принимать простой вид, в зависимости от принадлежности $x$, что даёт простое
        интегрирование и доказывает теорему
        \item Теперь докажем для счётного набора (объединения) клеток.
        Задача сводится к предыдущей
        \item Найдём меру множества, являющегося счётным пересечением объединения счётного числа клеток
        \item Определим убывающую последовательность множеств
        \item Далее считаем меры составляющих, по ходу дела используя теорему Лебега об ограниченной сходимости
        \item Теперь рассмотрим случай множества нулевой меры и с помощью всяких сравнений и пределов докажем требуемое
        \item В конце рассмотрим общий случай конечно измеримого множества
        \item Используем все предыдущие леммы и случаи и получаем требуемое при почти всех $x$
    \end{enumerate}
    
    \subsection{Теорема Фубини}
    
    \textbf{Th} \textit{О геометрическом смысле интеграла}
    
    \textbf{Th} \textit{Фубини}
    
    \section{Теорема о замене переменных в кратном интеграле}
    
    \textbf{Опр} \textit{$C^k$-гладкий диффеоморфизм}
    
    \textbf{Опр} \textit{Носитель функции}
    
    \textbf{Th} \textit{О замене переменных в кратном интеграле}
    
    \textbf{Л2}
    \textcolor{blue}{Теорема справедлива, если функция $f$ непрерывна на $Y$, а её носитель компактен и лежит в  $Y$}
    
    \begin{enumerate}
        \item Убрав условие 3, мы сделали теорему локальной (для каждой точки существует окрестность, где выполнено
        условие 3)
        \item Воспользуемся теоремой о расщеплении отображений, о неявной функции, критерием компактности, теоремой о
        разбиении единицы
        \item Это позволяет разбить функцию на сумму.
        Утверждение для фиксированного индекса (на его области значений) верно по предыдущей лемме
    \end{enumerate}
    
    \textbf{Л4}
    \textcolor{blue}{Теорема справедлива, если функция $f$ непрерывна на $Y$}
    
    \begin{enumerate}
        \item Рассмотрим неотрицательно значные функции и введём хитрые множества $Y_k$ и функции $f_k$
        \item Докажем, что $\supp \subset f_k$ исходя из определения $f_k$.
        Получили ограниченность и замкнутость $\supp f_k$
        \item Из построения множеств следуют включения, а за ними и неравенства
        \item Теперь покажем, что $f_k$ стремятся к $f$ через определения и построения условий
        \item Запишем следствия из предела и перейдём и завершим доказательство с помощью теоремы Б.~Леви
        \item В общем случае разобьём $f$ на $f_+$ и $f_-$ и получим искомое равенство
    \end{enumerate}
    
    \section{Теорема о построении криволинейной системы координат исходя из её части}
    
    \textbf{Опр} \textit{Криволинейная система координат на множестве $A$}
    
    \textbf{Опр} \textit{Координатный набор}
    
    \textbf{Th} \textit{О построении криволинейной системы координат исходя из ее части}
    
    \begin{enumerate}
        \item Рассмотрим отображение из известного набора функций и матрицу Якоби этого отображения
        \item Рассмотрим координатные строки и матрицу в точке и применим теорему о ранге матрицы
        \item Определим новые гладкие функции и всеобъемлющее отображение, рассмотрим новую матрицу Якоби
        \item Применим теорему об обратном отображении и получим требуемое
    \end{enumerate}
    
    \section{Комплексные числа.
    Модуль и аргумент, тригонометрическая форма.
    Арифметические операции с комплексными числами.
    Извлечение корня.
    Экспонента с комплексным показателем.
    Основная теорема алгебры.
    Разложение многочлена с комплексными коэффициентами на линейные множители.
    Разложение правильной дроби в сумму простейших дробей}

%    \subsection{Комплексные числа}
    
    \textbf{Опр} \textit{$\mathbb{C}$}
    
    \textbf{Опр} \textit{$\operatorname{Re}z, \operatorname{Im}z$}
    
    \textbf{Утв} \textit{Равенство, сложение и вычитание, умножение комплексных чисел}
    
    \textbf{Утв} \textit{Свойство операций с комплексными числами} \textcolor{gray}{Справедливы переместительный,
        сочетательный и распределительные законы}
    
    \textbf{Утв} \textit{Геометрическая интерпретация комплексных чисел} \textcolor{gray}{Точка на декартовой
    плоскоти-квадрате вещественных чисел}
    
    \subsection{Модуль и аргумент, тригонометрическая форма}
    
    \textbf{Опр} \textit{Модуль и аргумент комплексного числа} \textcolor{gray}{Надо вспонмнить их геометрический
    смысл и не забыть указать их обозначения} \\
    
    Любое комплексное число имеет модуль и аргумент.
    Для доказательства требуется рассмотреть тривиальный нулевой
    случай или поделить на корень из сумм квадратов координат, чтобы воспользоваться определением \\
    
    \textbf{Утв} \textit{Тригонометрическая форма комплексного числа} \textcolor{gray}{Модуль определён однозначно,
        аргумент -- нет}
    
    \subsection{Извлечение корня}
    
    \textbf{Опр} \textit{$n-$ая степень комплексного числа} \textcolor{gray}{Всё как с вещественными числами, но
    отсюда следует формула Муавра} \\
    
    Таким образом можно запросто решать комплексные уравнения, которые будут иметь $n$ корней.
    Для этого достаточно
    представить числа в экспоненциальной форме, а потом приравнять модули и аргументы
    
    \subsection{Экспонента с комплексным показателем}
    
    \textbf{Опр} \textit{Экспонента комплексного числа} \textcolor{gray}{Не забыть про домножение на $e^x$}
    
    \textbf{Опр} \textit{Формула Эйлера} \\
    
    Таким образом, любое комплексное число может быть представлено в экспоненциальной форме \\
    
    \textbf{Лемма} \textit{Свойство экспоненты} \textcolor{gray}{При перемножении показатели складываются}
    
    \begin{enumerate}
        \item Вспомним про стандартную запись комплексного числа, а затем про её экспоненту
        \item Перемножим синусы и косинусы и сгруппируем их, воспользовавшись формулами суммы синусов и косинусов
    \end{enumerate}
    
    \textbf{Следствие 1}
    \textcolor{blue}{Модуль и аргумент произведения есть произведение модулей и сумма аргументов}
    \textcolor{gray}{Достаточное перемножить экспоненциальные формы и выделить из результата требуемое}
    
    \textbf{Опр} \textit{Частное комплексного числа} \textcolor{gray}{Такое комплексное число, что ...}
    
    \textbf{Следствие 2}
    \textcolor{blue}{Частное существует и единственно, притом модуль и аргумент частного есть частное модулей и
    разность аргументов}
    \textcolor{gray}{Достаточно воспользоваться определением частного и равенства комплексных чисел}
    
    \subsection{Основная теорема алгебры}
    
    Смотреть в рукописном конспекте
    
    \subsection{Разложение многочлена с комплексными коэффициентами на линейные множители}
    
    Смотреть в рукописном конспекте
    
    \subsection{Разложение правильной дроби в сумму простейших дробей}
    
    Смотреть в рукописном конспекте
    
    \section{Первообразная и неопределенный интеграл.
    Линейность неопределенного интеграла, замена переменных и интегрирование по частям.
    Интегрирование рациональных функций.
    Основные приемы интегрирования иррациональных и трансцендентных функций}

%    \subsection{Первообразная и неопределенный интеграл}

\textbf{Опр} \textit{Первообразная} \textcolor{gray}{Функция, производная которой есть наша}

\textbf{Л1} \textcolor{blue}{Производная константы есть ноль}

\begin{enumerate}
    \item Зафиксируем произвольную точку $x_0$ и обозначим $C = f(x_0)$
    \item Применим теорему Лагранжа о среднем и внимательно посмотрим на числитель
\end{enumerate}

\textbf{Th.1} \textit{О структуре множества первообразных} \textcolor{gray}{Критерий первообразной: все
первообразные функции отличаются на константу}

\begin{enumerate}
    \item В одну сторону очевидно (достаточно всего лишь продифференцировать)
    \item В другую запишем разность производных первообразных и применим Л1
\end{enumerate}

\textbf{Опр} \textit{Неопределённый интеграл} \textcolor{gray}{Множество всех первообразных}

\textbf{Th.2} \textit{Обозначение и краткое обозначение неопределённого интеграла} \textcolor{gray}{Крючок равен
множетсву из $F(x) + C$} \\

Важно понимать, что неопределенный интеграл – это не одна функция, а множество функций.
Иначе говоря, константа $C$ , стоящая в правой части последней формулы, – не фиксированная константа, а параметр,
пробегающий $\mathbb{R}$ \\

\textbf{Л2} \textcolor{blue}{Операция взятия дифференциала и операция взятия неопределенного интеграла являются
взаимно обратными} \textcolor{gray}{А также верны два тождества для взаимно-обратных функций}

\begin{enumerate}
    \item Надо воспользоваться определением неопределённого интеграла и подставить его \("\)формулировку\("\)
    \item Надо обозначить $f = F^{'}$ и повторить рассуждения п.1
\end{enumerate}

\subsection{Линейность неопределенного интеграла, замена переменных и интегрирование по частям}

\textbf{Лемма} \textit{Свойство линейности неопределенного интеграла} \textcolor{gray}{Для доказательства достаточно
вспонимть определения первообразной, неопределенного интеграла и воспользоваться линейнойстью производной} \\

\textbf{Th.1} \textit{Замена переменной или метод интегрирования подстановкой} \textcolor{gray}{Для доказательства
достаточно воспользоваться инвариантностью (дифференциалы простой функции и сложной функции могут быть записаны в
одной и той же форме) первого дифференицала} \\

\textbf{Th.2} \textit{Метод интегрирования по частям} \textcolor{gray}{Для доказательства достаточно
воспользоваться формулой производной произведения и Л2}

\subsection{Интегрирование рациональных функций}

Алгоритм интегрирования рациональной дроби

\begin{enumerate}
    \item При необходимости, методом деления многочлена в столбик представить дробь в виде суммы многочлена и
    правильной рациональной дроби
    \item Найти корни знаменателя и разложить знаменатель на элементарные множители, а затем, методом неопределенных коэффициентов разложить правильную
    рациональную дробь в сумму элементарных дробей (это разложение существует и единственно)
    \item Проинтегрировать элементарные дроби и многочлен
\end{enumerate}

Методы интегрирования элементарных дробей

\begin{itemize}
    \item Интегралы вида $\int \frac{Adx}{(x-x_1)^k}$ считается табличным
    \item Интеграл вида $\int \frac{Bx + C}{(x^2 + px + q)^k} dx$ сводится к интегралу $\int \frac{dx}{(x^2 + px
    + q)^k}$ путём представления числителя в виде суммы дифференциала знаменателя и остатка до верного выражения
    \item Интеграл $\int \frac{dx}{(x^2 + px + q)^k} = I_k(t)$ при $k = 1$ вычисляется путём выделения полного
    квадрата в знаменателя и дальнейшего применения табличного интеграла [арктангенса]
    \item Интеграл $I_k(t)$ при $k > 1$ вычисляется рекуррентно, путём интегрирования по частям и с помощью
    умного нуля
\end{itemize}

\subsection{Основные приемы интегрирования иррациональных и трансцендентных функций}

\textbf{Опр} \textit{Одночлен, многочлен} \textcolor{gray}{Функция $n$ переменных и сумма таковых}

\textbf{Опр} \textit{Рациональная функция} \textcolor{gray}{Отношение многочленов}

\begin{itemize}
    \item Интеграл вида $\int R\left(x^{\frac{1}{n}}\right)dx$ сводится к интегралу от рациональной дроби с
    помощью подстановки $t = x^{\frac{1}{n}}$
    \item Интеграл вида $\int R\left(x, \left(\frac{ax+b}{cx+d} = y\right)^{\frac{1}{n}}\right)dx$ сводится к интегралу из п.1 с помощью подстановки $t = y^{\frac{1}{n}}$
    \item Интеграл вида $\int R\left(x, \sqrt{ax^2 + bx + c} \right)dx$ сводится к интегралу от рациональной дроби
    при помощи подстановки $t = \frac{x - x_1}{x - x_2}$ в случае наличия корней.
    В противном случае при $a < 0$ выражение не определено, а при $a > 0$ сводится подстановкой к интегралу от
    рациональной дроби
    \item Интеграл от дифференциального бинома $\int x^m (ax^n + b)^p dx$, где $m, n, p \in \mathbb{Q} $ сводится к
    интегралу от рациональной функции в случаях, если $p \in \mathbb{Z}$ (подстановка $t = x^{\frac{1}{\text{НОЗ}
    ~m~\text{и}~n}}$), $\frac{m+1}{n} \in \mathbb{Z}$ (подстановка $t = (ax^n + b)^{\frac{1}{\text{знаменатель
    дроби}~p}}$) или $\frac{m+1}{n} + p \in \mathbb{Z}$ (подстановка $t = \left(\frac{ax^n + b}{x^n}\right)^{\frac{1}{\text{
        знаменатель дроби}~p}}$). В противном случае через элементарные функции не выражается
    \item Интегралы с гиперболическими и тригонометрическими функциями универсальными подстановками $t = \tg \frac{x}{2}$ и $t = \th \frac{x}{2}$
    сводятся к интегралам от рациональной дроби
\end{itemize}
    
    \section{Числовые ряды.
    Знакопостоянные ряды.
    Признаки сравнения сходимости числовых рядов.
    Интегральный признак сходимости числового ряда.
    Признаки Даламбера и Коши.
    Знакопеременные ряды (Критерий Коши сходимости ряда).
    Сходимость и абсолютная сходимость.
    Признаки Дирихле, Лейбница и Абеля.
    Независимость суммы абсолютно сходящегося ряда от порядка слагаемых.
    Теорема Римана о перестановке членов условно сходящегося ряда.
    Перемножение абсолютно сходящихся рядов}

%    \subsection{Числовые ряды}

\textbf{Опр} \textit{Частичная сумма ряда} \textcolor{gray}{Сумма конечного числа членов ряда}

\textbf{Опр} \textit{Член ряда} \textcolor{gray}{Элемент последовательности суммирования}

\textbf{Опр} \textit{Сумма ряда} \textcolor{gray}{Предел частичных сумм}

\textbf{Опр} \textit{(Ра)сходящийся ряд} \textcolor{gray}{Предел частичных сумм (бес)конечен}

\textbf{Th.1} \textit{Необходимое условие сходимости ряда} \textcolor{gray}{Достаточно рассмотреть предел
разности двух соседних частичных сумм, что есть член ряда}

Отсюда следует, что ряд из геометрической прогрессии сходится при $\abs{q} < 1$

\textbf{Л1} \textit{Принцип локализации} \textcolor{gray}{Достаточно представить сумму ряда в виде частичной
суммы и остатка, а затем взять предел обеих частей, вспомнив свойство предела последовательности}

\textbf{Л2} \textit{Свойство линейности} \textcolor{gray}{Достаточно вспомнить свойства пределов
последовательностей, связанные с арифметическими неравенствами}

\textbf{Следствие} \textit{Свойство линейности}

\textcolor{blue}{Ряд, чьи члены есть сумма членов сходящегося и расходящегося рядов расходится}

\subsection{Знакопостоянные ряды}

\textbf{Th.1} \textit{О существовании суммы ряда с неотрицательными членами}

\textcolor{blue}{Сумма ряда с неотрицательными членами является неотрицательным числом или $\infty$, притом ряд
сходится $\Leftrightarrow \sum_{\mathbb{N}}^{}a_k < \infty$}

В силу неотрицательности членов, последовательность частичных сумм нестрого возрастает.
Поэтому по теореме Вейерштрасса о монотонной последовательности, предел будет существовать.
Он равен сумме ряда по её определению

\subsection{Признаки сравнения сходимости числовых рядов}

\textbf{Th.2} \textit{Первый признак сравнения} \textcolor{gray}{Следует из предыдущей теоремы}

\textbf{Опр} \textit{Эквивалентность по сходимости} \textcolor{gray}{Члены первого ряда миномажорируются членами
другого ряда и домножением на положительные $m$ и $M$, начиная с какго-то $k_0 \in \mathbb{N}$}

\textbf{Th.3} \textit{Второй признак сравнения}

\textcolor{blue}{Эквивалентные по сходимости ряды сходятся или
расходятся одновременно}

Доказательство следует из первого признака сравнения и принципа локализации

\subsection{Интегральный признак сходимости числового ряда}

Смотреть в рукописном конспекте

\subsection{Признаки Даламбера и Коши}

\textbf{Th} \textit{Признак Даламбера} \textcolor{blue}{Если после какого-то номера отношение соседних членов
ряда $< 1$, то ряд сходится. В противном случае расходится}

\begin{enumerate}
    \item По индукции (или необязательно) легко показать, что $a_k > a_{k_0} q^{k - k_0}$
    \item Затем осталось последовательно воспользоваться преобразованиями, сходимостью геометрической прогрессии
    признаком сравнения и принципом локализации
    \item В случае $\frac{a_{k+1}}{a_k} > 1$ не выполняется необходимое условие сходимости ряда
\end{enumerate}

\textbf{Следствие} \textit{Признак Даламбера в предельной форме} \textcolor{blue}{Если предел отношения соседних
членов ряда есть $q < 1$, то ряд сходится. Если $q > 1$, то ряд расходится, а в случае $q = 1$ ряд может
сходиться, а может и расходиться}

\begin{enumerate}
    \item $q^{'} = \frac{q+1}{2} < \frac{1 + 1}{2} < 1$, поэтому по определению предела теорема сводится к
    предыдущей
    \item Аналогично пользуемся определением предела
    \item В случае $a_k = \frac{1}{k^{\alpha}} \lim_{k\to\infty} \frac{a_{k+1}}{a_k} = \left(\frac{k}{k+1}\right)^{\alpha} = 1
    \forall \alpha \in \mathbb{R}$, однако данный ряд имеет разную сходимость в зависимости от $\alpha$
\end{enumerate}

\textbf{Th} \textit{Радикальный признак Коши}

\textcolor{blue}{Если после какого-то номера $\sqrt[k]{a_k} < 1$,
    то ряд сходится. В противном случае расходится}

\begin{enumerate}
    \item Достаточно возвести неравенство в квадрат и воспользоваться признаком сравнения и принципом локализации
    \item В случае $\sqrt[k]{a_k} > 1$ не выполняется необходимое условие сходимости ряда
\end{enumerate}

\textbf{Следствие} \textit{Признак Коши в предельной форме}

\textcolor{blue}{Если предел $q = \sqrt[k]{a_k} < 1$, то
ряд сходится. Если $q > 1$, то ряд расходится, а в случае $q = 1$ ряд может сходиться, а может и расходиться}

Доказательство аналогично доказательству признака Даламбера в предельной форме

\subsection{Знакопеременные ряды (Критерий Коши сходимости ряда)}

\textbf{Th} \textit{Критерий Коши сходимости ряда}

\textcolor{blue}{Ряд сходится тогда и только тогда, когда
выполняется условие Коши}

Для доказательства достаточно вспомнить определение сходимости ряда и критерий Коши для последовательностей (
сходимость $\Leftrightarrow$ фундаментальность)

\subsection{Сходимость и абсолютная сходимость}

\textbf{Опр} \textit{Абсолютно сходящийся ряд} \textcolor{gray}{Ряд из членов под модулем сходится}

\textbf{Опр} \textit{Условно сходящийся ряд} \textcolor{gray}{Ряд сходится, но не является абсолютно сходящимся}

\textbf{Th} \textit{Если ряд абсолютно сходится, то он сходится}

Для доказательства достаточно записать критерий Коши для ряда из членов под модулем, а затем воспользоваться
неравенством треугольника

\textbf{Th} \textit{Линейная комбинация абсолютно сходящихся рядов сходится}

Для доказательства достаточно последовательно воспользоваться свойство линейности, неравенством треугольника,
признаком сравнения и предыдущей теоремой

\subsection{Признаки Дирихле, Лейбница и Абеля}

Смотреть в рукописном конспекте

\subsection{Независимость суммы абсолютно сходящегося ряда от порядка слагаемых}

\textbf{Опр} \textit{Положительная и отрицательная составляющая члена} \textcolor{gray}{Полусумма и полуразность
модулей}

В зависимости от знака члена, он равен одной из своих составляющих, а другая при этом равна нулю.
Поэтому любой член ряда есть сумма его составляющих.
Когда речь идёт об абсолютно сходящимся ряде, то стоит взять модуль от обеих его частей (одна будет равна модулю
числа, другая нулю)

\textbf{Лемма} \textcolor{blue}{Сумма ряда из чисел одного знака не меняется при перестановке её элементов} \textcolor{gray}{В
силу переместительного закона сложения (I аксиома действительных чисел)}

\textbf{Th} \textcolor{blue}{Сумма абсолютно сходящегося ряда не зависит от перестановки слагаемых}

Для рядов с действительными элементами отдельно сумма положительных и сумма отрицательных элементов не
зависят от перестановок.
Поэтому сумма всех элементов, равная сумме обеих сумм, тоже не будет зависеть от перестановок

\subsection{Теорема Римана о перестановке членов условно сходящегося ряда}

\textbf{Лемма} \textcolor{blue}{Положительная и отрицательная суммы условно сходящегося ряда расходятся}

\begin{enumerate}
    \item Запишем ряд, как сумму его сумм в обычном и абсолютном виде:
    \begin{gather*}
        \sum_{\mathbb{N}}^{}a_k = \sum_{\mathbb{N}}^{}p_k + \sum_{\mathbb{N}}^{}n_k\\
        \sum_{\mathbb{N}}^{} \abs{a_k} = \sum_{\mathbb{N}}^{}p_k + \sum_{\mathbb{N}}^{}(-n_k)
    \end{gather*}
    \item Если хотя бы одна из полусумм сходится, то по первому равенству получим сходимость второй.
    Тогда по второму равенству сходится и ряд $\sum_{\mathbb{N}}^{} \abs{a_k}$, что противоречит условной сходимости ряда
\end{enumerate}

Из леммы следует, что положительная сумма стремиться к $+\infty$, а отрицательная -- к $-\infty$ \\

\textbf{Th} \textit{Римана о перестановке членов условно сходящегося ряда}

\textcolor{blue}{Если ряд сходится условно, то его члены можно переставить так, что сумма полученного ряда будет
равна любому наперёд заданному числу}

\begin{enumerate}
    \item Построим новый ряд, добавляя в него положительные члены, если его частичная сумма меньше заданной и
    отрицательные члены иначе
    \item Количество положительных и отрицательных членов стремится к бесконечности, потому как иначе, если хотя
    бы одно количество конечно, то по Л1 получаем расходимость исходного ряда.
    Поэтому любой член обоих полусумм будет присутствовать в новом ряде, поэтому такой ряд является перестановкой исходного ряда
    \item Новый ряд стремится к заданному числу, потому как из сходимости исходного ряда следует выполнение
    необходимого условия.
    Поэтому $\forall \varepsilon > 0~\exists N: \forall n > N~\abs{S^{'}_n - x} < \varepsilon$,
    что по определению свидетельствует о наличии требуемого предела
\end{enumerate}

\subsection{Перемножение абсолютно сходящихся рядов}

\textbf{Опр} \textit{Множество всевозможных пар натуральных чисел} \textcolor{gray}{Биекция
    $\mathbb{N} \rightarrow \mathbb{N}^2$}

\textbf{Th} \textit{О перемножении рядов}

\textcolor{blue}{Если два ряда сходятся абсолютно, то ряд, составленный из произведения соответвующих членов
сходится абсолютно, а его сумма равна произведению сумм}

\begin{enumerate}
    \item $\forall J \in \mathbb{N}$ определим $M_J = \max \{m_1, \dots, m_J\}$ и аналогично $N_J$.
    Тогда частичная сумма модулей членов нового ряда не превосходит произведения частичных сумм до $M_J$ и $N_J$
    рядов из модулей членов двух исходных рядов
    \item В силу абсолютной сходимости можно перейти к сравнению супремумов, что по теореме о существовании суммы
    ряда с неотрицательными членами сходится, то есть новый ряд сходится абсолютно
    \item Теперь найдём сумму нового ряда.
    В силу предыдущей теоремы мы имеет права воспользоваться другой перестановкой.
    Выберем перестановку методом вложенных квадратов.
    Она хороша тем, что за $N^2$ шагов мы считаем сумму первых $N^2$ членов, что задаёт ВОО.
    Частичная сумма нового ряда равна произведению старых частичных сумм
    \item Так как $S_{N^2}$ есть подпоследовательность последовательности $S_{n^2}$ с пределом $S$, то она
    стремится к такому же пределу, то есть требуемому
\end{enumerate}
    
    \section{Клеточные множества.
    Верхняя мepa Лебега и ее счетная полуаддитивность.
    Мера Лебега и ее счетная аддитивность.
    Непрерывность меры Лебега.
    Теорема о том, что семейство измеримых подмножеств $\mathbb{R}^n$ является $\sigma$-кольцом}

%    \subsection{Клеточные множества}

\textbf{Опр} \textit{Клетка} \textcolor{gray}{Декартово произведение ограниченных числовых промежутков}

\textbf{Опр} \textit{Мера клетки} \textcolor{gray}{Произведение длин её числовых промежутков}

Пустое множество будем считать клеткой по определению, а её меру -- равной нулю

\textbf{Опр} \textit{Гиперплоскость в $\mathbb{R}^n$} \textcolor{gray}{Линейная комбинация координат точки
из $\mathbb{R}^n$, равная константе}

\textbf{Опр} \textit{Разрез множества гиперплоскостью} \textcolor{gray}{Подобные гиперплоскости
множества с неравенствами разной строгости (два варианта разреза)}

\textbf{Л1} \textit{О мере клетки}

\textcolor{blue}{Мера клетки, являющейся дизъюнктным объединением клеток, равна сумме мер её составляющих}

\begin{enumerate}
    \item Разрежем клетку на две клетки гиперплоскостью $x_i = c$
    \item Если одна из новых клеток будет иметь нулевую меру, то требуемое равенство будет тривиально выполнено
    \item В случае $c \in w_i = (a_i, b_i)$ старую клетку удобно представить в виде двух клеток заменой $w_i = w^{'}_i + w^{''}_i$.
    Применяя эти рассуждения много раз получим доказываемое равенство
    \item В общем случае у нас может быть неграмотная последовательность разрезов, поэтому придётся дорезать.
    Разрежем клетку по концам числовых промежутков $w_{i_k}$, определяющих клетку $\sqcap_i$
    \item В любом случае получаем, что каждая клетка является дизъюнктным объединением клеток меньшего порядка,
    полученных в результате такого разрезания.
    Поэтому мера исходной клетки есть сумма сумм элементарных
\end{enumerate}

\textbf{Опр} \textit{Клеточное множество} \textcolor{gray}{Конечный дизъюнктный набор клеток}

\textbf{Л2} \textcolor{blue}{Мера клеточного множества не зависит от способа разбиения этого множества на клетки}

\begin{enumerate}
    \item Пусть есть две серии разрезов.
    Тогда рассмотрим клетки $\sqcap_{ij} = \sqcap_{i} \cap \sqcap_{j}$, среди которых могут быть и пустые клетки
    \item Поэтому $m(\sqcap_{i}) = \sum_{j = 1}^{J} m(\sqcap_{ij})$, а $m(\sqcap_{j}) = \sum_{i = 1}^{I} m(\sqcap_{ij})$
    \item Итого мера первых клеток равна сумма сумм элементарных и равна мере вторых клеток
\end{enumerate}

Перечислим свойства клеточных подмножеств $\mathbb{R}^n$

\begin{itemize}
    \item Замкнутость относительно разности множеств.
    Действительно, в случае разности достаточности рассмотреть
    разность клетки и клеточного множества.
    Дополнение клетки до клеточного множества будет дизъюнктным
    объединением конечного числа элементарных клеток, то есть клеточным множеством
    \item Замкнутость относительно пересечения в силу $A \cap B = A~\backslash~(A~\backslash~B)$
    \item Замкнутость относительно объединения, потому что объединение есть дизъюнктное объединение разности клеточных множеств (клеточного
    множества по п.1) и клеточного множества
    \item Из предыдущих пунктов следует, что семейство всех клеточных подмножеств $\mathbb{R}^n$ является кольцом
    \item Аддитивность, то есть $m(A \cup B) = m(A) + m(B) - m(A \cap B)$.
    Действительно, в тривиальном случае нулевого пересечения очевидно.
    В общем случае надо воспользоваться $m(A) = m(A~\backslash~B) + m(A \cap B)$ и $m(A \cup B) = m(A~\backslash~B) + m(B)$ и
    сравнить с доказываемым утверждением
    \item Монотонность, то есть мера надмножества больше подмножества.
    Свойство следует из предыдущего в силу $m(A) = m(A) + m(B~\backslash~A)$
\end{itemize}

\subsection{Верхняя мера Лебега и ее счетная полуаддитивность}

\textbf{Опр} \textit{Верхняя мера множества}

\textcolor{blue}{Инфимум сумм мер клеток по всем счетным наборам клеток, покрывающим множество}

Из этого определения следует монотонность верхних мер (в силу монотонности инфимума).
Также для клеточного множества его верхняя мера равна мере множества

\textbf{Th} \textit{Счётная полуаддитивность верхней меры} \textcolor{gray}{Полуаддитивность свидетельствует о
неравенстве (в одну сторону)}

\textcolor{blue}{Если множестыво покрыто не более чем счётным набором множеств $X_k$, то его мера не превосходит суммы
мер всех множеств из набора}

\begin{enumerate}
    \item В случае $\mu^*(X_k) = +\infty$ утверждение очевидно.
    Иначе будем считать, что все множества $X_k$ конечны
    \item Фиксируем $\forall \varepsilon > 0$.
    Тогда для каждого $X_k$ найдётся счётный набор клеток, такой что из определения инфимума разность меры этих клеток и
    нашего множества будет не превосходить $\frac{\varepsilon}{2^k}$
    \item В итоге, суммируя по всем клеткам и множествам $X_k$ получим разницу между верхними мерами множества и
    набора не более, чем в $\varepsilon$
\end{enumerate}

\subsection{Мера Лебега и ее счетная аддитивность}

\textbf{Опр} \textit{Симметрическая разность} \textcolor{gray}{Первое множество без второго в объединении с наоборот}

\textbf{Опр} \textit{Предел по мере} \textcolor{gray}{При $k \shortrightarrow \infty$ симметрическая разность
множества и $X_k$ стремится к нулю}

\textbf{Опр} \textit{Конечно измеримое множество} \textcolor{gray}{$\exists$ последовательность клеточных множеств, сходящихся к нашему}

\textbf{Опр} \textit{Измеримое по Лебегу множество} \textcolor{gray}{Объединение счётного набора конечно измеримых}

\textbf{Опр} \textit{Мера Лебега} \textcolor{gray}{Для измеримого множества равна его верхней мере}

\textbf{Опр} \textit{Сдвиг множества на вектор} \textcolor{gray}{Сдвиг не меняет меру}

\textbf{Л1} \textit{Об измеримых множествах}

\textcolor{blue}{Объединение, пересечение и разность конечно измеримых множеств измеримо, а также $\mu (X \cup Y) + \mu (X \cap Y) = \mu (X) + \mu (Y)$}

\begin{enumerate}
    \item По определению конечной измеримости найдутся сходящиеся к нашим последовательности клеточных множеств $X_k$ и $Y_k$
    \item Для клеточных множеств утверждение теоремы доказано раннее
    \item Также воспользуемся свойством аддитивности меры клеточных множеств и перейдём к пределу для доказательства
    последнего равенства
\end{enumerate}

Из этой леммы следует, что семейство всех конечно измеримых множеств в $\mathbb{R}^n$ является кольцом

\textbf{Л2} \textit{Об представлении измеримого множества}

\textcolor{blue}{Измеримое множество можно представить в виде дизъюнктного объединения счётного набора конечно измеримых множеств}

\begin{enumerate}
    \item В силу измеримости нашего множеств существует не более чем счётный набор конечно измеримых множеств, покрывающих наше
    \item Из этого набора составим последовательность концентрических вложенных, но дизъюнктных множеств с помощью операций разности и объединения
    \item Тогда условия леммы выполнены (множества нового набора конечно измеримы, а их дизъюнктное объединение по набору покрывает наше)
\end{enumerate}

\textbf{Th} \textit{Счетная аддитивность меры Лебега}

\textcolor{blue}{Если множество является дизъюнктным объединением счетного набора измеримых множеств, то оно
измеримо, а его мера равна сумме мер множеств из набора}

\begin{enumerate}
    \item Рассмотрим случай конечно измеримых множеств $X_k$.
    Тогда наше множество измеримо по определению
    \item С одной стороны его мера не превосходит сумм мер покрытия в силу счётной полуаддитивности верхней меры, а
    с другой, сумма мер любого конечного набора $X_k$ не превосходит меры нашего множества (в силу определения покрытия)
    \item Переходя к пределу по числу $X_k$ в наборе получаем оценку для меры множества снизу.
    Итого, два неравенства дают требуемое равенство
    \item В общем случае надо воспользоваться предыдущей леммой $\forall X_k$ и просуммировать по двум уровням нарезки (по двум индексам)
\end{enumerate}

\subsection{Непрерывность меры Лебега}

\textbf{Th} \textit{Непрерывность меры Лебега}

\textcolor{blue}{Если у нас есть счетный набор измеримых множеств $X_k$ а наше множество покрывается этим набором, то
его мера равна пределу мер $X_k$} \\

Для доказательства применим идею из Л2 предыдущей темы и воспользуемся счётной аддитивностью меры

\subsection{Теорема о том, что семейство измеримых подмножеств $\mathbb{R}^n$ является $\sigma$-кольцом}

\textbf{Опр} \textit{Кольцо множеств} \textcolor{gray}{Система множеств, замкнутая относительно операций
пересечения и разности} \\

\textbf{Л1} \textcolor{blue}{Кольцо множеств замкнуто относительно пересечения} \textcolor{gray}{В
силу $A \cap B = A~\backslash~(A~\backslash~B)$} \\

\textbf{Опр} \textit{$\sigma$-кольцо множеств} \textcolor{gray}{Система множеств, замкнутая относительно операций
счётного пересечения} \\

\textbf{Л2} \textcolor{blue}{$\sigma$-кольцо множеств замкнуто относительно счётного пересечения} \textcolor{gray} {
    Для доказательства достаточно рассмотреть конкретное множество из пересечения, доказать, что его разность с
пересечением лежит в $\sigma$-кольце, и повторить рассуждения Л1} \\

\textbf{Л3} \textcolor{blue}{Пересечение $X$ счётного набора конечо измеримых множеств $X_k$ является конечно измеримым множеством} \textcolor{gray}

\begin{enumerate}
    \item По Л1 предыдущей темы $X_1 \backslash X_k$ измеримо, поэтому и $X_1 \backslash X$ измеримо как счётное
    объединение разностей конечно измеримых множеств
    \item $\mu (X_1 \backslash X) \leq \mu (X_1)$, \undetline{конечно} измеримого множество, то и $X_1 \backslash X$ конечно измеримо (монотонность меры)
    \item Разность множеств конечно измерима, а так как $X = X_1 \backslash (X_1 \backslash X)$, то и $X$ тоже
\end{enumerate}

\textbf{Л4} \textcolor{blue}{Семейство всех измеримых множеств в $\mathbb{R}^n$ является кольцом}

\begin{enumerate}
    \item Пусть $X,Y \subset \mathbb{R}^n$ измеримы
    Тогда они представимы в виде объединений и пересечений счётных наборов конечно измеримых множеств, как и $X \cap Y$
    \item $X_k \backslash Y$ конечно измеримо как пересечение конечно измеримых, а $X \backslash Y$ конечно измеримо как счётное объединение конечно измеримых
    \item Разность множеств конечно измерима, а так как $X = X_1 \backslash (X_1 \backslash X)$, то и $X$ тоже
\end{enumerate}

\textbf{Th} \textcolor{blue}{Семейство всех измеримых множеств в $\mathbb{R}^n$ является $\sigma$-кольцом}

\begin{enumerate}
    \item Каждое множество $X_k$ счётного набора можно представить в виде счётного объединения набора конечно измеримых
    \item Тогда счётное объединение по всему такому набору будет конечно \undetline{измеримым} множеством
    \item В силу Л4 и определения $\sigma$-кольца, получаем требуемое равенство
\end{enumerate}
    
    \section{Измеримые функции.
    Измеримость суммы и поточечного предела измеримых функций.
    Интеграл Лебега для счетно-ступенчатых и для измеримых функций, линейность интеграла Лебега.
    Теорема о существовании интеграла от неотрицательной измеримой функции.
    Связь интегрируемости функции и интегрируемости ее положительной и отрицательной составляющих.
    Связь интегрируемости функции и интегрируемости ее модуля.
    Интегральная теорема о среднем.
    Счетная аддитивность и непрерывность интеграла Лебега по множествам интегрирования}

%    \subsection{Измеримые функции}

\textbf{Опр} \textit{Измеримая функция} \textcolor{gray}{$\forall C \in \mathbb{R}~L_< = \{ x \in X: f(x) < C \}$
    измеримо}

\textbf{Л1} \textit{Об измеримых функциях}

\textcolor{blue}{Если функция измерима, то $L_{\leq}, L_{\geq}, L_>$ измеримы}

\begin{enumerate}
    \item Фиксируем $\forall C \in \mathbb{R}$ и доказываем измеримость $L_{\leq}$, пользуясь определением $\sigma$-кольца
    \item Остальные множества доказываются через измеримость разности
\end{enumerate}

\textbf{Л2} \textcolor{blue}{Открытые и замкнутые множества измеримы}

\begin{enumerate}
    \item Последовательно воспользуемся определением открытости и всюду плотностью $\mathbb{Q}$, чтобы покрыть нашу
    точку рациональной клеткой
    \item Так как мы брали рациональные числа, то набор различных клеток будет не более, чем счётный.
    Таким образом, открытое множество является объединением счетного набора измеримых множеств, то есть измеримым
    множеством
    \item Замкнутое множество является дополнением открытого множества, поэтому измеримо, так как $\mathbb{R}$ есть
    $\sigma$-кольцо
\end{enumerate}

Любая непрерывная функция измерима, потому что её $L_<$ открыто, а значит, измеримо по Лебегу

\subsection{Измеримость суммы и поточечного предела измеримых функций}

\textbf{Л1} \textcolor{blue}{Сумма измеримых функций измерима}

\begin{enumerate}
    \item Расписываем элемент множества $L_<$ для суммы функций, пользуемся всюду плотностью $\mathbb{Q}$
    Мы специально используем $\mathbb{Q}$ для счётности
    \item Затем переходим к пересечению множеств и объединению по $\mathbb{Q}$, что по определению $\sigma$-кольца означает
    измеримость $L_<$, а значит, и суммы функций
\end{enumerate}

Любая линейная комбинация измеримых функций является измеримой функцией.
Это следует из того, что операция сложения сохраняет измеримость функций, а операция умножения на число сохраняет
измеримость согласно определению измеримой функции

\textbf{Л2} \textcolor{blue}{Поточечный предел измеримых функций измерим}

\begin{enumerate}
    \item Меняем $C$ на $C^{'}$ и объявляем $k \geq N, k \in \mathbb{N}$ в силу определения предела, строгого
    неравенства и всюду плотности $\mathbb{Q}$.
    То есть $x \in L_<$ теперь лежит в пересечении по всем $k \geq N$
    \item Затем через операции объединения по $C^{'}$ и $N$ приходим к измеримости нашего множества в силу того, что
    $\mathbb{R}$ есть $\sigma$-кольцо
\end{enumerate}

\subsection{Интеграл Лебега для счетно-ступенчатых и для измеримых функций, линейность интеграла Лебега}

\textbf{Опр} \textit{Счётно- и конечно-ступенчатая функция} \textcolor{gray}{Множество её значений счётно или
конечно}

\textbf{Опр} \textit{Счётное и конечное разбиение множества} \textcolor{gray}{Дизъюнткное покрытие
исходного множества}

\textbf{Опр} \textit{Измеримое разбиение} \textcolor{gray}{Все множества разбияния измеримы по Лебегу}

Функция называется счётно-ступенчатой, еси существует счётное разбиение области определения и соответсвующий
набор значений функций, одинаковых на конкретном множестве из разбиения

Если все множества набора измеримы, то функция тоже будет измерима (из определения измеримости функции)

\textbf{Опр} \textit{Интеграл Лебега для счётно-ступенчатой функции}

\textcolor{blue}{Сумма произведений мер $X_i$ на значении функции $f_i$ на этом $X_i$}

Даже если $f_i = $, $X_i = +\infty$, то их произведение всё равно 0.
А если в сумме содержатся разные по бесконечности слагаемые или не существует конечной / бесконечной суммы, то интеграл Лебега для этой функции не
существует

\textbf{Опр} \textit{Интегрируемая по Лебегу функция}

\textcolor{blue}{Если ряд Лебега для неё сходится абсолютно}

Нам существенна абсолютная сходимость, потому как иначе интегрируемость зависела бы от способа разбиения области
определения.
Это доказывает следующая лемма

\textbf{Л1} \textcolor{blue}{Интеграл Лебега не зависит от измеримого разбиения}

\begin{enumerate}
    \item Пусть есть два различных разбиения.
    Тогда рассмотрим клетки $X_{ij} = X_{i} \cap X_{j}$, среди которых могут быть и пустые клетки
    \item Просуммируем по всем таким множествам, пользуясь счётной аддитивностью меры Лебега.
    Мера $X_{ij}$ не зависит от порядка суммирования (к тому же, ряд сходится абсолютно), поэтому и домножение на $
    f_i$ и $f_j$ (равные на пересечении), тоже не повлияет на сумму ряда
    \item Поэтому можно запросто совершить переход (под знаком равенства) от одной суммы к другой
\end{enumerate}

\textbf{Утв} \textit{Геометрический смысл интеграла Лебега} \textcolor{gray}{Декартово произведение ($x, y$) в
    $\mathbb{R}^{n+1}$}

\textbf{Л2} \textit{Линейность интеграла Лебега для счетно-ступенчатых функций}

Доказывается той же идеей, что и предыдущая лемма, используя по ходу дела соответствующие свойства для абсолютно
сходящихся рядов

\textbf{Опр} \textit{Почти всюду, почти для всех} \textcolor{gray}{За исключением множества нулевой меры}

\textbf{Опр} \textit{Верхний и нижний интегралы Лебега} \textcolor{gray}{Инфимум (супремум) интеграла по
счётно-ступенчатым больше (меньше) нашей}

\textbf{Опр} \textit{Интеграл Лебега} \textcolor{gray}{Равное значение верхнего и нижнего интегралов Лебега}

\textbf{Опр} \textit{Интегрируемая по Лебегу функция} \textcolor{gray}{Измеримая с конечным интегралом Лебега}

Для счётно-ступенчатой функции её интеграл Лебега и общий интеграл Лебега для неё эквивалентны

\textbf{Л1} \textcolor{blue}{Если значения двух функций совпадают для почти всех аргументов, то их интегралы Лебега
существуют или не существуют одновременно, а если существуют, то совпадают}

Это следует из определений верхнего и нижнего интегралов

\textbf{Л2} \textcolor{blue}{Если функция интегрируема, то для почти всех аргументов её значение на них конечно}

Это следует из определений верхнего и нижнего интегралов

\subsection{Теорема о существовании интеграла от неотрицательной измеримой функции}

\textbf{Лемма} \textcolor{blue}{$\forall \varepsilon > 0~\exists$ измеримая и отдельно интегрируемая СС функции такие
    , что первая не больше нашей, а их сумма не меньше нашей, притом вторая бесконечно мала}

Заметим, что от нашей функции, как и от миноранты, мы не требуем быть интегрируемой.
Ведь даже $f(x) = x, x \in [0, +\infty]$ не имеет интегрируемой функции, удовлетворяющей условиям теоремы

\begin{enumerate}
    \item Рассмотрим случай конечной меры области определения $X$ нашей функции.
    Тогда выберем $N \in \mathbb{N}: \frac{\mu(X)}{N} < \varepsilon, \varphi(x) = \frac{1}{N} = const$, а $g(x)$ зададим
    как стандартную СС, что доказывает теорему
    \item В общем случае представим $X$ в виде дизъюнктного объединения счетного набора конечно измеримых
    множеств, для каждого из которых применим предыдущий пункт
    \item В силу бесконечной малости $\varphi_k(x)$, сделаем её меньше $\frac{\varepsilon}{2^k}$ (чтобы при
    интегрировании по всем функциям получить необходимую малость), а затем соберём обе функции в СС-ые, доказав
    общий случай
    \item Так как все члены $\varphi_k(x) \mu (X_k)$ неотрицательны, то ряд сходится абсолютно и, следовательно, функция
    $\varphi_k(x)$ интегрируема по определению
\end{enumerate}

\textbf{Th} \textit{О существовании интеграла от неотрицательной измеримой функции} \textcolor{gray}{
    Неотрицательная измеримая функция имеет (бес)конечный интеграл Лебега}

\begin{enumerate}
    \item В тривиальном случае бесконечного нижнего интеграла получаем бесконечные верхний и интеграл Лебега
    \item В общем случае зафиксируем $\forall \varepsilon > 0$ и применим результат предыдущей леммы, притом так как $g$
    измерима и СС, то будем считать, что область его определения $X$ можно представить в виде счетного
    дизъюнктного объединения конечно-измеримых множеств, то будем считать, что множества $X_k$ конечно-измеримы
    \item Покажем, что $g$ интегрируема от противного.
    Если у нас это удастся сделать, то мы мгновенно докажем теорему в силу равенства крайних интегралов Лебега.
    Предположим, что для $g~\exists\int = +\infty $
    \item Если каждый член суммы конечен, то для любой константы найдётся частичная сумма, большая её.
    Тогда мы можем построить СС (даже конечно-ступенчатую) функцию $g^{'}< g$ на $X$, которая будет интегрируема
    \item В случае наличия бесконечного члена, определим функцию $g^{'}< g$ на $X$ как либо этот конечный член,
    либо 0
    \item В любом случае построена $g^{'}< g$, чей бесконечный интеграл по $X$ не меньше нижнего нашей функции,
    что отрицает её интегрируемость по определению
    \item Итого, $g$ интегрируема, $g + \varphi$ тоже, а их разность может быть сколь угодно малой (в силу малости
    $\int \varphi dx$), что из неравенств, не меньше разности крайних интегралов
    \item В силу произвольности $\varepsilon > 0$ интегралы совпадают, поэтому наша функция интегрируема
\end{enumerate}

\subsection{Связь интегрируемости функции и интегрируемости ее положительной и отрицательной составляющих}

\textbf{Опр} \textit{Положительная и отрицательная составляющие функции} \textcolor{gray}{Соответсвующие максимумы}

Притом сама функция равна сумме этих составляющих

\textbf{Лемма} \textcolor{blue}{Функция измерима $\Leftrightarrow$ обе её составляющие интегрируемы}

\begin{enumerate}
    \item Из интегрируемости $f_+, f_-$ мгновенно следует интегрируемость функции в силу линейности интеграла Лебега
    \item В другую сторону рассмотрим $f_+$, чьё $L_<$ совпадает с $L_<$ функции (в случае $C \geq 0$ очевидно, а в
    случае $C < 0~L_<$ пусто, но всё равно измеримо по определению)
    \item Из интегрируемости $f$ следует, что $\exists h: f \leq h$ почти всюду на $X$, притом $h$ интегрируема.
    Аналогичное
    равенство
    справедливо и для положительных составляющих
    \item Так как $h$ интегрируема, то её положительная составляющая тоже по признаку сравнения для абсолютно
    сходящегося ряда
    \item По теореме предыдущей темы $f_+$ интегрируема, притом в силу теоремы об интегрировании неравенств
    (доказывается через крайние интегралы), интеграл конечен
    \item Произведя аналогичные рассуждения для $f_-$, получим доказательство в другую сторону, то есть полное
    доказательство
\end{enumerate}

\subsection{Связь интегрируемости функции и интегрируемости ее модуля}

\textbf{Th.1} \textit{Признак сравнения}

\textcolor{blue}{Если измеримая функция по модулю не превосходит интегрируемой по Лебегу функции для почти всех
аргументов, то она интегрируема}

\begin{enumerate}
    \item Из измеримости $f$ следует измеримость $f_+, f_-$
    \item Затем достаточно воспользоваться теоремой об интегрировании неравенств дважды и доказать
    интегрируемость $f_+, f_-$
    \item Из предыдущей леммы следует интегрируемость $f$
\end{enumerate}

\textbf{Th.2} \textit{Связь интегрируемость функции и её модуля}

\textcolor{blue}{$f$ интегрируема по Лебегу $\Leftrightarrow$ $f$ измерима и её модуль интегрируем по Лебегу} \\

Из интегрируемости $f$ следует интегрируемость $f_+, f_-$, то есть и их суммы, которая и есть модуль.
Для доказательства в обратную сторону достаточно воспользоваться Th.1

\subsection{Интегральная теорема о среднем}

\textbf{Th} \textit{Интегральная теорема о среднем} \textcolor{gray}{Если $X$ -- линейнос связный компакт
в $\mathbb{R}^n; f, g: X \rightarrow \mathbb{R}$, притом $f$ измерима, а $g$ интегрируема с сохранением знака, то
существует точка из $X$ ...}

\begin{enumerate}
    \item БОО будем полагать $g \geq 0$
    \item В силу теоремы Вейерштрасса, $f$ достигает экстремумов (потому как на компакте), то есть произведение
    функций полуограниченно
    \item В силу признака сравнения, произведение функций интегрируемо, а также, в силу интегрирования неравенств
    и свойства линейности интеграла имеем оценку и для интеграла произведения функций
    \item Если интеграл равен нулю, то теорема справедлива для любого аргумента.
    В противном случае возьмём отношение и посмотрим на неравенство для $C$
    \item По теореме о промежуточном значении, найдётся $\xi \in X: f(\xi) = C$
\end{enumerate}

\subsection{Счетная аддитивность и непрерывность интеграла Лебега по множествам интегрирования}

\textbf{Л1} \textit{Об интегрируемости на подмножестве} \textcolor{gray}{Интегрируемая функция интегрируема на
измеримом подмножестве}

\begin{enumerate}
    \item Из интегрируемости функции следует интегрируемость её модуля, то есть его верхний интеграл Лебега конечен
    \item Тогда существует интегрируемая СС, которая не меньше модуля для почти всех аргументов.
    Она будет интегрируема на подмножестве по определению интеграла от СС функции
    \item В силу признака сравнения и наша функция интегрируема на подмножестве
\end{enumerate}

\textbf{Л2} \textit{Конечная аддитивность интеграла Лебега по множествам} \textcolor{gray}{На непересекающихся
измеримых множествах, интегрируемая на них функция интегрируема на их объединении, притом её интеграл есть сумма
интегралов на множествах}

\begin{enumerate}
    \item Распишем определение верхнего интеграла Лебега и построим новую верхнюю СС для объединения
    \item Осталось применить аддитивность интеграла для СС и получить нижний интеграл для объединения множеств
    \item Аналогично для верхнего интеграла Лебега.
    В итоге в силу конечности суммы конечных слагаемых и определения интеграла Лебега, получаем доказываемое утверждение
\end{enumerate}

\textbf{Th.1} \textit{Непрерывность интеграла по множествам} \textcolor{gray}{Для счётного набора измеримых по
Лебегу вложенных множеств и интегрируемой по Лебегу функции интеграл по счётному объединению равен пределу
интегралов по множествам}

\begin{enumerate}
    \item Рассмотрим случай СС функции и измеримое разбиение области определения.
    Из этого набора составим последовательность концентрических вложенных, но дизъюнктных множеств с помощью
    операций разности
    \item Распишем интеграл функции на множестве, используя счётную аддитивность интеграла от СС функции и конечную
    аддитивность интеграла Лебега
    \item В общем случае интегрируемой по Лебегу функции воспользуемся конечной аддитивностью интеграла Лебега и
    разобьём множество на подмножество и его дополнение.
    Требуется доказать, что интеграл на дополнении есть ноль
    \item Для этого воспользуемся определением интеграла Лебега, теоремой об интегрировании неравенств,
    результатами для СС функций и теоремой о трёх последовательностях
\end{enumerate}

\textbf{Th.2} \textit{Счетная аддитивность интеграла Лебега} \textcolor{gray}{На измеримом разбиении
множества интегрируемая на них функция имеет интеграл, равный счётной сумме по множествам}

\begin{enumerate}
    \item Определим новый набор множеств как частичное объединение старых
    \item Далее используем непрерывность и конечную аддитивность интеграла по множествам
\end{enumerate}
    
    \section{Непрерывность интеграла как функции верхнего предела.
    Существование первообразной для непрерывной на отрезке функции.
    Формула Ньютона-Лейбница.
    Формулы замены переменных в интеграле и интегрирования по частям}
    
    \subsection{Непрерывность интеграла как функции верхнего предела}
    
    \textbf{Утв} \textit{Обозначения для интеграла Лебега} \textcolor{gray}{Множество интегрирования, связь с
    обратным и множество нулевой меры}
    
    \textbf{Лемма} \textcolor{blue}{Если $f$ интегрируем на отрезке, содержащим три точки, то её интеграл можно разбить
    на два}
    
    Доказывается через интегрируемость функции на подмножестве и с помощью конечной аддитивности интеграла по множествам
    
    \textbf{Th} \textit{Непрерывность интеграла как функции верхнего предела} \textcolor{gray}{Если на числовом
    промежутке функция интегрируема, то её $F(x) = \int_a^x f(t)dt$ непрерывна на $(a, b)$}
    
    \begin{enumerate}
        \item Зафиксируем произвольную точку отрезка и строго возрастающую последовательность с пределом в нашей точке
        \item Воспользуемся определением $F(x_0)$ и непрерывностью интеграла по множествам, а также тем, что
        предел слева совпадает с обычным на внутренностях
        \item Аналогичные рассуждения с убывающей последовательностью доказывают требуемую непрерывность (потому как и
        справа, и слева)
    \end{enumerate}
    
    \subsection{Существование первообразной для непрерывной на отрезке функции}
    
    \textbf{Th} \textcolor{blue}{Если функция интегрируема на отрезке и непрерывна его точке, то для её $
    F(x) \exist \frac{d}{dx}: F^{'}(x_0) = f(x_0)$, притом на концах отрезка речь идёт об односторонних производных}
    
    \begin{enumerate}
        \item Зафиксируем произвольную точку отрезка справа
        В силу аддитивность интеграла имеем $F(x) - F(x_0) = \int_{x_0}^x f(t)dt$
        \item Применим интегральную теорему о среднем для $f(x), g(x) = 1 \geq 0$, получим отношение.
        Тогда устремив аргумент к нашей точке, получим определение производной справа
        \item Аналогичные рассуждения дадут нам производную слева, аз значит, и доказываемую теорему
    \end{enumerate}
    
    Из этой теоремы следует существование первообразной для непрерывной для отрезке функции, а также, совместно с
    теоремой о структуре первообразных, их отличие на константу
    
    \subsection{Формула Ньютона-Лейбница}
    
    \textbf{Th} \textit{Формула Ньютона-Лейбница}
    
    Для доказательства достаточно расписать первообразную на множестве нулевой меры, на втором конце и взять разность
    
    \subsection{Формулы замены переменных в интеграле и интегрирования по частям}
    
    \textbf{Th.1} \textit{Замена переменной в определённом интеграле}
    
    \textcolor{blue}{Если функция $x = \varphi([a, b])$ имеет непрерывную производную на отрезке $[a, b]$, а $f$
        непрерывна на $\varphi([a, b])$, то справедливо равенство ...}
    
    \begin{enumerate}
        \item В силу непрерывности $f$ на $\varphi([a, b])$, для неё существует первообразная.
        Воспользуемся для неё формулой Ньютона-Лейбница
        \item Продифференцируем первообразную и определим, для какой функции она таковой является.
        Применим формулу Ньютона-Лейбница уже для неё (обратное равенство) и получим требуемое
    \end{enumerate}
    
    \textbf{Th.2} \textit{Интегрирование по частям}
    
    \textcolor{blue}{Если функции непрерывно дифференцируемы, то они могут быть проинтегрированы по частям} \\
    
    Для доказательства достаточно воспользоваться линейностью интеграла и формулой Ньютона–Лейбница
    
    \section{Мера декартова произведения двух конечно измеримых множеств.
    Выражение меры множества под графиком интегрируемой функции через интеграл.
    Площадь круга.
    Выражение объема тела вращения и длины кривой через интегралы.
    Связь интегрируемости по Риману и интегрируемости по Лебегу.
    Интегрируемость по Риману непрерывной на отрезке функции}

%    \subsection{Мера декартова произведения двух конечно измеримых множеств}

\textbf{Th} \textcolor{blue}{Если два множества конечно измеримы в своих надмножествах, то их декартово
произведение конечно измеримо в соотвествующем надмножестве с мерой, равной произведению мер}

\begin{enumerate}
    \item В тривиальном случае клеток равенство следует из определения
    \item В случае, если конечно измеримые множества представимы в виде счетного дизъюнктного объединения клеток,
    разобьём их на эти клетки, а потом, в силу теоремы о перемножении абсолютно сходящихся рядов, получим требуемое
    \item Покажем, что для любых конечно измеримых множеств мера их декартового произведения не превосходит
    произведения мер.
    Для этого зафиксируем $\forall \varepsilon > 0$ и счётные покрытия наших множеств клетками (они найдутся по
    определению верхней меры), притом разность мер покрытия и наших множеств не будет
    превосходить $\varepsilon$.
    Тогда распишем неравенство для верхней меры декартова произведения и, устремив $\varepsilon \rightarrow 0$,
    получим требуемое неравенство
    \item Теперь покажем, что если существуют множества, сходящиеся по мере к нашим (с конечной верхней мерой), то
    их декартово произведение также будет сходиться к декартову произведению наших.
    Действительно, для этого надо расписать неравенство для верхней меры симметрической разности, используя
    предыдущий пункт и понять, что она стремится к нулю
    \item В общем случае по определению конечно измеримого множества найдутся последовательности клеточных
    множеств, сходящиеся по мере к нашим.
    Тогда надо последовательно воспользоваться п.4 и п.1, а затем перейти к пределу
\end{enumerate}

Из теоремы следует, тчо декартово произведение множества нулевой меры и произвольного имеет нулевую меру

 \subsection{Выражение меры множества под графиком интегрируемой функции через интеграл}

\textbf{Лемма} \textit{Теорема о трёх последовательностях для конечно измеримых множеств}

\textcolor{blue}{Если задано наше множество и существуют конечно измеримые последовательности
миномажорант для него, которые в пределе имеют одинаковую меру, то наше множество измеримо и имеет ту же меру} \\

Для доказательства нам потребуется перейти от верхней меры (заданной для всех, в том числе для неизвестного нашего
множества) к клеточным множествам, для которых уже есть понятие предела по мере.
Иначе наши рассуждения могли бы быть неприменимы

\begin{enumerate}
    \item Рассмотрим \undetline{верхнюю} меру симметрической разность нашего и последовательности миноранты.
    Из неравенств будет следовать, что она стремится к нулю
    \item Теперь рассмотрим симметрическую разность клеточных множеств $A_{ik}$, покрывающих $A_k$, и саму $A_k$.
    Применив неравенство треугольника, получим, что клеточные множества $A_{ik}$ сходятся по мере к нашему
    \item Аналогичные рассуждения для мажорант доказывают теорему
\end{enumerate}

\textbf{Th} \textit{О геометрическом смысле интеграла}

\textcolor{blue}{Если область определения интегрируемой функции $X$ измерима, то площадь под графиком функции в
соотвествующем надмножестве конечно измерим с мерой равной интегралу лебега этой функции по $X$}

\begin{enumerate}
    \item В тривиальном случае СС функции можно разбить график на дизъюнктное объединение множеств и в силу
    счётной аддитивности интеграла Лебега получить требуемое утверждение
    \item В общем случае обозначим интеграл как $J$ и зафиксируем $\forall \varepsilon > 0$
    \item Воспользуемся определением верхних интегралов и запишем две серии неравенств (для СС-функций и их
    интегралов)
    \item При необходимости заменим значения миномажорант-СС-функций на множестве нулевой мере (чтобы
    доказываемое утверждение было справедливо для всего $X$)
    \item На предыдущем шаге записываем меру площадей графиков функции под миномажорантами и приходим к
    очевидному двойному вложению
    \item Так как в силу произвольности $\varepsilon > 0$ их площади стремятся к $J$ , то в силу леммы, площадь под
    графиком измерима с мерой $J$
\end{enumerate}

\subsection{Площадь круга}

\textbf{Лемма} \textcolor{blue}{Круг измерим с площадь $\pi r^2$}

\begin{enumerate}
    \item Напишем множество верхнего полукруга и после преобразований выразим $y$: $0 \leq y \leq \sqrt{
        r^2 - x^2}$
    \item По предыдущей теореме верхний полукруг измерим с интегралом в половину искомого (интеграл считается
    через замену).
    Аналогично для нижнего полукруга
    \item Так как две части круга имеют нулевую меру пересечения, то по формуле включений-исключений, мера круга
    равна $\pi r^2$
\end{enumerate}

\subsection{Выражение объема тела вращения и длины кривой через интегралы}

\textbf{Опр} \textit{Тело вращения вокруг оси} \textcolor{gray}{Если на отрезке задана неотрицательная функция, то
множество ...}

\textbf{Th.1} \textcolor{blue}{Если неотрицательная функция измерима и ограничена, то тело вращения измеримо...}

\begin{enumerate}
    \item Зафиксируем супремум ограниченной функции, число $N \in \mathbb{N}$, на которое мы разобьём наш отрезок
    множествами $X_k$ и измеримые конечно-ступенчатые функции-миномажаронты
    \item Распишем объём тел вращения для миноранты в терминах декартова произведения площади круга на меру $X_k$ c помощью определения интеграла для СС-функции
    \item Запишем неравенства для полученных объёмов и устремим $N \rightarrow +\infty$
    \item Аналогично распишем для мажоранты
    \item В силу вложенности и стремления по мере в пределе получим объём тела вращения для нашей функции
\end{enumerate}

\textbf{Th.2} \textit{Вычисление длины кривой}

\textcolor{blue}{Если кривая параметризована непрерывно дифференицируемой вектор-функцией, то её длина выражается
формулой ...} \\

Для доказательства достаточно рассмотреть переменную длину дуги, вспомнить теорему о производной переменной длины
дуги и применить формулу Ньютона-Лейбница

\subsection{Связь интегрируемости по Риману и интегрируемости по Лебегу}

\textbf{Опр} \textit{Разбиение отрезка, отрезки разбиения} \textcolor{gray}{Конечный набор точек}

\textbf{Опр} \textit{Выборка} \textcolor{gray}{Набор точек из отрезков разбиения}

\textbf{Опр} \textit{Интегральная сумма Римана} \textcolor{gray}{Сумма конечного числа слагаемых, зависит от
функции, разбиения и выборки}

\textbf{Опр} \textit{Мелкость разбиения} \textcolor{gray}{Максимальный отрезок разбиения}

\textbf{Опр} \textit{Интеграл Римана} \textcolor{gray}{Предел интегральных сумм Римана} \\

Заметим, что этот интеграл всегда конечен в силу работы на компакте (отрезке) \\

\textbf{Опр} \textit{Интегрируемая по Риману функция} \textcolor{gray}{$\exists$ интеграл Римана для этой функции
на этом отрезке}

\textbf{Th.1} \textit{Достаточное условие интегрируемости}

\textcolor{blue}{Если функция непрерывна на компакте, то она интегрируема на нём}

\begin{enumerate}
    \item Так как для любого $C \in \mathbb{R}~L_{\leq}$ замкнуто (а значит, измеримо), то функция измерима на
    компакте
    \item В силу теоремы Вейерштрасса функция ограничена на компакте некоторой константой
    \item Так как константа интегрируема на компакте, то по признаку сравнения функция тоже интегрируема
\end{enumerate}

\textbf{Th.2} \textcolor{blue}{Если функция интегрируема по Риману, то она интегрируема и по Лебегу и интегралы
совпадают}

\begin{enumerate}
    \item Зафиксируем $\forall \varepsilon > 0$ и достаточно мелкое разбиение отрезка
    \item Перепишем предельное неравенство в терминах инфимума и введём новые обозначения, чтобы ввести
    конечно-ступенчатую функцию
    \item Тогда интеграл для минорант будет интегралом Римана функции (записанным в терминах инфимума).
    Поэтому нижний интеграл будет не меньше Риманова
    \item Аналогично верхний интеграл не больше Риманова
    \item Объединив все полученные неравенства в одну строку, получим равенство крайних интегралов и интеграл
    Лебега по определению
\end{enumerate}

\subsection{Интегрируемость по Риману непрерывной на отрезке функции}

\textbf{Th} \textcolor{blue}{Для непрерывной на отрезке функции $f$ интеграл Римана существует и совпадает с
интегралом Лебега}

\begin{enumerate}
    \item Сначала надо воспользоваться теоремой Кантора, определением равномерной непрерывности
    \item Затем зафиксировать разбиение и выборку, определить конечно-ступенчатую функцию
    \item Вспомнить определение интеграла для СС функции и модуля непрерывности
    \item По Th.1 $f$ интегрируема по Лебегу, как и разность $f$ и СС функции в силу линейности интеграла
    \item Переходя к пределу при мелкости разбиения, получаем что интеграл Римана существует по определению,
    притом из рассуждений следует, что он совпадает с интегралом Лебега
\end{enumerate}
    
    \section{Теорема Б. Леви о монотонной сходимости.
    Теорема Лебега об ограниченной сходимости}
    
    Отличие следующих теорем от непрерывности интеграла по множествам состоит в том, что теперь предельный переход
    выполняется для функций, а не множеств
    
    \subsection{Теорема Б. Леви о монотонной сходимости}
    
    \textbf{Th} \textcolor{blue}{Если последовательность измеримых функций $f_k \geq 0$ монотонна и сходится к $f$,
        то $f$ измерима с интегралом, равным пределу интегралов $f_k$}
    
    \begin{enumerate}
        \item Измеримость функции следует из леммы о поточечной сходимости, а интегрируемость в силу
        существования интеграла от неотрицательной измеримой функции (интеграл может быть бесконечным)
        \item Рассмотрим случай конечного интеграла, предварительно выкинув множества нулевой меры, на котором он
        бесконечен
        \item Зафиксируем $\forall \varepsilon > 0$ и рассмотрим множества $X_k$ c $(1 - \varepsilon)$ внутри
        \item В силу монотонности функции, $X_k$ будут монотонны по включению и покрывать всю область определения
        \item Вспомним про непрерывность интеграл по множествам и определение предела
        \item Затем распишем неравенства, устремим $\varepsilon \rightarrow 0$ и получим доказываемое соотношение
        \item В случае бесконечного интеграла фиксируем $\forall C > 0$ и миноранту, чей интеграл на том же множестве
        будет $> C$ (она существует из определения нижнего интеграла) и выкинем множества нулевой меры, на
        которых миноранта больше $f$
        \item Рассмотри измеримые функции $g_k = \min(f_k, g)$, которые в пределе равны миноранте (показывается через
        определения предела для $f$ и минимума)
        \item Как показано в конечном случае, предел для миноранты будет больше $> C$, а в силу неравенства, для $f$
        тоже.
        В силу произвольности $C$ получаем необходимое равенство
    \end{enumerate}
    
    \subsection{Теорема Лебега об ограниченной сходимости}
    
    \textbf{Th} \textcolor{blue}{Если последовательность интегрируемых функций $f_k$, каждый член которой ограничен по
    модулю интегрируемой функцией $\varphi$ почти всюду на $X$ и поточечно сходится к $f$, то $f$ интегрируема с
    интегралом, равным пределу интегралов $f_k$}
    
    \begin{enumerate}
        \item Измеримость $f$ следует из леммы о поточечной сходимости, а интегрируемость в силу предельного перехода и
        признака сравнения
        \item Выкинем множества нулевой меры, на которых условие теоремы не выполняется
        \item Зафиксируем $\forall \varepsilon > 0$ и рассмотрим множества $X_k$ c $\varepsilon \varphi(x)$ внутри
        \item $X_k$ будут покрывать $X$ (включение в одну сторону очевидно, а в другое надо рассмотреть два случая
        для $\varphi(x)$, расписать определение предела).
        Также $X_k$ будут монотонны по включению
        \item Распишем предел для $\int_{X_k} \varphi$ с помощью непрерывности и аддитивности интеграла по множествам
        \item Теперь распишем неравенство для разности интегралов $f$ и $f_k$, воспользовавшись неравенством
        треугольника, определением $X_k$ и конечностью интеграла для $\varphi$
        \item В итоге, устремив $\varepsilon \rightarrow 0$, завершим доказательство теоремы
    \end{enumerate}
    
    \section{Несобственный интеграл.
    Связь сходимости несобственного интеграла и интегрируемости функции по Лебегу.
    Критерий Коши.
    Признаки Дирихле и Абеля сходимости несобственных интегралов}
    
    \subsection{Несобственный интеграл}
    
    \textbf{Опр} \textit{Несобственный интеграл, особенность} \textcolor{gray}{Односторонний предел интегрального конца}
    
    \textbf{Опр} \textit{(Рас)ходящийся несобственный интеграл} \textcolor{gray}{Если (не)существует конечный предел}
    
    \textbf{Опр} \textit{Собственный интеграл} \textcolor{gray}{Интеграл Лебега, который был до этого}
    
    \textbf{Опр} \textit{Абсолютно сходящийся несобственный интеграл} \textcolor{gray}{Аналогично рядам}
    
    \textbf{Опр} \textit{(Сходящийся) несобственный интеграл с двумя особенностями} \textcolor{gray}{Разбить на два
    интеграла с одной особеностью (и утверждать сходимость только в случае сходимости обоих интегралов)}
    
    \subsection{Связь сходимости несобственного интеграла и интегрируемости функции по Лебегу}
    
    \textbf{Th.1} \textcolor{blue}{Если $f$ интегрируема по Лебегу, на любом открытом промежутке, она
    интегрируема на всём промежутке $\Leftrightarrow$ соответсвующих несобственный интеграл сходится абсолютно}
    
    \begin{enumerate}
        \item $\Rightarrow$: согласно лемме об интегрируемости на подмножестве $f$ интегрируема на любом открытом
        промежутке, как и её модуль (по эквивалентности)
        \item Из аддитивности интеграла по множествам следует нестрогое возрастание функции $F(b^{'}) = \int_a^{b^{'}} \abs{f(x
            )} dx $
        \item По теореме существует предел слева, поэтому несобственный интеграл сходится абсолютно
        \item $\Leftarrow$: зафиксируем возрастающую последовательность $\{b_k\} \rightarrow b$
        \item Определим индикаторную последовательность функций $f_k(x)$.
        Она сходится к $f$, что докажет измеримость $f$ на всём интервале
        \item Затем введём новую функциональную последовательность $g(x) = \abs{f_k(x)}$.
        Она будет возрастать и в пределе равна $\abs{f(x)}$, поэтому применима теорема о монотонной сходимости
        \item Из неё следует интегрируемость $\abs{f(x)}$ на интервале, то есть и $f$
    \end{enumerate}
    
    \textbf{Th.2} \textcolor{blue}{Если $f$ интегрируема в собственном смысле, то несобственный интеграл сходится и
    его значение равна интегралу Лебега на том же интервале}
    
    Доказательство состоит в применении теоремы о непрерывности интеграла как функции верхнего предела
    
    \subsection{Критерий Коши}
    
    \textbf{Th} \textit{Критерий Коши}
    
    \textcolor{blue}{Если на числовом промежутке $f$ интегрируема по Лебегу на любом открытом промежутке, то
    несобственный интеграл этой функции сходится $\Leftrightarrow$ выполняется условие Коши}
    
    \begin{enumerate}
        \item Определим $F(t) = \int_a^t f(x)dx$.
        Несобственный интеграл c особенностью в верхнем конце будет сходиться, если у этой функции существует
        конечный предел при $t \rightarrow b - 0$
        \item Далее сведём задачу к КК существования предела функции и воспользуемся формулой Ньютона -- Лейбница
    \end{enumerate}
    
    \subsection{Признаки Дирихле и Абеля сходимости несобственных интегралов}
    
    Смотреть в рукописном конспекте
    
    \section{Связь поточечной и равномерной сходимостей для функциональной последовательности.
    Критерий Коши равномерной сходимости функциональной последовательности.
    Обобщенный признак сравнения для функциональных рядов.
    Признак Вейерштрасса равномерной сходимости функционального ряда.
    Признаки Дирихле и Лейбница равномерной сходимости функционального ряда.
    Признак Абеля равномерной сходимости функционального ряда.
    Непрерывность равномерного предела, непрерывных функций и суммы равномерно сходящегося функционального ряда с
    непрерывными слагаемыми.
    Почленное интегрирование функциональных последовательностей и рядов.
    Дифференцирование предельной функции и почленное дифференцирование функционального ряда}

%    \subsection{Связь поточечной и равномерной сходимостей для функциональной последовательности}

\textbf{Опр} \textit{Поточечный предел функциональной последовательности} \textcolor{gray}{Предел в привычном
понимании}

\textbf{Опр} \textit{Равномерный предел функциональной последовательности} \textcolor{gray}{$N \in \mathbb{N}$ не зависит от аргумента}

Из равномерной сходимости следует поточечная, но не наоборот

\textbf{Опр} \textit{Равномерно ограниченная функциональная последовательность} \textcolor{gray}{$N \in \mathbb{N}$
    не зависит от аргумента}

\subsection{Критерий Коши равномерной сходимости функциональной последовательности}

\textbf{Th} \textit{Критерий Коши}

\textcolor{blue}{Последовательность сходится равномерно $\Leftrightarrow$ выполняется условие Коши}

\begin{enumerate}
    \item $\Rightarrow$: дважды применить определение равномерной сходимости и воспользоваться неравенством
    треугольника
    \item $\Leftarrow$: требуется доказать равномерную сходимость из выполнения условия Коши числовой
    последовательности для любого фиксированного $x \in X$.
    В силу КК для числовой последовательности $\lim_{k\to\infty} f_k = f$
    \item Далее надо в силу $\forall p \in \mathbb{N}$ устремить его к $+\infty$ и по теореме о предельном
    переходе в неравенствах получить определение равномерной сходимости
\end{enumerate}

\subsection{Обобщенный признак сравнения для функциональных рядов}

\textbf{Опр} \textit{Поточечный предел функционального ряда} \textcolor{gray}{Сходимость ряда в привычном понимании}

\textbf{Опр} \textit{Равномерный предел функционального ряда} \textcolor{gray}{Если последовательность его
частичных сумм сходится равномерно на том же множестве}

\textbf{Опр} \textit{Остаток поточечно сходящегося функционального ряда} \textcolor{gray}{Разность суммы и
частичной суммы ряда}

\textbf{Th} \textit{Обобщенный признак сравнения}

\textcolor{blue}{Если каждый член нашего ряда по модулю не превосходит члена равномерно сходящегося на том же
множестве ряда, то и наш ряд сходится равномерно}

Доказательство состоит в двукратном применении КК

Из признака следует, что из равномерной абсолютной сходимости ряда следует равномерная сводимость ряда на том же
множестве

\subsection{Признак Вейерштрасса равномерной сходимости функционального ряда}

\textbf{Th} \textit{Признак Вейерштрасса}

\textcolor{blue}{Если каждый член нашего ряда по модулю не превосходит члена сходящегося ряда, то наш ряд
сходится равномерно на том же множестве}

Доказательство состоит в применении обобщенного признака сравнения.
Заметьте, что мы не требуем равномерной сходимости от ряда-мажоранты

\subsection{Признаки Дирихле и Лейбница равномерной сходимости функционального ряда}

Смотреть в рукописном конспекте

\subsection{Признак Абеля равномерной сходимости функционального ряда}

Смотреть в рукописном конспекте

\subsection{Непрерывность равномерного предела, непрерывных функций и суммы равномерно сходящегося функционального ряда с
непрерывными слагаемыми}

\textbf{Th.1} \textit{О непрерывности предельной функции}

\textcolor{blue}{Если последовательность $f_k$ непрерывных на множестве $X$ функций сходится равномерно на
множестве $X$, то $f$ непрерывна на $X$}

\begin{enumerate}
    \item Зафиксируем $\forall \varepsilon > 0$ и $x_0 \in X$
    \item Далее для доказательства достаточно дважды записать определения равномерной сходимости и
    один раз непрерывности функции $f_N (x)$ для нужных долей  $\varepsilon$ и воспользоваться неравенством
    треугольника
\end{enumerate}

\textbf{Th.2} \textit{О непрерывности суммы ряда}

\textcolor{blue}{Если функциональный ряд $u_k$ непрерывных на множестве $X$ функций сходится
равномерно на множестве $X$, то сумма ряда непрерывна на $X$}

Доказательство состоит в применение Th.1 последовательности частичных сумм ряда

\subsection{Почленное интегрирование функциональных последовательностей и рядов}

\textbf{Th.1} \textit{Об интегрировании предельной функции}

\textcolor{blue}{Если последовательность $f_k$ интегрируемых на конечно измеримом множестве $X$ функций сходится
равномерно на множестве $X$ к интегрируемой функции $f$, то интеграл этой функции есть предел интегралов}

\begin{enumerate}
    \item Воспользуемся sup-критерием для $\varepsilon = 1$.
    Тогда из неравенства следует интегрируемость $f$ по признаку сравнения
    \item Расписав супремум для разности интегралов в пределе получим 0, что завершает доказательство
\end{enumerate}

\textbf{Следствие} \textcolor{blue}{Если последовательность непрерывных на компакте $X$ функций $f_k$ сходится
равномерно к функции $f$, то интеграл этой функции есть предел интегралов} \\

Непрерывность $f$ следует из теоремы предыдущей темы, а интегрируемость из достаточного условия интегрируемости,
что позволяет применить предыдущую теорему и доказать утверждение \\

\textbf{Th.2} \textit{Об почленном интегрировании ряда}

\textcolor{blue}{Если функциональный ряд $u_k$ непрерывных на компакте $X$ функций сходится
равномерно, то сумма интеграла есть интеграл суммы} \\

Доказательство состоит в применение следствия из предыдущей теоремы к последовательности частичных сумм ряда с
использованием линейности интеграла

\subsection{Дифференцирование предельной функции и почленное дифференцирование функционального ряда}

\textbf{Th.1} \textit{О дифференцировании предельной функции}

\textcolor{blue}{Если последовательность $f_k$ непрерывно дифференцируемых на отрезке $[a, b]$ функций сходится
хотя бы в одной точке $x_0$, а последовательность производных $f_k^{'}$ сходится равномерно на $[a, b]$, то
последовательность $f_k$ сходится равномерное на $[a, b]$ к некоторой непрерывно дифференицируемой функции $f$,
    притом производная предела есть предел производных}

\begin{enumerate}
    \item Обозначим предельную функцию для $f_k^{'}$ за $\varphi (x)$, непрерывную по теореме, и предел $f_k(x_0)$
    за $A$
    \item Далее определим $f(x) =  A + \int_{x_0}^x \varphi (t) dt$ и $f_k (x) = f_k (x_0) + \int_{x_0}^x f_k^{'}(
    t) dt$
    \item Затем пара хитрых замечаний, работа с супремумом, использование sup-критерия
    \item В итоге получаем равномерную сходимость $f_k$ и требуемое равенство с учётом построения $f(x)$
\end{enumerate}

\textbf{Th.2} \textit{О почленном дифференцировании ряда} \\

\textcolor{blue}{Если функциональный ряд $u_k$ непрерывно дифференцируемых на отрезке $[a, b]$ функций сходится
хотя бы в одной точке $x_0$, а ряд производных $u_k^{'}$ сходится равномерно на $[a, b]$, то
справделива формула почленного дифференицрования ряда, то есть производная суммы ряда есть сумма производных}

Доказательство состоит в применение Th.1 к последовательности частичных сумм ряда
    
    \section{Степенные ряды.
    Формула Коши-Адамара для радиуса сходимости.
    Теорема о круге сходимости степенного ряда.
    Первая теорема Абеля.
    Теорема о равномерной сходимости степенного ряда.
    Вторая теорема Абеля.
    Сохранение радиуса сходимости при почленном дифференцировании степенного ряда.
    Теоремы о почленном интегрировании и дифференцировании степенного ряда.
    Единственность разложения функции в степенной ряд, ряд Тейлора.
    Достаточное условие аналитичности функции.
    Пример бесконечно дифференцируемой, но неаналитической функции.
    Представление экспоненты комплексного аргумента степенным рядом.
    Формулы Эйлера.
    Формула Тейлора с остаточным членом в интегральной форме.
    Представление степенной и логарифмической функций степенными рядами}

%    \subsection{Степенные ряды}

\textbf{Опр} \textit{Предел последовательности комплексных чисел} \textcolor{gray}{Предел модуля разности равен
нулю}

Заметим, что комплексный предел эквивалентен двум вещественным (для действительной и мнимой части)

\textbf{Опр} \textit{Сходящийся комплексный ряд} \textcolor{gray}{Существует конечный предел последовательности
частичных сумм этого ряда}

\textbf{Опр} \textit{Абсолютно сходящийся комплексный ряд} \textcolor{gray}{Сходится вещественный ряд модулей
членов ряда}

И вновь сходимость комплексного ряда эквивалентна сходимости двух вещественных рядов

\textbf{Опр} \textit{Равномерно сходящийся комплекснозначная функциональная последовательность} \textcolor{gray}{
    Вещественнозначная последовательность модулей разности предельной функции и элементов последовательности
    равномерно сходится к нулю на том же множестве}

\textbf{Опр} \textit{Равномерно сходящийся комплексный функциональный ряд} \textcolor{gray}{Последовательность
частичных сумм этого ряда равномерно сходится к сумме этого ряда на том же множестве}

\textbf{Опр} \textit{Степенной ряд} \textcolor{gray}{Если задана последовательность комплексных чисел
и комплексное число, то ...}

Однако удобнее (и мы в дальнейшем будем так делать) работать с рядом без степенной разности, сделав замену
комплексной переменной

\subsection{Формула Коши-Адамара для радиуса сходимости}

\textbf{Опр} \textit{Радиус сходимости степенного ряда} \textcolor{gray}{Неотрицательное число (или бесконечность
    ), определяемое формулой Коши-Адамара}

Притом для этой формулы мы расширили операцию деления

\subsection{Теорема о круге сходимости степенного ряда}

\textbf{Опр} \textit{Круг сходимости степенного ряда} \textcolor{gray}{Круг на комплексной плоскости с центром
в $w_0 (0)$ и радиусом равным радиусу сходимости}

Если радиус сходимости бесконечен, то кругом сходимости считается вся комплексная плоскость

\textbf{Th} \textit{О круге сходимости}

\textcolor{blue}{Степенной ряд абсолютно сходится внутри круга сходимости и расходится вне его}

\begin{enumerate}
    \item Зафиксируем произвольное комплексное число $z_0 \neq 0$, обозначим $q = \frac{z_0}{R}$ и исследуем
    сходимость с помощью обобщённого признака Коши
    \item В тривиально случае $z_0 = 0$ ряд сходится абсолютно
    \item В случае $0 < \abs{z_0} < R$ в силу обобщённого признака Коши ряд сходится абсолютно
    \item В случае $\abs{z_0} > R$ в силу обобщённого признака Коши члены абсолютного ряда не стремятся к нулю,
    как и исходного ряда, а значит, он расходится по отрицанию необходимого условия
\end{enumerate}

\subsection{Первая теорема Абеля}

\textbf{Th} \textit{Первая теорема Абеля}

\textcolor{blue}{Если степенной ряд сходится в точке $z_0$, то он сходится абсолюто в любой точке по модулю
меньшей}

Доказательство следует от противного в силу п.4 теоремы о круге сходимости

\subsection{Теорема о равномерной сходимости степенного ряда}

\textbf{Th} \textit{О равномерной сходимости степенного ряда}

\textcolor{blue}{$\forall r \in (0, R)$ ряд $\sum_{\mathbb{N}}_0 c_k z^k$ сходится равномерно в круге радиуса $r$}

Доказывается через неравенство, применением теоремы о круге сходимости и по признаку Вейерштрасса равномерной
сходимости комплексного ряда

\begin{enumerate}
    \item Зафиксируем произвольное комплексное число $z_0 \neq 0$, обозначим $q = \frac{z_0}{R}$ и исследуем
    сходимость с помощью обобщённого признака Коши
    \item В тривиально случае $z_0 = 0$ ряд сходится абсолютно
    \item В случае $0 < \abs{z_0} < R$ в силу обобщённого признака Коши ряд сходится абсолютно
    \item В случае $\abs{z_0} > R$ в силу обобщённого признака Коши члены абсолютного ряда не стремятся к нулю,
    как и исходного ряда, а значит, он расходится по отрицанию необходимого условия
\end{enumerate}

\subsection{Вторая теорема Абеля}

\textbf{Th} \textit{Вторая теорема Абеля}

\textcolor{blue}{Если степенной ряд сходится в точке $z_0$, то он сходится равномерно на отрезке $[0, z_0]$}

\begin{enumerate}
    \item Разобьём члены ряда на произведение членов произведения с помощью параметра $t \in [0, 1]$
    \item Первый ряд сходится по условию (а значит, по предыдущей теореме, ещё и равномерно)
    \item Второй ряд равномерно ограничен на отрезке и монотонен по индексу
    \item Поэтому два вещественных ряда сходятся равномерно на $[0, 1]$, как и исходный ряд на $[0, z_0]$
\end{enumerate}

\subsection{Сохранение радиуса сходимости при почленном дифференцировании степенного ряда}

\textbf{Th} \textcolor{blue}{Радиусы сходимости степенных рядов, полученные формальным дифференцированием и
интегрированием исходного, совпадают с его радиусом сходимости}

\begin{enumerate}
    \item Радиусы сходимости исходного и продифференцированного рядов совпадают в силу формулы Коши-Адамара
    \item Также они сходятся или расходятся одновременно, потому как при $z = 0$ это очевидно, а в противном
    случае они отличаются на ненулевую константу (как и их пределы)
    \item Так как исходный ряд получается почленным дифференцированием интегрального, то и их радиусы сходимости
    совпадают
\end{enumerate}

\subsection{Теоремы о почленном интегрировании и дифференцировании степенного ряда}

\textbf{Th} \textit{Об интегрировании и дифференцировании степенного ряда}

\textcolor{blue}{Если вещественный степенной ряд имеет ненулевой радиус сходимости, то внутри интервала
сходимости
    \begin{itemize}
        \item справедливы формулы почленного интегрирования
        \item функция ряда имеет производные любого порядка, получаемые почленным дифференцированием ряда
        \item коэффициенты степенного ряда однозначно определяются по обрывку формулы Тейлора
    \end{itemize}   }

\begin{enumerate}
    \item Для почленного интегрирования достаточно ввести новую переменную и воспользоваться теоремами о
    равномерной сходимости степенного ряда и о почленном интегрировании равномерно сходящегося функционального ряда
    \item Для производных достаточно ввести новую переменную и воспользоваться теоремами о сохранении радиуса
    сходимости, о равномерной сходимости степенного ряда и о почленном дифференцировании функционального ряда
    \item Проводя те же рассуждения по индукции, доказываем второе утверждение теоремы
    \item Доказывается аналогично лемме первого семестра перед формулой Тейлора
\end{enumerate}

\subsection{Единственность разложения функции в степенной ряд, ряд Тейлора}

\textbf{Опр} \textit{Бесконечно дифференцируемая функция в точке} \textcolor{gray}{В этой точке существуют
производные функции любого порядка}

\textbf{Опр} \textit{Ряд Тейлора} \textcolor{gray}{Ряд бесконечно дифференцруемой функции в точке с членами ...}

\textbf{Опр} \textit{Регулярная функция в точке $z_0$} \textcolor{gray}{Ряд Тейлора функции в точке $z_0$
    сходится к функции в некоторой окрестности $z_0$}

Из теоремы об интегрировании и дифференцировании степенного ряда следует, что если функция может быть
представлена как сумма степенного ряда $\sum_{\mathbb{N}_0} a_k (z - z_0)^k$ с ненулевым радиусом сходимости, то
этот ряд является рядом Тейлора функции в точке $z_0$.
В этом случае функция является регулярной в точке $z_0$

\textbf{Опр} \textit{Остаточный член формулы Тейлора} \textcolor{gray}{Разность $n$ раз дифференцируемой функции
и формулы Тейлора}

Непосредственно из определений следует, что функция является регулярной в точке
$\Leftrightarrow \lim_{n\to\infty} r_n(x) = 0$.
Притом для доказательства регулярности недостаточно показать ненулевой радиус сходимости функции, надо ещё проверить
её остаток

\subsection{Достаточное условие аналитичности функции}

\textbf{Th} \textit{Достаточное условие регулярности}

\textcolor{blue}{Если $\exists U_\delta (x_0)$, где функция бесконечно дифференцируема и последовательность её
производных равномерно ограничена константой $C > 0$, то функция регулярна в точке и $\forall x \in U_\delta (x_0)$
    раскладывается в ряд Тейлора}

\begin{enumerate}
    \item Применим формулу Тейлора с остаточным членом в форме Лагранжа.
    Тогда остаточный член формулы Тейлора $\leq M \frac{\delta^{n+1}}{(n+1)!}$
    \item Так как факториал растёт быстрее показательной (доказывается через принцип Архимеда, определение
    факториала, цепочку неравенств и предельный переход), то остаточный член стремится к нулю
    \item Поэтому функция регулярна, потому как раскладывается в ряд Тейлора в $x_0$
\end{enumerate}

\subsection{Пример бесконечно дифференцируемой, но неаналитической функции}

\begin{equation}
    f(x) =
    \begin{cases}
        e^{-\frac{1}{x^2}}, x \neq 0; \\
        0, x = 0.
    \end{cases}
\end{equation}

Ряд Тейлора этой бесконечно дифференцируемой в точке $x_0 = 0$ сходится не к функции $f(x)$, а к
некоторой другой функции, не совпадающей с $f(x)$ в сколь угодно малой окрестности точки

\[ \forall k \in \mathbb{N} \lim_{x \to 0} \frac{1}{x^k} e^{-\frac{1}{x^2}} = \lim_{t \to +\infty} t^{\frac{k}{2}} e^{-t} = 0 \]

По индукции легко показать, что если $P_{3n} (t)$ -- многочлен степени $3n$ от $t$, то

\begin{equation}
    f^{(n)}(x) =
    \begin{cases}
        P_{3n} (\frac{1}{x}) e^{-\frac{1}{x^2}}, x \neq 0; \\
        0, x = 0.
    \end{cases}
\end{equation}

Следовательно, все коэффициенты ряда Тейлора функции $f(x)$ в точке $x_0 = 0$ равны нулю.
Поэтому сумма ряда Тейлора функции $f(x)$ в точке $x_0$ равна нулю и не совпадает с функцией $f(x)$ в сколь угодно
малой окрестности точки $x_0$.
Таким образом, хотя функция и бесконечно дифференцируема, она не является регулярной в нуле

\subsection{Представление экспоненты комплексного аргумента степенным рядом}

\textbf{Опр} \textit{Ряд Маклорена} \textcolor{gray}{Ряд Тейлора функции в нуле}

\textbf{Th.1} \textcolor{blue}{Ряды маклорена функций $e^x, \sin(x), \cos(x), \sh(x), \ch(x)$ сходятся к этим
функциям на всей числовой прямой}

\begin{enumerate}
    \item $\forall \delta > 0~\forall x \in U_\delta (0)~e^x < e^\delta$, поэтому выполнено достаточное условие
    регулярности
    \item Аналогично, используя ограниченность последовательности всех производных оставшихся функций доказываем
    их разложения
\end{enumerate}

\textbf{Th.2} \textcolor{blue}{Для комплексной экспоненты её ряд Тейлора не отличается от вещественного}

\begin{enumerate}
    \item В силу предыдущей теоремы радиус сходимости степенного ряда-претендента сходится на всём $\mathbb{C}$,
    поэтому по теореме о круге сходимости он сходится абсолютно для любого $z \in \mathbb{C}$
    \item Зафиксируем произвольное комплексное число в алгебраической форме и воспользуемся определением
    экспоненты комплексного числа, чтобы зафиксировать доказываемое равенство
    \item Покажем, что функция-ряд-претендент обладает свойством экспоненты.
    Для этого воспользуемся теоремой о перемножении абсолютно сходящихся рядов, которая для комплексных рядов
    доказывается точно так же, как и для вещественных (только здесь надо использовать метод \("\)диагоналей\("\))
    \item В результате преобразований получим сумму сумм, которую распределим по этим суммам, и применим формулу
    бинома Ньютона, завершив доказательство свойства
    \item Далее рассмотрим функцию кандидат на чисто мнимом аргументе и путём разложения на чётную и нечётную
    суммы получим выражение для чисто мнимой экспоненты
    \item В итоге, применив свойство экспоненты и убедившись, что функция работает на вещественных аргументах,
    получим разложение комплексной экспоненты в ряд Тейлора в силу единственности
\end{enumerate}

\subsection{Формулы Эйлера}

\textbf{Лемма} \textcolor{blue}{Для любого $z \in \mathbb{C}$ справедливы формулы Эйлера} \textcolor{gray}{Они
используют новопостроенные комплексные функции и подравнивают комплексную тригонометрию к вещественной гиперболике}

\begin{enumerate}
    \item Для доказательства формулы гиперкомплексной экспоненты достаточно разделить сумм на чётную и нечётную,
    а затем воспользоваться $i^2 = -1$
    \item Остальные формулы следуют из первой
\end{enumerate}

\subsection{Формула Тейлора с остаточным членом в интегральной форме}

\textbf{Th} \textit{Формула Тейлора с остаточным членом в интегральной форме}

\textcolor{blue}{Если функция в $U_\delta (x_0)$ имеет непрерывные производные по $n+1$ порядок, то для
остаточного члена формулы Тейлора справедливо представление в интегральной форме: $r_n (x) = \frac{1}{n!} \int_{
    x_0}^x (x - t)^n f^{n+1}(t)dt \forall x \in U_\delta (x_0) $}

\begin{enumerate}
    \item При $n = 0$ теорема справедлива в силу формулы Ньютона -- Лейбница
    \item Пусть теорема справедлива для $n = s - 1$.
    Тогда проинтегрируем $r_{s-1}$ по частям
    \item Затем, расписав $r_s$ по определению, подставим проинтегрированное выражение и получим требуемое равенство
    \item Таким образом, теорема доказана по индукции
\end{enumerate}

\subsection{Представление степенной и логарифмической функций степенными рядами}

\textbf{Th} \textcolor{blue}{Ряд Маклорена степенной функции сходится к этой функции на интервале единичного радиуса}

\begin{enumerate}
    \item Зафиксируем $x \in (-1; 1)$ и учитывая выражение для $f^{n}$ распишем остаточный член в интегральной
    форме, походу дела вынося константы, вводя новые обозначения и переменные интегрирования
    \item Затем воспользуемся ограниченностью $x$ для оценки.
    Осталось показать, что $\lambda_n \rightarrow 0$
    \item В тривиальных случаях $x = 0$ и $\alpha = m \in \mathbb{N}_0, m < n$ утверждение очевидно
    \item В общем случае найдём предел отношения и воспользуемся схожими рассуждениями с доказательством признака
    Даламбера (сравнение с геометрической прогрессией)
\end{enumerate}

Заметим, что при $m \geq n$ ряд Маклорена совпадает с конечной суммой \\

Из доказанного и теоремы о почленном интегрировании степенного ряда при $\abs{x} < 1$ (не забывая про замену
индекса суммирования) получаем ряд Маклорена для логарифма.
Данное разложение справедливо и при $x = 1$.
Действительно, данный ряд будет сходиться по признаку Лейбница.
Следовательно, в силу второй теоремы Абеля этот ряд сходится равномерно на отрезке $[0; 1]$.
Согласно теореме о непрерывности суммы равномерно сходящегося функционального ряда частичные суммы этого ряда будет
непрерывны на отрезке $[0; 1]$.
Поэтому существует требуемый предел

\end{document}
