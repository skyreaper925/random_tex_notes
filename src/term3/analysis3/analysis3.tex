%! Author = user
%! Date = 28.12.2023

\documentclass[a4paper, 14pt]{article}
%\documentclass[draft]{article}

\usepackage[T2A]{fontenc}
\usepackage[utf8]{inputenc}
\usepackage[english, russian]{babel}
\usepackage[top = 2cm, bottom = 2cm, left = 2cm, right = 2cm]{geometry}
\usepackage{indentfirst}
\usepackage{xcolor}
\usepackage{hyperref}
\usepackage{gensymb}
\usepackage{pgfplots}
\usepackage{amsmath, amsfonts, amsthm, mathtools}
\usepackage{amssymb}
\usepackage{physics, multirow, float}
\usepackage{wrapfig, tabularx}
\usepackage{icomma} % Clever comma: 0,2 - number while 0, 2 - two numbers
\usepackage{tikz, standalone}
\usepackage{fancyhdr,fancybox}
\usepackage{booktabs}
\usepackage{listings}
\usepackage{lstmisc}
\usepackage{stmaryrd}

%\полуторный интервал
\onehalfspacing

\hypersetup
{   colorlinks = false,
    linkcolor = blue,
    pdftitle = {analysis},
    pdfauthor = {Володин Максим},
    allcolors = [RGB]{010 090 200}
}

%\gravarphicspath{{images/}}
%\DeclareGravarphicsExtensions{.pdf,.png,.jpg}

\restylefloat{table}
\usetikzlibrary{external}

\mathtoolsset{showonlyrefs = true} % Numbers will appear only where \eqref{} in the text LINKED
\pagestyle{fancy}

\fancyhf{}
\fancyhead[L]{КИТП}
\fancyhead[R]{Конспект билетов}
\fancyfoot[L]{}
\usepackage{lastpage}
\fancyfoot[R]{\thepage /\pageref{LastPage}}


\pgfplotsset{compat=1.18}

\begin{document}
    
    \tableofcontents \newpage
    
    \section{Теорема о выражении меры множества через интеграл от меры сечений.
    Теорема Фубини}
    
    \subsection{Теорема о выражении меры множества через интеграл от меры сечений}
    
    \textbf{Л1}
    
    \textcolor{blue}{Пусть есть счётный набор конечно измеримых убывающих вложенных множеств $X_i$.
    Мера множества, являющегося счётным пересечением есть предел мер}
    
    \textbf{Л2}
    
    \textcolor{blue}{Если интеграл неотрицательной функции по множеству равен нулю, то сама функция равна нулю почти
    на всём множестве}
    
    \textbf{Th} \textit{О выражении меры множества через интеграл от меры сечений}
    
    \begin{enumerate}
        \item Для начала докажем для клетки.
        Её сечение будет принимать простой вид, в зависимости от принадлежности $x$, что даёт простое
        интегрирование и доказывает теорему
        \item Теперь докажем для счётного набора (объединения) клеток.
        Задача сводится к предыдущей.
        \item Найдём меру множества, являющегося счётным пересечением объединения счётного числа клеток
        \item Определим убывающую последовательность множеств
        \item Далее считаем меры составляющих, по ходу дела используя теорему Лебега об ограниченной сходимости
        \item Теперь рассмотрим случай множества нулевой меры и, с помощью всяких сравнений и пределов, докажем
        требуемое.
        \item В конце рассмотрим общий случай конечно измеримого множества.
        \item Используем все предыдущие леммы и случаи и получаем требуемое при почти всех $x$
    \end{enumerate}
    
    \subsection{Теорема Фубини}
    
    \textbf{Th} \textit{О геометрическом смысле интеграла}
    
    \textbf{Th} \textit{Фубини}
    
    \section{Теорема о замене переменных в кратном интеграле}
    
    \textbf{Опр} \textit{$C^k$-гладкий диффеоморфизм}
    
    \textbf{Опр} \textit{Носитель функции}
    
    \textbf{Th} \textit{О замене переменных в кратном интеграле}
    
    \textbf{Л2}
    \textcolor{blue}{Теорема справедлива, если функция $f$ непрерывна на $Y$, а её носитель компактен и лежит в  $Y$}
    
    \begin{enumerate}
        \item Убрав условие 3, мы сделали теорему локальной (для каждой точки существует окрестность, где выполнено
        условие 3)
        \item Воспользуемся теоремой о расщеплении отображений, о неявной функции, критерием компактности, теоремой о
        разбиении единицы
        \item Это позволяет разбить функцию на сумму.
        Утверждение для фиксированного индекса (на его области значений) верно по предыдущей лемме
    \end{enumerate}
    
    \textbf{Л4}
    \textcolor{blue}{Теорема справедлива, если функция $f$ непрерывна на $Y$}
    
    \begin{enumerate}
        \item Рассмотрим неотрицательно значные функции и введём хитрые множества $Y_k$ и функции $f_k$
        \item Докажем, что $\supp \subset f_k$ исходя из определения $f_k$.
        Получили ограниченность и замкнутость $\supp f_k$
        \item Из построения множеств следуют включения, а за ними и неравенства
        \item Теперь покажем, что $f_k$ стремятся к $f$ через определения и построения условий.
        \item Запишем следствия из предела и перейдём и завершим доказательство с помощью теоремы Б.~Леви
        \item В общем случае разобьём $f$ на $f_+$ и $f_-$ и получим искомое равенство
    \end{enumerate}
    
    \section{Теорема о построении криволинейной системы координат исходя из её части}
    
    \textbf{Опр} \textit{Криволинейная система координат на множестве $A$}
    
    \textbf{Опр} \textit{Координатный набор}
    
    \textbf{Th} \textit{О построении криволинейной системы координат исходя из ее части.}
    
    \begin{enumerate}
        \item Рассмотрим отображение из известного набора функций и матрицу Якоби этого отображения.
        \item Рассмотрим координатные строки и матрицу в точке и применим теорему о ранге матрицы
        \item Определим новые гладкие функции и всеобъемлющее отображение, рассмотрим новую матрицу Якоби
        \item Применим теорему об обратном отображении и получим требуемое
    \end{enumerate}
    
    \section{Гладкие подмногообразия пространства $R^N$.
    Теорема о гладком подмногообразии пространства $R^N$, заданном системой уравнений}
    
    \subsection{Гладкие подмногообразия пространства $R^N$}
    
    \textbf{Опр} \textit{Гладкое $n$-мерное подмногообразием пространства $\mathbb{R}^N_p$ в точке $P \in M$}
    
    \textbf{Опр} \textit{Канонический и выпрямляющий диффеоморфизм}
    
    \textbf{Утв} \textit{Гладкое $n$-мерное подмногообразие пространства}
    
    \subsection{Теорема о гладком подмногообразии пространства $R^N$, заданном системой уравнений}
    
    \textbf{Th} \textit{О гладком подмногообразии, заданном системой уравнений}
    
    \begin{enumerate}
        \item Сначала достроим отображения до гладкого диффеоморфизма по теореме о построении криволинейной системы
        координат, исходя из её части
        \item Докажем, что выпрямляемость обратного диффеоморфизма.
        Это делается через анализ множеств и из их свойств
    \end{enumerate}
    
    \section{Геометрический касательный вектор к подмножеству пространства $\mathbb{R}^n$.
    Теоремы о структуре множества $T_P (M)$ геометрических касательных векторов к гладкому подмногообразию $M$
        пространства $\mathbb{R}^n$ в общем случае и в случае, когда $M$ заданно системой уравнений}
    
    \subsection{Геометрический касательный вектор к подмножеству пространства $\mathbb{R}^n$}
    
    \textbf{Опр} \textit{Геометрический касательный вектор к множеству в точке}
    
    \textbf{Опр} \textit{Геометрическое касательное пространство}
    
    \subsection{Теоремы о структуре множества $T_P (M)$ геометрических касательных векторов к гладкому
    подмногообразию $M$ пространства $\mathbb{R}^n$ в общем случае и в случае, когда $M$ заданно системой уравнений}
    
    \textbf{Th.1} \textit{О структуре множества геометрических касательных векторов к гладкому подмногообразию.}
    
    \begin{enumerate}
        \item Воспользуемся определением канонического диффеоморфизма, леммой о $T_P (M)$ к линейному пространству и
        о локальности $T_P (M)$.
        \item Запишем вид $T_P (M)$ и перейдём к локальной параметризации (из-за правил умножения матриц)
        \item Итого, касательные векторы есть л.к. столбцов матрицы Якоби, а само $T_P (M)$ является $n$-мерным
        линейным подпространством $\mathbb{R}^N$
    \end{enumerate}
    
    \textbf{Th. 2} \textit{О структуре множества геометрических касательных векторов к подмногообразию, заданному
    системой уравнений.}
    
    \begin{enumerate}
        \item Пишем те же рассуждениями, что и в теореме о гладком подмногообразии, заданном системой уравнений.
        \item Введём новую переменную и воспользуемся предыдущей теоремой
    \end{enumerate}
    
    \section{Необходимые условия безусловного экстремума.
    Достаточные условия безусловного экстремума}
    
    \subsection{Необходимые условия безусловного экстремума}
    
    \textbf{Опр} \textit{Точка экстремума}
    
    \textbf{Опр} \textit{Точка (не)строгого локального минимума (максимума) функции на множестве}
    
    \textbf{Th} \textit{Необходимое условие экстремума}
    
    Для доказательства воспользуемся определением градиента и рассмотрим функцию одной переменной, где применим теорему
    Ферма
    
    \subsection{Достаточные условия безусловного экстремума}
    
    \textbf{Опр} \textit{Стационарная точка}
    
    \textbf{Th} \textit{Достаточные условия экстремума.}
    
    \begin{enumerate}
        \item Разложим $\Delta f$ по формуле Тейлора с использованием определения стационарной точки.
        \item Воспользуемся леммой и определением $o$-малого и предела
        \item Теперь перепишем $\Delta f$ и получим требуемое.
        В случае отрицательной определённости рассуждения аналогичны ($f$ меняется на $-f$)
        \item В знаконеопределённом случае рассмотрим выделенные направления и запишем $\Delta f$.
        \item Воспользуемся определением предела и выберем достаточно малые $t_i$ для построения противоречия
        \item В последнем случае приводятся два контрпримера $f = x^4$ и $f = x^3$
    \end{enumerate}
    
    \section{Метод Лагранжа нахождения точек условного экстремума.
    Необходимые условия условного экстремума.
    Достаточные условия условного экстремума}
    
    \subsection{Метод Лагранжа нахождения точек условного экстремума}
    
    \textbf{Опр} \textit{Функция Лагранжа}
    
    \textbf{Опр} \textit{Множители Лагранжа}
    
    \subsection{Необходимые условия условного экстремума}
    
    \textbf{Th} \textit{Необходимые условия экстремума}
    
    Сделаем общие построения для доказательства теорем.
    
    \begin{enumerate}
        \item Воспользуемся теоремой о построении криволинейной системы координат, исходя из её части и введём новые
        обозначения для обратных функций
        \item Тогда можно расписать обратную функцию Лагранжа и показать, что новая задача эквивалентна старой
    \end{enumerate}
    
    Теперь докажем саму теорему.
    
    \begin{enumerate}
        \item Воспользуемся теоремой о необходимом условии безусловного экстремума и выберем специальные множители
        Лагранжа.
        \item Вернёмся к исходным переменным, воспользовавшись теоремой о дифференцировании сложной функции и значениями
        $\frac{\partial L}{\partial \lambda_i}$
        \item Вышеперечисленное показывает, что $x_0$ -- стационарная точка функции Лагранжа, как и нашей исходной функции
    \end{enumerate}
    
    \subsection{Достаточные условия условного экстремума}
    
    \textbf{Th} \textit{Достаточные условия экстремума.}
    
    \begin{enumerate}
        \item Перейдём к обратной функции Лагранжа, для которой точка $y_0$ стационарна
        \item Исходная точка будет стационарной, если в этой точке обратная функция совпадёт с обратной функцией
        Лагранжа, то есть $k'$ для обратной функции Лагранжа будет отрицательная определена
        \item Покажем, что это эквивалентно отрицательной определённости $k$.
        Для этого воспользуемся инвариантностью первого дифференциала, распишем второй и воспользуемся
        стационарностью точки
        \item Согласно теореме о структуре множества геометрических касательных векторов к подмногообразию, заданному
        системой уравнений, получаем эквивалентность структур форм, что нам и требуется
    \end{enumerate}
    
    \section{Топологическое пространство.
    Индуцированная топология.
    Карта и атлас на топологическом пространстве.
    Общие (абстрактные) определения многообразия и гладкого многобразия.
    Классы гладкости $C^k$ отображений из одного гладкого многообразия в другое}
    
    \subsection{Топологическое пространство}
    
    \textbf{Опр} \textit{Топологическое пространство}
    
    \textbf{Опр} \textit{Топология}
    
    \textbf{Опр} \textit{Открытое в топологическом пространстве множество}
    
    \textbf{Опр} \textit{Семейство всех открытых подмножеств метрического пространства с метрикой}
    
    \textbf{Опр} \textit{Окрестность точки}
    
    \textbf{Опр} \textit{Внутренность множества}
    
    \textbf{Опр} \textit{Замыкание множества}
    
    \subsection{Индуцированная топология}
    
    \textbf{Опр} \textit{Индуцированная топология}
    
    \textbf{Л1}
    
    Доказывается по определению, с привлечением старых множеств, породивших новую топологию
    
    \textbf{Опр} \textit{Хаусдорфово топологическое пространство}
    
    \textbf{Опр} \textit{Предел по топологическому пространству}
    
    Заметим, что у хаусдорфова пространства не может быть двух различных пределов; иначе может
    
    \textbf{Л2}
    
    Доказывается выбором специальных окрестностей, которые не пересекаются
    
    \textbf{Опр} \textit{База топологического пространства}
    
    \textbf{Л3}
    
    Возьмём пересечения открытых шаров с рациональными радиусами и координатами
    
    \textbf{Опр} \textit{Непрерывное отображение}
    
    \textbf{Опр} \textit{Секвенциально непрерывное в точке отображение}
    
    \textbf{Опр} \textit{(Секвенциально) непрерывное в точке отображение}
    
    \textbf{Опр} \textit{Гомеоморфизм}
    
    \textbf{Л4}
    
    Докажем от частного к общему с помощью непрерывности и открытости объединения открытых
    
    \textbf{Опр} \textit{Компактное топологическое пространство}
    
    \textbf{Опр} \textit{Секвенциально компактное топологическое пространство}
    
    \textbf{Опр} \textit{Секвенциально компактное множество}
    
    \textbf{Л5}
    
    Возьмём открытое покрытие множества, перейдём к прообразам, выберем там открытое покрытие и конечное подпокрытие
    
    \textbf{Опр} \textit{Гомеоморфные множества}
    
    \textbf{Опр} \textit{Топологический инвариант}
    
    \textbf{Опр} \textit{Линейно-связное топологическое пространство}
    
    Компактность и линейная связность являются топологическими инвариантами, поскольку они сохраняются при любом
    непрерывном отображении
    
    \subsection{Карта и атлас на топологическом пространстве}
    
    \textbf{Опр} \textit{$n$-мерная карта на топологическом пространстве}
    
    \textbf{Опр} \textit{Гомеоморфизм карты}
    
    \textbf{Опр} \textit{Район действия карты}
    
    \textbf{Опр} \textit{Область параметров карты}
    
    \textbf{Опр} \textit{Атлас на топологическом пространстве}
    
    \subsection{Общие (абстрактные) определения многообразия и гладкого многобразия}
    
    \textbf{Опр} \textit{$n$-мерное абстрактное многообразие}
    
    \textbf{Опр} \textit{Замена координат, отображение перехода, отображение склейки}
    
    \subsection{Классы гладкости $C^k$ отображений из одного гладкого многообразия в другое}
    
    \textbf{Опр} \textit{Гладкий диффеоморфизм}
    
    \textbf{Опр} \textit{Гладкий атлас}
    
    Атлас на многообразии, состоящий из одной карты, считается гладким
    
    \textbf{Опр} \textit{Эквивалентные гладкие атласы}
    
    \textbf{Опр} \textit{Гладкая структура, определяемая атласом}
    
    \textbf{Опр} \textit{Гладкое $n$-мерное многообразие}
    
    \textbf{Опр} \textit{Карта на гладком многообразии}
    
    \textbf{Опр} \textit{Локальная система координат карты}
    
    \textbf{Опр} \textit{Класс $C^k$-гладких отображений}
    
    \textbf{Опр} \textit{Координатное представление отображения}
    
    \textbf{Опр} \textit{Диффеоморфные гладкие многообразия}
    
    \section{Теорема о гладком атласе на гладком подмногообразии пространства $\mathbb{R}^N$.
    Достаточное условие гладкости подмногообразия пространства $\mathbb{R}^N$ в терминах карты}
    
    \subsection{Теорема о гладком атласе на гладком подмногообразии пространства $\mathbb{R}^N$}
    
    \textbf{Th} \textit{О гладком атласе на гладком подмногообразии}
    
    \begin{enumerate}
        \item По лемме, $\forall P$ найдётся карта на топологическом пространстве $M$, порождённая каноническим
        диффеоморфизмом, район действия которой содержит точку $P$.
        Семейство всех таких карт составляет атлас; покажем, что он гладкий
        \item Фиксируем $\forall P$, вводим новые обозначения и рассматриваем отображения замены координат
        \item Они будут состоять из суперпозиции гладких диффеоморфизмов, что докажет и диффеоморфность замены координат
    \end{enumerate}
    
    \subsection{Достаточное условие гладкости подмногообразия пространства $\mathbb{R}^N$ в терминах карты}
    
    \textbf{Опр} \textit{Порождённая каноническим диффеоморфизмом карта}
    
    \textbf{Опр} \textit{Порождённая каноническим диффеоморфизмом в некоторой окрестности точки карта}
    
    \textbf{Th.1} \textit{Достаточное условие гладкости подмногообразия в терминах карты.}
    
    \begin{enumerate}
        \item Считаем $V$ открытым подмножеством согласно лемме и воспользуемся теоремой о ранге матрицы
        \item От исходного отображения перейдём к $f(x)$ с новыми обозначениями и запишем Матрицу Якоби отображения
        \item Увидим, что в $x_0$ матрица невырождена, что позволяет использовать теорему об обратном отображении
        \item $M$ будет задано простой системой уравнений, то есть имеет вид подпространства или полуподпространства
        \item Теперь докажем, что $M$ подмногообразие.
        В случае внутренней точки рассматриваем сужения и пересечения, вводя новые обозначения.
        \item Осталось показать, что параметры находились в линейной части пространства.
        В случае внутренней точки доказываем сначала прямое, а потом и обратное включения с помощью шаманства
        \item Случай граничной точки следует заменой подпространства на полуподпространство
        \item Согласно определению отображения $f$, справедливо равенство, которое и завершает доказательство
    \end{enumerate}
    
    \section{Касательный вектор к абстрактному гладкому многообразию как оператор дифференцирования.
    Теорема о структуре множества $T_P (M)$ касательных векторов.
    Изменение координат касательного вектора при замене локальной системы координат}
    
    \subsection{Касательный вектор к абстрактному гладкому многообразию как оператор дифференцирования}
    
    \textbf{Опр} \textit{Производная функции по вектору в точке}
    
    \textbf{Опр} \textit{Касательный вектор}
    
    Оператор обладает свойством локальности
    
    \textbf{Обозначение} \textit{Множество всех касательных векторов}
    
    \textbf{Опр} \textit{Соответствующие абстрактный и обычный касательный векторы}
    
    \textbf{Опр} \textit{Координаты касательного вектора}
    
    Касательный вектор является линейным оператором.
    Это следует из свойства линейности и правила Лейбница для производной функции одной переменной
    
    \subsection{Теорема о структуре множества $T_P (M)$ касательных векторов}
    
    \textbf{Th} \textit{О структуре множества касательных векторов.}
    
    \begin{enumerate}
        \item Фиксируем произвольную ЛСК в окрестности точки и введём новые обозначения.
        \item Перейдём к равенствам для любой функции $f$.
        В итоге получим линейное пространство.
        \item Покажем, что коэффициенты разложения вектора по системе векторов определены однозначно
        Это следует из существования соответствующего геометрического касательного вектора
        \item Итого операторный набор составляет базис в $T_P (M)$
    \end{enumerate}
    
    \textbf{Опр} \textit{Производная функции по геометрическому касательному вектору}
    
    \textbf{Опр} \textit{Изоморфизмом линейных пространств}
    
    \textbf{Опр} \textit{Изоморфные линейные пространства}
    
    \subsection{Изменение координат касательного вектора при замене локальной системы координат}
    
    \textbf{Лемма}
    
    Доказывается записью вектора в двух базисах и с помощью теоремы о дифференцировании сложной функции
    
    \section{Край многообразия.
    Теорема о независимости краевой точки карты от карты}
    
    \subsection{Край многообразия}
    
    \textbf{Опр} \textit{Краем допустимой области параметров}
    
    Край, вообще говоря, не совпадает с границей множества, потому как граничные точки могут не принадлежать множеству
    
    \textbf{Опр} \textit{Краевая точка карты}
    
    \subsection{Теорема о независимости краевой точки карты от карты}
    
    \textbf{Th} \textit{О независимости краевой точки от карты.}
    
    \begin{enumerate}
        \item Докажем от противного: пусть краевая для одной карты и нет для другой
        \item $\exists$ две окрестности, операции с которым показывают, что $x_2$ внутренняя точка множества $V_2$.
        \item Сделаем замену координат, а потом и тождественное изображение.
        Получим невырожденность замены координат.
        \item Воспользуемся теоремой о неявной функции и получим окрестность точки $x_1$ в $V_1$ первой карты
        Также $X_k$ будут монотонны по включению
        \item Таким образом, точка $x_1$ не лежит на границе области, а значит, $P$ не краевая точка карты, противоречие
    \end{enumerate}
    
    \textbf{Опр} \textit{Край гладкого многообразия}
    
    \section{Ориентация гладкого многообразия.
    Существование ровно двух ориентаций линейно-связного ориентируемого многообразия}
    
    \subsection{Ориентация гладкого многообразия}
    
    \textbf{Опр} \textit{Согласованные (по ориентации) карты}
    
    \textbf{Опр} \textit{Ориентирующий атлас}
    
    \textbf{Опр} \textit{(Не)ориентируемое многообразие}
    
    \textbf{Опр} \textit{Согласованные атласы}
    
    \textbf{Опр} \textit{Ориентация многообразия}
    
    \textbf{Опр} \textit{Ориентированное многообразие}
    
    \textbf{Опр} \textit{Карты, соответствующие ориентации или согласованные с ориентацией гладкого многообразия}
    
    \subsection{Существование ровно двух ориентаций линейно-связного ориентируемого многообразия}
    
    \textbf{Th} \textit{О двух ориентациях многообразия.}
    
    \begin{enumerate}
        \item Возьмём ориентирующий атлас и сделаем из него атлас с противоположной ориентацией.
        Заметим, что допустимая область параметров таковой остаётся
        \item Таким образом, существуют по крайней мере две различные ориентации многообразия -- это класс всех
        атласов, согласованных с первым
        \item Покажем, что третьей ориентации многообразия не существует.
        Фиксируем ориентирующий атлас.
        \item Выберем $\forall P$ и проанализируем знак якобиана замены координат
        \item Знак якобиана непрерывно зависит от точки, притом не может обращаться в ноль как якобиан диффеоморфизма
        \item Итак, в зависимости от знака якобиана, кандидат согласован либо с первым, либо со вторым атласом
    \end{enumerate}
    
    \section{Ориентация гладкого $(N - 1)$-мерного подмногообразия пространства $\mathbb{R}^N$.
    Теорема о непрерывной нормали}
    
    \subsection{Ориентация гладкого $(N - 1)$-мерного подмногообразия пространства $\mathbb{R}^N$}
    
    \textbf{Опр} \textit{Единичный вектор нормали к многообразию}
    
    \textbf{Опр} \textit{Согласованные единичный вектор нормали и карта}
    
    \textbf{Опр} \textit{Согласованный с ориентацией многообразия вектор нормали}
    
    \subsection{Теорема о непрерывной нормали}
    
    \textbf{Th} \textit{О непрерывной нормали}
    
    \textcolor{blue}{Последовательность сходится равномерно $\Leftrightarrow$ выполняется условие Коши}
    
    \begin{enumerate}
        \item $\Rightarrow$: покажем непрерывность функции.
        Зафиксируем $\forall P_0$ и покажем, что частичный предел последовательности стремящихся точек единственен.
        \item Зафиксируем карту с $P_0$ и перейдём к пределу в скалярных произведениях и по модулю
        \item Условие согласованности нормали и карты $\Leftrightarrow \det > 0$.
        Перейдя к пределу, получим равенство пределов нормалей, притом функция нормали непрерывна в любой точке $M$
        \item $\Leftarrow$: перейдём к карте и возьмём правый базис (или изменим до правого)
        \item Пользуемся непрерывностью и линейной связностью для получения правого базиса в каждой точке
        \item Полученный атлас будет ориентирующим, так как все карты имеют правые тройки, поэтому они согласованы
    \end{enumerate}
    
    \section{Теорема о построении ориентирующего атласа для края многообразия на основе ориентирующего атласа
    исходного многообразия.
    Согласование ориентации гладкого многообразия и ориентации его края}
    
    \subsection{Теорема о построении ориентирующего атласа для края многообразия на основе ориентирующего атласа
    исходного многообразия}
    
    \textbf{Л2}
    
    \begin{enumerate}
        \item Зафиксируем гладкий атлас и перейдём от обычной окрестности к открытому шару или полушару
        \item Из предыдущей леммы $\Rightarrow \exists$ диффеоморфизм между область параметров и (полу)пространством
        \item Введём новые обозначения, получим гладкий атлас, а затем и гладкий атлас из (полу)пространства
    \end{enumerate}
    
    \textbf{Л3}
    
    \begin{enumerate}
        \item Пользуясь предыдущей леммой, получим гладкий атлас с нужной областью параметров
        \item У всех карт, ориентации которых не совпадают с ориентацией $M$, поменяем вторую координату (область
        параметров, заметим, не изменится)
    \end{enumerate}
    
    \textbf{Th} \textit{О построении ориентирующего атласа для края многообразия}
    
    \begin{enumerate}
        \item Сначала получим гладкий атлас на крае, а затем покажем, что $\forall$ две карты атласа согласованы
        \item От открытого образа перейдём к открытому прообразу и рассмотрим отображение замены координат.
        Покажем, что якобиан сужения на крае положителен (то есть атлас ориентирующий)
        \item Зафиксируем краевую точку и посчитаем частные производные по координатам.
        \item Перейдём к якобианам и получим, что атлас на краю действительно ориентирующий (исходный таковой по условию)
    \end{enumerate}
    
    \subsection{Согласование ориентации гладкого многообразия и ориентации его края}
    
    \textbf{Опр} \textit{Согласованная ориентация края и многообразия}
    
    \textbf{Опр} \textit{Перенос ориентации между многообразиями}
    
    \section{Тензорное поле на многообразии.
    Изменение компонент тензорного поля при замене локальной системы координат.
    Выражение тензорного поля через его компоненты с помощью операции тензорного произведения}
    
    \subsection{Тензорное поле на многообразии}
    
    \textbf{Опр} \textit{Тензор на пространстве}
    
    \textbf{Опр} \textit{Компоненты (координаты) тензора}
    
    \textbf{Опр} \textit{Тензорное произведение тензора}
    
    Оно не коммутативно
    
    \textbf{Опр} \textit{Векторное поле}
    
    \textbf{Опр} \textit{Ковекторное поле}
    
    \textbf{Опр} \textit{Скалярное поле}
    
    \textbf{Опр} \textit{Тензорное поле}
    
    \textbf{Опр} \textit{Компоненты тензорного поля}
    
    \subsection{Изменение компонент тензорного поля при замене локальной системы координат}
    
    \textbf{Th} \textit{О законе изменения компонент тензорного поля при замене ЛСК}
    
    Перейдём к другим координатам по формуле, подставим это в старый тензор, вынесем константы и получим новую формулу
    
    \subsection{Выражение тензорного поля через его компоненты с помощью операции тензорного произведения}
    
    \textbf{Опр} \textit{Сумма двух тензорных полей (одинакового типа)}
    
    \textbf{Опр} \textit{Произведение тензорного и скалярного полей}
    
    \textbf{Опр} \textit{Тензорное произведение тензорных полей}
    
    \textbf{Th}
    
    \section{Дифференциальные формы на гладком многообразии, их представление через внешнее произведение
    дифференциалов координатных функций.
    Внешний дифференциал дифференциальной формы, его независимость от локальной системы координат}
    
    \subsection{Дифференциальные формы на гладком многообразии, их представление через внешнее произведение
    дифференциалов координатных функций}
    
    \textbf{Опр} \textit{Внешняя форма}
    
    \textbf{Опр} \textit{Перестановка чисел}
    
    \textbf{Опр} \textit{Транспозиция}
    
    \textbf{Опр} \textit{Знак перестановки}
    
    \textbf{Опр} \textit{Перестановка набора элементов}
    
    \textbf{Опр} \textit{Альтернирование тензора}
    
    \textbf{Опр} \textit{Внешнее произведение внешних форм}
    
    \textbf{Опр} \textit{Дифференциальная форма}
    
    \textbf{Опр} \textit{Множество гладких дифференциальных форм}
    
    \textbf{Опр} \textit{Внешнее произведение дифференциальных форм}
    
    \textbf{Опр} \textit{Моном}
    
    \textbf{Опр} \textit{Внешний дифференциал}
    
    Из определения внешнего дифференциала и линейности дифференциала скалярного поля следует линейность внешнего
    дифференциала формы
    
    \subsection{Внешний дифференциал дифференциальной формы, его независимость от локальной системы координат}
    
    \textbf{Th} \textit{Об инвариантности внешнего дифференциала формы относительно ЛСК}
    
    \begin{enumerate}
        \item Сначала рассмотрим скалярное поле ($q = 0$), для которого утверждение верно из прошлых семестров
        \item В случае $q = 1$ покажем, что мы доказываем и заменим компоненты тензорного поля при замене ЛСК.
        \item Сведём к доказываемому, преобразуем и проверим требуемое равенство
        \item Переобозначим индексы суммирования, воспользуемся кососимметричностью и получим требуемое
        \item Случай $q > 1$ рассматривается аналогично
    \end{enumerate}
    
    \section{Правило Лейбница для внешнего дифференциала внешнего произведения двух дифференциальных форм}
    
    \textbf{Th} \textit{Правило Лейбница для внешнего дифференциала.}
    
    \begin{enumerate}
        \item Запишем общий вид дифференциальной формы.
        В силу линейности достаточно рассмотреть суммы из одного слагаемого.
        \item Используем определение внешнего дифференциала и правило Лейбница для скалярных полей
        \item Далее используем свойство дифференциальных форм по перестановкам и получаем требуемое
    \end{enumerate}
    
    \section{Перенос касательных векторов.
    Выражение для переноса базисного вектора касательного пространства через частные производные координатных функций
    отображения}
    
    \subsection{Перенос касательных векторов}
    
    \textbf{Опр} \textit{Прямой перенос}
    
    \subsection{Выражение для переноса базисного вектора касательного пространства через частные производные
    координатных функций отображения}

    \textbf{Th} \textit{Выражение для прямого переноса базисного вектора касательного пространства через частные
    производные координатных функций отображения.}
    
    \begin{enumerate}
        \item Запишем переход к другой системе координат
        \item Последовательно используем определение прямого переноса, требованием, чтобы функция зависела от
        аргумента производной при дифференцировании, производной сложной функции и ещё раз определением прямого переноса
    \end{enumerate}
    
\end{document}
