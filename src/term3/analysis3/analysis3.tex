%! Author = user
%! Date = 28.12.2023

\documentclass[a4paper, 14pt]{article}
%\documentclass[draft]{article}

\usepackage[T2A]{fontenc}
\usepackage[utf8]{inputenc}
\usepackage[english, russian]{babel}
\usepackage[top = 2cm, bottom = 2cm, left = 2cm, right = 2cm]{geometry}
\usepackage{indentfirst}
\usepackage{xcolor}
\usepackage{hyperref}
\usepackage{gensymb}
\usepackage{pgfplots}
\usepackage{amsmath, amsfonts, amsthm, mathtools}
\usepackage{amssymb}
\usepackage{physics, multirow, float}
\usepackage{wrapfig, tabularx}
\usepackage{icomma} % Clever comma: 0,2 - number while 0, 2 - two numbers
\usepackage{tikz, standalone}
\usepackage{fancyhdr,fancybox}
\usepackage{booktabs}
\usepackage{listings}
\usepackage{lstmisc}
\usepackage{stmaryrd}

%\полуторный интервал
\onehalfspacing

\hypersetup
{   colorlinks = false,
    linkcolor = blue,
    pdftitle = {analysis},
    pdfauthor = {Володин Максим},
    allcolors = [RGB]{010 090 200}
}

%\gravarphicspath{{images/}}
%\DeclareGravarphicsExtensions{.pdf,.png,.jpg}

\restylefloat{table}
\usetikzlibrary{external}

\mathtoolsset{showonlyrefs = true} % Numbers will appear only where \eqref{} in the text LINKED
\pagestyle{fancy}

\fancyhf{}
\fancyhead[L]{КИТП}
\fancyhead[R]{Конспект билетов}
\fancyfoot[L]{}
\usepackage{lastpage}
\fancyfoot[R]{\thepage /\pageref{LastPage}}


\pgfplotsset{compat=1.18}

\begin{document}
    
    \tableofcontents \newpage
    
    \section{Теорема о выражении меры множества через интеграл от меры сечений.
    Теорема Фубини}
    
    \subsection{Теорема о выражении меры множества через интеграл от меры сечений}
    
    \textbf{Л1}
    
    \textcolor{blue}{Пусть есть счётный набор конечно измеримых убывающих вложенных множеств $X_i$.
    Мера множества, являющегося счётным пересечением есть предел мер}
    
    \textbf{Л2}
    
    \textcolor{blue}{Если интеграл неотрицательной функции по множеству равен нулю, то сама функция равна нулю почти
    на всём множестве}
    
    \textbf{Th} \textit{О выражении меры множества через интеграл от меры сечений}
    
    \begin{enumerate}
        \item Для начала докажем для клетки.
        Её сечение будет принимать простой вид, в зависимости от принадлежности $x$, что даёт простое
        интегрирование и доказывает теорему
        \item Теперь докажем для счётного набора (объединения) клеток.
        Задача сводится к предыдущей.
        \item Найдём меру множества, являющегося счётным пересечением объединения счётного числа клеток
        \item Определим убывающую последовательность множеств
        \item Далее считаем меры составляющих, по ходу дела используя теорему Лебега об ограниченной сходимости
        \item Теперь рассмотрим случай множества нулевой меры и, с помощью всяких сравнений и пределов, докажем
        требуемое.
        \item В конце рассмотрим общий случай конечно измеримого множества.
        \item Используем все предыдущие леммы и случаи и получаем требуемое при почти всех $x$
    \end{enumerate}
    
    \subsection{Теорема Фубини}
    
    \textbf{Th} \textit{О геометрическом смысле интеграла}
    
    \textbf{Th} \textit{Фубини}
    
    \section{Теорема о замене переменных в кратном интеграле}
    
    \textbf{Опр} \textit{$C^k$-гладкий диффеоморфизм}
    
    \textbf{Опр} \textit{Носитель функции}
    
    \textbf{Th} \textit{О замене переменных в кратном интеграле}
    
    \textbf{Л2}
    \textcolor{blue}{Теорема справедлива, если функция $f$ непрерывна на $Y$, а её носитель компактен и лежит в  $Y$}
    
    \begin{enumerate}
        \item Убрав условие 3, мы сделали теорему локальной (для каждой точки существует окрестность, где выполнено
        условие 3)
        \item Воспользуемся теоремой о расщеплении отображений, о неявной функции, критерием компактности, теоремой о
        разбиении единицы
        \item Это позволяет разбить функцию на сумму.
        Утверждение для фиксированного индекса (на его области значений) верно по предыдущей лемме
    \end{enumerate}
    
    \textbf{Л4}
    \textcolor{blue}{Теорема справедлива, если функция $f$ непрерывна на $Y$}
    
    \begin{enumerate}
        \item Рассмотрим неотрицательно значные функции и введём хитрые множества $Y_k$ и функции $f_k$
        \item Докажем, что $\supp \subset f_k$ исходя из определения $f_k$.
        Получили ограниченность и замкнутость $\supp f_k$
        \item Из построения множеств следуют включения, а за ними и неравенства
        \item Теперь покажем, что $f_k$ стремятся к $f$ через определения и построения условий.
        \item Запишем следствия из предела и перейдём и завершим доказательство с помощью теоремы Б.~Леви
        \item В общем случае разобьём $f$ на $f_+$ и $f_-$ и получим искомое равенство
    \end{enumerate}
    
    \section{Теорема о построении криволинейной системы координат исходя из её части}
    
    \textbf{Опр} \textit{Криволинейная система координат на множестве $A$}
    
    \textbf{Опр} \textit{Координатный набор}
    
    \textbf{Th} \textit{О построении криволинейной системы координат исходя из ее части.}
    
    \begin{enumerate}
        \item Рассмотрим отображение из известного набора функций и матрицу Якоби этого отображения.
        \item Рассмотрим координатные строки и матрицу в точке и применим теорему о ранге матрицы
        \item Определим новые гладкие функции и всеобъемлющее отображение, рассмотрим новую матрицу Якоби
        \item Применим теорему об обратном отображении и получим требуемое
    \end{enumerate}
    
    \section{Гладкие подмногообразия пространства $R^N$.
    Теорема о гладком подмногообразии пространства $R^N$, заданном системой уравнений}
    
    \subsection{Гладкие подмногообразия пространства $R^N$}
    
    \textbf{Опр} \textit{Гладкое $n$-мерное подмногообразием пространства $\mathbb{R}^N_p$ в точке $P \in M$}
    
    \textbf{Опр} \textit{Канонический и выпрямляющий диффеоморфизм}
    
    \textbf{Утв} \textit{Гладкое $n$-мерное подмногообразие пространства}
    
    \subsection{Теорема о гладком подмногообразии пространства $R^N$, заданном системой уравнений}
    
    \textbf{Th} \textit{О гладком подмногообразии, заданном системой уравнений}
    
    \begin{enumerate}
        \item Сначала достроим отображения до гладкого диффеоморфизма по теореме о построении криволинейной системы
        координат, исходя из её части
        \item Докажем, что выпрямляемость обратного диффеоморфизма.
        Это делается через анализ множеств и из их свойств
    \end{enumerate}
    
    \section{Геометрический касательный вектор к подмножеству пространства $\mathbb{R}^n$.
    Теоремы о структуре множества $T_P (M)$ геометрических касательных векторов к гладкому подмногообразию $M$
        пространства $\mathbb{R}^n$ в общем случае и в случае, когда $M$ заданно системой уравнений}
    
    \subsection{Геометрический касательный вектор к подмножеству пространства $\mathbb{R}^n$}
    
    \textbf{Опр} \textit{Геометрический касательный вектор к множеству в точке}
    
    \textbf{Опр} \textit{Геометрическое касательное пространство}
    
    \subsection{Теоремы о структуре множества $T_P (M)$ геометрических касательных векторов к гладкому
    подмногообразию $M$ пространства $\mathbb{R}^n$ в общем случае и в случае, когда $M$ заданно системой уравнений}
    
    \textbf{Th.1} \textit{О структуре множества геометрических касательных векторов к гладкому подмногообразию.}
    
    \begin{enumerate}
        \item Воспользуемся определением канонического диффеоморфизма, леммой о $T_P (M)$ к линейному пространству и
        о локальности $T_P (M)$.
        \item Запишем вид $T_P (M)$ и перейдём к локальной параметризации (из-за правил умножения матриц)
        \item Итого, касательные векторы есть л.к. столбцов матрицы Якоби, а само $T_P (M)$ является $n$-мерным
        линейным подпространством $\mathbb{R}^N$
    \end{enumerate}
    
    \textbf{Th. 2} \textit{О структуре множества геометрических касательных векторов к подмногообразию, заданному
    системой уравнений.}
    
    \begin{enumerate}
        \item Пишем те же рассуждениями, что и в теореме о гладком подмногообразии, заданном системой уравнений.
        \item Введём новую переменную и воспользуемся предыдущей теоремой
    \end{enumerate}
    
    \section{Необходимые условия безусловного экстремума.
    Достаточные условия безусловного экстремума}
    
    \subsection{Необходимые условия безусловного экстремума}
    
    \textbf{Опр} \textit{Точка экстремума}
    
    \textbf{Опр} \textit{Точка (не)строгого локального минимума (максимума) функции на множестве}
    
    \textbf{Th} \textit{Необходимое условие экстремума}
    
    Для доказательства воспользуемся определением градиента и рассмотрим функцию одной переменной, где применим теорему
    Ферма
    
    \subsection{Достаточные условия безусловного экстремума}
    
    \textbf{Опр} \textit{Стационарная точка}
    
    \textbf{Th} \textit{Достаточные условия экстремума.}
    
    \begin{enumerate}
        \item Разложим $\Delta f$ по формуле Тейлора с использованием определения стационарной точки.
        \item Воспользуемся леммой и определением $o$-малого и предела
        \item Теперь перепишем $\Delta f$ и получим требуемое.
        В случае отрицательной определённости рассуждения аналогичны ($f$ меняется на $-f$)
        \item В знаконеопределённом случае рассмотрим выделенные направления и запишем $\Delta f$.
        \item Воспользуемся определением предела и выберем достаточно малые $t_i$ для построения противоречия
        \item В последнем случае приводятся два контрпримера $f = x^4$ и $f = x^3$
    \end{enumerate}
    
    \section{Метод Лагранжа нахождения точек условного экстремума.
    Необходимые условия условного экстремума.
    Достаточные условия условного экстремума}
    
    \subsection{Метод Лагранжа нахождения точек условного экстремума}
    
    \textbf{Опр} \textit{Функция Лагранжа}
    
    \textbf{Опр} \textit{Множители Лагранжа}
    
    \subsection{Необходимые условия условного экстремума}
    
    \textbf{Th} \textit{Необходимые условия экстремума}
    
    Сделаем общие построения для доказательства теорем.
    
    \begin{enumerate}
        \item Воспользуемся теоремой о построении криволинейной системы координат, исходя из её части и введём новые
        обозначения для обратных функций
        \item Тогда можно расписать обратную функцию Лагранжа и показать, что новая задача эквивалентна старой
    \end{enumerate}
    
    Теперь докажем саму теорему.
    
    \begin{enumerate}
        \item Воспользуемся теоремой о необходимом условии безусловного экстремума и выберем специальные множители
        Лагранжа.
        \item Вернёмся к исходным переменным, воспользовавшись теоремой о дифференцировании сложной функции и значениями
        $\frac{\partial L}{\partial \lambda_i}$
        \item Вышеперечисленное показывает, что $x_0$ -- стационарная точка функции Лагранжа, как и нашей исходной функции
    \end{enumerate}
    
    \subsection{Достаточные условия условного экстремума}
    
    \textbf{Th} \textit{Достаточные условия экстремума.}
    
    \begin{enumerate}
        \item Перейдём к обратной функции Лагранжа, для которой точка $y_0$ стационарна
        \item Исходная точка будет стационарной, если в этой точке обратная функция совпадёт с обратной функцией
        Лагранжа, то есть $k'$ для обратной функции Лагранжа будет отрицательная определена
        \item Покажем, что это эквивалентно отрицательной определённости $k$.
        Для этого воспользуемся инвариантностью первого дифференциала, распишем второй и воспользуемся
        стационарностью точки
        \item Согласно теореме о структуре множества геометрических касательных векторов к подмногообразию, заданному
        системой уравнений, получаем эквивалентность структур форм, что нам и требуется
    \end{enumerate}
    
    \section{Топологическое пространство.
    Индуцированная топология.
    Карта и атлас на топологическом пространстве.
    Общие (абстрактные) определения многообразия и гладкого многобразия.
    Классы гладкости $C^k$ отображений из одного гладкого многообразия в другое}
    
    \subsection{Топологическое пространство}
    
    \textbf{Опр} \textit{Топологическое пространство}
    
    \textbf{Опр} \textit{Топология}
    
    \textbf{Опр} \textit{Открытое в топологическом пространстве множество}
    
    \textbf{Опр} \textit{Семейство всех открытых подмножеств метрического пространства с метрикой}
    
    \textbf{Опр} \textit{Окрестность точки}
    
    \textbf{Опр} \textit{Внутренность множества}
    
    \textbf{Опр} \textit{Замыкание множества}
    
    \subsection{Индуцированная топология}
    
    \textbf{Опр} \textit{Индуцированная топология}
    
    \textbf{Л1}
    
    Доказывается по определению, с привлечением старых множеств, породивших новую топологию
    
    \textbf{Опр} \textit{Хаусдорфово топологическое пространство}
    
    \textbf{Опр} \textit{Предел по топологическому пространству}
    
    Заметим, что у хаусдорфова пространства не может быть двух различных пределов; иначе может
    
    \textbf{Л2}
    
    Доказывается выбором специальных окрестностей, которые не пересекаются
    
    \textbf{Опр} \textit{База топологического пространства}
    
    \textbf{Л3}
    
    Возьмём пересечения открытых шаров с рациональными радиусами и координатами
    
    \textbf{Опр} \textit{Непрерывное отображение}
    
    \textbf{Опр} \textit{Секвенциально непрерывное в точке отображение}
    
    \textbf{Опр} \textit{(Секвенциально) непрерывное в точке отображение}
    
    \textbf{Опр} \textit{Гомеоморфизм}
    
    \textbf{Л4}
    
    Докажем от частного к общему с помощью непрерывности и открытости объединения открытых
    
    \textbf{Опр} \textit{Компактное топологическое пространство}
    
    \textbf{Опр} \textit{Секвенциально компактное топологическое пространство}
    
    \textbf{Опр} \textit{Секвенциально компактное множество}
    
    \textbf{Л5}
    
    Возьмём открытое покрытие множества, перейдём к прообразам, выберем там открытое покрытие и конечное подпокрытие
    
    \textbf{Опр} \textit{Гомеоморфные множества}
    
    \textbf{Опр} \textit{Топологический инвариант}
    
    \textbf{Опр} \textit{Линейно-связное топологическое пространство}
    
    Компактность и линейная связность являются топологическими инвариантами, поскольку они сохраняются при любом
    непрерывном отображении
    
    \subsection{Карта и атлас на топологическом пространстве}
    
    \textbf{Опр} \textit{$n$-мерная карта на топологическом пространстве}
    
    \textbf{Опр} \textit{Гомеоморфизм карты}
    
    \textbf{Опр} \textit{Район действия карты}
    
    \textbf{Опр} \textit{Область параметров карты}
    
    \textbf{Опр} \textit{Атлас на топологическом пространстве}
    
    \subsection{Общие (абстрактные) определения многообразия и гладкого многобразия}
    
    \textbf{Опр} \textit{$n$-мерное абстрактное многообразие}
    
    \textbf{Опр} \textit{Замена координат, отображение перехода, отображение склейки}
    
    \subsection{Классы гладкости $C^k$ отображений из одного гладкого многообразия в другое}
    
    \textbf{Опр} \textit{Гладкий диффеоморфизм}
    
    \textbf{Опр} \textit{Гладкий атлас}
    
    Атлас на многообразии, состоящий из одной карты, считается гладким
    
    \textbf{Опр} \textit{Эквивалентные гладкие атласы}
    
    \textbf{Опр} \textit{Гладкая структура, определяемая атласом}
    
    \textbf{Опр} \textit{Гладкое $n$-мерное многообразие}
    
    \textbf{Опр} \textit{Карта на гладком многообразии}
    
    \textbf{Опр} \textit{Локальная система координат карты}
    
    \textbf{Опр} \textit{Класс $C^k$-гладких отображений}
    
    \textbf{Опр} \textit{Координатное представление отображения}
    
    \textbf{Опр} \textit{Диффеоморфные гладкие многообразия}
    
    \section{Теорема о гладком атласе на гладком подмногообразии пространства $\mathbb{R}^N$.
    Достаточное условие гладкости подмногообразия пространства $\mathbb{R}^N$ в терминах карты}
    
    \subsection{Теорема о гладком атласе на гладком подмногообразии пространства $\mathbb{R}^N$}
    
    \textbf{Th} \textit{О гладком атласе на гладком подмногообразии}
    
    \begin{enumerate}
        \item По лемме, $\forall P$ найдётся карта на топологическом пространстве $M$, порождённая каноническим
        диффеоморфизмом, район действия которой содержит точку $P$.
        Семейство всех таких карт составляет атлас; покажем, что он гладкий
        \item Фиксируем $\forall P$, вводим новые обозначения и рассматриваем отображения замены координат
        \item Они будут состоять из суперпозиции гладких диффеоморфизмов, что докажет и диффеоморфность замены координат
    \end{enumerate}
    
    \subsection{Достаточное условие гладкости подмногообразия пространства $\mathbb{R}^N$ в терминах карты}
    
    \textbf{Опр} \textit{Порождённая каноническим диффеоморфизмом карта}
    
    \textbf{Опр} \textit{Порождённая каноническим диффеоморфизмом в некоторой окрестности точки карта}
    
    \textbf{Th.1} \textit{Достаточное условие гладкости подмногообразия в терминах карты.}
    
    \begin{enumerate}
        \item Считаем $V$ открытым подмножеством согласно лемме и воспользуемся теоремой о ранге матрицы
        \item От исходного отображения перейдём к $f(x)$ с новыми обозначениями и запишем Матрицу Якоби отображения
        \item Увидим, что в $x_0$ матрица невырождена, что позволяет использовать теорему об обратном отображении
        \item $M$ будет задано простой системой уравнений, то есть имеет вид подпространства или полуподпространства
        \item Теперь докажем, что $M$ подмногообразие.
        В случае внутренней точки рассматриваем сужения и пересечения, вводя новые обозначения.
        \item Осталось показать, что параметры находились в линейной части пространства.
        В случае внутренней точки доказываем сначала прямое, а потом и обратное включения с помощью шаманства
        \item Случай граничной точки следует заменой подпространства на полуподпространство
        \item Согласно определению отображения $f$, справедливо равенство, которое и завершает доказательство
    \end{enumerate}
    
    \section{Касательный вектор к абстрактному гладкому многообразию как оператор дифференцирования.
    Теорема о структуре множества $T_P (M)$ касательных векторов.
    Изменение координат касательного вектора при замене локальной системы координат}
    
    \subsection{Касательный вектор к абстрактному гладкому многообразию как оператор дифференцирования}
    
    \textbf{Опр} \textit{Производная функции по вектору в точке}
    
    \textbf{Опр} \textit{Касательный вектор}
    
    Оператор обладает свойством локальности
    
    \textbf{Обозначение} \textit{Множество всех касательных векторов}
    
    \textbf{Опр} \textit{Соответствующие абстрактный и обычный касательный векторы}
    
    \textbf{Опр} \textit{Координаты касательного вектора}
    
    Касательный вектор является линейным оператором.
    Это следует из свойства линейности и правила Лейбница для производной функции одной переменной
    
    \subsection{Теорема о структуре множества $T_P (M)$ касательных векторов}
    
    \textbf{Th} \textit{О структуре множества касательных векторов.}
    
    \begin{enumerate}
        \item Фиксируем произвольную ЛСК в окрестности точки и введём новые обозначения.
        \item Перейдём к равенствам для любой функции $f$.
        В итоге получим линейное пространство.
        \item Покажем, что коэффициенты разложения вектора по системе векторов определены однозначно
        Это следует из существования соответствующего геометрического касательного вектора
        \item Итого операторный набор составляет базис в $T_P (M)$
    \end{enumerate}
    
    \textbf{Опр} \textit{Производная функции по геометрическому касательному вектору}
    
    \textbf{Опр} \textit{Изоморфизмом линейных пространств}
    
    \textbf{Опр} \textit{Изоморфные линейные пространства}
    
    \subsection{Изменение координат касательного вектора при замене локальной системы координат}
    
    \textbf{Лемма}
    
    Доказывается записью вектора в двух базисах и с помощью теоремы о дифференцировании сложной функции
    
    \section{Край многообразия.
    Теорема о независимости краевой точки карты от карты}
    
    \subsection{Край многообразия}
    
    \textbf{Опр} \textit{Краем допустимой области параметров}
    
    Край, вообще говоря, не совпадает с границей множества, потому как граничные точки могут не принадлежать множеству
    
    \textbf{Опр} \textit{Краевая точка карты}
    
    \subsection{Теорема о независимости краевой точки карты от карты}
    
    \textbf{Th} \textit{О независимости краевой точки от карты.}
    
    \begin{enumerate}
        \item Докажем от противного: пусть краевая для одной карты и нет для другой
        \item $\exists$ две окрестности, операции с которым показывают, что $x_2$ внутренняя точка множества $V_2$.
        \item Сделаем замену координат, а потом и тождественное изображение.
        Получим невырожденность замены координат.
        \item Воспользуемся теоремой о неявной функции и получим окрестность точки $x_1$ в $V_1$ первой карты
        Также $X_k$ будут монотонны по включению
        \item Таким образом, точка $x_1$ не лежит на границе области, а значит, $P$ не краевая точка карты, противоречие
    \end{enumerate}
    
    \textbf{Опр} \textit{Край гладкого многообразия}
    
    \section{Ориентация гладкого многообразия.
    Существование ровно двух ориентаций линейно-связного ориентируемого многообразия}
    
    \subsection{Ориентация гладкого многообразия}
    
    \textbf{Опр} \textit{Согласованные (по ориентации) карты}
    
    \textbf{Опр} \textit{Ориентирующий атлас}
    
    \textbf{Опр} \textit{(Не)ориентируемое многообразие}
    
    \textbf{Опр} \textit{Согласованные атласы}
    
    \textbf{Опр} \textit{Ориентация многообразия}
    
    \textbf{Опр} \textit{Ориентированное многообразие}
    
    \textbf{Опр} \textit{Карты, соответствующие ориентации или согласованные с ориентацией гладкого многообразия}
    
    \subsection{Существование ровно двух ориентаций линейно-связного ориентируемого многообразия}
    
    \textbf{Th} \textit{О двух ориентациях многообразия.}
    
    \begin{enumerate}
        \item Возьмём ориентирующий атлас и сделаем из него атлас с противоположной ориентацией.
        Заметим, что допустимая область параметров таковой остаётся
        \item Таким образом, существуют по крайней мере две различные ориентации многообразия -- это класс всех
        атласов, согласованных с первым
        \item Покажем, что третьей ориентации многообразия не существует.
        Фиксируем ориентирующий атлас.
        \item Выберем $\forall P$ и проанализируем знак якобиана замены координат
        \item Знак якобиана непрерывно зависит от точки, притом не может обращаться в ноль как якобиан диффеоморфизма
        \item Итак, в зависимости от знака якобиана, кандидат согласован либо с первым, либо со вторым атласом
    \end{enumerate}
    
    \section{Связь поточечной и равномерной сходимостей для функциональной последовательности.
    Критерий Коши равномерной сходимости функциональной последовательности.
    Обобщенный признак сравнения для функциональных рядов.
    Признак Вейерштрасса равномерной сходимости функционального ряда.
    Признаки Дирихле и Лейбница равномерной сходимости функционального ряда.
    Признак Абеля равномерной сходимости функционального ряда.
    Непрерывность равномерного предела, непрерывных функций и суммы равномерно сходящегося функционального ряда с
    непрерывными слагаемыми.
    Почленное интегрирование функциональных последовательностей и рядов.
    Дифференцирование предельной функции и почленное дифференцирование функционального ряда}
    
    \subsection{Связь поточечной и равномерной сходимостей для функциональной последовательности}

\textbf{Опр} \textit{Поточечный предел функциональной последовательности} \textcolor{gray}{Предел в привычном
понимании}

\textbf{Опр} \textit{Равномерный предел функциональной последовательности} \textcolor{gray}{$N \in \mathbb{N}$ не зависит от аргумента}

Из равномерной сходимости следует поточечная, но не наоборот

\textbf{Опр} \textit{Равномерно ограниченная функциональная последовательность} \textcolor{gray}{$N \in \mathbb{N}$
    не зависит от аргумента}

\subsection{Критерий Коши равномерной сходимости функциональной последовательности}

\textbf{Th} \textit{Критерий Коши}

\textcolor{blue}{Последовательность сходится равномерно $\Leftrightarrow$ выполняется условие Коши}

\begin{enumerate}
    \item $\Rightarrow$: дважды применить определение равномерной сходимости и воспользоваться неравенством
    треугольника
    \item $\Leftarrow$: требуется доказать равномерную сходимость из выполнения условия Коши числовой
    последовательности для любого фиксированного $x \in X$.
    В силу КК для числовой последовательности $\lim_{k\to\infty} f_k = f$
    \item Далее надо в силу $\forall p \in \mathbb{N}$ устремить его к $+\infty$ и по теореме о предельном
    переходе в неравенствах получить определение равномерной сходимости
\end{enumerate}

\subsection{Обобщенный признак сравнения для функциональных рядов}

\textbf{Опр} \textit{Поточечный предел функционального ряда} \textcolor{gray}{Сходимость ряда в привычном понимании}

\textbf{Опр} \textit{Равномерный предел функционального ряда} \textcolor{gray}{Если последовательность его
частичных сумм сходится равномерно на том же множестве}

\textbf{Опр} \textit{Остаток поточечно сходящегося функционального ряда} \textcolor{gray}{Разность суммы и
частичной суммы ряда}

\textbf{Th} \textit{Обобщенный признак сравнения}

\textcolor{blue}{Если каждый член нашего ряда по модулю не превосходит члена равномерно сходящегося на том же
множестве ряда, то и наш ряд сходится равномерно}

Доказательство состоит в двукратном применении КК

Из признака следует, что из равномерной абсолютной сходимости ряда следует равномерная сводимость ряда на том же
множестве

\subsection{Признак Вейерштрасса равномерной сходимости функционального ряда}

\textbf{Th} \textit{Признак Вейерштрасса}

\textcolor{blue}{Если каждый член нашего ряда по модулю не превосходит члена сходящегося ряда, то наш ряд
сходится равномерно на том же множестве}

Доказательство состоит в применении обобщенного признака сравнения.
Заметьте, что мы не требуем равномерной сходимости от ряда-мажоранты

\subsection{Признаки Дирихле и Лейбница равномерной сходимости функционального ряда}

Смотреть в рукописном конспекте

\subsection{Признак Абеля равномерной сходимости функционального ряда}

Смотреть в рукописном конспекте

\subsection{Непрерывность равномерного предела, непрерывных функций и суммы равномерно сходящегося функционального ряда с
непрерывными слагаемыми}

\textbf{Th.1} \textit{О непрерывности предельной функции}

\textcolor{blue}{Если последовательность $f_k$ непрерывных на множестве $X$ функций сходится равномерно на
множестве $X$, то $f$ непрерывна на $X$}

\begin{enumerate}
    \item Зафиксируем $\forall \varepsilon > 0$ и $x_0 \in X$
    \item Далее для доказательства достаточно дважды записать определения равномерной сходимости и
    один раз непрерывности функции $f_N (x)$ для нужных долей  $\varepsilon$ и воспользоваться неравенством
    треугольника
\end{enumerate}

\textbf{Th.2} \textit{О непрерывности суммы ряда}

\textcolor{blue}{Если функциональный ряд $u_k$ непрерывных на множестве $X$ функций сходится
равномерно на множестве $X$, то сумма ряда непрерывна на $X$}

Доказательство состоит в применение Th.1 последовательности частичных сумм ряда

\subsection{Почленное интегрирование функциональных последовательностей и рядов}

\textbf{Th.1} \textit{Об интегрировании предельной функции}

\textcolor{blue}{Если последовательность $f_k$ интегрируемых на конечно измеримом множестве $X$ функций сходится
равномерно на множестве $X$ к интегрируемой функции $f$, то интеграл этой функции есть предел интегралов}

\begin{enumerate}
    \item Воспользуемся sup-критерием для $\varepsilon = 1$.
    Тогда из неравенства следует интегрируемость $f$ по признаку сравнения
    \item Расписав супремум для разности интегралов в пределе получим 0, что завершает доказательство
\end{enumerate}

\textbf{Следствие} \textcolor{blue}{Если последовательность непрерывных на компакте $X$ функций $f_k$ сходится
равномерно к функции $f$, то интеграл этой функции есть предел интегралов} \\

Непрерывность $f$ следует из теоремы предыдущей темы, а интегрируемость из достаточного условия интегрируемости,
что позволяет применить предыдущую теорему и доказать утверждение \\

\textbf{Th.2} \textit{Об почленном интегрировании ряда}

\textcolor{blue}{Если функциональный ряд $u_k$ непрерывных на компакте $X$ функций сходится
равномерно, то сумма интеграла есть интеграл суммы} \\

Доказательство состоит в применение следствия из предыдущей теоремы к последовательности частичных сумм ряда с
использованием линейности интеграла

\subsection{Дифференцирование предельной функции и почленное дифференцирование функционального ряда}

\textbf{Th.1} \textit{О дифференцировании предельной функции}

\textcolor{blue}{Если последовательность $f_k$ непрерывно дифференцируемых на отрезке $[a, b]$ функций сходится
хотя бы в одной точке $x_0$, а последовательность производных $f_k^{'}$ сходится равномерно на $[a, b]$, то
последовательность $f_k$ сходится равномерное на $[a, b]$ к некоторой непрерывно дифференицируемой функции $f$,
    притом производная предела есть предел производных}

\begin{enumerate}
    \item Обозначим предельную функцию для $f_k^{'}$ за $\varphi (x)$, непрерывную по теореме, и предел $f_k(x_0)$
    за $A$
    \item Далее определим $f(x) =  A + \int_{x_0}^x \varphi (t) dt$ и $f_k (x) = f_k (x_0) + \int_{x_0}^x f_k^{'}(
    t) dt$
    \item Затем пара хитрых замечаний, работа с супремумом, использование sup-критерия
    \item В итоге получаем равномерную сходимость $f_k$ и требуемое равенство с учётом построения $f(x)$
\end{enumerate}

\textbf{Th.2} \textit{О почленном дифференцировании ряда} \\

\textcolor{blue}{Если функциональный ряд $u_k$ непрерывно дифференцируемых на отрезке $[a, b]$ функций сходится
хотя бы в одной точке $x_0$, а ряд производных $u_k^{'}$ сходится равномерно на $[a, b]$, то
справделива формула почленного дифференицрования ряда, то есть производная суммы ряда есть сумма производных}

Доказательство состоит в применение Th.1 к последовательности частичных сумм ряда
    
    \section{Степенные ряды.
    Формула Коши-Адамара для радиуса сходимости.
    Теорема о круге сходимости степенного ряда.
    Первая теорема Абеля.
    Теорема о равномерной сходимости степенного ряда.
    Вторая теорема Абеля.
    Сохранение радиуса сходимости при почленном дифференцировании степенного ряда.
    Теоремы о почленном интегрировании и дифференцировании степенного ряда.
    Единственность разложения функции в степенной ряд, ряд Тейлора.
    Достаточное условие аналитичности функции.
    Пример бесконечно дифференцируемой, но неаналитической функции.
    Представление экспоненты комплексного аргумента степенным рядом.
    Формулы Эйлера.
    Формула Тейлора с остаточным членом в интегральной форме.
    Представление степенной и логарифмической функций степенными рядами}
    
    \subsection{Степенные ряды}

\textbf{Опр} \textit{Предел последовательности комплексных чисел} \textcolor{gray}{Предел модуля разности равен
нулю}

Заметим, что комплексный предел эквивалентен двум вещественным (для действительной и мнимой части)

\textbf{Опр} \textit{Сходящийся комплексный ряд} \textcolor{gray}{Существует конечный предел последовательности
частичных сумм этого ряда}

\textbf{Опр} \textit{Абсолютно сходящийся комплексный ряд} \textcolor{gray}{Сходится вещественный ряд модулей
членов ряда}

И вновь сходимость комплексного ряда эквивалентна сходимости двух вещественных рядов

\textbf{Опр} \textit{Равномерно сходящийся комплекснозначная функциональная последовательность} \textcolor{gray}{
    Вещественнозначная последовательность модулей разности предельной функции и элементов последовательности
    равномерно сходится к нулю на том же множестве}

\textbf{Опр} \textit{Равномерно сходящийся комплексный функциональный ряд} \textcolor{gray}{Последовательность
частичных сумм этого ряда равномерно сходится к сумме этого ряда на том же множестве}

\textbf{Опр} \textit{Степенной ряд} \textcolor{gray}{Если задана последовательность комплексных чисел
и комплексное число, то ...}

Однако удобнее (и мы в дальнейшем будем так делать) работать с рядом без степенной разности, сделав замену
комплексной переменной

\subsection{Формула Коши-Адамара для радиуса сходимости}

\textbf{Опр} \textit{Радиус сходимости степенного ряда} \textcolor{gray}{Неотрицательное число (или бесконечность
    ), определяемое формулой Коши-Адамара}

Притом для этой формулы мы расширили операцию деления

\subsection{Теорема о круге сходимости степенного ряда}

\textbf{Опр} \textit{Круг сходимости степенного ряда} \textcolor{gray}{Круг на комплексной плоскости с центром
в $w_0 (0)$ и радиусом равным радиусу сходимости}

Если радиус сходимости бесконечен, то кругом сходимости считается вся комплексная плоскость

\textbf{Th} \textit{О круге сходимости}

\textcolor{blue}{Степенной ряд абсолютно сходится внутри круга сходимости и расходится вне его}

\begin{enumerate}
    \item Зафиксируем произвольное комплексное число $z_0 \neq 0$, обозначим $q = \frac{z_0}{R}$ и исследуем
    сходимость с помощью обобщённого признака Коши
    \item В тривиально случае $z_0 = 0$ ряд сходится абсолютно
    \item В случае $0 < \abs{z_0} < R$ в силу обобщённого признака Коши ряд сходится абсолютно
    \item В случае $\abs{z_0} > R$ в силу обобщённого признака Коши члены абсолютного ряда не стремятся к нулю,
    как и исходного ряда, а значит, он расходится по отрицанию необходимого условия
\end{enumerate}

\subsection{Первая теорема Абеля}

\textbf{Th} \textit{Первая теорема Абеля}

\textcolor{blue}{Если степенной ряд сходится в точке $z_0$, то он сходится абсолюто в любой точке по модулю
меньшей}

Доказательство следует от противного в силу п.4 теоремы о круге сходимости

\subsection{Теорема о равномерной сходимости степенного ряда}

\textbf{Th} \textit{О равномерной сходимости степенного ряда}

\textcolor{blue}{$\forall r \in (0, R)$ ряд $\sum_{\mathbb{N}}_0 c_k z^k$ сходится равномерно в круге радиуса $r$}

Доказывается через неравенство, применением теоремы о круге сходимости и по признаку Вейерштрасса равномерной
сходимости комплексного ряда

\begin{enumerate}
    \item Зафиксируем произвольное комплексное число $z_0 \neq 0$, обозначим $q = \frac{z_0}{R}$ и исследуем
    сходимость с помощью обобщённого признака Коши
    \item В тривиально случае $z_0 = 0$ ряд сходится абсолютно
    \item В случае $0 < \abs{z_0} < R$ в силу обобщённого признака Коши ряд сходится абсолютно
    \item В случае $\abs{z_0} > R$ в силу обобщённого признака Коши члены абсолютного ряда не стремятся к нулю,
    как и исходного ряда, а значит, он расходится по отрицанию необходимого условия
\end{enumerate}

\subsection{Вторая теорема Абеля}

\textbf{Th} \textit{Вторая теорема Абеля}

\textcolor{blue}{Если степенной ряд сходится в точке $z_0$, то он сходится равномерно на отрезке $[0, z_0]$}

\begin{enumerate}
    \item Разобьём члены ряда на произведение членов произведения с помощью параметра $t \in [0, 1]$
    \item Первый ряд сходится по условию (а значит, по предыдущей теореме, ещё и равномерно)
    \item Второй ряд равномерно ограничен на отрезке и монотонен по индексу
    \item Поэтому два вещественных ряда сходятся равномерно на $[0, 1]$, как и исходный ряд на $[0, z_0]$
\end{enumerate}

\subsection{Сохранение радиуса сходимости при почленном дифференцировании степенного ряда}

\textbf{Th} \textcolor{blue}{Радиусы сходимости степенных рядов, полученные формальным дифференцированием и
интегрированием исходного, совпадают с его радиусом сходимости}

\begin{enumerate}
    \item Радиусы сходимости исходного и продифференцированного рядов совпадают в силу формулы Коши-Адамара
    \item Также они сходятся или расходятся одновременно, потому как при $z = 0$ это очевидно, а в противном
    случае они отличаются на ненулевую константу (как и их пределы)
    \item Так как исходный ряд получается почленным дифференцированием интегрального, то и их радиусы сходимости
    совпадают
\end{enumerate}

\subsection{Теоремы о почленном интегрировании и дифференцировании степенного ряда}

\textbf{Th} \textit{Об интегрировании и дифференцировании степенного ряда}

\textcolor{blue}{Если вещественный степенной ряд имеет ненулевой радиус сходимости, то внутри интервала
сходимости
    \begin{itemize}
        \item справедливы формулы почленного интегрирования
        \item функция ряда имеет производные любого порядка, получаемые почленным дифференцированием ряда
        \item коэффициенты степенного ряда однозначно определяются по обрывку формулы Тейлора
    \end{itemize}   }

\begin{enumerate}
    \item Для почленного интегрирования достаточно ввести новую переменную и воспользоваться теоремами о
    равномерной сходимости степенного ряда и о почленном интегрировании равномерно сходящегося функционального ряда
    \item Для производных достаточно ввести новую переменную и воспользоваться теоремами о сохранении радиуса
    сходимости, о равномерной сходимости степенного ряда и о почленном дифференцировании функционального ряда
    \item Проводя те же рассуждения по индукции, доказываем второе утверждение теоремы
    \item Доказывается аналогично лемме первого семестра перед формулой Тейлора
\end{enumerate}

\subsection{Единственность разложения функции в степенной ряд, ряд Тейлора}

\textbf{Опр} \textit{Бесконечно дифференцируемая функция в точке} \textcolor{gray}{В этой точке существуют
производные функции любого порядка}

\textbf{Опр} \textit{Ряд Тейлора} \textcolor{gray}{Ряд бесконечно дифференцруемой функции в точке с членами ...}

\textbf{Опр} \textit{Регулярная функция в точке $z_0$} \textcolor{gray}{Ряд Тейлора функции в точке $z_0$
    сходится к функции в некоторой окрестности $z_0$}

Из теоремы об интегрировании и дифференцировании степенного ряда следует, что если функция может быть
представлена как сумма степенного ряда $\sum_{\mathbb{N}_0} a_k (z - z_0)^k$ с ненулевым радиусом сходимости, то
этот ряд является рядом Тейлора функции в точке $z_0$.
В этом случае функция является регулярной в точке $z_0$

\textbf{Опр} \textit{Остаточный член формулы Тейлора} \textcolor{gray}{Разность $n$ раз дифференцируемой функции
и формулы Тейлора}

Непосредственно из определений следует, что функция является регулярной в точке
$\Leftrightarrow \lim_{n\to\infty} r_n(x) = 0$.
Притом для доказательства регулярности недостаточно показать ненулевой радиус сходимости функции, надо ещё проверить
её остаток

\subsection{Достаточное условие аналитичности функции}

\textbf{Th} \textit{Достаточное условие регулярности}

\textcolor{blue}{Если $\exists U_\delta (x_0)$, где функция бесконечно дифференцируема и последовательность её
производных равномерно ограничена константой $C > 0$, то функция регулярна в точке и $\forall x \in U_\delta (x_0)$
    раскладывается в ряд Тейлора}

\begin{enumerate}
    \item Применим формулу Тейлора с остаточным членом в форме Лагранжа.
    Тогда остаточный член формулы Тейлора $\leq M \frac{\delta^{n+1}}{(n+1)!}$
    \item Так как факториал растёт быстрее показательной (доказывается через принцип Архимеда, определение
    факториала, цепочку неравенств и предельный переход), то остаточный член стремится к нулю
    \item Поэтому функция регулярна, потому как раскладывается в ряд Тейлора в $x_0$
\end{enumerate}

\subsection{Пример бесконечно дифференцируемой, но неаналитической функции}

\begin{equation}
    f(x) =
    \begin{cases}
        e^{-\frac{1}{x^2}}, x \neq 0; \\
        0, x = 0.
    \end{cases}
\end{equation}

Ряд Тейлора этой бесконечно дифференцируемой в точке $x_0 = 0$ сходится не к функции $f(x)$, а к
некоторой другой функции, не совпадающей с $f(x)$ в сколь угодно малой окрестности точки

\[ \forall k \in \mathbb{N} \lim_{x \to 0} \frac{1}{x^k} e^{-\frac{1}{x^2}} = \lim_{t \to +\infty} t^{\frac{k}{2}} e^{-t} = 0 \]

По индукции легко показать, что если $P_{3n} (t)$ -- многочлен степени $3n$ от $t$, то

\begin{equation}
    f^{(n)}(x) =
    \begin{cases}
        P_{3n} (\frac{1}{x}) e^{-\frac{1}{x^2}}, x \neq 0; \\
        0, x = 0.
    \end{cases}
\end{equation}

Следовательно, все коэффициенты ряда Тейлора функции $f(x)$ в точке $x_0 = 0$ равны нулю.
Поэтому сумма ряда Тейлора функции $f(x)$ в точке $x_0$ равна нулю и не совпадает с функцией $f(x)$ в сколь угодно
малой окрестности точки $x_0$.
Таким образом, хотя функция и бесконечно дифференцируема, она не является регулярной в нуле

\subsection{Представление экспоненты комплексного аргумента степенным рядом}

\textbf{Опр} \textit{Ряд Маклорена} \textcolor{gray}{Ряд Тейлора функции в нуле}

\textbf{Th.1} \textcolor{blue}{Ряды маклорена функций $e^x, \sin(x), \cos(x), \sh(x), \ch(x)$ сходятся к этим
функциям на всей числовой прямой}

\begin{enumerate}
    \item $\forall \delta > 0~\forall x \in U_\delta (0)~e^x < e^\delta$, поэтому выполнено достаточное условие
    регулярности
    \item Аналогично, используя ограниченность последовательности всех производных оставшихся функций доказываем
    их разложения
\end{enumerate}

\textbf{Th.2} \textcolor{blue}{Для комплексной экспоненты её ряд Тейлора не отличается от вещественного}

\begin{enumerate}
    \item В силу предыдущей теоремы радиус сходимости степенного ряда-претендента сходится на всём $\mathbb{C}$,
    поэтому по теореме о круге сходимости он сходится абсолютно для любого $z \in \mathbb{C}$
    \item Зафиксируем произвольное комплексное число в алгебраической форме и воспользуемся определением
    экспоненты комплексного числа, чтобы зафиксировать доказываемое равенство
    \item Покажем, что функция-ряд-претендент обладает свойством экспоненты.
    Для этого воспользуемся теоремой о перемножении абсолютно сходящихся рядов, которая для комплексных рядов
    доказывается точно так же, как и для вещественных (только здесь надо использовать метод \("\)диагоналей\("\))
    \item В результате преобразований получим сумму сумм, которую распределим по этим суммам, и применим формулу
    бинома Ньютона, завершив доказательство свойства
    \item Далее рассмотрим функцию кандидат на чисто мнимом аргументе и путём разложения на чётную и нечётную
    суммы получим выражение для чисто мнимой экспоненты
    \item В итоге, применив свойство экспоненты и убедившись, что функция работает на вещественных аргументах,
    получим разложение комплексной экспоненты в ряд Тейлора в силу единственности
\end{enumerate}

\subsection{Формулы Эйлера}

\textbf{Лемма} \textcolor{blue}{Для любого $z \in \mathbb{C}$ справедливы формулы Эйлера} \textcolor{gray}{Они
используют новопостроенные комплексные функции и подравнивают комплексную тригонометрию к вещественной гиперболике}

\begin{enumerate}
    \item Для доказательства формулы гиперкомплексной экспоненты достаточно разделить сумм на чётную и нечётную,
    а затем воспользоваться $i^2 = -1$
    \item Остальные формулы следуют из первой
\end{enumerate}

\subsection{Формула Тейлора с остаточным членом в интегральной форме}

\textbf{Th} \textit{Формула Тейлора с остаточным членом в интегральной форме}

\textcolor{blue}{Если функция в $U_\delta (x_0)$ имеет непрерывные производные по $n+1$ порядок, то для
остаточного члена формулы Тейлора справедливо представление в интегральной форме: $r_n (x) = \frac{1}{n!} \int_{
    x_0}^x (x - t)^n f^{n+1}(t)dt \forall x \in U_\delta (x_0) $}

\begin{enumerate}
    \item При $n = 0$ теорема справедлива в силу формулы Ньютона -- Лейбница
    \item Пусть теорема справедлива для $n = s - 1$.
    Тогда проинтегрируем $r_{s-1}$ по частям
    \item Затем, расписав $r_s$ по определению, подставим проинтегрированное выражение и получим требуемое равенство
    \item Таким образом, теорема доказана по индукции
\end{enumerate}

\subsection{Представление степенной и логарифмической функций степенными рядами}

\textbf{Th} \textcolor{blue}{Ряд Маклорена степенной функции сходится к этой функции на интервале единичного радиуса}

\begin{enumerate}
    \item Зафиксируем $x \in (-1; 1)$ и учитывая выражение для $f^{n}$ распишем остаточный член в интегральной
    форме, походу дела вынося константы, вводя новые обозначения и переменные интегрирования
    \item Затем воспользуемся ограниченностью $x$ для оценки.
    Осталось показать, что $\lambda_n \rightarrow 0$
    \item В тривиальных случаях $x = 0$ и $\alpha = m \in \mathbb{N}_0, m < n$ утверждение очевидно
    \item В общем случае найдём предел отношения и воспользуемся схожими рассуждениями с доказательством признака
    Даламбера (сравнение с геометрической прогрессией)
\end{enumerate}

Заметим, что при $m \geq n$ ряд Маклорена совпадает с конечной суммой \\

Из доказанного и теоремы о почленном интегрировании степенного ряда при $\abs{x} < 1$ (не забывая про замену
индекса суммирования) получаем ряд Маклорена для логарифма.
Данное разложение справедливо и при $x = 1$.
Действительно, данный ряд будет сходиться по признаку Лейбница.
Следовательно, в силу второй теоремы Абеля этот ряд сходится равномерно на отрезке $[0; 1]$.
Согласно теореме о непрерывности суммы равномерно сходящегося функционального ряда частичные суммы этого ряда будет
непрерывны на отрезке $[0; 1]$.
Поэтому существует требуемый предел

\end{document}
