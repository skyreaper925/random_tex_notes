\textbf{Опр} \textit{$C^k$-гладкий диффеоморфизм}

\textbf{Опр} \textit{Носитель функции}

\textbf{Th} \textit{О замене переменных в кратном интеграле}

\textbf{Л2}
\textcolor{blue}{Теорема справедлива, если функция $f$ непрерывна на $Y$, а её носитель компактен и лежит в  $Y$}

\begin{enumerate}
    \item Убрав условие 3, мы сделали теорему локальной (для каждой точки существует окрестность, где выполнено
    условие 3)
    \item Воспользуемся теоремой о расщеплении отображений, о неявной функции, критерием компактности, теоремой о
    разбиении единицы
    \item Это позволяет разбить функцию на сумму.
    Утверждение для фиксированного индекса (на его области значений) верно по предыдущей лемме
\end{enumerate}


\textbf{Л4}
\textcolor{blue}{Теорема справедлива, если функция $f$ непрерывна на $Y$}

\begin{enumerate}
    \item Рассмотрим неотрицательно значные функции и введём хитрые множества $Y_k$ и функции $f_k$
    \item Докажем, что $\supp \subset f_k$ исходя из определения $f_k$.
    Получили ограниченность и замкнутость $\supp f_k$
    \item Из построения множеств следуют включения, а за ними и неравенства
\end{enumerate}



\textcolor{gray}{Все частные производные порядка $(k - 1)$ функции определены в окрестности и дифференцируемы в точке}

\textbf{Опр} \textit{Дифференциал $k$ порядка}
\textcolor{gray}{Дифференциал от предыдущего порядка с подстановкой точки $x_0$}
\begin{enumerate}
    \item Докажем для второго дифференциала, используя запись первого дифференциала в виде суммы частных производных
    \item В том же виде запишем второй дифференциал
    \item Получим его в виде суммы сумм начиная со внешних и заканчивая внутренними частными производными функции
    \item Для дифференциалов высших порядков доказываемое утверждение следует по индукции
\end{enumerate}

\subsection{Формула Тейлора для ФМП с остаточным членов в формах Лагранжа и Пеано}

\textbf{Th.1} \textit{Формула Тейлора с остаточным членом в форме Лагранжа}
\textcolor{gray}{$f(x_0) +$ сумма дифференциалов нормированных на факториал порядка +
следующий дифференициал в неизвестной точке отрезка}

\begin{enumerate}
    \item Зафиксируем $\forall x \in U_{\delta}(x_0)$ и рассмотрим функцию $\varphi(t) = f(x_0 + t\Delta x)$
    \item $\forall k \in \overline{1, m+1}$ продифференцируем её как \underline{сложную} функцию
    \item Применяем формулу Тейлора с остаточным членом в форме Лагранжа для функции одной переменной $\varphi(t)$
    и используем равенства $\varphi (1) = f(x), \varphi (0) = f(x_0)$
    \item Полученное выражение доказывает теорему
\end{enumerate}

\textbf{Опр} \textit{Многочлен Тейлора порядка $m$ функции $f$ в точке $x_0$}
\textcolor{gray}{$f(x_0) +$ сумма дифференциалов нормированных на факториал порядка}

\textbf{Th.2} \textit{Формула Тейлора с остаточным членом в форме Пеано}
\textcolor{gray}{Функция представима в виде суммы многочлена Тейлора и o-малого при \underline{непрерывных} в
окрестности частных производных}

\begin{enumerate}
    \item Применяем формулу Тейлора с остаточным членом в форме Лагранжа и раскладываем до $m-1$ порядка
    (на один меньше, чем требуется).
    Требуется доказать, что разность остаточного члена Лагранжа и дифференциала порядка $m$ есть остаток в форме Пеано
    \item Записываем эти дифференциалы через суммы сумм частных производных
    \item Модуль суммы не превосходит суммы модулей.
    Записываем это неравенство
    \item При $x \rightarrow x_0 $ в силу непрерывности каждого из дифференциалов получаем требуемую запись
\end{enumerate}

Аналогично первому семестру можно доказать единственность разложения в форме Пеано