\subsection{Дифференцируемость, дифференциал, градиент и производная по вектору для ФМП их связь геометрический смысл}

\textbf{Опр} \textit{Дифференцируемая в точке функция} \textcolor{gray}{Функция приближается линейной с точностью до o}

\textbf{Опр} \textit{Дифференциал} \textcolor{gray}{Линейная часть в предыдущем определении}

\textbf{Опр} \textit{Градиент} \textcolor{gray}{Вектор коэффициентов дифференциала}

\begin{gather*}
    df(x_0)[x - x_0] = (grad f(x_0), x - x_0)\\
    grad f(x_0) = (A_1, \dots, A_n) \Rightarrow df(x_0)[x - x_0] = \sum_{i=1}^{n} A_i (x^i - x_0^i)\\
    f(x) - f(x_0) = (grad f(x_0), x - x_0) + o(\abs{x - x_0}) = \sum_{i=1}^{n} A_i (x^i - x_0^i) + o(\abs{x - x_0})\\
\end{gather*}

\textbf{Утв} \textit{Геометрический смысл градиента и дифференциала}

\begin{enumerate}
    \item Фиксируем внутреннюю точку и проводим невертикальную плоскость $\alpha$
    \item Выразим из уравнения через нормальный вектор координату $z$
    \item Введём новые обозначения для нормального вектора
\end{enumerate}

\textbf{Опр} \textit{Касательная плоскость}
\textcolor{gray}{Плоскость, которая в окрестности этой точки приближает наш график с точностью до
    $o(\varrho), \varrho = \sqrt{(x - x_0)^2 + (y - y_0)^2}$} \\

\textbf{Th. 1} \textit{О геометрическом смысле градиента и дифференциала}

\begin{itemize}
    \color{blue}
    \item Касательная плоскость существует $\Leftrightarrow f$ дифференцируема в $(x_0, y_0)$
    \item $(N_x, N_y) = grad f(x_0, y_0)$
    \item $df(x_0, y_0)[x - x_0, y - y_0] = z_\alpha (x, y) - z_\alpha (x_0, y_0)$, то есть дифференциал в точке равен приращению аппликаты касательной плоскости
\end{itemize}

\begin{enumerate}
    \item Подставим в определение касательной плоскости уравнение плоскости $z_\alpha$, проходящей через $(x_0, z_0)$
    \item Запишем условие существования такой плоскости (существуют соответствующие коэффициенты)
    \item Сравниваем это условие с условием дифференцируемости в $(x_0, z_0)$
    \item Записываем вид градиента с учётом дифференцируемости
    \item С учётом всех пунктов записываем дифференциал и определение касательной плоскости
    \item Не забываем, что касательная плоскость проходит через точку касания
\end{enumerate}

\subsection{Достаточное условие дифференцируемости ФМП}

\textbf{Th} \textit{Достаточное условие дифференцируемости}
\textcolor{gray}{Все частные производные определены в окрестности и непрерывны в точке}

\begin{enumerate}
    \item Доказательство проведем для функции двух переменных
    \item Представим приращение функции $f$ как сумму приращений по каждой переменной (чтобы потом свести это к
    частным производным).
    Полученное выражение должно напоминать определение дифференцируемости
    \item Фиксируем $y$ и применяем теорему Лагранжа о среднем к функции $\varphi (x) = f(x, y)$
    \item Записываем полученное выражение в терминах $\theta \in (0, 1)$, зависящее от $x$ и $y$
    \item Записываем приращение с $x$ через частную производную в терминах $\theta$
    \item Вводим новое обозначение с умным нулём в виде $\frac{\partial f}{\partial x} (x_0, y_0)$ (нам же нужно всё
    свести к частным производным)
    \item Пользуемся непрерывностью частной производной, вводим $o(\varrho), \varrho = \sqrt{(x - x_0)^2 + (y - y_0)^2}$
    \item Итого, $f(x, y) - f(x_0, y) =  \frac{\partial f}{\partial x} (x_0, y_0) \cdot (x - x_0 + o(\varrho)$
    \item Аналогично с $f(x_0, y) - f(x_0, y_0)$
    \item Подставляя полученные выражения в п.2 получаем определение дифференцируемости в $(x_0, y_0)$
\end{enumerate}

\subsection{Матрица Якоби. Дифференцирование суперпозиции вектор-функций}

\textbf{Опр} \textit{Дифференцируемая вектор-функция}
\textcolor{gray}{Существует линейное отображение (\textit{дифференциал}), приближающее нашу функцию с точностью до o}

\textbf{Опр} \textit{Матрица Якоби $\mathcal{D} f(x_0)$ в точке}
\textcolor{gray}{Матрица линейного отображения $df(x_0): \mathds{R}_n \rightarrow \mathds{R}_m$}

\textbf{Л1} \textit{Связь дифференцируемости и $\mathcal{D} f(x_0)$}

\begin{itemize}
    \color{blue}
    \item Вектор-функция дифференцируема $\Leftrightarrow f$ дифференцируема каждая её компонента
    \item $\mathcal{D} f(x_0)$ записывается в виде склейки столбцов, каждый из которых соответствует конкретной координате
\end{itemize}

\begin{enumerate}
    \item Предположим дифференцируемость и будем рассматривать $df(x_0)[x - x_0]$ как произведение $\mathcal{D} f(x_0)$
    на столбец $(x - x_0)$
    \item Обозначим элементы $\mathcal{D} f(x_0)$ как $a_i^j$
    \item Переходим к построчной интерпретации предыдущего векторного равенства.
    Тогда $grad f_k(x_0)$ имеет компоненты $a_i^k, \dots, a_n^k$
    \item Вспоминая связь градиента и частных производных приходим к требуемому равенству
    \item Обратно, пусть   все компоненты вектор-функции дифференцируемы в точке.
    Тогда   по   определению   дифференцируемости   скалярной функции для  любого  $k in \overrightarrow{1, m}$
    существуют числа $a_i^k, \dots, a_n^k$
    \item Из системы линейных уравнений переходим к векторному, получая требуемое утверждение
\end{enumerate}

\textbf{Th} \textit{О дифференцировании сложной функции}

\begin{itemize}
    \color{blue}
    \item Композиция дифференцируемых вектор-функций дифференцируема
    \item Суперпозиция дифференциалов есть дифференциал суперпозиции
    \item Матрица Якоби суперпозиции есть произведение матриц Якоби
\end{itemize}

\begin{enumerate}
    \item Запишем определение дифференцируемости вектор-функции для $f(x)$ и $g(y)$
    \item Подставляем в формулу для $g(y) y = f(x)$ и в дифференциал, и в $\overrightarrow{o}$.
    Получили сложный дифференциал и $\overrightarrow{o}(\abs{f(x) - f(x_0)})$
    \item Пользуемся дистрибутивностью и линейностью дифференциала (о-малость сохраняется для $x$)
    \item Получили требуемое равенство, где суперпозиция дифференциалов -- линейное отображение, приближающее функцию с
    точностью до  $\overrightarrow{o}(\abs{x - x_0})$
    \item Пользуясь тем, что $df(x_0)[x - x_0] = \mathcal{D} f(x_0) \cdot (x - x_0)$, получаем равенство для матриц Якоби
\end{enumerate}