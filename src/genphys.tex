%! Author = user
%! Date = 06.06.2023

\documentclass[a4paper, 14pt]{article}
%\documentclass[draft]{article}

\usepackage[T2A]{fontenc}
\usepackage[utf8]{inputenc}
\usepackage[english, russian]{babel}
\usepackage[top = 2cm, bottom = 2cm, left = 2cm, right = 2cm]{geometry}
\usepackage{indentfirst}
\usepackage{xcolor}
\usepackage{hyperref}
\usepackage{gensymb}
\usepackage{pgfplots}
\usepackage{amsmath, amsfonts, amsthm, mathtools}
\usepackage{amssymb}
\usepackage{physics, multirow, float}
\usepackage{wrapfig, tabularx}
\usepackage{icomma} % Clever comma: 0,2 - number while 0, 2 - two numbers
\usepackage{tikz, standalone}
\usepackage{fancyhdr,fancybox}
\usepackage{lastpage}
\usepackage{booktabs}
\usepackage{listings}
\usepackage{lstmisc}
\usepackage{stmaryrd}

%\полуторный интервал
\onehalfspacing

\hypersetup
{   colorlinks = false,
    linkcolor = blue,
    pdftitle = {genphys},
    pdfauthor = {Володин Максим},
    allcolors = [RGB]{010 090 200}
}

%\gravarphicspath{{images/}}
%\DeclareGravarphicsExtensions{.pdf,.png,.jpg}

\restylefloat{table}
\usetikzlibrary{external}

\mathtoolsset{showonlyrefs = true} % Numbers will appear only where \eqref{} in the text LINKED
\pagestyle{fancy}

\fancyhf{}
\fancyhead[L]{Общая физика}
\fancyhead[R]{Конспект билетов}
\fancyfoot[L]{}
\fancyfoot[R]{\thepage /\pageref{LastPage}}

\pgfplotsset{compat=1.18}

\begin{document}

    \tableofcontents \newpage

    \section{Термодинамическая система.
    Микроскопические и макроскопические параметры.
    Уравнение состояния (термическое и калорическое).
    Равновесные и неравновесные состояния и процессы}

    \subsection{Термодинамическая система}

    \textbf{Опр} \textit{Система, термодинамическая система} \textcolor{gray}{Совокупность рассматриваемых тел ...}

    \textbf{Опр} \textit{Изолированная, закытая и открытая термодинамическая система} \textcolor{gray}{Обмен ...}

    \subsection{Микроскопические и макроскопические параметры}

    \textbf{Утв} \textit{Существуют микроскопические и макроскопические состояния} \textcolor{gray}{+ их другие имена}

    \textbf{Опр} \textit{Микроскопическое и макроскопическое состояния} \textcolor{gray}{Состояние системы,}

    \textbf{Опр} \textit{Микроскопические параметры} \textcolor{gray}{Величины, характризующий макросостояние}

    \subsection{Уравнение состояния (термическое и калорическое)}

    \textbf{Опр} \textit{Уравнение состояния} \textcolor{gray}{Состояние, отражающее для конкретного класса величин ...}

    \textbf{Опр} \textit{Термодинамическое, калорическое уравнение состояния} \textcolor{gray}{$f(P, V, T) = 0, j(\dots)$}

    \subsection{Равновесные и неравновесные состояния и процессы}

    \textbf{Опр} \textit{Термодинамическое равновесие} \textcolor{gray}{Все макроскопические процессы прекращаются, ...}

    \textbf{Опр} \textit{Основное (общее) начало термодинамики} \textcolor{gray}{Предоставленная самой себе ...}

    \textbf{Опр} \textit{-I начало ТД} \textcolor{gray}{Три условия на любую изолированную систему}

    \textbf{Опр} \textit{Неравновесное состояние} \textcolor{gray}{Предоставленная самой себе ...}

    \textbf{Опр} \textit{Релаксация, время релаксации} \textcolor{gray}{Переход из состояния, в котором система ...}

    \textbf{Опр} \textit{Траектория процесса} \textcolor{gray}{Состояния системы и переходы между ними}

    \textbf{Опр} \textit{Равновесный (квазистатический) процесс} \textcolor{gray}{По ходу процесса система ...}

    \textbf{Опр} \textit{Неравновесное состояние} \textcolor{gray}{На траектории процесса встречаются ...}

    \section{Идеальный газ.
    Уравнение состояния идеального газа.
    Идеально-газовое определение температуры.
    Связь давления и температуры идеального газа с кинетической энергией его молекул}

    \subsection{Идеальный газ}

    \textbf{Опр} \textit{Идеальный газ} \textcolor{gray}{Газ, у которого взаимодействием молекул между собой можно ...}

    \subsection{Уравнение состояния идеального газа}

    \textbf{Закон} \textit{Бойля -- Мариотта}

    \textcolor{blue}{$PV = const, const$ однозначно определяется количеством газа и степенью его "нагретости"}

    \textbf{Опр} \textit{Газовая постоянная} \textcolor{gray}{Определяется из тройной точки воды. Измеряется в ...}

    \textbf{Опр} \textit{Постоянная Больцмана} \textcolor{gray}{$k \frac{R}{N_A} = 6,022 \cdot 10^{23} \frac{1}{moles}$}

    \textbf{Закон} \textit{Уравнение состояния идеального газа Менделеева -- Клапейрона}

    \textcolor{blue}{$PV = \mu RT = NkT = \frac{m}{\mu} RT$}

    \subsection{Идеально-газовое определение температуры}

    Отсюда можно определить температуру по идеально-газовой шкале $T = \frac{PV}{\mu R}$

    \subsection{Связь давления и температуры идеального газа с кинетической энергией его молекул}

    В результате перехода от микрорассмотрения к макро, получим $P = \frac{1}{3}nvp = \dots \Rightarrow U = N \frac{3
    }{2}kT$

    Из полной кинетической энергии газа в воздухе $E = N \overline{\varepsilon}$ можно получить $\overline{\varepsilon}
    = \frac{3}{2}kT \Rightarrow P = nkT$

    \section{Работа, внутренняя энергия, теплота.
    Первое начало термодинамики.
    Внутренняя энергия идеального газа}

    \subsection{Работа, внутренняя энергия, теплота}

    \textbf{Опр} \textit{Функция состояния} \textcolor{gray}{Величина, принимающая определённое значение в каждом ...}

    \textbf{Опр} \textit{Работа, совершённая системой и над ней} \textcolor{gray}{$PdV, P \in \{P_{in}, P_{out} \}$}

    Для квазистатического процесса $\delta A_{in} = - \delta A_{out}$.

    Заметим, что работа не является функцией состояния

    \textbf{Опр} \textit{Адиабатическая оболочка} \textcolor{gray}{При любых изменениях температуры окружающих ...}

    Если система заключена в адиабатическую оболочку, то работа внешних сил не зависит от траектории процесса, а
    определяется только начальным и конечным состояниями системы: $A_{12} = U_2 - U_1, U$ -- внутренняя энергия,
    функция состояния

    \textbf{Опр} \textit{Количество теплоты} \textcolor{gray}{Если система заключена в жёсткую ...}

    $Q_{in} = U_2 - U_1 = -Q_{out}$

    Первое начало термодинамики

    \subsection{Первое начало термодинамики}

    \textbf{Закон} \textit{Первое начало термодинамики}

    \textcolor{blue}{ЗСЭ, записываемый как $\delta Q_{in} = dU + A_{in} \Rightarrow dU = \delta Q - PdV$}

    \subsection{Внутренняя энергия идеального газа}

    \textbf{Опр} \textit{Внутренняя энергия ИГ} \textcolor{gray}{Функция только температуры, так как определяется ...}

    \[ dU = c_V dT, U = \int c_V dT = \nu N_A \overline{\varepsilon} = \frac{i}{2} \nu RT \]

    \section{Теплоёмкость.
    Теплоёмкости при постоянном объёме и давлении.
    Связь между $c_V$ и $c_P$ для идеального газа (соотношение Майера)}

    \subsection{Теплоёмкость}

    \textbf{Опр} \textit{Теплоёмкость} \textcolor{gray}{$c = \frac{\delta Q_{in}}{dT}$}

    \subsection{Теплоёмкости при постоянном объёме и давлении}

    Для получения указанных формул, достаточно записать I начало ТД, выполнить преобразования и записать определение
    , не забыв, что $H = U + PV$ есть энтальпия

    \subsection{Связь между $c_V$ и $c_P$ для идеального газа (соотношение Майера)}

    \[ c_P dT = dU = d(U + PV) = (c_V + \nu R) dT \Rightarrow c_P = c_V + \nu R\]

    \section{Политропический и адиабатический процессы.
    Уравнение адиабаты и политропы идеального газа.
    Скорость звука в газах}

    \subsection{Политропический и адиабатический процессы}

    \textbf{Опр} \textit{Политропический процесс} \textcolor{gray}{Процесс, в котором теплоёмкость остаётся ...}

    \textbf{Опр} \textit{Адиабатический процесс} \textcolor{gray}{Процесс, происходящий в теплоизолированной ...}

    \subsection{Уравнение адиабаты и политропы идеального газа}

    \textbf{Опр} \textit{Показатель адиабаты, политропы} \textcolor{gray}{$\gamma = \frac{c_P}{c_V}$, $n = \dots$}

    Из уравнения состояния ИГ можно вывести уравнение адиабатического и политропического процессов

    Существует четыре основных политропических процесса: адиабата, изохора, -бара, -терма.
    У каждого из них свои теплоёмкости, показатели политропы $n$ и уравнения

    \subsection{Скорость звука в газах}

    \textbf{Опр} \textit{Скорость звука} \textcolor{gray}{Фазовая скорость продольных волн в бесконечной ...}

    Скорость звука можно запросто вывести из соответствующего уравнения механики

    \textbf{Опр} \textit{Адиабатическая скорость звука} \textcolor{gray}{За время прохождения звука на ...}

    Адиабатическую скорость звука выражается через ту же конечную формулу, с использованием уравнения адиабаты и
    $\rho = \frac{P \mu}{RT}$

    \section{Тепловые машины.
    Цикл Карно.
    КПД машины Карно.
    Теоремы Карно.
    Холодильная машина и тепловой насос.
    Коэффициенты эффективности идеальной холодильной машины и идеального теплового насоса}

    \subsection{Тепловые машины}

    \textbf{Опр} \textit{Тепловая машина} \textcolor{gray}{Устройство, которое преобразует теплоту в работу или ...}

    \subsection{Цикл Карно}

    \textbf{Опр} \textit{Машина Карно} \textcolor{gray}{Тепловая машина, работающая по циклу Карно}

    \textbf{Опр} \textit{Цикл Карно} \textcolor{gray}{Обратимый цикл из двух изотерм и адиабат}

    \subsection{КПД машины Карно}

    \textbf{Опр} \textit{КПД тепловой машины} \textcolor{gray}{Отношение работы, произведённой машиной за один цикл ...}

    \subsection{Теоремы Карно}

    \textbf{Th} \textit{Первая теорема Карно}

    \textcolor{blue}{КПД любой тепловой машины, работающей между между двумя заданными термостатами, не может
    превышать КПД машины Карно, работающей между теми же резервуарами}

    \begin{enumerate}
        \item от противного: пусть у необратимой машины КПД больше.
        Рассмотрим работу этих двух машин в разных направлениях на одних и тех же резервуарах
        \item Подберём $Q_{+1}, Q_{+2}: Q_{+1} = Q_{+2}$.
        Тогда рассмотрим суммарные теплоты и работы за цикл (ведь две тепловые машины всё равно что одна
        многофункциональная)
        \item Итого, получилось что единственным результатом цикла большой машины есть производство работы за счёт
        охлаждение холодильника, $w$ со II началом ТД
    \end{enumerate}

    \textbf{Th} \textit{Вторая теорема Карно}

    \textcolor{blue}{КПД любых идеальных машин, работающих по циклу Карно между двумя заданными термостатами, равны и
    не зависят от устройства машин и рабочего тела}

    Это следствие первой теоремы: надо применить её к двум конкретным машинам Карно и поменять их местами.
    Система двух неравенств эквивалентна равенстве.
    Независимость от параметров достигнута за счёт рассмотрения общего случая

    Чтобы найти точное значение КПД машины Карно, работающей с телами с \underline{температурами} $T_1, T_2$, надо
    рассмотреть идеальный газ как рабочее тело, вспомнить определение цикла Карно, модифицированное уравнение
    адиабаты и работу на изотерме

    \subsection{Холодильная машина и тепловой насос}

    \textbf{Опр} \textit{Холодильный цикл} \textcolor{gray}{Имеет в результате потребление работы через отбирание ...}

    \textbf{Опр} \textit{Холодильная машина} \textcolor{gray}{Машина, работающая по холодильному циклу}

    \textbf{Опр} \textit{Тепловой насос} \textcolor{gray}{Машина для передачи тепла к более нагретому телу от ...}

    \subsection{Коэффициенты эффективности идеальной холодильной машины и идеального теплового насоса}

    \textbf{Опр} \textit{Эффективность холодильной машины} \textcolor{gray}{Отношение тепла холодильника к работе ...}

    Чтобы найти эффективность идеальной холодильной машины, надо воспользоваться теоремами Карно.
    Аналогично для идеального теплового насоса

    \section{Второе начало термодинамики.
    Энтропия (термодинамическое определение).
    Неравенство Клаузиуса.
    Энтропия идеального газа}

    \subsection{Второе начало термодинамики}

    \textbf{Опр} \textit{Машина Клаузиуса} \textcolor{gray}{Машина, работающая по круговому циклу, в результате ...}

    \textbf{Закон} \textit{Второе начало термодинамики в формулировке Клаузиуса}

    \textcolor{blue}{Машина Клаузиуса невозможна}

    \textbf{Опр} \textit{Машина Томсона} \textcolor{gray}{Машина, работающая по круговому циклу, в результате ...}

    \textbf{Закон} \textit{Второе начало термодинамики в формулировке Томсона (лорда Кельвина)}

    \textcolor{blue}{Машина Томсона невозможна}

    \textbf{Th} \textcolor{blue}{Формулировки Клаузиуса и Томсона эквивалентны}

    $\Leftarrow:$ производимую мТ работу можно целиком передать нагревателю, создав мК

    $\Rightarrow:$ рассмотрим две машины между заданными термостатами: обыкновенную и мК.
    Результат одновременной работы этих двух машин есть мТ

    \subsection{Энтропия (термодинамическое определение)}

    \textbf{Опр} \textit{Термодинамическая энтропия}

    Рассматривается произвольный обратимый круговой процесс, проходящий через фиксированные точки, рассматривается
    интеграл по циклу $\frac{\delta Q}{T}$ и путём преобразований (и, возможно, неравенства Клаузиуса) доказывается,
    что его величина не зависит от пути между точками

    Тогда перед нами функция состояния по определению.
    Назовём её энтропией и будем обозначать как \ldots

    \subsection{Неравенство Клаузиуса}

    \textbf{Утв} \textit{Неравенство Клаузиуса}

    \textcolor{blue}{$\oint \frac{\delta Q_i}{T_i} \leq 0$}

    Иногда данное неравенство записывают в дискретной форме.

    По второму началу термодинамики получим, что данный интеграл в обратимых процессах эквивалентен $-\delta S = 0$ в
    силу того, что обратимыми являются лишь машины, работающие по циклу Карно, а для них данное равенство выполнено
    из выражения для КПД цикла Карно.

    В случае неравновесного процесса, запишем его КПД и сравним с КПД цикла Карно.
    Тогда получим, что интеграл Клаузиуса процесса будет меньше, чем интеграл Клаузиуса цикла Карно, то есть $\leq 0$

    \subsection{Энтропия идеального газа}

    Из I начала ТД и определения энтропии, получаем её выражение для ИГ

    \section{Обратимые и необратимые процессы.
    Закон возрастания энтропии.
    Неравновесное расширение газа в пустоту}

    \subsection{Обратимые и необратимые процессы}

    \textbf{Опр} \textit{(Не)обратимые процессы} \textcolor{gray}{Процесс (не)мб проведён в обратном направлении ...}

    \subsection{Закон возрастания энтропии}

    \textbf{Закон} \textit{Неубывания энтропии}

    Для его доказательство достаточно рассмотреть круговой процесс с обратимой и нет частью, воспользоваться
    определением энтропии и неравенством Клаузиуса

    \textbf{Утв} \textit{Постулат Гиббса}

    \textcolor{blue}{Энтропия максимальна в состоянии равновесия}

    \subsection{Неравновесное расширение газа в пустоту}

    В теплоизолированной системе при расширении газа в вакуум по I началу ТД $\Delta Q = 0$ в силу
    теплоизолированности $A = 0$, потому что не над чем совершать работу, поэтому и $\Delta T = 0$.
    Из выражения энтропии для ИГ $\Delta S = \nu R \ln \left(\frac{V_2}{V_1}\right) > 0$.
    По-другому, возрастание энтропии можно объяснить необратимостью процесса в замкнутой системе

    \section{Термодинамические потенциалы: внутренняя энергия, энтальпия, свободная энергия, термодинамический
    потенциал Гиббса.
    Метод получения соотношений Максвелла (соотношений взаимности)}

    \subsection{Термодинамические потенциалы: внутренняя энергия, энтальпия, свободная энергия, термодинамический
    потенциал Гиббса}

    \textbf{Опр} \textit{Термодинамические потенциалы} \textcolor{gray}{Функции определённых наборов ТД параметров, ...}

    Всего есть четыре основных ТД потенциала: внутренняя энергия, энтальпия ($+PV$), свободная энергия ($-TS$) и
    потенциал Гиббса (совокупность двух предыдущих).
    У каждого есть свой набор параметров, полный дифференциал, а также, конкретные частные производные.
    При желании, для их вычисления в случае ИГ можно указать явные формулы через его энтропию

    \subsection{Метод получения соотношений Максвелла (соотношений взаимности)}

    В силу того, что все ТД функции непрерывны, верна теорема Шварца о равенстве смешанных частных производных.
    Это позволяет получить четыре новых равенства, называемых соотношениями взаимности Максвелла

    \section{Фазовые переходы первого рода.
    Уравнение Клапейрона—Клаузиуса.
    Фазовое равновесие «жидкость—пар», зависимость давления насыщенного пара от температуры}

    \subsection{Фазовые переходы первого рода}

    \textbf{Опр} \textit{Фаза} \textcolor{gray}{Физически однородная часть системы, отличающаяся своими физическими ...}

    \textbf{Опр} \textit{Химический потенциал} \textcolor{gray}{Величина, определяющая изменение энергии системы ...}

    Химический потенциал можно приписать каждому из известных до этого ТД потенциалов

    \textbf{Опр} \textit{Экстенсивные и интенсивные параметры} \textcolor{gray}{Величины, пропорциональные и не- ...}

    \textbf{Утв} \textit{Условия равновесия фаз}

    \textcolor{blue}{В состоянии равновесия во всём веществе выполнены условия
        \begin{itemize}
            \item Механического равновесия ($P = const$)
            \item Теплового равновесия ($T = const$)
            \item Равновесия по отношению перехода частиц между различными фазами ($\mu = const$)
        \end{itemize}}

    Притом можно показать, что из выполнения первых двух условий следует третье

    В случае отсутствия равновесия частицы переходят в фазу с меньшим химическим потенциалом

    \subsection{Уравнение Клапейрона—Клаузиуса}

    Фазы могут существовать, если только давление и температура лежать на кривой фазового равновесия

    \textbf{Опр} \textit{Фазовая диаграмма} \textcolor{gray}{Координаты $(P, T)$, на которой нанесены различные ...}

    Из равенства химический потенциалов следует равенство их дифференциалов из потенциал Гиббса.
    Преобразуя равенство и введя теплоту фазового перехода в расчёте на одну частицу (энергозатраты на осуществление
    перехода), получим уравнение Клапейрона -- Клаузиуса

    Также можно показать, что введённая теплота смены фазы есть разница энтальпий

    \subsection{Фазовое равновесие «жидкость -- пар», зависимость давления насыщенного пара от температуры}

    Уравнение кривой равновесия «жидкость -- пар» может быть получено с помощью уравнения ИГ, факта, что $v_2 \gg
    v_1$ и в предположении $q = const$.
    В таком случае зависимость выражается в экспоненциальной форме в двух случаях задания $q$ (удельно или молярно)

    \textbf{Опр} \textit{Насыщенный пар} \textcolor{gray}{Пар, находящийся в равновесии с собственной жидкостью}

    \section{Теорема Б. Леви о монотонной сходимости.
    Теорема Лебега об ограниченной сходимости}

    Отличие следующих теорем от непрерывности интеграла по множествам состоит в том, что теперь предельный переход
    выполняется для функций, а не множеств

    \subsection{Теорема Б. Леви о монотонной сходимости}

    \textbf{Th} \textcolor{blue}{Если последовательность измеримых функций $f_k \geq 0$ монотонна и сходится к $f$,
        то $f$ измерима с интегралом, равным пределу интегралов $f_k$}

    \begin{enumerate}
        \item Измеримость функции следует из леммы о поточечной сходимости, а интегрируемость в силу
        существования интеграла от неотрицательной измеримой функции (интеграл может быть бесконечным)
        \item Рассмотрим случай конечного интеграла, предварительно выкинув множества нулевой меры, на котором он
        бесконечен
        \item Зафиксируем $\forall \varepsilon > 0$ и рассмотрим множества $X_k$ c $(1 - \varepsilon)$ внутри
        \item В силу монотонности функции, $X_k$ будут монотонны по включению и покрывать всю область определения
        \item Вспомним про непрерывность интеграл по множествам и определение предела
        \item Затем распишем неравенства, устремим $\varepsilon \rightarrow 0$ и получим доказываемое соотношение
        \item В случае бесконечного интеграла фиксируем $\forall C > 0$ и миноранту, чей интеграл на том же множестве
        будет $> C$ (она существует из определения нижнего интеграла) и выкинем множества нулевой меры, на
        которых миноранта больше $f$
        \item Рассмотри измеримые функции $g_k = \min(f_k, g)$, которые в пределе равны миноранте (показывается через
        определения предела для $f$ и минимума)
        \item Как показано в конечном случае, предел для миноранты будет больше $> C$, а в силу неравенства, для $f$
        тоже.
        В силу произвольности $C$ получаем необходимое равенство
    \end{enumerate}

    \subsection{Теорема Лебега об ограниченной сходимости}

    \textbf{Th} \textcolor{blue}{Если последовательность интегрируемых функций $f_k$, каждый член которой ограничен по
    модулю интегрируемой функцией $\varphi$ почти всюду на $X$ и поточечно сходится к $f$, то $f$ интегрируема с
    интегралом, равным пределу интегралов $f_k$}

    \begin{enumerate}
        \item Измеримость $f$ следует из леммы о поточечной сходимости, а интегрируемость в силу предельного перехода и
        признака сравнения
        \item Выкинем множества нулевой меры, на которых условие теоремы не выполняется
        \item Зафиксируем $\forall \varepsilon > 0$ и рассмотрим множества $X_k$ c $\varepsilon \varphi(x)$ внутри
        \item $X_k$ будут покрывать $X$ (включение в одну сторону очевидно, а в другое надо рассмотреть два случая
        для $\varphi(x)$, расписать определение предела).
        Также $X_k$ будут монотонны по включению
        \item Распишем предел для $\int_{X_k} \varphi$ с помощью непрерывности и аддитивности интеграла по множествам
        \item Теперь распишем неравенство для разности интегралов $f$ и $f_k$, воспользовавшись неравенством
        треугольника, определением $X_k$ и конечностью интеграла для $\varphi$
        \item В итоге, устремив $\varepsilon \rightarrow 0$, завершим доказательство теоремы
    \end{enumerate}

    \section{Несобственный интеграл.
    Связь сходимости несобственного интеграла и интегрируемости функции по Лебегу.
    Критерий Коши.
    Признаки Дирихле и Абеля сходимости несобственных интегралов}

    \subsection{Несобственный интеграл}

    \textbf{Опр} \textit{Несобственный интеграл, особенность} \textcolor{gray}{Односторонний предел интегрального конца}

    \textbf{Опр} \textit{(Рас)ходящийся несобственный интеграл} \textcolor{gray}{Если (не)существует конечный предел}

    \textbf{Опр} \textit{Собственный интеграл} \textcolor{gray}{Интеграл Лебега, который был до этого}

    \textbf{Опр} \textit{Абсолютно сходящийся несобственный интеграл} \textcolor{gray}{Аналогично рядам}

    \textbf{Опр} \textit{(Сходящийся) несобственный интеграл с двумя особенностями} \textcolor{gray}{Разбить на два
    интеграла с одной особеностью (и утверждать сходимость только в случае сходимости обоих интегралов)}

    \subsection{Связь сходимости несобственного интеграла и интегрируемости функции по Лебегу}

    \textbf{Th.1} \textcolor{blue}{Если $f$ интегрируема по Лебегу, на любом открытом промежутке, она
    интегрируема на всём промежутке $\Leftrightarrow$ соответсвующих несобственный интеграл сходится абсолютно}

    \begin{enumerate}
        \item $\Rightarrow$: согласно лемме об интегрируемости на подмножестве $f$ интегрируема на любом открытом
        промежутке, как и её модуль (по эквивалентности)
        \item Из аддитивности интеграла по множествам следует нестрогое возрастание функции $F(b^{'}) = \int_a^{b^{'}} \abs{f(x
            )} dx $
        \item По теореме существует предел слева, поэтому несобственный интеграл сходится абсолютно
        \item $\Leftarrow$: зафиксируем возрастающую последовательность $\{b_k\} \rightarrow b$
        \item Определим индикаторную последовательность функций $f_k(x)$.
        Она сходится к $f$, что докажет измеримость $f$ на всём интервале
        \item Затем введём новую функциональную последовательность $g(x) = \abs{f_k(x)}$.
        Она будет возрастать и в пределе равна $\abs{f(x)}$, поэтому применима теорема о монотонной сходимости
        \item Из неё следует интегрируемость $\abs{f(x)}$ на интервале, то есть и $f$
    \end{enumerate}

    \textbf{Th.2} \textcolor{blue}{Если $f$ интегрируема в собственном смысле, то несобственный интеграл сходится и
    его значение равна интегралу Лебега на том же интервале}

    Доказательство состоит в применении теоремы о непрерывности интеграла как функции верхнего предела

    \subsection{Критерий Коши}

    \textbf{Th} \textit{Критерий Коши}

    \textcolor{blue}{Если на числовом промежутке $f$ интегрируема по Лебегу на любом открытом промежутке, то
    несобственный интеграл этой функции сходится $\Leftrightarrow$ выполняется условие Коши}

    \begin{enumerate}
        \item Определим $F(t) = \int_a^t f(x)dx$.
        Несобственный интеграл c особенностью в верхнем конце будет сходиться, если у этой функции существует
        конечный предел при $t \rightarrow b - 0$
        \item Далее сведём задачу к КК существования предела функции и воспользуемся формулой Ньютона -- Лейбница
    \end{enumerate}

    \subsection{Признаки Дирихле и Абеля сходимости несобственных интегралов}

    Смотреть в рукописном конспекте

    \section{Связь поточечной и равномерной сходимостей для функциональной последовательности.
    Критерий Коши равномерной сходимости функциональной последовательности.
    Обобщенный признак сравнения для функциональных рядов.
    Признак Вейерштрасса равномерной сходимости функционального ряда.
    Признаки Дирихле и Лейбница равномерной сходимости функционального ряда.
    Признак Абеля равномерной сходимости функционального ряда.
    Непрерывность равномерного предела, непрерывных функций и суммы равномерно сходящегося функционального ряда с
    непрерывными слагаемыми.
    Почленное интегрирование функциональных последовательностей и рядов.
    Дифференцирование предельной функции и почленное дифференцирование функционального ряда}

    \subsection{Связь поточечной и равномерной сходимостей для функциональной последовательности}

\textbf{Опр} \textit{Поточечный предел функциональной последовательности} \textcolor{gray}{Предел в привычном
понимании}

\textbf{Опр} \textit{Равномерный предел функциональной последовательности} \textcolor{gray}{$N \in \mathbb{N}$ не зависит от аргумента}

Из равномерной сходимости следует поточечная, но не наоборот

\textbf{Опр} \textit{Равномерно ограниченная функциональная последовательность} \textcolor{gray}{$N \in \mathbb{N}$
    не зависит от аргумента}

\subsection{Критерий Коши равномерной сходимости функциональной последовательности}

\textbf{Th} \textit{Критерий Коши}

\textcolor{blue}{Последовательность сходится равномерно $\Leftrightarrow$ выполняется условие Коши}

\begin{enumerate}
    \item $\Rightarrow$: дважды применить определение равномерной сходимости и воспользоваться неравенством
    треугольника
    \item $\Leftarrow$: требуется доказать равномерную сходимость из выполнения условия Коши числовой
    последовательности для любого фиксированного $x \in X$.
    В силу КК для числовой последовательности $\lim_{k\to\infty} f_k = f$
    \item Далее надо в силу $\forall p \in \mathbb{N}$ устремить его к $+\infty$ и по теореме о предельном
    переходе в неравенствах получить определение равномерной сходимости
\end{enumerate}

\subsection{Обобщенный признак сравнения для функциональных рядов}

\textbf{Опр} \textit{Поточечный предел функционального ряда} \textcolor{gray}{Сходимость ряда в привычном понимании}

\textbf{Опр} \textit{Равномерный предел функционального ряда} \textcolor{gray}{Если последовательность его
частичных сумм сходится равномерно на том же множестве}

\textbf{Опр} \textit{Остаток поточечно сходящегося функционального ряда} \textcolor{gray}{Разность суммы и
частичной суммы ряда}

\textbf{Th} \textit{Обобщенный признак сравнения}

\textcolor{blue}{Если каждый член нашего ряда по модулю не превосходит члена равномерно сходящегося на том же
множестве ряда, то и наш ряд сходится равномерно}

Доказательство состоит в двукратном применении КК

Из признака следует, что из равномерной абсолютной сходимости ряда следует равномерная сводимость ряда на том же
множестве

\subsection{Признак Вейерштрасса равномерной сходимости функционального ряда}

\textbf{Th} \textit{Признак Вейерштрасса}

\textcolor{blue}{Если каждый член нашего ряда по модулю не превосходит члена сходящегося ряда, то наш ряд
сходится равномерно на том же множестве}

Доказательство состоит в применении обобщенного признака сравнения.
Заметьте, что мы не требуем равномерной сходимости от ряда-мажоранты

\subsection{Признаки Дирихле и Лейбница равномерной сходимости функционального ряда}

Смотреть в рукописном конспекте

\subsection{Признак Абеля равномерной сходимости функционального ряда}

Смотреть в рукописном конспекте

\subsection{Непрерывность равномерного предела, непрерывных функций и суммы равномерно сходящегося функционального ряда с
непрерывными слагаемыми}

\textbf{Th.1} \textit{О непрерывности предельной функции}

\textcolor{blue}{Если последовательность $f_k$ непрерывных на множестве $X$ функций сходится равномерно на
множестве $X$, то $f$ непрерывна на $X$}

\begin{enumerate}
    \item Зафиксируем $\forall \varepsilon > 0$ и $x_0 \in X$
    \item Далее для доказательства достаточно дважды записать определения равномерной сходимости и
    один раз непрерывности функции $f_N (x)$ для нужных долей  $\varepsilon$ и воспользоваться неравенством
    треугольника
\end{enumerate}

\textbf{Th.2} \textit{О непрерывности суммы ряда}

\textcolor{blue}{Если функциональный ряд $u_k$ непрерывных на множестве $X$ функций сходится
равномерно на множестве $X$, то сумма ряда непрерывна на $X$}

Доказательство состоит в применение Th.1 последовательности частичных сумм ряда

\subsection{Почленное интегрирование функциональных последовательностей и рядов}

\textbf{Th.1} \textit{Об интегрировании предельной функции}

\textcolor{blue}{Если последовательность $f_k$ интегрируемых на конечно измеримом множестве $X$ функций сходится
равномерно на множестве $X$ к интегрируемой функции $f$, то интеграл этой функции есть предел интегралов}

\begin{enumerate}
    \item Воспользуемся sup-критерием для $\varepsilon = 1$.
    Тогда из неравенства следует интегрируемость $f$ по признаку сравнения
    \item Расписав супремум для разности интегралов в пределе получим 0, что завершает доказательство
\end{enumerate}

\textbf{Следствие} \textcolor{blue}{Если последовательность непрерывных на компакте $X$ функций $f_k$ сходится
равномерно к функции $f$, то интеграл этой функции есть предел интегралов} \\

Непрерывность $f$ следует из теоремы предыдущей темы, а интегрируемость из достаточного условия интегрируемости,
что позволяет применить предыдущую теорему и доказать утверждение \\

\textbf{Th.2} \textit{Об почленном интегрировании ряда}

\textcolor{blue}{Если функциональный ряд $u_k$ непрерывных на компакте $X$ функций сходится
равномерно, то сумма интеграла есть интеграл суммы} \\

Доказательство состоит в применение следствия из предыдущей теоремы к последовательности частичных сумм ряда с
использованием линейности интеграла

\subsection{Дифференцирование предельной функции и почленное дифференцирование функционального ряда}

\textbf{Th.1} \textit{О дифференцировании предельной функции}

\textcolor{blue}{Если последовательность $f_k$ непрерывно дифференцируемых на отрезке $[a, b]$ функций сходится
хотя бы в одной точке $x_0$, а последовательность производных $f_k^{'}$ сходится равномерно на $[a, b]$, то
последовательность $f_k$ сходится равномерное на $[a, b]$ к некоторой непрерывно дифференицируемой функции $f$,
    притом производная предела есть предел производных}

\begin{enumerate}
    \item Обозначим предельную функцию для $f_k^{'}$ за $\varphi (x)$, непрерывную по теореме, и предел $f_k(x_0)$
    за $A$
    \item Далее определим $f(x) =  A + \int_{x_0}^x \varphi (t) dt$ и $f_k (x) = f_k (x_0) + \int_{x_0}^x f_k^{'}(
    t) dt$
    \item Затем пара хитрых замечаний, работа с супремумом, использование sup-критерия
    \item В итоге получаем равномерную сходимость $f_k$ и требуемое равенство с учётом построения $f(x)$
\end{enumerate}

\textbf{Th.2} \textit{О почленном дифференцировании ряда} \\

\textcolor{blue}{Если функциональный ряд $u_k$ непрерывно дифференцируемых на отрезке $[a, b]$ функций сходится
хотя бы в одной точке $x_0$, а ряд производных $u_k^{'}$ сходится равномерно на $[a, b]$, то
справделива формула почленного дифференицрования ряда, то есть производная суммы ряда есть сумма производных}

Доказательство состоит в применение Th.1 к последовательности частичных сумм ряда

    \section{Степенные ряды.
    Формула Коши-Адамара для радиуса сходимости.
    Теорема о круге сходимости степенного ряда.
    Первая теорема Абеля.
    Теорема о равномерной сходимости степенного ряда.
    Вторая теорема Абеля.
    Сохранение радиуса сходимости при почленном дифференцировании степенного ряда.
    Теоремы о почленном интегрировании и дифференцировании степенного ряда.
    Единственность разложения функции в степенной ряд, ряд Тейлора.
    Достаточное условие аналитичности функции.
    Пример бесконечно дифференцируемой, но неаналитической функции.
    Представление экспоненты комплексного аргумента степенным рядом.
    Формулы Эйлера.
    Формула Тейлора с остаточным членом в интегральной форме.
    Представление степенной и логарифмической функций степенными рядами}

    \subsection{Степенные ряды}

\textbf{Опр} \textit{Предел последовательности комплексных чисел} \textcolor{gray}{Предел модуля разности равен
нулю}

Заметим, что комплексный предел эквивалентен двум вещественным (для действительной и мнимой части)

\textbf{Опр} \textit{Сходящийся комплексный ряд} \textcolor{gray}{Существует конечный предел последовательности
частичных сумм этого ряда}

\textbf{Опр} \textit{Абсолютно сходящийся комплексный ряд} \textcolor{gray}{Сходится вещественный ряд модулей
членов ряда}

И вновь сходимость комплексного ряда эквивалентна сходимости двух вещественных рядов

\textbf{Опр} \textit{Равномерно сходящийся комплекснозначная функциональная последовательность} \textcolor{gray}{
    Вещественнозначная последовательность модулей разности предельной функции и элементов последовательности
    равномерно сходится к нулю на том же множестве}

\textbf{Опр} \textit{Равномерно сходящийся комплексный функциональный ряд} \textcolor{gray}{Последовательность
частичных сумм этого ряда равномерно сходится к сумме этого ряда на том же множестве}

\textbf{Опр} \textit{Степенной ряд} \textcolor{gray}{Если задана последовательность комплексных чисел
и комплексное число, то ...}

Однако удобнее (и мы в дальнейшем будем так делать) работать с рядом без степенной разности, сделав замену
комплексной переменной

\subsection{Формула Коши-Адамара для радиуса сходимости}

\textbf{Опр} \textit{Радиус сходимости степенного ряда} \textcolor{gray}{Неотрицательное число (или бесконечность
    ), определяемое формулой Коши-Адамара}

Притом для этой формулы мы расширили операцию деления

\subsection{Теорема о круге сходимости степенного ряда}

\textbf{Опр} \textit{Круг сходимости степенного ряда} \textcolor{gray}{Круг на комплексной плоскости с центром
в $w_0 (0)$ и радиусом равным радиусу сходимости}

Если радиус сходимости бесконечен, то кругом сходимости считается вся комплексная плоскость

\textbf{Th} \textit{О круге сходимости}

\textcolor{blue}{Степенной ряд абсолютно сходится внутри круга сходимости и расходится вне его}

\begin{enumerate}
    \item Зафиксируем произвольное комплексное число $z_0 \neq 0$, обозначим $q = \frac{z_0}{R}$ и исследуем
    сходимость с помощью обобщённого признака Коши
    \item В тривиально случае $z_0 = 0$ ряд сходится абсолютно
    \item В случае $0 < \abs{z_0} < R$ в силу обобщённого признака Коши ряд сходится абсолютно
    \item В случае $\abs{z_0} > R$ в силу обобщённого признака Коши члены абсолютного ряда не стремятся к нулю,
    как и исходного ряда, а значит, он расходится по отрицанию необходимого условия
\end{enumerate}

\subsection{Первая теорема Абеля}

\textbf{Th} \textit{Первая теорема Абеля}

\textcolor{blue}{Если степенной ряд сходится в точке $z_0$, то он сходится абсолюто в любой точке по модулю
меньшей}

Доказательство следует от противного в силу п.4 теоремы о круге сходимости

\subsection{Теорема о равномерной сходимости степенного ряда}

\textbf{Th} \textit{О равномерной сходимости степенного ряда}

\textcolor{blue}{$\forall r \in (0, R)$ ряд $\sum_{\mathbb{N}}_0 c_k z^k$ сходится равномерно в круге радиуса $r$}

Доказывается через неравенство, применением теоремы о круге сходимости и по признаку Вейерштрасса равномерной
сходимости комплексного ряда

\begin{enumerate}
    \item Зафиксируем произвольное комплексное число $z_0 \neq 0$, обозначим $q = \frac{z_0}{R}$ и исследуем
    сходимость с помощью обобщённого признака Коши
    \item В тривиально случае $z_0 = 0$ ряд сходится абсолютно
    \item В случае $0 < \abs{z_0} < R$ в силу обобщённого признака Коши ряд сходится абсолютно
    \item В случае $\abs{z_0} > R$ в силу обобщённого признака Коши члены абсолютного ряда не стремятся к нулю,
    как и исходного ряда, а значит, он расходится по отрицанию необходимого условия
\end{enumerate}

\subsection{Вторая теорема Абеля}

\textbf{Th} \textit{Вторая теорема Абеля}

\textcolor{blue}{Если степенной ряд сходится в точке $z_0$, то он сходится равномерно на отрезке $[0, z_0]$}

\begin{enumerate}
    \item Разобьём члены ряда на произведение членов произведения с помощью параметра $t \in [0, 1]$
    \item Первый ряд сходится по условию (а значит, по предыдущей теореме, ещё и равномерно)
    \item Второй ряд равномерно ограничен на отрезке и монотонен по индексу
    \item Поэтому два вещественных ряда сходятся равномерно на $[0, 1]$, как и исходный ряд на $[0, z_0]$
\end{enumerate}

\subsection{Сохранение радиуса сходимости при почленном дифференцировании степенного ряда}

\textbf{Th} \textcolor{blue}{Радиусы сходимости степенных рядов, полученные формальным дифференцированием и
интегрированием исходного, совпадают с его радиусом сходимости}

\begin{enumerate}
    \item Радиусы сходимости исходного и продифференцированного рядов совпадают в силу формулы Коши-Адамара
    \item Также они сходятся или расходятся одновременно, потому как при $z = 0$ это очевидно, а в противном
    случае они отличаются на ненулевую константу (как и их пределы)
    \item Так как исходный ряд получается почленным дифференцированием интегрального, то и их радиусы сходимости
    совпадают
\end{enumerate}

\subsection{Теоремы о почленном интегрировании и дифференцировании степенного ряда}

\textbf{Th} \textit{Об интегрировании и дифференцировании степенного ряда}

\textcolor{blue}{Если вещественный степенной ряд имеет ненулевой радиус сходимости, то внутри интервала
сходимости
    \begin{itemize}
        \item справедливы формулы почленного интегрирования
        \item функция ряда имеет производные любого порядка, получаемые почленным дифференцированием ряда
        \item коэффициенты степенного ряда однозначно определяются по обрывку формулы Тейлора
    \end{itemize}   }

\begin{enumerate}
    \item Для почленного интегрирования достаточно ввести новую переменную и воспользоваться теоремами о
    равномерной сходимости степенного ряда и о почленном интегрировании равномерно сходящегося функционального ряда
    \item Для производных достаточно ввести новую переменную и воспользоваться теоремами о сохранении радиуса
    сходимости, о равномерной сходимости степенного ряда и о почленном дифференцировании функционального ряда
    \item Проводя те же рассуждения по индукции, доказываем второе утверждение теоремы
    \item Доказывается аналогично лемме первого семестра перед формулой Тейлора
\end{enumerate}

\subsection{Единственность разложения функции в степенной ряд, ряд Тейлора}

\textbf{Опр} \textit{Бесконечно дифференцируемая функция в точке} \textcolor{gray}{В этой точке существуют
производные функции любого порядка}

\textbf{Опр} \textit{Ряд Тейлора} \textcolor{gray}{Ряд бесконечно дифференцруемой функции в точке с членами ...}

\textbf{Опр} \textit{Регулярная функция в точке $z_0$} \textcolor{gray}{Ряд Тейлора функции в точке $z_0$
    сходится к функции в некоторой окрестности $z_0$}

Из теоремы об интегрировании и дифференцировании степенного ряда следует, что если функция может быть
представлена как сумма степенного ряда $\sum_{\mathbb{N}_0} a_k (z - z_0)^k$ с ненулевым радиусом сходимости, то
этот ряд является рядом Тейлора функции в точке $z_0$.
В этом случае функция является регулярной в точке $z_0$

\textbf{Опр} \textit{Остаточный член формулы Тейлора} \textcolor{gray}{Разность $n$ раз дифференцируемой функции
и формулы Тейлора}

Непосредственно из определений следует, что функция является регулярной в точке
$\Leftrightarrow \lim_{n\to\infty} r_n(x) = 0$.
Притом для доказательства регулярности недостаточно показать ненулевой радиус сходимости функции, надо ещё проверить
её остаток

\subsection{Достаточное условие аналитичности функции}

\textbf{Th} \textit{Достаточное условие регулярности}

\textcolor{blue}{Если $\exists U_\delta (x_0)$, где функция бесконечно дифференцируема и последовательность её
производных равномерно ограничена константой $C > 0$, то функция регулярна в точке и $\forall x \in U_\delta (x_0)$
    раскладывается в ряд Тейлора}

\begin{enumerate}
    \item Применим формулу Тейлора с остаточным членом в форме Лагранжа.
    Тогда остаточный член формулы Тейлора $\leq M \frac{\delta^{n+1}}{(n+1)!}$
    \item Так как факториал растёт быстрее показательной (доказывается через принцип Архимеда, определение
    факториала, цепочку неравенств и предельный переход), то остаточный член стремится к нулю
    \item Поэтому функция регулярна, потому как раскладывается в ряд Тейлора в $x_0$
\end{enumerate}

\subsection{Пример бесконечно дифференцируемой, но неаналитической функции}

\begin{equation}
    f(x) =
    \begin{cases}
        e^{-\frac{1}{x^2}}, x \neq 0; \\
        0, x = 0.
    \end{cases}
\end{equation}

Ряд Тейлора этой бесконечно дифференцируемой в точке $x_0 = 0$ сходится не к функции $f(x)$, а к
некоторой другой функции, не совпадающей с $f(x)$ в сколь угодно малой окрестности точки

\[ \forall k \in \mathbb{N} \lim_{x \to 0} \frac{1}{x^k} e^{-\frac{1}{x^2}} = \lim_{t \to +\infty} t^{\frac{k}{2}} e^{-t} = 0 \]

По индукции легко показать, что если $P_{3n} (t)$ -- многочлен степени $3n$ от $t$, то

\begin{equation}
    f^{(n)}(x) =
    \begin{cases}
        P_{3n} (\frac{1}{x}) e^{-\frac{1}{x^2}}, x \neq 0; \\
        0, x = 0.
    \end{cases}
\end{equation}

Следовательно, все коэффициенты ряда Тейлора функции $f(x)$ в точке $x_0 = 0$ равны нулю.
Поэтому сумма ряда Тейлора функции $f(x)$ в точке $x_0$ равна нулю и не совпадает с функцией $f(x)$ в сколь угодно
малой окрестности точки $x_0$.
Таким образом, хотя функция и бесконечно дифференцируема, она не является регулярной в нуле

\subsection{Представление экспоненты комплексного аргумента степенным рядом}

\textbf{Опр} \textit{Ряд Маклорена} \textcolor{gray}{Ряд Тейлора функции в нуле}

\textbf{Th.1} \textcolor{blue}{Ряды маклорена функций $e^x, \sin(x), \cos(x), \sh(x), \ch(x)$ сходятся к этим
функциям на всей числовой прямой}

\begin{enumerate}
    \item $\forall \delta > 0~\forall x \in U_\delta (0)~e^x < e^\delta$, поэтому выполнено достаточное условие
    регулярности
    \item Аналогично, используя ограниченность последовательности всех производных оставшихся функций доказываем
    их разложения
\end{enumerate}

\textbf{Th.2} \textcolor{blue}{Для комплексной экспоненты её ряд Тейлора не отличается от вещественного}

\begin{enumerate}
    \item В силу предыдущей теоремы радиус сходимости степенного ряда-претендента сходится на всём $\mathbb{C}$,
    поэтому по теореме о круге сходимости он сходится абсолютно для любого $z \in \mathbb{C}$
    \item Зафиксируем произвольное комплексное число в алгебраической форме и воспользуемся определением
    экспоненты комплексного числа, чтобы зафиксировать доказываемое равенство
    \item Покажем, что функция-ряд-претендент обладает свойством экспоненты.
    Для этого воспользуемся теоремой о перемножении абсолютно сходящихся рядов, которая для комплексных рядов
    доказывается точно так же, как и для вещественных (только здесь надо использовать метод \("\)диагоналей\("\))
    \item В результате преобразований получим сумму сумм, которую распределим по этим суммам, и применим формулу
    бинома Ньютона, завершив доказательство свойства
    \item Далее рассмотрим функцию кандидат на чисто мнимом аргументе и путём разложения на чётную и нечётную
    суммы получим выражение для чисто мнимой экспоненты
    \item В итоге, применив свойство экспоненты и убедившись, что функция работает на вещественных аргументах,
    получим разложение комплексной экспоненты в ряд Тейлора в силу единственности
\end{enumerate}

\subsection{Формулы Эйлера}

\textbf{Лемма} \textcolor{blue}{Для любого $z \in \mathbb{C}$ справедливы формулы Эйлера} \textcolor{gray}{Они
используют новопостроенные комплексные функции и подравнивают комплексную тригонометрию к вещественной гиперболике}

\begin{enumerate}
    \item Для доказательства формулы гиперкомплексной экспоненты достаточно разделить сумм на чётную и нечётную,
    а затем воспользоваться $i^2 = -1$
    \item Остальные формулы следуют из первой
\end{enumerate}

\subsection{Формула Тейлора с остаточным членом в интегральной форме}

\textbf{Th} \textit{Формула Тейлора с остаточным членом в интегральной форме}

\textcolor{blue}{Если функция в $U_\delta (x_0)$ имеет непрерывные производные по $n+1$ порядок, то для
остаточного члена формулы Тейлора справедливо представление в интегральной форме: $r_n (x) = \frac{1}{n!} \int_{
    x_0}^x (x - t)^n f^{n+1}(t)dt \forall x \in U_\delta (x_0) $}

\begin{enumerate}
    \item При $n = 0$ теорема справедлива в силу формулы Ньютона -- Лейбница
    \item Пусть теорема справедлива для $n = s - 1$.
    Тогда проинтегрируем $r_{s-1}$ по частям
    \item Затем, расписав $r_s$ по определению, подставим проинтегрированное выражение и получим требуемое равенство
    \item Таким образом, теорема доказана по индукции
\end{enumerate}

\subsection{Представление степенной и логарифмической функций степенными рядами}

\textbf{Th} \textcolor{blue}{Ряд Маклорена степенной функции сходится к этой функции на интервале единичного радиуса}

\begin{enumerate}
    \item Зафиксируем $x \in (-1; 1)$ и учитывая выражение для $f^{n}$ распишем остаточный член в интегральной
    форме, походу дела вынося константы, вводя новые обозначения и переменные интегрирования
    \item Затем воспользуемся ограниченностью $x$ для оценки.
    Осталось показать, что $\lambda_n \rightarrow 0$
    \item В тривиальных случаях $x = 0$ и $\alpha = m \in \mathbb{N}_0, m < n$ утверждение очевидно
    \item В общем случае найдём предел отношения и воспользуемся схожими рассуждениями с доказательством признака
    Даламбера (сравнение с геометрической прогрессией)
\end{enumerate}

Заметим, что при $m \geq n$ ряд Маклорена совпадает с конечной суммой \\

Из доказанного и теоремы о почленном интегрировании степенного ряда при $\abs{x} < 1$ (не забывая про замену
индекса суммирования) получаем ряд Маклорена для логарифма.
Данное разложение справедливо и при $x = 1$.
Действительно, данный ряд будет сходиться по признаку Лейбница.
Следовательно, в силу второй теоремы Абеля этот ряд сходится равномерно на отрезке $[0; 1]$.
Согласно теореме о непрерывности суммы равномерно сходящегося функционального ряда частичные суммы этого ряда будет
непрерывны на отрезке $[0; 1]$.
Поэтому существует требуемый предел

\end{document}
