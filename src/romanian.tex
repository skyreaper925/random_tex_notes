%! Author = user
%! Date = 17.05.2023

\documentclass[a4paper, 14pt]{article}
%\documentclass[draft]{article}

\usepackage[T2A]{fontenc}
\usepackage[utf8]{inputenc}
\usepackage[english, russian]{babel}
\usepackage[top = 2cm, bottom = 2cm, left = 2cm, right = 2cm]{geometry}
\usepackage{indentfirst}
\usepackage{xcolor}
\usepackage{hyperref}
\usepackage{gensymb}
\usepackage{pgfplots}
\usepackage{amsmath, amsfonts, amsthm, mathtools}
\usepackage{amssymb}
\usepackage{physics, multirow, float}
\usepackage{wrapfig, tabularx}
\usepackage{icomma} % Clever comma: 0,2 - number while 0, 2 - two numbers
\usepackage{tikz, standalone}
\usepackage{fancyhdr,fancybox}
\usepackage{lastpage}
\usepackage{booktabs}
\usepackage{listings}
\usepackage{lstmisc}
\usepackage{stmaryrd}

%\полуторный интервал
\onehalfspacing

\hypersetup
{   colorlinks = false,
    linkcolor = blue,
    pdftitle = {analysis},
    pdfauthor = {Володин Максим},
    allcolors = [RGB]{010 090 200}
}

%\gravarphicspath{{images/}}
%\DeclareGravarphicsExtensions{.pdf,.png,.jpg}

\restylefloat{table}
\usetikzlibrary{external}

\mathtoolsset{showonlyrefs = true} % Numbers will appear only where \eqref{} in the text LINKED
\pagestyle{fancy}

\pgfplotsset{compat=1.18}
\title{Открытая командная румынская олимпиада по физике, 2017}
\date{Теоретическая часть - Октябрь 27, 2017}

\begin{document}

    \maketitle

    \section*{Теоретическая задача 1: Два парадокса}
    Парадокс - это утверждение, которое явно ложно или противоречиво, если пытаться его рассмотреть с двух сторон
    \subsection*{А. Где недостающая энергия?}
    \subsubsection*{4 балла}
    Тело (рассматриваемое как материальная точка) с массой $m$ покоится на склоне (рис 1).
    При отпускании тело начинает скатываться к основанию склона и продолжает двигаться вправо по горизонтальной
    плоскости.
    Угол между наклонной и горизонтальной плоскостями гладок (величина скорости тела не меняется, меняется лишь
    направление).
    Вначале тело находится на высоте $h$ над землёй.
    Высота $h$ и величина ускорения свободного падения $g$ известны.
    Трения нет

    \begin{description}
        \item [A1]
        Найдите скорость тела $v_0$ на горизонтальной плоскости (0,3 балла)
    \end{description}

    Рассмотрим теперь тот же процесс, но в системе отсчета, движущейся вправо с постоянной скоростью $v_0$
    относительно земли.
    В такой модели конечная энергия тела равна нулю, а его начальная энергия положительна

    \begin{description}
        \item [A2] Куда «исчезла» энергия?
        Подробно распишите компоненты недостающей энергии (3,7 балла)
    \end{description}

    \subsection*{B. Где недостающий импульс?}
    \subsubsection*{6 баллов}
    \textbf{B1 Импульс электромагнитных волн}

    \vspace{\baselineskip}
    Энергия, передаваемая электромагнитной волной за единицу времени на единицу поверхности, называется вектором
    Пойнтинга и имеет формулу $\vec{S}$ = $\frac{1}{\mu_0}$ $\vec{E}$ $\times$ $\vec{B}$, где $\mu_0$ - магнитная
    проницаемость воздуха.
    Электрическая проницаемость воздуха ${\upvarepsilon_0}$ также известна

    \begin{description}
        \item [B1] Выведите объемную плотность импульса электромагнитной
        волны $p_V$.
        Выразите результат в форме вектора (\overrightarrow {$p_V$}) (0,9 балла)
    \end{description}

    \vspace{\baselineskip}
    \textbf{B2 Парадокс Фейнмана}

    \vspace{\baselineskip}
    На Рис.~2 показаны две длинные концентрические цилиндрические оболочки длиной $l$.
    Внутренний цилиндр имеет
    радиус $a$ и электрический заряд $+Q$, равномерно распределенный по его поверхности.
    Внешний цилиндр имеет радиус $b$ ($b$ $\ll$ $l$)
    и электрический заряд $-Q$ равномерно распределенный по его поверхности.
    Цилиндры изготовлены из одинакового материала, имеющего поверхностную плотность $\upvarsigma$.
    Концентрически с ними расположен длинный соленоид радиусом $R$ ($a$ < $R$ < $b$), имеющий $n$ оборотов на единицу
    длины и несущий электрический ток зарядом $i$.
    Соленоид фиксирован, но цилиндрические оболочки могут свободно и независимо вращаться вокруг своей общей оси.
    Изначально вся система покоится
    \newpage
    \vspace{\baselineskip}
    \textbf{B2.1 Угловые скорости}

    \begin{description}
        \item [B2.1] Когда ток в соленоиде начинает постепенно снижаться до нуля, цилиндры начинают вращаться.
        Найдите значение конечных угловых скоростей (их величину и направление) обоих цилиндров, когда тока на
        соленоиде не будет (2,6 балла)
    \end{description}

    \textbf{Учтите:} цилиндрические оболочки настолько тяжелые, что не создают собственное магнитное поле вращением

    \vspace{\baselineskip}
    \textbf{B2.2 Парадокс Фейнмана}

    \begin{description}
        \item [B2.2] Поскольку ни одна внешняя сила не действовала на систему, ее импульс должен был сохраниться.
        Откуда «появился» момент импульса?
        Подробно распишите этот парадокс численно (1,3 балла)
    \end{description}

    \textbf{B2.3 Радиальная спица}

    \begin{description}
        \item [B2.3] Вместо того, чтобы уменьшать ток через соленоид, цилиндры жестко связаны с радиальной спицей
        незначительной массы (нам сейчас не интересен способ осуществить это на практике).
        Спица является слабым проводником, чтобы пренебречь током смещения.
        Определите общий импульс цилиндров в этом случае, а также их угловые скорости (1,2 балла)
    \end{description}
    \author{
        \begin{flushright}
            предложено

            \textbf{Доц. Проф. Себастьяном ПОПЕСКУ, д-ром наук}

            Физический факультет Ясского университета имени Александру Иона Кузы, РУМЫНИЯ
        \end{flushright}
    }

\end{document}
