    \subsection{Линейная независимость собственных подпространств, отвечающих различным собственным значениям}

    \textbf{Th} \textcolor{blue}{$\lambda$ -- собственное значение $\Leftrightarrow \lambda$ характеристическое число}

    \begin{enumerate}
        \item $\lambda$ -- собственное значение $\Leftrightarrow \ker (\varphi - \lambda) \neq O$
        \item $\Leftrightarrow$ соответствующая СЛУ имеет нетривиальное решение
        \item $\Leftrightarrow$ соответствующая квадратная матрица вырождена
        \item $\Leftrightarrow$ соответствующий определитель равен нулю
        \item $\Leftrightarrow \lambda$ характеристическое число
    \end{enumerate}

    \textbf{Th} \textcolor{blue}{Собственные векторы различных собственных значений л.н.з.}

    \begin{enumerate}
        \item Доказывается по индукции.
        База очевидна
        \item Докажем переход.
        Для этого рассмотрим $k+1$ собственный вектор, из которых $k$ заведомо л.н.з
        \item Применим к их л.к. $\varphi$.
        Из неё вычтем правильную л.к. первых $k$ векторов (чтобы обнулить $\alpha_{k+1}$)
        \item Итого, $k$ коэффициентов нули, а значит, и $k+1$ тоже, то есть система осталась л.н.з.
    \end{enumerate}

    \subsection{Алгебраическая и геометрическая кратность собственного значения}

    \textbf{Опр} \textit{Геометрическая кратность} \textcolor{gray}{Размерность собственного подпространства}

    \textbf{Th} \textcolor{blue}{Геометрическая кратность не превосходит алгебраическую}

    \begin{enumerate}
        \item Рассмотрим собственное пространство размерности $s$ и произвольный базис в нём.
        Дополним его до базиса во всём пространстве
        \item Запишем матрицу линейного оператора.
        Она будет иметь блочно-диагональный вид
        \item Вычислим характеристический многочлен матрицы и непосредственно убедимся в доказываемом (потому как в
        оставшемся многочлене собственное значение может быть корнем; в противном случае достигается
        равенство)
    \end{enumerate}

     \subsection{Критерий диагонализируемости преобразования}

    \textbf{Опр} \textit{Диагонализируемое преобразование} \textcolor{gray}{Существует базис, в котором достигается
    диагональный вид}

    \textbf{Th} \textit{Первый критерий диагонализируемости}

    \textcolor{blue}{Если $\varphi \in L(V, V)$ имеет попарно различные собтсвенные значения $\lambda_i$
        кратнойстей $s_i$, то следующие условия эквивалентны:
    \begin{enumerate}
        \item $\varphi$ диагонализируем
        \item В пространстве существует базис из собственных векторов
        \item $\dim V_{\lambda_i} = s_i$
        \item $V = \oplus_i V_{\lambda_i}$
    \end{enumerate}             }

\begin{itemize}
    \item $1 \Leftrightarrow 2:$ в силу того, что матрица $A$ есть склейка применения $\varphi$ на базисные векторы
    \item $2 \Rightarrow 3:$ суммируем $t_i \leq \dim V_{\lambda_i} \leq s_i$ по $i$
    \item $3 \Rightarrow 4:$ так как собственные пространства разных собственных значений не пересекаются, то они
    разлагаются в прямую сумму.
    Сумма их размерностей будет $\sum_i s_i = n$, то есть всего пространства
    \item $4 \Rightarrow 2:$ достаточно выбрать базис в каждом подпространстве и объединить.
    Ранее доказывалось, что он будет базисом во всём пространстве (второй критерий прямой суммы)
\end{itemize}

     \textbf{Следствие} \textit{Достаточное условие диагонализируемости}

    \textcolor{blue}{Если характеристический многочлен имеет $n$ различных корней из поля, то $\varphi$
        диагонализируем}

    Действительно, в таком случае у каждого собственного подпространство размерность единица и они располагаются на
    главной диагонали