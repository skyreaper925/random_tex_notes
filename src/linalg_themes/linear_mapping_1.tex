\subsection{Линейные отображения и линейные преобразования векторных пространств (линейные операторы)}

\textbf{Опр} \textit{Линейное отображение} \textcolor{gray}{Отображение, удовлетворяющее двум аксиомам}

Отсюда следуют конечная линейность, отображение нулевого и противоположного вектора

Множество всех линейных отображений обозначается как $L(V, W)$.
В случае $W = V$ линейное отображение называют линейным преобразованием (оператором)

\textbf{Опр} \textit{Линейная функция (функционал)} \textcolor{gray}{Случай $\dim W = 1 (W = \mathbb{K})$}

\textbf{Утв} \textcolor{blue}{Под действием линйного отображения л.з система остаётся л.з}

Достаточно записать нетривиальную линейную комбинацию и взять её образ, используя уже известные аксиомы

\textbf{Утв} \textcolor{blue}{Ранг системы под действием линейного отображения не возрастает}

Это следует из определения ранга и противного к предыдущему утверждению.
В силу равенства ранга и размерности в конечномерном случае, получаем аналогичное неравенство для размерностей

\textbf{Утв} \textit{Образ подпространства}

\textcolor{blue}{Образ линейной оболочки есть линеная оболочка образов}

Действительно, если записать определение линейной оболочки (множество всех линейных комбинаций) и подействовать
отображением, то получится требуемое.
В частном случае, если взять базис (его линейная оболочка есть всё пространство), то образ пространства есть
линейная оболочка линейная оболочка образов базисных векторов

\textbf{Опр} \textit{Линейное вложение} \textcolor{gray}{Инъективное линейное отображение}

\textbf{Утв} \textcolor{blue}{В случае линейного вложения л.н.з. система остаётся л.н.з.}

Действительно, если записать л.к. образов и \("\)вынести $\varphi$ за скобки\("\), то в силу инъективности
получим, л.к. исходных векторов.
В силу её линейной независимости, эта л.к. тривиальна, как и л.к. образов
В частном случае, если взять базис, то получим равенства рангов $U$ и $\varphi (U)$, как и размерностей

\textbf{Th} \textcolor{blue}{Если взять базис $e_i$ в $V$ и произвольные векторы $c_i$ в $W$, то
    $\exists ! \varphi: \varphi(e_i) = c_i$. Дополнительно, $\varphi$ инъективно $c_i$ л.н.з.}

\begin{enumerate}
    \item Для начала докажем единственность.
    Зафиксируем произвольный вектор $a$ пространства, разложим его по базису и рассмотрим $\varphi (a)$, имеющего
    единственные коэффициенты.
    В силу произвольности $a$ теорема справедлива
    \item Для доказательства существования, достаточно взять два произвольных вектора из пространства,
    подействовать на них отображением (с учётом $\varphi(e_i) = c_i$), затем проверить аксиомы линейного отображения
    \item $\Rightarrow$: следует из предыдущего утверждения
    \item $\Leftarrow$: от противного, с использованием определения инъективности, разложения $a - b$ по базису и
    $\varphi(e_i) = c_i$
\end{enumerate}

\subsection{Операции над линейными отображениями, линейное пространство линейных отображений}

\textbf{Опр} \textit{Сумма отображений} \textcolor{gray}{Такое отображение, что ...}

\textbf{Опр} \textit{Произведение отображения на скаляр} \textcolor{gray}{Такое отображение, что ...}

В комплексном случае скаляр заменяется на комплексно-сопряжённый.

Нетрудно проверить, что оба нововведённых отображения линейны.
Также проверкой доказывается ассоциативность, дистрибутивность и линейность в случае композиции отображений

\subsection{Алгебра линейных операторов}

Так как на множестве $L(V, V)$ определены операции сложения, умножения на скаляр и умножения, то $L(V, V)$ имеет
структуру ассоциативной алгебры (непустое множество (носитель) с заданным на нём набором операций и отношений (
сигнатурой)).
Ассоциативная потому как заданы операции ассоциативного умножения, то есть $\forall k, l \in \mathbb{F}$ и
$\forall a, b, c \in A$ справедливо

\begin{enumerate}
    \item $a(b + c) = ab + ac$
    \item $(a + b)c = ac + bc$
    \item $(k+l)a = ka + la$
    \item $k(a + b) = ka + kb$
    \item $k(la) = (kl)a$
    \item $k(ab) = (ka)b = a(kb)$
    \item $1a = a$, где 1 -- единица кольца $\mathbb{K}$
\end{enumerate}

\textbf{Опр} \textit{Аннулирующий многочлен для оператора} \textcolor{gray}{$P(\varphi) = 0$}

\textbf{Опр} \textit{Минимальный многочлен} \textcolor{gray}{Аннулирующий многочлен с минимальной степенью}

\textbf{Утв} \textcolor{blue}{Пусть $\mu$ -- минимальный многочлен оператора $\varphi$,а $P \in \mathbb{F}$ --
произвольный.
Тогда $P$ аннулирует $\varphi \Leftrightarrow f \vdots \mu$ в кольце многочленов над $\mathbb{F}$}

\begin{enumerate}
    \item Разделим $P$ на $\mu$ с остатком и подставим в полученное равенство $\varphi$
    \item Воспользуемся условием и получим $P(\varphi) = 0 \Leftrightarrow r(\varphi) = 0$
    \item В таком случае остаток должен быть аннулирующим для $\varphi$, то есть его степень меньше степени
    минимального многочлена, поэтому $w$ не возникает только в случае $r \equiv 0$
    \item Таким образом, эквивалентность доказана
\end{enumerate}

Отсюда следует, что минимальный многочлен единственен с точностью до умножения на константу

\subsection{Изоморфизмы}

\textbf{Опр} \textit{Изоморфизм} \textcolor{gray}{Линейное биективное отображение}

\textbf{Опр} \textit{Изоморфные векторные пространства} \textcolor{gray}{Между ними существует изоморфизм}

\textbf{Утв} \textcolor{blue}{Обратный к изоморфизму изоморфизм}

\begin{enumerate}
    \item Биективность следует из тождеств для обратных функций
    \item Далее берутся векторы из образа и на них проверяются аксиомы линейного отображения
    \item Итого, обратный к изоморфизму изоморфизм по определению
\end{enumerate}

\textbf{Th} \textit{Классификация конечномерных векторных пространств}

\textcolor{blue}{Пространства изоморфны $\Rightarrow$ их размерности совпадают}

\begin{enumerate}
    \item $\Rightarrow:$ из изоморфности следует инъективность, а для инъективных отображений равенство доказано ранее
    \item $\leftarrow:$ построим изоморфизм между элементами каждого пространствами и их координатными столбцами в
    них по фиксированному базису.
    Ранее было доказано, что такое разложение единственно.
    Достаточно обратить какое-то из отображений (по предыдущему утверждению оно тоже будет изоморфизмом).
    Итого, мы получили композицию изоморфизмов, то есть изоморфизм
\end{enumerate}

\textbf{Th} \textcolor{blue}{Если конечномерные пространства $U, V: \dim U = \dim V; e_i$ -- базис в $U$,
    $\varphi \in L(U,V)$. Тогда следующие условия эквивалентны:
    \begin{enumerate}
        \item $\varphi$ -- изоморфизм
        \item $\varphi$ инъективен
        \item $\varphi$ сюръективен
        \item $\varphi (e_i)$ есть базис в $V$
    \end{enumerate}         }

\begin{itemize}
    \item $1 \Rightarrow 2:$ по определению
    \item $2 \Rightarrow 3:$ из инъективности следует $\dim (\varphi(U)) = \dim U \rightarrow \varphi(V) \cong U
    \rightarrow \varphi$ сюръективно
    \item $3 \Rightarrow 4:$ это следует из свойства линейной оболочки образов базисных векторов, связи
    размерности и ранга, определения ранга и базиса
    \item $4 \Rightarrow 1:$ по критерию инъективности, в силу л.н.з. $\varphi (e_i), \varphi$ будет инъективно,
    а из свойства линейной оболочки следует сюръективность $\varphi$
\end{itemize}