%! Author = user
%! Date = 17.05.2023

\documentclass[a4paper, 14pt]{article}

%\hypersetup
%{   colorlinks,
%    pdftitle={C++ themes},
%    pdfauthor={Володин Максим},
%    allcolors=[RGB]{010 090 200}
%}

\usepackage[T2A]{fontenc}
\usepackage[utf8]{inputenc}
\usepackage[english, russian]{babel}
\usepackage[top = 2cm, bottom = 2cm, left = 2cm, right = 2cm]{geometry}
\usepackage{indentfirst}
\usepackage{xcolor}
\usepackage{hyperref}
\usepackage{graphicx}
\usepackage{gensymb}
\usepackage{pgfplots}
\usepackage{amsmath, amsfonts, amsthm, mathtools}
\usepackage{amssymb}
\usepackage{physics, multirow, float}
\usepackage{wrapfig, tabularx}
\usepackage{icomma} % Clever comma: 0,2 - number while 0, 2 - two numbers
\usepackage{tikz, standalone}
\usepackage{fancyhdr,fancybox}
\usepackage{lastpage}
\usepackage{booktabs}
\usepackage{listings}
\usepackage{lstmisc}

\graphicspath{{images/}}
\DeclareGraphicsExtensions{.pdf,.png,.jpg}

\restylefloat{table}
\usetikzlibrary{external}

\mathtoolsset{showonlyrefs = true} % Numbers will appear only where \eqref{} in the text LINKED
\pagestyle{fancy}

\fancyhf{}
\fancyhead[R]{Конспект билетов}
\fancyfoot[R]{\thepage /\pageref{LastPage}}
\fancyhead[L]{Линейная алгебра}

\pgfplotsset{compat=1.18}

\begin{document}

    \tableofcontents \newpage

    \addcontentsline{toc}{section}{Векторные пространства} \part*{Векторные пространства}

    \section{Векторное пространство.
    Подпространство.
    Линейная оболочка системы векторов.
    Линейно (не)зависимые системы векторов.
    Конечномерные линейные пространства}

    \subsection{Векторное пространство}

    \textbf{Опр} \textit{Унарная, бинараная операция на множестве над полем} \textcolor{gray}{Ставит в соответсвие
    элементу (элементам) из множества другой элемент из множества}

    \textbf{Опр} \textit{Векторное пространство над полем} \textcolor{gray}{Помимо унарности и бинарности, по 4 аксиомы
    сложения и умножения}

    \subsection{Подпространство}

    \textbf{Опр} \textit{Подпространство} \textcolor{gray}{Требуются лишь унарность и бинарность}

    \subsection{Линейная оболочка системы векторов}

    \textbf{Опр} \textit{Линейная оболочка} \textcolor{gray}{Все векторы, которые линейно выражаются через минимальную
    систему, покрывающую пространство}

    \subsection{Линейно (не)зависимые системы векторов}

    \textbf{Опр} \textit{Линейная комбинация}
    \textcolor{gray}{Сумма векторов с коэффциентами из поля}

    \textbf{Опр} \textit{Линейно (не)зависимая система векторов}
    \textcolor{gray}{Нетривиальная линейная комбинация (не) равна нулю}

    \subsection{Конечномерные линейные пространства}

    \textbf{Опр} \textit{Ранг (не)пустой системы векторов}
    \textcolor{gray}{Любой набор векторов, чьё число большее чем ранг, будет линейно зависим. Ранг пустой считаем
    нулевым}

    \textbf{Опр} \textit{Размерность}
    \textcolor{gray}{Более употребительное название для ранга в случае работы с подпространсвом}

    \textbf{Опр} \textit{(Бес)конечномерные линейные пространства}
    \textcolor{gray}{Если их размерность (бес)конечна}

    \textbf{Л1} \textcolor{blue}{Любой вектор системы векторов ранга $r$ раскладывается по $r$ л.н.з векторам}

    \begin{enumerate}
        \item Возьмём вектора из линейной оболочки и добавим к ним произвольный вектор системы $a$.
        Эта система будет л.з. из определения ранга
        \item Тогда найдутся коэффициента для нетривиальной линейной комбинации, притом коэффициент перед
        $\lambda_a \neq 0$ (иначе линейная оболочка была бы зависима)
        \item Из линейной комбинации выразим $a$, поделив все вектора на $\lambda_a$
    \end{enumerate}

    \textbf{Л2} \textcolor{blue}{Если вектор $b$ принадлежит линейной оболочке $a_1, \dots, a_k$ других векторов, то
    он не влияет на её ранг}

    \begin{enumerate}
        \item От противного: пусть $\exists$ л.н.з система из $r+1$ векторов (она будет содержать $b$, иначе $w$ с
        определением ранга)
        \item Итак, пусть система $b, a_1, \dots, a_r$ л.н.з. Тогда система $a_1, \dots, a_r$ тоже будет л.н.з. Так
        их $r$ штук, то все вектора $a_1, \dots, a_k$ будут выражаться через $a_1, \dots, a_r$
        \item Если мы заменим $a_1, \dots, a_k$ на их выражения через $a_1, \dots, a_r$, то получится, что $b$
        выражается по ним, что $w$ л.н.з $b, a_1, \dots, a_r$
    \end{enumerate}

    \textbf{Th} \textit{Основная теорема о рангах}
    \textcolor{blue}{Ранг подсистемы и системы совпадает $\Leftrightarrow$ $\forall$ вектор системы
    раскладывается по линейной обололочке подсистемы}

    $\Rightarrow$: мгновенно следует из Л1 \textcolor{gray}{В моей формулировке}

    $\Leftarrow$:
    \begin{enumerate}
        \item От противного: пусть $\exists$ л.н.з система из $r+1$ векторов
        \item Её ранг будет не меньше ранга её и любого количества л.н.з векторов из подсистемы
        \item С другой стороны, многократно применяя Л2, получим что её ранг не превышает ранга подсистемы, $w$
    \end{enumerate}

    \textbf{Следствие 1}
    \textcolor{blue}{Для любой подсистемы векторного пространства ранг равен размерности линейной оболочки}

    \textbf{Следствие 2}
    \textcolor{blue}{Если размерности вложенных подпространств совпадают, то они равны}

    \section{Базис и размерность конечномерного линейного пространства, корректность ее определения
        (лемма Штайница).
        Дополнение линейно независимой системы векторов до базиса.
        Координаты вектора в базисе, запись операций над векторами через координаты.
        Изменение координат вектора при замене базиса.
        Матрица перехода}

    \subsection{Базис и размерность конечномерного линейного пространства, корректность ее определения
        (лемма Штайница)}

    \textbf{Опр} \textit{Базис} \textcolor{gray}{Система л.н.з векторов, являющаяся линейной оболочкой}

    \textbf{Л} \textit{Штайница}
    \textcolor{blue}{Пусть система векторов $a_1, \dots, a_n$ порождает пространство $V$, а система
    векторов $b_1, \dots, b_m$ л.н.з. Тогда $n \geq m$}

    \begin{enumerate}
        \item Возьмём $b_1$. Он будет выражаться через $a_1, \dots, a_n$ по определению линейной оболочки. БОО первый
        коэффициент в его разложении по $a_1, \dots, a_n$ ненулевой (иначе мы их переупорядочим)
        \item Выразим из этого разложения $a_1$. Тогда $V = <b_1, a_2, \dots, a_n>$. Так, действуя по индукции,
        заменим все вектора $a_i$
        \item В случае $n < m$ получим противоречие с л.н.з. $b_1, \dots, b_m$ (потому что всего $n$ векторов
        порождают пространство). Таким образом $n \geq m$, притом недостающие до линейно оболочки вектора можно взять
        из $a_1, \dots, a_n$
    \end{enumerate}

    \subsection{Дополнение линейно независимой системы векторов до базиса}

    \textbf{Утв} \textcolor{blue}{Систему л.н.з векторов можно дополнить до базиса}

    \begin{enumerate}
        \item Ранг подсистемы меньше ранга системы, поэтому выполняется обратное к основной теореме о рангах
        утверждение ($\exists~\overline{x}$), не лежащей в л.н.з подсистеме)
        \item Если мы добавим $\overline{x}$ к подсистеме и она станет зависимой, то в нетривиальной линейной комбинации равен
        нулю либо новый коэффициент ($w$ с л.н.з исходной подсистемы), либо какой-то из старых (тогда $\overline{x}$
        выражается через вектора линейной оболочки и принадлежит ей)
        \item Продолжая процесс и далее, дополним систему до базиса
    \end{enumerate}

    \subsection{Координаты вектора в базисе, запись операций над векторами через координаты}

    \textbf{Опр} \textit{Координаты вектора в базисе} \textcolor{gray}{Коэффициенты в разложении по базису}

    При сложении векторов и домножении на число, координаты изменяются покомпонентно

    \section{Теорема о неявной функции для одного уравнения.
    Теорема Лагранжа о среднем для вектор-функции нескольких переменных.
    Принцип Банаха сжимающих отображений.
    Теорема о неявной функции для системы уравнений.
    Теорема об обратном отображении.
    Теорема о расщеплении отображений}

\end{document}
