%! Author = user
%! Date = 17.05.2023

\documentclass[a4paper, 14pt]{article}

%\hypersetup
%{   colorlinks,
%    pdftitle={analysis_themeslinalg summary},
%    pdfauthor={Володин Максим},
%    allcolors=[RGB]{010 090 200}
%}

\usepackage[T2A]{fontenc}
\usepackage[utf8]{inputenc}
\usepackage[english, russian]{babel}
\usepackage[top = 2cm, bottom = 2cm, left = 2cm, right = 2cm]{geometry}
\usepackage{indentfirst}
\usepackage{xcolor}
\usepackage{hyperref}
\usepackage{graphicx}
\usepackage{gensymb}
\usepackage{pgfplots}
\usepackage{amsmath, amsfonts, amsthm, mathtools}
\usepackage{amssymb}
\usepackage{physics, multirow, float}
\usepackage{wrapfig, tabularx}
\usepackage{icomma} % Clever comma: 0,2 - number while 0, 2 - two numbers
\usepackage{tikz, standalone}
\usepackage{fancyhdr,fancybox}
\usepackage{lastpage}
\usepackage{booktabs}
\usepackage{listings}
\usepackage{lstmisc}
\usepackage{mathtools}

\graphicspath{{images/}}
\DeclareGraphicsExtensions{.pdf,.png,.jpg}

\restylefloat{table}
\usetikzlibrary{external}

\mathtoolsset{showonlyrefs = true} % Numbers will appear only where \eqref{} in the text LINKED
\pagestyle{fancy}

\fancyhf{}
\fancyhead[R]{Конспект билетов}
\fancyfoot[R]{\thepage /\pageref{LastPage}}
\fancyhead[L]{Линейная алгебра}

\pgfplotsset{compat=1.18}

\begin{document}

    \tableofcontents \newpage

    \addcontentsline{toc}{section}{Векторные пространства} \part*{Векторные пространства}

    \section{Векторное пространство.
    Подпространство.
    Линейная оболочка системы векторов.
    Линейно (не)зависимые системы векторов.
    Конечномерные линейные пространства}

    \subsection{Векторное пространство}

    \textbf{Опр} \textit{Унарная, бинараная операция на множестве над полем} \textcolor{gray}{Ставит в соответсвие
    элементу (элементам) из множества другой элемент из множества}

    \textbf{Опр} \textit{Векторное пространство над полем} \textcolor{gray}{Помимо унарности и бинарности, по 4 аксиомы
    сложения и умножения}

    \subsection{Подпространство}

    \textbf{Опр} \textit{Подпространство} \textcolor{gray}{Требуются лишь унарность и бинарность}

    \subsection{Линейная оболочка системы векторов}

    \textbf{Опр} \textit{Линейная оболочка} \textcolor{gray}{Все векторы, которые линейно выражаются через минимальную
    систему, покрывающую пространство}

    \subsection{Линейно (не)зависимые системы векторов}

    \textbf{Опр} \textit{Линейная комбинация}
    \textcolor{gray}{Сумма векторов с коэффциентами из поля}

    \textbf{Опр} \textit{Линейно (не)зависимая система векторов}
    \textcolor{gray}{Нетривиальная линейная комбинация (не) равна нулю}

    \subsection{Конечномерные линейные пространства}

    \textbf{Опр} \textit{Ранг (не)пустой системы векторов}
    \textcolor{gray}{Любой набор векторов, чьё число большее чем ранг, будет линейно зависим. Ранг пустой считаем
    нулевым}

    \textbf{Опр} \textit{Размерность}
    \textcolor{gray}{Более употребительное название для ранга в случае работы с подпространсвом}

    \textbf{Опр} \textit{(Бес)конечномерные линейные пространства}
    \textcolor{gray}{Если их размерность (бес)конечна}

    \textbf{Л1} \textcolor{blue}{Любой вектор системы векторов ранга $r$ раскладывается по $r$ л.н.з векторам}

    \begin{enumerate}
        \item Возьмём вектора из линейной оболочки и добавим к ним произвольный вектор системы $a$.
        Эта система будет л.з. из определения ранга
        \item Тогда найдутся коэффициента для нетривиальной линейной комбинации, притом коэффициент перед
        $\lambda_a \neq 0$ (иначе линейная оболочка была бы зависима)
        \item Из линейной комбинации выразим $a$, поделив все вектора на $\lambda_a$
    \end{enumerate}

    \textbf{Л2} \textcolor{blue}{Если вектор $b$ принадлежит линейной оболочке $a_1, \dots, a_k$ других векторов, то
    он не влияет на её ранг}

    \begin{enumerate}
        \item От противного: пусть $\exists$ л.н.з система из $r+1$ векторов (она будет содержать $b$, иначе $w$ с
        определением ранга)
        \item Итак, пусть система $b, a_1, \dots, a_r$ л.н.з. Тогда система $a_1, \dots, a_r$ тоже будет л.н.з. Так
        их $r$ штук, то все вектора $a_1, \dots, a_k$ будут выражаться через $a_1, \dots, a_r$
        \item Если мы заменим $a_1, \dots, a_k$ на их выражения через $a_1, \dots, a_r$, то получится, что $b$
        выражается по ним, что $w$ л.н.з $b, a_1, \dots, a_r$
    \end{enumerate}

    \textbf{Th} \textit{Основная теорема о рангах}
    \textcolor{blue}{Ранг подсистемы и системы совпадает $\Leftrightarrow$ $\forall$ вектор системы
    раскладывается по линейной обололочке подсистемы}

    $\Rightarrow$: мгновенно следует из Л1 \textcolor{gray}{В моей формулировке}

    $\Leftarrow$:
    \begin{enumerate}
        \item От противного: пусть $\exists$ л.н.з система из $r+1$ векторов
        \item Её ранг будет не меньше ранга её и любого количества л.н.з векторов из подсистемы
        \item С другой стороны, многократно применяя Л2, получим что её ранг не превышает ранга подсистемы, $w$
    \end{enumerate}

    \textbf{Следствие 1}
    \textcolor{blue}{Для любой подсистемы векторного пространства ранг равен размерности линейной оболочки}

    \textbf{Следствие 2}
    \textcolor{blue}{Если размерности вложенных подпространств совпадают, то они равны}

    \section{Базис и размерность конечномерного линейного пространства, корректность ее определения
        (лемма Штайница).
        Дополнение линейно независимой системы векторов до базиса.
        Координаты вектора в базисе, запись операций над векторами через координаты.
        Изменение координат вектора при замене базиса (матрица перехода)}

    \subsection{Базис и размерность конечномерного линейного пространства, корректность ее определения (лемма Штайница)}

    \textbf{Опр} \textit{Базис} \textcolor{gray}{Система л.н.з векторов, являющаяся линейной оболочкой}

    \textbf{Лемма} \textit{Штайница}
    \textcolor{blue}{Пусть система векторов $a_1, \dots, a_n$ порождает пространство $V$, а система
    векторов $b_1, \dots, b_m$ л.н.з. Тогда $n \geq m$}

    \begin{enumerate}
        \item Возьмём $b_1$.
        Он будет выражаться через $a_1, \dots, a_n$ по определению линейной оболочки.
        БОО первый коэффициент в его разложении по $a_1, \dots, a_n$ ненулевой (иначе мы их переупорядочим)
        \item Выразим из этого разложения $a_1$.
        Тогда $V = <b_1, a_2, \dots, a_n>$.
        Так, действуя по индукции, заменим все вектора $a_i$
        \item В случае $n < m$ получим противоречие с л.н.з. $b_1, \dots, b_m$ (потому что всего $n$ векторов
        порождают пространство).
        Таким образом $n \geq m$, притом недостающие до линейно оболочки вектора можно взять из $a_1, \dots, a_n$
    \end{enumerate}

    \subsection{Дополнение линейно независимой системы векторов до базиса}

    \textbf{Утв} \textcolor{blue}{Систему л.н.з векторов можно дополнить до базиса}

    \begin{enumerate}
        \item Ранг подсистемы меньше ранга системы, поэтому выполняется обратное к основной теореме о рангах
        утверждение ($\exists~\overline{x}$), не лежащей в л.н.з подсистеме)
        \item Если мы добавим $\overline{x}$ к подсистеме и она станет зависимой, то в нетривиальной линейной комбинации равен
        нулю либо новый коэффициент ($w$ с л.н.з исходной подсистемы), либо какой-то из старых (тогда $\overline{x}$
        выражается через векторы линейной оболочки и принадлежит ей)
        \item Продолжая процесс и далее, дополним систему до базиса
    \end{enumerate}

    \subsection{Координаты вектора в базисе, запись операций над векторами через координаты}

    \textbf{Опр} \textit{Координаты вектора в базисе} \textcolor{gray}{Коэффициенты в разложении по базису}

    При сложении векторов и домножении на число, координаты изменяются покомпонентно

    \subsection{Изменение координат вектора при замене базиса (матрица перехода)}

    \textbf{Опр} \textit{Матрица перехода} \textcolor{gray}{Матрица координатных столбцов новых базисных векторов
    относительного старых}

    \textbf{Th} \textcolor{blue}{$S$ чья-то матрица перехода $\Leftrightarrow S$ невырождена}

    Это следует из того, что координатные столбцы $X_k$ векторов $a_k$ линейно независимы $\Leftrightarrow a_k$ л.н.з.
    ($\lambda^k a_k = 0 \Leftrightarrow \lambda^k X_k = 0$).
    Поэтому столбцы матрица невырождена (её столбцы л.н.з.) $\Leftrightarrow$ векторы л.н.з.

    \textbf{Th} \textcolor{blue}{Если вектор имеет в базисах $e, e^{'}$ координатные столбцы $X, X^{'}$, то $X = SX^{'}$}

    Достаточно расписать один вектор в обоих базисах и сравнить записи

    \textbf{Утв} \textcolor{blue}{Если $\exists e, e^{'}, e^{''}; e^{'} = eC, e^{''} = e^{'}D$, то $e^{''} = eCD$}

    Последовательно подставляем матрицы перехода

    \section{Сумма и пересечение подпространств.
    Линейно независимые подпространства, прямая сумма подпространств, её характеризации, прямое дополнение
    подпространства, проекция на подпространство вдоль прямого дополнения.
    Связь размерностей суммы и пересечения подпространств (формула Грассмана).
    Понятие факторпространства, его базис и размерность}

    \subsection{Сумма и пересечение подпространств}

\textbf{Опр} \textit{Пересечение подпространств} \textcolor{gray}{Множество элементов, которые являются их
обычным теоретико-множественным пересечением как подмножеств}

\textbf{Опр} \textit{Сумма подмножеств по Минковскому} \textcolor{gray}{Множество-сумма векторов всех векторов $
a_i$ из каждого $A_i \subset V$, то есть каждый вектор из суммы раскладывается по векторам из всех пространств}

Для суммы Минковского выполняется коммутативность и ассоциативность

\textbf{Утв} \textcolor{blue}{Сумма линейных оболочек есть линейная оболочка объединения}

Каждый элемент суммы есть линейная оболочка какого-то подпространства, поэтому если мы сложим все линейные
оболочки по Минковскому, то получим линейную оболочку объединения (совокупности)

\textbf{Следствие} \textcolor{blue}{Сумма конечного числа подпространств есть подпространство}

Потому как сумма есть минимальное по включению подпространство, содержащее в себе каждое из пространств

\textbf{Утв} \textcolor{blue}{Размерность суммы не превосходит суммы размерностей}

Размерность равна рангу, а для рангов ранг объединения не превосходит суммы рангов (доказывается от противного)

\subsection{Линейно независимые подпространства, прямая сумма подпространств, её характеризации, прямое дополнение
подпространства, проекция на подпространство вдоль прямого дополнения}

\textbf{Опр} \textit{Прямая (внешняя) сумма} \textcolor{gray}{Декартово произвдение $(a_1, \dots, a_n)$}

Сложение и умножение на скаляр определены для внешней суммы покомпонентно

\textbf{Опр} \textit{Прямая (внутренняя) сумма} \textcolor{gray}{Сумма, вектора $
a_i$ в разложении которой из каждого $A_i \subset V$ определены однозначно}

Например, всё пространство разлагается в прямую сумму своих базисных векторов

Через $\overline{U_i}$ обозначим сумму всех рассматриваемых пространств, за исключением $U_i: U_1 + \dots + U_{i-1} +
U_{i+1} + \dots + U_n$

\textbf{Th.1} \textit{Первый критерий прямой суммы}

\textcolor{blue}{Сумма подпространств прямая $\Leftrightarrow U_i \cap \overline{U_i}$}

\begin{itemize}
    \item $\Rightarrow:$ от противного.
    Пусть БОО условие не выполнено для $U_1$.
    Тогда там есть ненулевой вектор, который принадлежит $U_1$ и раскладывается по остальным пространствам.
    Тогда у нас существует два представления нулевого вектора (одно тривиальное, другое новое), $w$ с прямостью суммы
    \item $\Leftarrow:$ от противного.
    Пусть существует два различных разложения $v$ по $a_i$ и $b_i$.
    БОО хотя бы $a_1 \neq b_1$, поэтому если возьмём их разность, то с одной стороны она $\in U_1$, а с другой ---
    $\in \overline{U_i}$, то есть пересечение не тривиально, $w$
\end{itemize}

\textbf{Th.2} \textit{Второй критерий прямой суммы}

\textcolor{blue}{Для конечномерных подпространств следующие условия эквивалентны:
    \begin{enumerate}
        \item Сумма $U = \oplus U_i$ прямая
        \item Система из $\sum_i \dim U_i$ векторов из объединения базисов есть базис в $U$
        \item $\dim U = \sum_i \dim U_i$
    \end{enumerate} }

\begin{itemize}
    \item $2 \Leftrightarrow 3:$ по следствию основной теоремы о рангах $\dim U = rg e = rg \cup_i e^i$, поэтому
    $ \dim U = \sum_i \dim U_i \Leftrightarrow rg e = \sum_i \dim U_i \Leftrightarrow e $ л.н.з система
    \item $1 \Rightarrow 2:$ от противного.
    Пусть $e$ л.з. система.
    Запишем нетривиальную л.к. всей суммы.
    Так как хотя бы одна компонента нетривиальная, то БОО $l_1 \neq 0$.
    Тогда эта комбинация одновременно принадлежит и $U_1$, и $\overline{U_1}$, по той же идее, что и в Th.1, это $w$
    \item $2 \Rightarrow 1:$ от противного.
    Воспользуемся Th.1 получим ненулевое пересечение, распишем вектор оттуда (нетривиальная л.к.) и получим $w$ с л.н
    .з. $e$
\end{itemize}

\textbf{Опр} \textit{Прямое дополнение подпространства} \textcolor{gray}{Недостающий член в прямой сумме до всего
пространства}

В случае двух подпространств, они входят в определение симметрично

\textbf{Утв} \textcolor{blue}{Сумма размерностей подпространства и любоего его прямого дополнения равна
размерности всего пространства $V$}

\textbf{Утв} \textcolor{blue}{У любого подпространства конечномерного простраснтва $V$ существует прямое дополнение}

Для нахождения дополнения достаточно выбрать базис в подпространстве и дополнить его до базиса в пространстве.
Тогда по Th.2 эта система и будет прямым дополнением

Если $a = a_1 + a_2$, то $a_1$ называется проекцией $a$ на $U_1$ вдоль (параллельно) $U_2$

\subsection{Связь размерностей суммы и пересечения подпространств}

Смотреть в рукописном конспекте

\subsection{Понятие факторпространства, его базис и размерность}

\textbf{Опр} \textit{Факторпространство} \textcolor{gray}{Фактор-множество (множество всех классов
эквивалентности для заданного отношения эквивалентности) $a \sim b \Leftrightarrow b - a \in U$}

Элементы факторпространства есть смежные классы вида $a + U$
\begin{gather*}
    (a + U) + (b + U) = (a + b) + U\\
    \lambda (a + U) = \lambda a + U\\
\end{gather*}
Если $W$ - прямое дополнение $U$, то существует естественный изоморфизм $W \rightarrow V/U (a \mapsto a + U)$.
Он является ограничением на $W$ линейного отображения $\pi: V \rightarrow V/U, \pi (v) = v + U$ и называется
канонической проекцией.
Действительно, из определения прямого дополнения следует единственность $u \in U: v = u + w$.
Применим $\pi$ и получим $v + U = w + U$, что влечёт биективность $\pi$

Отсюда следует, что дополнение произвольного базиса в $U$ до базиса в $V$ после применения к ней $\pi$ будет
базисом в $V/U$, притом $\dim V/U = \dim V - \dim U$, что следует из теоремы о сумме размерностей ядра и образа


    \section{Понятие аффинного пространства, связь между аффинным и векторным пространством}

    \textbf{Опр} \textit{Афинное пространство} \textcolor{gray}{Отображение из точек-векторов в точки (откладывание
    от точки векторы)}

    Афинное пространство удовлетворяет трём аксиомам (ассоциативность, существование нуля и единственность \("\)
    дополнения\("\)). Также справедливо \("\)правило треугольника\("\)

    \textbf{Th} \textit{О замене координат}

    \textcolor{blue}{Если существует две ДСК и $S$ -- матрица перехода от старой к новой, $\gamma$
        -- координатный столбец начала координат новой системы в старой, то справедливо $X = SX^{'} + \gamma$}

    Достаточно рассмотреть произвольный вектор как сумму сдвигов в ноль и в точку, перейти к базису и координатным
    столбцам, вспомнить определение матрицы перехода и сократить базис

    \addcontentsline{toc}{section}{Линейные отображения} \part*{Линейные отображения}

    \section{Линейные отображения и линейные преобразования векторных пространств (линейные операторы).
    Операции над линейными отображениями, линейное пространство линейных отображений.
    Алгебра линейных операторов.
    Изоморфизмы}

    \subsection{Линейные отображения и линейные преобразования векторных пространств (линейные операторы)}

\textbf{Опр} \textit{Линейное отображение} \textcolor{gray}{Отображение, удовлетворяющее двум аксиомам}

Отсюда следуют конечная линейность, отображение нулевого и противоположного вектора

Множество всех линейных отображений обозначается как $L(V, W)$.
В случае $W = V$ линейное отображение называют линейным преобразованием (оператором)

\textbf{Опр} \textit{Линейная функция (функционал)} \textcolor{gray}{Случай $\dim W = 1 (W = \mathbb{K})$}

\textbf{Утв} \textcolor{blue}{Под действием линйного отображения л.з система остаётся л.з}

Достаточно записать нетривиальную линейную комбинацию и взять её образ, используя уже известные аксиомы

\textbf{Утв} \textcolor{blue}{Ранг системы под действием линейного отображения не возрастает}

Это следует из определения ранга и противного к предыдущему утверждению.
В силу равенства ранга и размерности в конечномерном случае, получаем аналогичное неравенство для размерностей

\textbf{Утв} \textit{Образ подпространства}

\textcolor{blue}{Образ линейной оболочки есть линеная оболочка образов}

Действительно, если записать определение линейной оболочки (множество всех линейных комбинаций) и подействовать
отображением, то получится требуемое.
В частном случае, если взять базис (его линейная оболочка есть всё пространство), то образ пространства есть
линейная оболочка линейная оболочка образов базисных векторов

\textbf{Опр} \textit{Линейное вложение} \textcolor{gray}{Инъективное линейное отображение}

\textbf{Утв} \textcolor{blue}{В случае линейного вложения л.н.з. система остаётся л.н.з.}

Действительно, если записать л.к. образов и \("\)вынести $\varphi$ за скобки\("\), то в силу инъективности
получим, л.к. исходных векторов.
В силу её линейной независимости, эта л.к. тривиальна, как и л.к. образов
В частном случае, если взять базис, то получим равенства рангов $U$ и $\varphi (U)$, как и размерностей

\textbf{Th} \textcolor{blue}{Если взять базис $e_i$ в $V$ и произвольные векторы $c_i$ в $W$, то
    $\exists ! \varphi: \varphi(e_i) = c_i$. Дополнительно, $\varphi$ инъективно $c_i$ л.н.з.}

\begin{enumerate}
    \item Для начала докажем единственность.
    Зафиксируем произвольный вектор $a$ пространства, разложим его по базису и рассмотрим $\varphi (a)$, имеющего
    единственные коэффициенты.
    В силу произвольности $a$ теорема справедлива
    \item Для доказательства существования, достаточно взять два произвольных вектора из пространства,
    подействовать на них отображением (с учётом $\varphi(e_i) = c_i$), затем проверить аксиомы линейного отображения
    \item $\Rightarrow$: следует из предыдущего утверждения
    \item $\Leftarrow$: от противного, с использованием определения инъективности, разложения $a - b$ по базису и
    $\varphi(e_i) = c_i$
\end{enumerate}

\subsection{Операции над линейными отображениями, линейное пространство линейных отображений}

\textbf{Опр} \textit{Сумма отображений} \textcolor{gray}{Такое отображение, что ...}

\textbf{Опр} \textit{Произведение отображения на скаляр} \textcolor{gray}{Такое отображение, что ...}

В комплексном случае скаляр заменяется на комплексно-сопряжённый.

Нетрудно проверить, что оба нововведённых отображения линейны.
Также проверкой доказывается ассоциативность, дистрибутивность и линейность в случае композиции отображений

\subsection{Алгебра линейных операторов}

Так как на множестве $L(V, V)$ определены операции сложения, умножения на скаляр и умножения, то $L(V, V)$ имеет
структуру ассоциативной алгебры (непустое множество (носитель) с заданным на нём набором операций и отношений (
сигнатурой)).
Ассоциативная потому как заданы операции ассоциативного умножения, то есть $\forall k, l \in \mathbb{F}$ и
$\forall a, b, c \in A$ справедливо

\begin{enumerate}
    \item $a(b + c) = ab + ac$
    \item $(a + b)c = ac + bc$
    \item $(k+l)a = ka + la$
    \item $k(a + b) = ka + kb$
    \item $k(la) = (kl)a$
    \item $k(ab) = (ka)b = a(kb)$
    \item $1a = a$, где 1 -- единица кольца $\mathbb{K}$
\end{enumerate}

\textbf{Опр} \textit{Аннулирующий многочлен для оператора} \textcolor{gray}{$P(\varphi) = 0$}

\textbf{Опр} \textit{Минимальный многочлен} \textcolor{gray}{Аннулирующий многочлен с минимальной степенью}

\textbf{Утв} \textcolor{blue}{Пусть $\mu$ -- минимальный многочлен оператора $\varphi$,а $P \in \mathbb{F}$ --
произвольный.
Тогда $P$ аннулирует $\varphi \Leftrightarrow f \vdots \mu$ в кольце многочленов над $\mathbb{F}$}

\begin{enumerate}
    \item Разделим $P$ на $\mu$ с остатком и подставим в полученное равенство $\varphi$
    \item Воспользуемся условием и получим $P(\varphi) = 0 \Leftrightarrow r(\varphi) = 0$
    \item В таком случае остаток должен быть аннулирующим для $\varphi$, то есть его степень меньше степени
    минимального многочлена, поэтому $w$ не возникает только в случае $r \equiv 0$
    \item Таким образом, эквивалентность доказана
\end{enumerate}

Отсюда следует, что минимальный многочлен единственен с точностью до умножения на константу

\subsection{Изоморфизмы}

\textbf{Опр} \textit{Изоморфизм} \textcolor{gray}{Линейное биективное отображение}

\textbf{Опр} \textit{Изоморфные векторные пространства} \textcolor{gray}{Между ними существует изоморфизм}

\textbf{Утв} \textcolor{blue}{Обратный к изоморфизму изоморфизм}

\begin{enumerate}
    \item Биективность следует из тождеств для обратных функций
    \item Далее берутся векторы из образа и на них проверяются аксиомы линейного отображения
    \item Итого, обратный к изоморфизму изоморфизм по определению
\end{enumerate}

\textbf{Th} \textit{Классификация конечномерных векторных пространств}

\textcolor{blue}{Пространства изоморфны $\Rightarrow$ их размерности совпадают}

\begin{enumerate}
    \item $\Rightarrow:$ из изоморфности следует инъективность, а для инъективных отображений равенство доказано ранее
    \item $\leftarrow:$ построим изоморфизм между элементами каждого пространствами и их координатными столбцами в
    них по фиксированному базису.
    Ранее было доказано, что такое разложение единственно.
    Достаточно обратить какое-то из отображений (по предыдущему утверждению оно тоже будет изоморфизмом).
    Итого, мы получили композицию изоморфизмов, то есть изоморфизм
\end{enumerate}

\textbf{Th} \textcolor{blue}{Если конечномерные пространства $U, V: \dim U = \dim V; e_i$ -- базис в $U$,
    $\varphi \in L(U,V)$. Тогда следующие условия эквивалентны:
    \begin{enumerate}
        \item $\varphi$ -- изоморфизм
        \item $\varphi$ инъективен
        \item $\varphi$ сюръективен
        \item $\varphi (e_i)$ есть базис в $V$
    \end{enumerate}         }

\begin{itemize}
    \item $1 \Rightarrow 2:$ по определению
    \item $2 \Rightarrow 3:$ из инъективности следует $\dim (\varphi(U)) = \dim U \rightarrow \varphi(V) \cong U
    \rightarrow \varphi$ сюръективно
    \item $3 \Rightarrow 4:$ это следует из свойства линейной оболочки образов базисных векторов, связи
    размерности и ранга, определения ранга и базиса
    \item $4 \Rightarrow 1:$ по критерию инъективности, в силу л.н.з. $\varphi (e_i), \varphi$ будет инъективно,
    а из свойства линейной оболочки следует сюръективность $\varphi$
\end{itemize}

    \section{Матрица линейного отображения.
    Координатная запись линейного отображения.
    Связь операций над матрицами и над линейными отображениями.
    Изменение матрицы линейного отображения (преобразования) при замене базиса}

    \subsection{Матрица линейного отображения}

    \textbf{Опр} \textit{Матрица линейного отображения} \textcolor{gray}{Как и в обычном случае, координатные столбцы
    векторов $\varphi(e_i)$}

    Таким образом, существует биекция между $L(V, W)$ и $Mat(m, n)$, как и изоморфизм (проверяется).
    Отсюда следует, что размерность линейных операторов есть $mn$

    \subsection{Координатная запись линейного отображения}

    \textbf{Th} \textcolor{blue}{Если $\varphi \leftrightarrow A; a = eX, \varphi(a) = fY$, то $y = Ax$}

    Для доказательства достаточно записать определение координатного столбца, применить к ней $\varphi$ и в силу
    коммутируемости, поменять местами строки и столбцы, чтобы увидеть запись матричного умножения, что доказывает
    равенство

    \textbf{Следствие} \textcolor{blue}{Если дано неизвестно в плане линейности отображение $\varphi$, такое, что под
    его действием $y = Ax$, то оно линейное}

    В силу наличия биекции между матрица и линейными отображениями, найдём такое $\phi$.
    В таком случае по Th., они будут иметь одинаковую координатную запись $y = Ax$, то есть равны и $\varphi$ линейно

    \textbf{Th} \textcolor{blue}{Если линейное преобразование $\varphi$ таково, что $\varphi(e_i) = e_i^{'}$, то
        $A: A \leftrightarrow \varphi$ есть матрица перехода между базисами}

    Следует из определений (матрица линейного преобразования будет удовлетворять определению матрицы перехода)

    \subsection{Связь операций над матрицами и над линейными отображениями}

    \textbf{Утв} \textcolor{blue}{Композиции линейных отображений соответствует произведение соответствующих матриц}

    Доказывается по определению (подстановкой)

    \textbf{Следствие} \textcolor{blue}{Обратному отображению соотвествует обратная матрица}

    Следует из предыдущего утверждения и того, что тождественному отображению соответствует единичная матрица

    \textbf{Следствие} \textcolor{blue}{$P(\varphi) \leftrightarrow P(A)$}

    \textbf{Следствие} \textcolor{blue}{$\varphi \leftrightarrow A$ задаёт изоморфизм алгебр линейных преобразований
    и квадратных матриц}

    То есть изоморфизм группы биективных линейных преобразований и группы невырожденных матриц

    \subsection{Изменение матрицы линейного отображения (преобразования) при замене базиса}

    \textbf{Th} \textcolor{blue}{Если $L(V, W); S: e^{'} = eS; y = Ax, y^{'} = Ax^{'}; f^{'} = fR$, то
        $A^{'} = R^{-1}AS$ (матрица линейного отображения в другом базисе)}

    Доказывается путём подстановок и комбинаций равенств

    В частном случае $L(V, V) A^{'} = S^{-1}AS$ \\

    \textbf{Следствие} \textcolor{blue}{Ранг матрицы линейного отображения не зависит от выбора базисов в
    пространствах}

    Потому что мы домножаем слева и справа на невырожденные матрицы

    \section{Ядро и образ, их описание в терминах матрицы линейного отображения.
    Критерий инъективности.
    Связь между размерностями ядра и образа}

    \subsection{Ядро и образ, их описание в терминах матрицы линейного отображения}

    \textbf{Опр} \textit{Образ линейного отображения} \textcolor{gray}{Множество всех векторов $V$ под
    действием $\varphi \in L(V, W)$}

    \textbf{Th} \textit{Координатное описание образа}

    \textcolor{blue}{Если $\varphi \in L(V, W)$, а $b \in W: b =
    fY$, то $b \in \Im \varphi \Leftrightarrow Y \in <a_{.1}, \dots, a_{.n}>$}

    Это следует из записи образа через линейную оболочку действия $\varphi$ на базисные векторы, определения матрицы
    линейного отображения и того факта, что $b = fY$ есть л.к. столбцов $Y$

    Отсюда также следует, что размерность образа равна рангу матрицы линейного отображения

    \textbf{Утв} \textcolor{blue}{В случае $\varphi \in L(V, W)$ прообразы образов в подмножестве $W$ являются
    подмножеством $V$}

    \textbf{Опр} \textit{Ядро линейного отображения} \textcolor{gray}{Множество всех векторов $V$, которые зануляются
    под действием $\varphi \in L(V, W)$}

    То есть ядро есть подмножества $V$.
    Также ядро можно охарактеризовать как полный прообраз нулевого пространства, поэтому если ядро пусто, то
    оператор невырожден

    \textbf{Th} \textit{Координатное описание ядра}

    \textcolor{blue}{Если $\varphi \in L(V, W)$, а $a \in V: a = eX$, то $a \in Ker \varphi \Leftrightarrow AX = 0$}

    В обе стороны по определению ядра

    Другими словами, в терминах координатных столбцов ядро задается как общее решение однородной СЛУ $Ax = 0$

    \textbf{Следствие} \textcolor{blue}{$\dim Ker \varphi = n - rg A$}

    \subsection{Критерий инъективности}

    \textbf{Th} \textit{Критерий инъективности}

    \textcolor{blue}{Если $\varphi \in L(V, W)$, инъективно
        $\Leftrightarrow Ker \varphi = 0$}

    $\rightarrow:$ пользуемся $\varphi(0) = 0$
    $\leftarrow:$ от противного с использованием определения инъективности

    \subsection{Связь между размерностями ядра и образа}

    \textbf{Th} \textcolor{blue}{В конечномерных пространствах $\dim Ker \varphi + \dim \Im \varphi = n$}

    Следует из $\dim \Im \varphi = rg A$ и $\dim Ker \varphi = n - rg A$

    \section{Аффинные преобразования, их свойства.
    Аффинная группа}

    \subsection{Аффинные преобразования, их свойства}

    \textbf{Опр} \textit{Аффинно-линейное преобразование}

    \textbf{Опр} \textit{Дифференциал отображения} \textcolor{gray}{Обозначение элемента $\varphi \in L(V, V)$}

    \textbf{Опр} \textit{Аффинное преобразование} \textcolor{gray}{Биективное преобразование}

    \textbf{Утв} \textcolor{blue}{Преобразование аффинно $\Leftrightarrow$ его дифференциал биективен}

    Для доказательства достаточно воспользоваться определением при одной фиксированной точке в нём

    \textbf{Утв} \textcolor{blue}{Композиция линейный и аффинных преобразований линейна и аффинна соответственно, а
    их дифференциал есть произведение дифференциалов}

    Следует из определения и того, что композиция биективных отображений биективна

    \textbf{Утв} \textcolor{blue}{Обратное к аффинному отображению отображение аффинно}

    Следует из определения

    \subsection{Аффинная группа}

    Аффинные преобразования образуют группу относительно композиции

    \textbf{Утв} \textcolor{blue}{$Y = AX + C$}

    Для доказательства достаточное взять в определении аффинно-линейного преобразования точку $M = 0$

    Отсюда следует, что любое аффинное преобразование задаётся параллельным переносом и поворотом вокруг неподвижной
    точки, то есть линейное преобразование однозначно задаётся точкой и двумя векторами

    \textbf{Th} \textcolor{blue}{Линейное преобразование $f$ аффинно $\Leftrightarrow$ переводит неколлинеарные
    точки в неколлинеарные}

    Построим ДСК на наших трёх точках, подействуем на них преобразование и получим новую ДСК. $f$ однозначно задано
    этой ДСК.
    Поэтому $f$ аффинно $\Leftrightarrow$ $f$ биективно $\Leftrightarrow$ неколлинеарная система (л.н.з)
    переходит в неколлинеарную (в частности, система три точки)

    \textbf{Th} \textit{Связь аффинного преобразования с заменой координат}

    \textcolor{blue}{При аффинном преобразовании координатный столбец вектора не меняется}

    Достаточно воспользоваться координатной запись вектора, а потом к концевым точкам применить аффинное преобразование

    \textbf{Th} \textcolor{blue}{При аффинном преобразовании
        \begin{enumerate}
            \item прямая переходит в прямую
            \item параллельные прямые переходят в параллельные
            \item отношения длин отрезков сохраняются
            \item центральная симметрия сохраняется
        \end{enumerate}           }

    \begin{enumerate}
        \item достаточно параметризовать прямую и применить определение к концевым точкам
        \item аналогичным образом в силу линейности (из определения) сохраняются отношения (длин отрезков и
        расстояния между прямыми)
        \item отношения длин отрезков сохраняются
        \item при центральной симметрии для любых двух симметричных точек центр есть середина соответствующего
        отрезка, а так как отношения сохраняются, то получаем сохранение определения
    \end{enumerate}

    \textbf{Th} \textit{Изменение площадей}

    \textcolor{blue}{При аффинном преобразовании, чей дифференциал имеет матрицу $A$, площадь фигуры умножается
    на $\abs{\det A}$}

    Покажем на примере параллелограмма.
    Достаточно расписать определение ориентированной площади, применить преобразование и взять модуль (настоящая
    площадь неотрицательна)

    \textbf{Th} \textcolor{blue}{При аффинном преобразовании порядок алгебраической кривой не меняется}

    Так как при аффинном преобразовании координаты не меняются, то не поменяется и многочлен, задающий кривую (
    скалярное произведение коэффициентов на переменные), как и его порядок

    \addcontentsline{toc}{section}{Структура линейного преобразования} \part*{Структура линейного преобразования}

    \section{Инвариантные подпространства.
    Ограничение оператора на инвариантное подпространство.
    Фактороператор}

    \subsection{Инвариантные подпространства}

    \textbf{Опр} \textit{Инвариантное подпространство} \textcolor{gray}{Образ лежит в нём же}

    \textbf{Утв} \textcolor{blue}{Сумма и пересечение инвариантных подпространств инвариантно}

    Доказывается поэлементной проверкой определения

    \textbf{Утв} \textcolor{blue}{В случае коммутирующих преобразований ядро и образ одного инвариантно относительно
    другого}

    Доказывается по определению

    \textbf{Следствие} \textcolor{blue}{Ядро и образ многочлена $f(\varphi)$инвариантны относительно $\varphi \in L(V, V)$}

    \textbf{Утв} \textcolor{blue}{$U$ инвариантнно относительно $\varphi \Leftrightarrow U$ инвариантнно
    относительно $\varphi - \lambda, \lambda \in \mathbb{F}$}

    Доказывается проверкой в одну сторону и путём взятия другой $\lambda$ в обратную

    Таким образом, в случае $\Im (\varphi - \lambda) \subset U \rightarrow U$ инвариантно $\varphi$

    \subsection{Ограничение оператора на инвариантное подпространство}

    \textbf{Утв} \textcolor{blue}{Если $U$ инвариантно относительно $\varphi$ -- изоморфизма, то $U$ инвариантно
    относительно $\varphi^{-1}$}

    Достаточно рассмотреть сужение $\varphi$ на $U$.
    В силу инъективности это будет изоморфизм.
    Тогда обращаем его и получаем требуемое

    \textbf{Утв} \textcolor{blue}{Если $U_k$ инвариантно относительно линейное оболочки первых $k$ векторов
        $\Leftrightarrow a_{ij} = 0, i \in \overline{k+1, n}, j \in \overline{1, k}$, то есть матрица имеет
        блочно-диагональный вид, где второй квадрант есть сужение $\varphi$ на $U_k$}

    Достаточно воспользоваться определением матрицы линейного преобразования и вспомнить, что у нас базис не меняется

    \textbf{Утв} \textcolor{blue}{Если $\varphi \in L(V, V)$, а $P(\varphi), \deg P = k$ вырожден, то существует не
    более чем $k$-мерное инвариантное подпространство $V$ относительно $\varphi$}

    \begin{enumerate}
        \item Возьмём произвольный элемент ядра $a$ и покажем, что $U = <a, \varphi(a), \dots, \varphi^{k-1}(a)>$
        инвариантно относительно $\varphi$
        \item В силу индуктивности $\varphi^j$, достаточно доказать лишь что $\varphi^k (a) \in U$
        \item Подставляем $\varphi(a)$ в многочлен и получаем, что $\varphi^k (a)$ линейно выражается через остальные
        члены, что доказывает инвариантность и утверждение
    \end{enumerate}

    \subsection{Фактороператор}

    \textbf{Опр} \textit{Фактороператор} \textcolor{gray}{Линейный оператор, определённый формулой $\overline{\uppsi
    } (v + U) = \varphi (v) + U, \forall v \in V$}

    \section{Собственные векторы и собственные значения.
    Собственные подпространства.
    Характеристический многочлен и его инвариантность.
    Определитель и след преобразования}

    \subsection{Собственные векторы и собственные значения}

\textbf{Опр} \textit{Собственное значение} \textcolor{gray}{Существует $a \in V:$}

\textbf{Опр} \textit{Собственный значение} \textcolor{gray}{Ненулевой вектор $a$ преобразования ...}

\textbf{Утв} \textcolor{blue}{Ненулевой вектор $a$ собственный для $\varphi \Leftrightarrow <a>$ инвариантна
относительно $\varphi$}

В силу эквивалентности инвариантности наличию собственного значения

\textbf{Утв} \textcolor{blue}{Ненулевой вектор $a$ собственный для $\varphi$ с собственным
значением $\lambda \Leftrightarrow a \in \ker (\varphi - \lambda)$}

Достаточно вспомнить определение ядра

\subsection{Собственные подпространства}

\textbf{Опр} \textit{Собственное подпространство} \textcolor{gray}{Ядро $\ker (\varphi - \lambda)$, содержащее ...}

\textbf{Утв} \textcolor{blue}{Сумма подпространств $V_{\lambda_i}$ прямая}

\begin{enumerate}
    \item От противного: возьмём $a_1 \in V_{\lambda_1} \cap \sum_i V_{\lambda_i}$, то есть $a_1 = \sum_i a_i$
    \item Применим к этому равенству преобразование $\sqcap_2^k (\varphi - \lambda_k)$
    \item Справа у нас получится ноль, а слева -- нет, $w$
\end{enumerate}

\subsection{Характеристический многочлен и его инвариантность}

\textbf{Опр} \textit{Характеристический многочлен} \textcolor{gray}{Функция от константы. Не забыть про обозначение}

\textbf{Опр} \textit{Характеристическое уравнение} \textcolor{gray}{Равенства многочлена нулю}

\textbf{Опр} \textit{Характеристические числа} \textcolor{gray}{Корни арактерестического многочлена}

Характеристический многочлен можно записать и с учётом алгебраической кратности его корней

\textbf{Утв} \textcolor{blue}{Характерестический многочлен имеет
вид $(-1)^n \lambda^n + (-1)^{n-1} tr A + \dots + \abs{A}$}

Достаточно знать, что определитель есть функция от всех элементов матрицы, затем просто расписать коэффициенты
перед требуемыми степенями

Отсюда, в соответствии с теоремой Виета, сумма всех характеристических чисел равна следу, а произведение есть $\det A$


Стоит учесть, что данное утверждения верно лишь в $\mathbb{C}$.
В $\mathbb{R}$ собственные значения есть только вещественные характеристические числа

\textbf{Th} \textcolor{blue}{
    \begin{enumerate}
        \item В $\mathbb{C}$ у $\varphi~\exists$ одномерное инвариантное подпространство
        \item В $\mathbb{R}$ у $\varphi~\exists$ одномерное инвариантное подпространство в случае нечётного $n$
        \item У $\varphi~\exists$ ненулевое инвариантное подпространство размерности не выше 2
    \end{enumerate}         }

\begin{enumerate}
    \item По основной теореме алгебры у любого многочлена есть по крайней мере один комплексный корень
    \item Из анализа известно, что в таком случае у многочлена есть по крайней мере один вещественный корень
    \item Если у многочлена есть вещественный корень, то у него есть и одномерное инвариантное подпространство.
    Иначе рассмотрим комплексный корень.
    Из анализа известно, что его сопряжённый тоже будет корнем характеристического многочлена.
    Тогда многочлен $P$ с этими коэффициентами будет вещественен, а $\det P(A) = 0$ в силу наличия соответствущих
    собственных значений.
    В таком случае ранее было доказано, что у $\varphi~\exists$ двумерное инвариантное подпространство
\end{enumerate}

\subsection{Определитель и след преобразования}

\textbf{Утв} \textcolor{blue}{Если матрица оператора верхнетреугольна, то характеристические числа
характеристического многочлена совпадают с диагональными элементами}

Верно в силу того, что определитель верхнетреугольной матрицы равен произведению диагональных элементов

\textbf{Th} \textit{Инвариантность характеристического многочлена}

\textcolor{blue}{Характеристический многочлен не зависит от выбора базиса}

Достаточно записать характеристическое уравнение в двух базисах, перейти от одного к другому с помощью матрицы
перехода и преобразовать выражение

\textbf{Следствие} \textcolor{blue}{Определитель, след, набор характеристических чисел матрицы оператора не
зависят от выбора базиса}

Все вышеперечисленные термины выражаются через коэффициенты характеристического многочлена

\end{document}
