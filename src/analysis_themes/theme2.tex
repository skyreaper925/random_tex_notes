\subsection{Частные производные высших порядков}

\textbf{Опр} \textit{Частная производная порядка $k$}
\textcolor{gray}{Частная производная от частной производной предыдущего порядка}

\textbf{Опр} \textit{Смешанная частная производная} \\

Смешанный производные могут зависеть от порядка дифференцирования

\subsection{Теорема о независимости смешанных производных от порядка дифференцирования}

\textbf{Th (Шварца)} \textit{О равенстве смешанных производных}
\textcolor{gray}{Равны в случае существования в окрестности и непрерывности в точке}

\begin{enumerate}
    \item Перейдём от круглой окрестности к квадратной путём взятия $\delta^{'} = \frac{\delta}{\sqrt{2}}$
    \item Записываем вторую разность функции $w(t), t \in \left(-\frac{\delta}{\sqrt{2}}, \frac{\delta}{\sqrt{2}}\right)$
    (прибавляем побочную диагональ и вычитаем главную)
    \item Записываем вторую разность через разность первых разностей $g(x) = f(x, y_0 + t) - f(x, y_0)$ и применяем к
    ней th.
    Лагранжа о среднем, записав её через $\theta_1 \in (0, 1)$
    \item В силу $g^{'}(x) = \frac{\partial f}{\partial x} (x, y_0 + t) - \frac{\partial f}{\partial x} (x, y_0)$
    перепишем $w(t)$
    \item Теперь применим th.
    Лагранжа о среднем для $\varphi (y) := \frac{\partial f}{\partial x} (x_0 + \theta_1 t, y)$ записав её через
    $\theta_2 \in (0, 1)$
    \item Итого, $w(t) = \frac{\partial^2 f}{\partial y \partial x} (x_0 + \theta_1 t, y_0 + \theta_2 t)t^2$
    \item Затем делим $w(t)$ на $t^2$ и пользуемся непрерывностью частных производных в окрестности точки
    \item Аналогичный процесс произведём с $f(y, x)$ и получим требуемое равенство
\end{enumerate}

По аналогии можно доказать равенство смешанных частных производных высших порядков

\subsection{Дифференциалы высших порядков}

\textbf{Опр} \textit{$k$ раз дифференцируемая функция}
\textcolor{gray}{Все частные производные порядка $(k - 1)$ функции определены в окрестности и дифференцируемы в точке}

\textbf{Опр} \textit{Дифференциал $k$ порядка}
\textcolor{gray}{Дифференциал от предыдущего порядка с подстановкой точки $x_0$}

\textbf{Лемма} \textit{О виде дифференциал $k$ порядка}
\textcolor{gray}{Дифференциал $k$ порядка есть сумма сумм начиная со всех внешних и заканчивая всеми внутренними
частными производными функции}

\begin{enumerate}
    \item Докажем для второго дифференциала, используя запись первого дифференциала в виде суммы частных производных
    \item В том же виде запишем второй дифференциал
    \item Получим его в виде суммы сумм начиная со внешних и заканчивая внутренними частными производными функции
    \item Для дифференциалов высших порядков доказываемое утверждение следует по индукции
\end{enumerate}

\subsection{Формула Тейлора для ФМП с остаточным членов в формах Лагранжа и Пеано}

\textbf{Th.1} \textit{Формула Тейлора с остаточным членом в форме Лагранжа}
\textcolor{gray}{$f(x_0) +$ сумма дифференциалов нормированных на факториал порядка +
следующий дифференициал в неизвестной точке отрезка}

\begin{enumerate}
    \item Зафиксируем $\forall x \in U_{\delta}(x_0)$ и рассмотрим функцию $\varphi(t) = f(x_0 + t\Delta x)$
    \item $\forall k \in \overline{1, m+1}$ продифференцируем её как \underline{сложную} функцию
    \item Применяем формулу Тейлора с остаточным членом в форме Лагранжа для функции одной переменной $\varphi(t)$
    и используем равенства $\varphi (1) = f(x), \varphi (0) = f(x_0)$
    \item Полученное выражение доказывает теорему
\end{enumerate}

\textbf{Опр} \textit{Многочлен Тейлора порядка $m$ функции $f$ в точке $x_0$}
\textcolor{gray}{$f(x_0) +$ сумма дифференциалов нормированных на факториал порядка}

\textbf{Th.2} \textit{Формула Тейлора с остаточным членом в форме Пеано}
\textcolor{gray}{Функция представима в виде суммы многочлена Тейлора и o-малого при \underline{непрерывных} в
окрестности частных производных}

\begin{enumerate}
    \item Применяем формулу Тейлора с остаточным членом в форме Лагранжа и раскладываем до $m-1$ порядка
    (на один меньше, чем требуется).
    Требуется доказать, что разность остаточного члена Лагранжа и дифференциала порядка $m$ есть остаток в форме Пеано
    \item Записываем эти дифференциалы через суммы сумм частных производных
    \item Модуль суммы не превосходит суммы модулей.
    Записываем это неравенство
    \item При $x \rightarrow x_0 $ в силу непрерывности каждого из дифференциалов получаем требуемую запись
\end{enumerate}

Аналогично первому семестру можно доказать единственность разложения в форме Пеано