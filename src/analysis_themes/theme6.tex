\subsection{Числовые ряды}

\textbf{Опр} \textit{Частичная сумма ряда} \textcolor{gray}{Сумма конечного числа членов ряда}

\textbf{Опр} \textit{Член ряда} \textcolor{gray}{Элемент последовательности суммирования}

\textbf{Опр} \textit{Сумма ряда} \textcolor{gray}{Предел частичных сумм}

\textbf{Опр} \textit{(Рас)ходящийся ряд} \textcolor{gray}{Предел частичных сумм (бес)конечен}

\textbf{Th} \textit{Необходимое условие сходимости ряда} \textcolor{gray}{Достаточно рассмотреть предел
разности двух соседних частичных сумм, что есть член ряда} \\

Отсюда следует, что ряд из геометрической прогрессии сходится при $\abs{q} < 1$ \\

\textbf{Л1} \textit{Принцип локализации} \textcolor{gray}{Достаточно представить сумму ряда в виде частичной
суммы и остатка, а затем взять предел обеих частей, вспомнив свойство предела последовательности}

\textbf{Л2} \textit{Свойство линейности} \textcolor{gray}{Достаточно вспомнить свойства пределов
последовательностей, связанные с арифметическими неравенствами}

\textbf{Следствие} \textit{Свойство линейности}

\textcolor{blue}{Ряд, чьи члены есть сумма членов сходящегося и расходящегося рядов расходится}

\subsection{Знакопостоянные ряды}

\textbf{Th} \textit{О существовании суммы ряда с неотрицательными членами}

\textcolor{blue}{Сумма ряда с неотрицательными членами является неотрицательным числом или $\infty$, притом ряд
сходится $\Leftrightarrow \sum_{\mathbb{N}}^{}a_k < \infty$} \\

В силу неотрицательности членов, последовательность частичных сумм нестрого возрастает.
Поэтому по теореме Вейерштрасса о монотонной последовательности, предел будет существовать.
Он равен сумме ряда по её определению

\subsection{Признаки сравнения сходимости числовых рядов}

\textbf{Th.1} \textit{Первый признак сравнения} \textcolor{gray}{Следует из предыдущей теоремы}

\textbf{Опр} \textit{Эквивалентность по сходимости} \textcolor{gray}{Члены первого ряда миномажорируются членами
другого ряда и домножением на положительные $m$ и $M$, начиная с какго-то $k_0 \in \mathbb{N}$}

\textbf{Th.2} \textit{Второй признак сравнения}

\textcolor{blue}{Эквивалентные по сходимости ряды сходятся или
расходятся одновременно} \\

Доказательство следует из первого признака сравнения и принципа локализации

\subsection{Интегральный признак сходимости числового ряда}

Смотреть в рукописном конспекте

\subsection{Признаки Даламбера и Коши}

\textbf{Th.1} \textit{Признак Даламбера} \textcolor{blue}{Если после какого-то номера отношение соседних членов
ряда $< 1$, то ряд сходится. В противном случае расходится}

\begin{enumerate}
    \item По индукции (или необязательно) легко показать, что $a_k \leq a_{k_0} q^{k - k_0}$
    \item Затем осталось последовательно воспользоваться преобразованиями, сходимостью геометрической прогрессии
    признаком сравнения и принципом локализации
    \item В случае $\frac{a_{k+1}}{a_k} > 1$ не выполняется необходимое условие сходимости ряда
\end{enumerate}

\textbf{Следствие} \textit{Признак Даламбера в предельной форме} \textcolor{blue}{Если предел отношения соседних
членов ряда есть $q < 1$, то ряд сходится. Если $q > 1$, то ряд расходится, а в случае $q = 1$ ряд может
сходиться, а может и расходиться}

\begin{enumerate}
    \item $q^{'} = \frac{q+1}{2} < \frac{1 + 1}{2} < 1$, поэтому по определению предела теорема сводится к
    предыдущей
    \item Аналогично пользуемся определением предела
    \item В случае $a_k = \frac{1}{k^{\alpha}} \lim_{k\to\infty} \frac{a_{k+1}}{a_k} = \left(\frac{k}{k+1}\right)^{\alpha} = 1~
    \forall \alpha \in \mathbb{R}$, однако данный ряд имеет разную сходимость в зависимости от $\alpha$
\end{enumerate}

\textbf{Th.2} \textit{Радикальный признак Коши}

\textcolor{blue}{Если после какого-то номера $\sqrt[k]{a_k} < 1$,
    то ряд сходится. В противном случае расходится}

\begin{enumerate}
    \item Достаточно возвести неравенство в квадрат и воспользоваться признаком сравнения и принципом локализации
    \item В случае $\sqrt[k]{a_k} > 1$ не выполняется необходимое условие сходимости ряда
\end{enumerate}

\textbf{Следствие} \textit{Признак Коши в предельной форме}

\textcolor{blue}{Если предел $q = \sqrt[k]{a_k} < 1$, то
ряд сходится. Если $q > 1$, то ряд расходится, а в случае $q = 1$ ряд может сходиться, а может и расходиться} \\

Доказательство аналогично доказательству признака Даламбера в предельной форме

\subsection{Знакопеременные ряды (Критерий Коши сходимости ряда)}

\textbf{Th} \textit{Критерий Коши сходимости ряда}

\textcolor{blue}{Ряд сходится тогда и только тогда, когда выполняется условие Коши} \\

Для доказательства достаточно вспомнить определение сходимости ряда и критерий Коши для последовательностей (сходимость $\Leftrightarrow$ фундаментальность)

\subsection{Сходимость и абсолютная сходимость}

\textbf{Опр} \textit{Абсолютно сходящийся ряд} \textcolor{gray}{Ряд из членов под модулем сходится}

\textbf{Опр} \textit{Условно сходящийся ряд} \textcolor{gray}{Ряд сходится, но не является абсолютно сходящимся}

\textbf{Th} \textit{Если ряд абсолютно сходится, то он сходится} \\

Для доказательства достаточно записать критерий Коши для ряда из членов под модулем, а затем воспользоваться
неравенством треугольника \\

\textbf{Th} \textit{Линейная комбинация абсолютно сходящихся рядов сходится} \\

Для доказательства достаточно последовательно воспользоваться свойством линейности, неравенством треугольника,
признаком сравнения и предыдущей теоремой

\subsection{Признаки Дирихле, Лейбница и Абеля}

Смотреть в рукописном конспекте

\subsection{Независимость суммы абсолютно сходящегося ряда от порядка слагаемых}

\textbf{Опр} \textit{Положительная и отрицательная составляющая члена} \textcolor{gray}{Полусумма и полуразность
модулей}

В зависимости от знака члена, он равен одной из своих составляющих, а другая при этом равна нулю.
Поэтому любой член ряда есть сумма его составляющих.
Когда речь идёт об абсолютно сходящимся ряде, то стоит взять модуль от обеих его частей (одна будет равна модулю
числа, другая нулю)

\textbf{Лемма} \textcolor{blue}{Сумма ряда из чисел одного знака не меняется при перестановке её элементов} \textcolor{gray}{В
силу переместительного закона сложения (I аксиома действительных чисел)}

\textbf{Th} \textcolor{blue}{Сумма абсолютно сходящегося ряда не зависит от перестановки слагаемых}

Для рядов с действительными элементами отдельно сумма положительных и сумма отрицательных элементов не
зависят от перестановок.
Поэтому сумма всех элементов, равная сумме обеих сумм, тоже не будет зависеть от перестановок

\subsection{Теорема Римана о перестановке членов условно сходящегося ряда}

\textbf{Лемма} \textcolor{blue}{Положительная и отрицательная суммы условно сходящегося ряда расходятся}

\begin{enumerate}
    \item Запишем ряд, как сумму его сумм в обычном и абсолютном виде:
    \begin{gather*}
        \sum_{\mathbb{N}}^{}a_k = \sum_{\mathbb{N}}^{}p_k + \sum_{\mathbb{N}}^{}n_k\\
        \sum_{\mathbb{N}}^{} \abs{a_k} = \sum_{\mathbb{N}}^{}p_k + \sum_{\mathbb{N}}^{}(-n_k)
    \end{gather*}
    \item Если хотя бы одна из полусумм сходится, то по первому равенству получим сходимость второй.
    Тогда по второму равенству сходится и ряд $\sum_{\mathbb{N}}^{} \abs{a_k}$, что противоречит условной сходимости ряда
\end{enumerate}

Из леммы следует, что положительная сумма стремиться к $+\infty$, а отрицательная -- к $-\infty$ \\

\textbf{Th} \textit{Римана о перестановке членов условно сходящегося ряда}

\textcolor{blue}{Если ряд сходится условно, то его члены можно переставить так, что сумма полученного ряда будет
равна любому наперёд заданному числу}

\begin{enumerate}
    \item Построим новый ряд, добавляя в него положительные члены, если его частичная сумма меньше заданной и
    отрицательные члены иначе
    \item Количество положительных и отрицательных членов стремится к бесконечности, потому как иначе, если хотя
    бы одно количество конечно, то по Л1 получаем расходимость исходного ряда.
    Поэтому любой член обоих полусумм будет присутствовать в новом ряде, поэтому такой ряд является перестановкой исходного ряда
    \item Новый ряд стремится к заданному числу, потому как из сходимости исходного ряда следует выполнение
    необходимого условия.
    Поэтому $\forall \varepsilon > 0~\exists N: \forall n > N~\abs{S^{'}_n - x} < \varepsilon$,
    что по определению свидетельствует о наличии требуемого предела
\end{enumerate}

\subsection{Перемножение абсолютно сходящихся рядов}

\textbf{Опр} \textit{Множество всевозможных пар натуральных чисел} \textcolor{gray}{Биекция
    $\mathbb{N} \rightarrow \mathbb{N}^2$}

\textbf{Th} \textit{О перемножении рядов}

\textcolor{blue}{Если два ряда сходятся абсолютно, то ряд, составленный из произведения соответвующих членов
сходится абсолютно, а его сумма равна произведению сумм}

\begin{enumerate}
    \item $\forall J \in \mathbb{N}$ определим $M_J = \max \{m_1, \dots, m_J\}$ и аналогично $N_J$.
    Тогда частичная сумма модулей членов нового ряда не превосходит произведения частичных сумм до $M_J$ и $N_J$
    рядов из модулей членов двух исходных рядов
    \item В силу абсолютной сходимости можно перейти к сравнению супремумов, что по теореме о существовании суммы
    ряда с неотрицательными членами сходится, то есть новый ряд сходится абсолютно
    \item Теперь найдём сумму нового ряда.
    В силу предыдущей теоремы мы имеет права воспользоваться другой перестановкой.
    Выберем перестановку методом вложенных квадратов.
    Она хороша тем, что за $N^2$ шагов мы считаем сумму первых $N^2$ членов, что задаёт ВОО.
    Частичная сумма нового ряда равна произведению старых частичных сумм
    \item Так как $S_{N^2}$ есть подпоследовательность последовательности $S_{n^2}$ с пределом $S$, то она
    стремится к такому же пределу, то есть требуемому
\end{enumerate}