\documentclass[a4paper, 14pt]{article}

%\hypersetup
%{   colorlinks,
%    pdftitle={2.1.6. Journal},
%    pdfauthor={Володин Максим},
%    allcolors=[RGB]{010 090 200}
%}

\usepackage[T2A]{fontenc}
\usepackage[utf8]{inputenc}
\usepackage[english, russian]{babel}
\usepackage[top = 2cm, bottom = 2cm, left = 2cm, right = 2cm]{geometry}
\usepackage{indentfirst}
\usepackage{xcolor}
\usepackage{hyperref}
\usepackage{graphicx}
\usepackage{gensymb}
\usepackage{pgfplots}
\usepackage{amsmath, amsfonts, amssymb, amsthm, mathtools}
\usepackage{physics, multirow, float}
\usepackage{wrapfig, tabularx}
\usepackage{icomma} % Clever comma: 0,2 - number while 0, 2 - two numbers
\usepackage{tikz, standalone}
\usepackage{fancyhdr,fancybox}
\usepackage{lastpage}
\usepackage{booktabs}
\usepackage{listings}
\usepackage{lstmisc}

\graphicspath{{images/}}
\DeclareGraphicsExtensions{.pdf,.png,.jpg}

\restylefloat{table}
\usetikzlibrary{external}

\mathtoolsset{showonlyrefs = true} % Numbers will appear only where \eqref{} in the text LINKED
\pagestyle{fancy}

\fancyhf{}
\fancyhead[R]{Семестровые задания}
\fancyfoot[R]{\thepage /\pageref{LastPage}}
%\fancyhead[L]{2.5.1}

\pgfplotsset{compat=1.18}

\begin{document}

    \tableofcontents

    \addcontentsline{toc}{part}{Линейная алгебра} \part*{Линейная алгебра}

    \addcontentsline{toc}{section}{Первое задание} \section*{Первое задание}

    \addcontentsline{toc}{subsection}{Плотность матричного множества} \subsection*{Плотность матричного множества}

Для начала, давайте определим, что значит \("\)множество диагонализируемых квадратных матриц плотно над полем
комплексных чисел\("\).
Это означает, что любая квадратная матрица $A$ размера $n \times n$ с элементами из поля комплексных чисел может
быть приближена произвольно близко другой диагональной матрицей $D$ размера $n \times n$ с элементами из того же
поля, то есть существует последовательность диагонализируемых матриц $A_k$ таких, что $\lim_{k \to \infty} A_k = D$.

Для доказательства этого факта можно воспользоваться теоремой о жордановой нормальной форме.
Эта теорема утверждает, что любая квадратная матрица $A$ размера $n \times n$ с элементами из поля комплексных
чисел подобна матрице вида $J = \mathrm{diag}(J_1, J_2, \ldots, J_k)$, где каждый блок $J_i$ имеет вид
\[
    J_i = \begin{pmatrix}
              \lambda_i & 1         & 0         & \cdots & 0         \\
              0         & \lambda_i & 1         & \cdots & 0         \\
              0         & 0         & \lambda_i & \cdots & 0         \\
              \vdots    & \vdots    & \vdots    & \ddots & \vdots    \\
              0         & 0         & 0         & \cdots & \lambda_i
    \end{pmatrix},
\]
где $\lambda_i$ -- собственное значение матрицы $A$, а размерность блока $J_i$ равна количеству жордановых клеток,
соответствующих этому собственному значению.

Теперь, если мы хотим приблизить матрицу $A$ диагональной матрицей $D$, мы можем заменить каждый блок $J_i$
матрицы $J$ на диагональную матрицу $\mathrm{diag}(\lambda_i, \lambda_i, \ldots, \lambda_i)$, что даст нам диагональную матрицу $D'$.
Однако, если мы теперь рассмотрим матрицу $A'$, которая получается из матрицы $A$ заменой блоков $J_i$ на
соответствующие диагональные матрицы, то матрицы $A'$ и $D'$ не обязательно будут подобны, и мы не можем
гарантировать, что последовательность $A_k$ сходится к $D$.

Однако, мы можем заменить каждый блок $J_i$ матрицы $J$ на матрицу вида
\[
    J_i' = \begin{pmatrix}
               \lambda_i & \epsilon_i & 0          & \cdots & 0         \\
               0         & \lambda_i  & \epsilon_i & \cdots & 0         \\
               0         & 0          & \lambda_i  & \cdots & 0         \\
               \vdots    & \vdots     & \vdots     & \ddots & \vdots    \\
               0         & 0          & 0          & \cdots & \lambda_i
    \end{pmatrix},
\]
где $\epsilon_i$ -- произвольно малое комплексное число, и затем рассмотреть матрицу $A''$, которая получается из
матрицы $A$ заменой блоков $J_i$ на соответствующие матрицы $J_i'$.
В этом случае матрицы $A''$ и $D'$ будут подобны, и мы можем гарантировать, что последовательность $A_k$ сходится
к $D$ при $\epsilon_i \to 0$ для всех $i = 1, \ldots, k$.

Таким образом, мы доказали, что множество диагонализируемых квадратных матриц плотно над полем комплексных чисел


\addcontentsline{toc}{subsection}{Многочлены} \subsection*{Многочлены}
\textit{Докажите, что характеристический многочлен линейного преобразования делится на характеристический
многочлен его ограничения на инвариантном подпространстве}

Это утверждение называется \("\)основной теоремой о блочном виде матрицы линейного оператора\("\).
Пусть $V$ -- векторное пространство над полем $F$, $L: V \rightarrow V$ -- линейный оператор, $U$ -- инвариантное
подпространство пространства $V$, т.е. $L(U) \subseteq U$.
Обозначим через $L_U$ ограничение оператора $L$ на подпространство $U$.

Тогда существуют такие базисы $e_1, \ldots, e_k$ в $U$ и $f_1, \ldots, f_m$ в $V$, что матрица оператора $L$ в
базисе $e_1, \ldots, e_k, f_1, \ldots, f_m$ имеет блочно-диагональную форму:
\[
    \begin{pmatrix}
        A_1 & 0   \\
        0   & A_2
    \end{pmatrix},
\]
где $A_1$ -- матрица оператора $L_U$ в базисе $e_1, \ldots, e_k$, а $A_2$ -- матрица оператора $L_{U^\perp}$ в
базисе $f_1, \ldots, f_m$.

Теперь заметим, что характеристический многочлен матрицы оператора $L$ равен произведению характеристических
многочленов матриц $A_1$ и $A_2$, т.е.
\[ \chi_L(t) = \chi_{L_U}(t) \cdot \chi_{L_{U^\perp}}(t). \]
Это следует из того, что характеристический многочлен матрицы блочно-диагональной формы равен произведению
характеристических многочленов диагональных блоков.

Таким образом, мы доказали, что характеристический многочлен линейного оператора $L$ делится на
характеристический многочлен его ограничения $L_U$ на инвариантном подпространстве $U$.

\addcontentsline{toc}{subsection}{Поворот на угол} \subsection*{Поворот на угол}

\textit{Найти подпространства трёхмерного геометрического пространства, инвариантные относительно поворота на
ненулевой угол $\alpha$ вокруг прямой $x = ta$}

Пусть $V$ -- трёхмерное геометрическое пространство, $L$ -- прямая в $V$ с направляющим вектором $a$.
Тогда оператор поворота $R_{\alpha}$ вокруг прямой $L$ на угол $\alpha$ можно выразить через матрицу поворота
вокруг оси $z$ в базисе, связанном с прямой $L$.
Для этого нужно выбрать ортонормированный базис $e_1, e_2, e_3$, где $e_3 = a$, и заменить матрицу поворота в
базисе $e_1, e_2, e_3$ на матрицу поворота в базисе $e_1', e_2', e_3$, где $e_1' = e_1$, $e_2' = e_2$,
$e_3' = R_{\alpha}e_3$.
Матрица перехода между этими базисами имеет вид:

\[
    S = \begin{pmatrix}
            \cos\alpha & -\sin\alpha & 0 \\
            \sin\alpha & \cos\alpha  & 0 \\
            0          & 0           & 1
    \end{pmatrix}.
\]

Теперь пусть $U$ -- подпространство $V$, инвариантное относительно поворота $R_{\alpha}$.
Тогда любой вектор $v \in U$ может быть представлен в виде $v = x + ty$, где $x \in L$, $y \in L^{\perp}$,
$t \in \mathbb{R}$.
Заметим, что $R_{\alpha}x = x$, так как $x$ лежит в прямой $L$, инвариантной относительно поворота.
Кроме того, $R_{\alpha}y \in L^{\perp}$, так как прямая $L^{\perp}$ ортогональна к направлению поворота.
Следовательно, $R_{\alpha}v = R_{\alpha}(x + ty) = x + tR_{\alpha}y \in U$.
Таким образом, $U$ содержит все векторы вида $x + ty$, где $x \in L$ и $y \in L^{\perp}$, и является инвариантным
относительно поворота $R_{\alpha}$.

Таким образом, мы получили, что любое подпространство $U$ в $V$, инвариантное относительно поворота на ненулевой
угол $\alpha$ вокруг прямой $L$, является прямой $L$ вместе с её ортогональным дополнением $L^{\perp}$.

\addcontentsline{toc}{subsection}{Подпространства} \subsection*{Подпространства}

Пусть дано линейное преобразование $T: V \rightarrow V$ над линейным пространством $V$ размерности $n$ с $n$
попарно различными собственными значениями $\lambda_1, \lambda_2, \dots, \lambda_n$.
Для любого собственного значения $\lambda_i$ рассмотрим его собственное подпространство $V_i = \{v \in V: T(v) = \lambda_i v\}$.
Так как собственные значения попарно различны, то собственные подпространства линейно независимы.
Кроме того, сумма всех собственных подпространств равна всему пространству $V$, так как каждый вектор $v \in V$
может быть разложен в сумму векторов из собственных подпространств, соответствующих попарно различным собственным
значениям.
Следовательно, мы получили разложение пространства $V$ в прямую сумму собственных подпространств:

\[V = V_1 \oplus V_2 \oplus \dots \oplus V_n.\]

Каждое из подпространств $V_i$ инвариантно относительно $T$, так как для любого вектора $v \in V_i$ выполнено $T(
v) = \lambda_i v \in V_i$.
Количество инвариантных подпространств, как мы видим, равно числу всех возможных комбинаций прямых сумм
собственных подпространств, то есть $2^n - 1$, так как каждое из $n$ собственных подпространств может входить или
не входить в данную прямую сумму.


\addcontentsline{toc}{subsection}{Инварианты} \subsection*{Инварианты}

Найти все инвариантные подпространства оператора, матрица которого в некотором базисе равна жордановой клетке

Для того чтобы найти все инвариантные подпространства оператора, матрица которого в некотором базисе равна
жордановой клетке, нужно сначала найти жорданов базис для этой матрицы.

Жорданов базис - это базис, в котором матрица оператора имеет жорданову форму.
Жорданова форма матрицы является блочно-диагональной матрицей, где каждый блок - это жорданова клетка.
Каждая жорданова клетка соответствует одному из собственных значений оператора и содержит на главной диагонали
это собственное значение, а также единицы на диагонали над главной диагональю.

Как только вы найдете жорданов базис для матрицы оператора, вы можете определить все инвариантные подпространства, их
размерности, а также базисы для каждого подпространства.

Для примера, давайте рассмотрим жорданову клетку размера $3 \times 3$ с собственным значением $\lambda$.
Его жорданов базис состоит из трех векторов: $v_1$, $v_2$ и $v_3$.
Жорданова клетка для этого собственного значения имеет следующий вид:

\[
    \begin{pmatrix}
        \lambda & 1       & 0       \\
        0       & \lambda & 1       \\
        0       & 0       & \lambda \\
    \end{pmatrix}
\]

Для определения инвариантных подпространств можно рассмотреть подпространства, порожденные комбинациями этих
векторов.
Возможны следующие случаи:
\begin{enumerate}
    \item Подпространство, порожденное одним вектором $v_i$ (размерность 1).
    Это подпространство инвариантно относительно оператора, так как $Av_i = \lambda v_i + w$, где $w$ - линейная
    комбинация $v_1$, $v_2$ и $v_3$ с коэффициентами, не равными нулю.
    Таким образом, $Av_i$ принадлежит тому же подпространству, что и $v_i$.
    \item Подпространство, порожденное двумя векторами $v_i$ и $v_j$ (размерность 2).
    Это подпространство инвариантно
    относительно оператора, если $v_i$ и $v_j$ являются собственными векторами для одного и того же собственного
    значения, то есть $\lambda_i = \lambda_j$.
    В этом случае $Av_i = \lambda_i v_i + w_1$ и $Av_j = \lambda_j v_j + w_2$, где $w_1$ и $w_2$ - линейные
    комбинации $v_1$, $v_2$ и $v_3$ с коэффициентами, не равными нулю.
    Тогда $A(v_i + v_j) = (\lambda_i + \lambda_j)(v_i + v_j) + (w_1 + w_2)$, что означает, что $v_i + v_j$ также
    является собственным вектором для собственного значения $\lambda_i + \lambda_j$.
    \item Подпространство, порожденное всеми тремя векторами $v_1$, $v_2$ и $v_3$ (размерность 3).
    Это подпространство
    инвариантно относительно оператора, так как $A(v_1 + v_2 + v_3) = \lambda_1 v_1 + \lambda_2 v_2 + \lambda_3 v_3 +
    w$, где $w$ -- линейная комбинация $v_1$, $v_2$ и $v_3$ с коэффициентами, не равными нулю.
    Таким образом, $v_1 + v_2 + v_3$ также является собственным вектором для собственного значения $\lambda_1 + \lambda_2 + \lambda_3$.
\end{enumerate}

Таким образом, мы нашли все три инвариантных подпространства для данной жордановой клетки.
Аналогично можно определить инвариантные подпространства для матрицы оператора, которая имеет жорданову форму в
некотором базисе.

\addcontentsline{toc}{subsection}{Перестановки} \subsection*{Перестановки}

Для того, чтобы доказать, что два перестановочных линейных преобразования комплексного пространства имеют общий
собственный вектор, мы можем воспользоваться следующим фактом:

Если два линейных преобразования перестановочные, то они коммутируют с любым многочленом от них.
В частности, они коммутируют с минимальным многочленом каждого из них.

Допустим, что у нас есть два перестановочных линейных преобразования A и B комплексного пространства V. Пусть λ -
собственное значение A, и v - соответствующий ему собственный вектор, то есть Av = λv.
Тогда, поскольку A и B перестановочные, мы можем записать:

ABv = BA v = B(λv) = λBv

Таким образом, Bv также является собственным вектором A, соответствующим собственному значению λ.
Если λ не является собственным значением B, то мы можем повторить этот процесс, используя минимальный многочлен B
вместо A. Таким образом, мы получим общий собственный вектор для $A$ и $B$.

Таким образом, мы доказали, что два перестановочных линейных преобразования комплексного пространства имеют общий
собственный вектор.

\addcontentsline{toc}{subsection}{Теорема Гамильтона Кэли}
\subsection*{Теорема Гамильтона Кэли}

Теорема Гамильтона-Кэли утверждает, что любая матрица $A$ удовлетворяет своему характеристическому уравнению:

\[ \det(\lambda I - A) = 0, \]

где $I$ - единичная матрица, а $\lambda$ - собственное значение матрицы $A$.

Для доказательства этой теоремы воспользуемся фактом, что множество диагонализируемых матриц плотно над полем
комплексных чисел.
Это означает, что любая матрица $A$ может быть приближена диагональной матрицей $D$ с любой
точностью.
То есть существует последовательность диагонализируемых матриц $A_n$, которые сходятся к матрице $A$
при $n \rightarrow \infty$.
При этом, каждая матрица $A_n$ имеет собственные значения $\lambda_{1,n}, \lambda_{2,n}, \dots, \lambda_{m,n}$ и
соответствующие им собственные векторы $v_{1,n}, v_{2,n}, \dots, v_{m,n}$.

Рассмотрим характеристическое уравнение для матрицы $A_n$:

\[ \det(\lambda I - A_n) = (\lambda - \lambda_{1,n})(\lambda - \lambda_{2,n}) \dots (\lambda - \lambda_{m,n}). \]

Так как матрица $A_n$ диагонализируема, то существует невырожденная матрица $P_n$, такая что:

\[ A_n = P_n D_n P_n^{-1}, \]

где $D_n$ -- диагональная матрица, элементы на диагонали которой равны собственным значениям матрицы $A_n$.
Подставим это выражение в характеристическое уравнение:

\[ \det(\lambda I - A_n) = \det(\lambda I - P_n D_n P_n^{-1}) = \det(P_n(\lambda I - D_n)P_n^{-1}) =
\det(\lambda I - D_n). \]

Таким образом, характеристическое уравнение для матрицы $A_n$ сводится к уравнению для диагональной матрицы $D_n$.
Поскольку последовательность матриц $A_n$ сходится к матрице $A$, то последовательность диагональных матриц $D_n$
также сходится к диагональной матрице $D$, элементы на диагонали которой равны собственным значениям матрицы $A$.
Таким образом, характеристическое уравнение для матрицы $A$ равно:

\[ \det(\lambda I - A) = \lim_{n \rightarrow \infty} \det(\lambda I - A_n) = \lim_{n \rightarrow \infty} (\lambda -
\lambda_{1,n})(\lambda - \lambda_{2,n}) \dots (\lambda - \lambda_{m,n}). \]

Так как каждая матрица $A_n$ имеет собственные значения, то и матрица $A$ также имеет собственные значения.
Следовательно, характеристическое уравнение для матрицы $A$ имеет вид:

\[ \det(\lambda I - A) = (\lambda - \lambda_1)(\lambda - \lambda_2) \dots (\lambda - \lambda_m), \]

где $\lambda_1, \lambda_2, \dots, \lambda_m$ -- собственные значения матрицы $A$.
Следовательно, теорема Гамильтона-Кэли доказана.


\subsection*{Оператор трёхкратного дифференцирования} \addcontentsline{toc}{subsection}{Оператор трёхкратного
дифференцирования}

Для нахождения жордановой нормальной формы (ЖНФ) оператора трехкратного дифференцирования в пространстве
вещественных многочленов степени не выше 9, нужно выполнить следующие шаги:

\begin{enumerate}

    \item Найти все собственные значения оператора.
    Для этого решим характеристическое уравнение:

    \[
        \det(\lambda I - A) = \det \begin{pmatrix}
                                       \lambda & 0       & 0       & 0       & 0       & 0       & 0       & 0       & 0       \\
                                       0       & \lambda & 0       & 0       & 0       & 0       & 0       & 0       & 0       \\
                                       0       & 0       & \lambda & 0       & 0       & 0       & 0       & 0       & 0       \\
                                       0       & 0       & 0       & \lambda & 0       & 0       & 0       & 0       & 0       \\
                                       0       & 0       & 0       & 0       & \lambda & 0       & 0       & 0       & 0       \\
                                       0       & 0       & 0       & 0       & 0       & \lambda & 0       & 0       & 0       \\
                                       0       & 0       & 0       & 0       & 0       & 0       & \lambda & 0       & 0       \\
                                       0       & 0       & 0       & 0       & 0       & 0       & 0       & \lambda & 0       \\
                                       0       & 0       & 0       & 0       & 0       & 0       & 0       & 0       & \lambda \\
        \end{pmatrix} = \lambda^9.
    \]

    Отсюда видно, что у оператора есть только одно собственное значение - ноль.
    \item Найти размерности жордановых клеток для каждого собственного значения.
    Так как у нас только одно
    собственное значение, то достаточно найти размерность жордановых клеток для нулевого собственного значения.
    Для этого нужно найти ядро оператора $(A - \lambda I)^k$, где $\lambda = 0$ и $k$ - порядок клетки.
    Начнем с $k=1$:

    \[ (A - \lambda I)^1 = A - \mathbf{0} = A. \]

    Чтобы найти ядро оператора $A$, нужно решить систему уравнений:

    \[ Ax = \mathbf{0}. \]

    Пусть $x = a_9 x^9 + a_8 x^8 + \dots + a_1 x + a_0$, тогда

    \[ Ax = a_9 \frac{d^3}{dx^3} x^9 + a_8 \frac{d^3}{dx^3} x^8 + \dots + a_1 \frac{d^3}{dx^3} x + a_0 \frac{d^3}{dx^3}
    (1) = a_9 \cdot 84\cdot70\cdot56\cdot x^6 + \dots + a_1 \cdot6\cdot2\cdot x = \mathbf{0}. \]

    Отсюда следует, что $a_9 = a_8 = \dots = a_1 = a_0 = 0$, так как каждый множитель в выражении для $Ax$ содержит
    производную третьего порядка, а значит не может быть равен нулю для всех $x$.
    Следовательно, размерность жордановой клетки для собственного значения ноль равна единице
    \item Найти жорданов базис.
    Жорданов базис строится на основе жордановых клеток.
    Для каждой жордановой клетки
    размерности $m$ строится матрица $J_m$, которая имеет вид:

    \[
        J_m = \begin{pmatrix}
                  \lambda & 1      & \cdots & \cdots & \cdots & \cdots & \cdots & \cdots & \cdots  \\
                  \cdots  & \cdots & \cdots & \cdots & \cdots & \cdots & \cdots & \cdots & \cdots  \\
                  \cdots  & \cdots & \cdots & \cdots & \cdots & \cdots & \cdots & \cdots & \cdots  \\
                  \cdots  & \cdots & \cdots & \cdots & \cdots & \cdots & \cdots & \cdots & 1       \\
                  \cdots  & \cdots & \cdots & \cdots & \cdots & \cdots & \cdots & \cdots & \lambda \\
        \end{pmatrix},
    \]

    где $\lambda$ - собственное значение.
    Жорданов базис составляется из жордановых клеток путем объединения столбцов матрицы $J_m$.
    В нашем случае жорданов базис будет состоять из единственного многочлена $x^8$.
    \item Найти минимальный многочлен оператора.
    Минимальный многочлен оператора - это многочлен наименьшей степени, который обнуляет оператор.
    Так как у нашего оператора только одно собственное значение - ноль, то минимальный многочлен должен быть
    делителем многочлена $\lambda^9$.
    Также известно, что минимальный многочлен должен иметь такие же неприводимые множители, как и
    характеристический многочлен.
    Следовательно, минимальный многочлен оператора равен $\lambda^k$, где $k$ -- порядок наибольшей жордановой
    клетки.
    В нашем случае $k=1$, поэтому минимальный многочлен равен $\lambda$.
\end{enumerate}

    \section*{Второе задание}

    \addcontentsline{toc}{section}{Второе задание}

\addcontentsline{toc}{subsection}{Положительная билейная форма} \subsection*{Положительная билинейная форма}

Для начала, давайте вспомним определение положительно определенной квадратичной формы.
Квадратичная форма $Q$ над полем вещественных чисел $\mathbb{R}$ называется положительно определенной, если для
любого ненулевого вектора $x$ из $\mathbb{R}^n$ значение $Q(x)$ положительно, то есть $Q(x) > 0$.

Теперь рассмотрим матрицу $A$, соответствующую данной квадратичной форме.
Пусть $a_{ij}$ -- элементы этой матрицы.
Тогда квадратичная форма $Q(x)$ может быть записана в виде $Q(x) = x^T A x$.
Заметим, что матрица $A$ симметрична, так как $a_{ij} = a_{ji}$.

Предположим, что максимальный по модулю элемент матрицы $A$ отрицательный, то есть $|a_{ij}| > |a_{kl}|$ для всех $i,
j,k,l$.
Тогда рассмотрим вектор $x$, у которого все компоненты равны нулю, кроме $i$-ой и $j$-ой, которые равны
соответственно $1$ и $-1$.
Тогда $x^T A x = a_{ij} - a_{ji} < 0$, так как $a_{ij}$ отрицательный.
Но это противоречит тому, что квадратичная форма $Q(x)$ положительно определена, так как мы нашли вектор $x$, для
которого $Q(x) < 0$.

Следовательно, максимальный по модулю элемент матрицы $A$ положителен

\addcontentsline{toc}{subsection}{След квадрата матрицы} \subsection*{След квадрата матрицы}

Для начала, давайте запишем квадратичную форму, соответствующую следу квадрата матрицы порядка $n$.
Пусть $A$ - матрица порядка $n$, тогда $Q(A) = \operatorname{tr}(A^2)$.

Заметим, что матрица $A^2$ является симметрической, так как $(A^2)^T = A^T A^T = A A = A^2$.
Следовательно, квадратичная форма $Q(A)$ является квадратичной формой на симметрических матрицах порядка $n$.

Теперь давайте найдем ранг этой квадратичной формы.
Для этого нам нужно найти количество независимых переменных, от которых зависит квадратичная форма.
Поскольку квадратичная форма зависит только от матрицы $A$, у которой $n^2$ элементов, то ранг квадратичной формы
равен $n^2$.

Чтобы найти сигнатуру квадратичной формы, нужно найти количество положительных, отрицательных и нулевых собственных
значений матрицы $A^2$.
Заметим, что собственные значения матрицы $A^2$ всегда неотрицательны, так как $\operatorname{tr}(A^2)$ является
суммой квадратов собственных значений матрицы $A$.
Следовательно, у нас нет отрицательных собственных значений, и сигнатура квадратичной формы равна $(n^2, 0)$.

\addcontentsline{toc}{subsection}{Угловые миноры} \subsection*{Угловые миноры}

Для данной квадратичной формы на трехмерном вещественном пространстве, у которой угловые миноры равны $0$, $0$ и
$\alpha > 0$, положительный индекс инерции может быть равен $1$, а отрицательный индекс инерции равен $0$.

Индексы инерции квадратичной формы определяются количеством положительных и отрицательных собственных значений
матрицы квадратичной формы.
Так как у данной квадратичной формы имеется только одно ненулевое собственное значение, а его знак положителен, то
положительный индекс инерции равен 1, а отрицательный индекс инерции равен 0.

Таким образом, индексы инерции для данной квадратичной формы имеют вид $(1,0)$.

\addcontentsline{toc}{subsection}{Положительная определённость} \subsection*{Положительная определённость}

Рассмотрим пример, который покажет, что из положительной определенности двух ограничений пространства, являющейся
прямой сумма двух своих подпространств, не следует положительная определённость квадратичной формы.

Пусть $V = \mathbb{R}^2$ и $V_1 = \operatorname{span}\{(1,0)\}$, $V_2 = \operatorname{span}\{(0,1)\}$ - подпространства в $V$.
Рассмотрим два ограничения пространства $V$:

\[ f_1(x,y) = x, \quad f_2(x,y) = y \]

Проверим, что оба ограничения положительно определены:

\begin{gather*}
    a(x,y) = x > 0, \quad \forall (x,y) \in V_1, \quad (x,y) \neq (0,0)\\
    f_2(x,y) = y > 0, \quad \forall (x,y) \in V_2, \quad (x,y) \neq (0,0)\\
\end{gather*}

Теперь рассмотрим квадратичную форму $Q(x,y) = x^2 + y^2$ на $V$.
Можно заметить, что $Q(x,y)$ не является положительно определенной, так как она принимает отрицательные значения на
векторах $(x,y)$, не лежащих в $V_1$ или $V_2$.
Например, на векторе $(1,1)$:

\[ Q(1,1) = 1^2 + 1^2 = 2 > 0 \]

Таким образом, мы показали, что из положительной определенности двух ограничений пространства, являющейся прямой
сумма двух своих подпространств, не следует положительная определённость квадратичной формы.

\addcontentsline{toc}{subsection}{Кососимметричный определитель} \subsection*{Кососимметричный определитель}

Для того, чтобы доказать, что определитель целочисленной кососимметрической матрицы является квадратом целого числа,
мы можем использовать следующий факт:

Если $A$ -- кососимметрическая матрица, то $\det(A)$ является квадратом определителя матрицы $B$, где $B = iA$ и $i$
-- мнимая единица.

Итак, пусть $A$ -- целочисленная кососимметрическая матрица.
Тогда $B=iA$ также является кососимметрической матрицей.
Кроме того, элементы $B$ являются комплексными числами с мнимой частью, равной целому числу.

Таким образом, определитель $B$ является квадратом модуля его определителя.
Модуль комплексного числа с мнимой частью, равной целому числу, всегда является целым числом.

Следовательно, $\det(B)$ является квадратом целого числа.
Но $\det(B) = \det(iA) = i^n \det(A)$, где $n$ - порядок матрицы $A$.

Так как $A$ - кососимметрическая матрица, то $\det(A)$ является мнимым числом.
Поэтому $i^n\det(A)$ является квадратом целого числа.

Таким образом, мы доказали, что определитель целочисленной кососимметрической матрицы является квадратом целого числа

\addcontentsline{toc}{subsection}{Идемпотенция} \subsection*{Идемпотенция}

Самосопряженное преобразование -- это линейное преобразование, которое равно своему сопряженному.
Идемпотентное преобразование -- это линейное преобразование, которое при повторном применении к вектору даёт тот же
самый вектор.

Пусть $A$ -- матрица линейного преобразования.
Тогда самосопряжённость означает, что $A = A^*$, где $A^*$ -- сопряженная матрица.
Идемпотентность означает, что $A^2 = A$.

Рассмотрим матрицу $A$ размера $n \times n$.
Тогда $A$ самосопряжена, если $A = A^*$, то есть $a_{ij} = \overline{a_{ji}}$ для всех $i, j$.
Идемпотентность означает, что $A^2 = A$, то есть $A$ является проектором на некоторое подпространство
$V \subseteq \mathbb{C}^n$.

Пусть $A$ -- самосопряженный идемпотентный оператор.
Тогда $A^2 = A$ и $A = A^*$.
Рассмотрим собственные значения $\lambda$ матрицы $A$.
Так как $A$ идемпотентна, то $\lambda^2 = \lambda$, то есть $\lambda = 0$ или $\lambda = 1$.
Также, так как $A$ самосопряжена, то существует ортонормированный базис из собственных векторов матрицы $A$.
Пусть $V$ - подпространство, порожденное собственными векторами, соответствующими собственному значению $1$.
Тогда $A$ является проектором на $V$.

Таким образом, все самосопряженные идемпотентные операторы являются проекторами на некоторое подпространство
$V \subseteq \mathbb{C}^n$.
Обратно, любой проектор на подпространство $V$ является самосопряженным идемпотентным оператором.

\addcontentsline{toc}{subsection}{Общий ОНБ} \subsection*{Общий ОНБ}

Пусть $A$ и $B$ -- два самосопряженных оператора в евклидовом пространстве $V$.
Предположим, что у них есть общий ортонормированный базис из собственных векторов.
Тогда для любого вектора $v$ из этого базиса выполняется:
\begin{gather*}
    A = \lambda_v v,\\
    Bv = \mu_v v,\\
\end{gather*}
где $\lambda_v$ и $\mu_v$ -- собственные значения операторов $A$ и $B$ соответственно.
Таким образом, операторы $A$ и $B$ диагонализуемы и коммутируют в этом базисе:
\[ ABv = BA v = A(\mu_v v) = \mu_v Av = \mu_v \lambda_v v = \lambda_v Bv. \]
Обратно, предположим, что операторы $A$ и $B$ коммутируют.
Тогда они диагонализуемы в одном и том же ортонормированном базисе из собственных векторов.
Действительно, если $v_1, v_2, \dots, v_n$ -- ортонормированный базис из собственных векторов оператора $A$, то
каждый вектор $v_i$ является также собственным вектором оператора $B$, так как
\[ ABv_i = BA v_i = A(\mu_i v_i) = \mu_i Av_i = \mu_i \lambda_i v_i = \lambda_i Bv_i. \]
Таким образом, операторы $A$ и $B$ имеют общий ортонормированный базис из собственных векторов.

\addcontentsline{toc}{subsection}{Преобразование линейно} \subsection*{Преобразование линейно}

Пусть $f: V \to V$ -- отображение, сохраняющее скалярное произведение.
Тогда для любых векторов $u,v \in V$ выполняется
\[ \langle f(u), f(v) \rangle = \langle u,v \rangle. \]
Заметим, что это равенство можно переписать в виде
\[ \langle f(u+v), f(u+v) \rangle = \langle u+v,u+v \rangle. \]
Раскрывая скобки, получаем
\[ \langle f(u),f(u) \rangle + 2\langle f(u),f(v) \rangle + \langle f(v),f(v) \rangle = \langle u,u \rangle + 2\langle u,v \rangle + \langle v,v \rangle. \]
Так как $f$ сохраняет скалярное произведение, то $\langle f(u),f(u) \rangle = \langle u,u \rangle$ и $\langle f(v),f(v) \rangle = \langle v,v \rangle$.
Поэтому
\[ 2\langle f(u),f(v) \rangle = 2\langle u,v \rangle. \]
Таким образом, для любых векторов $u,v \in V$ выполнено $\langle f(u),f(v) \rangle = \langle u,v \rangle$, что
означает, что $f$ сохраняет углы и длины векторов.
Поэтому $f$ является изометрией евклидова пространства $V$.

Для доказательства линейности отображения $f$ достаточно заметить, что из сохранения скалярного произведения следует
линейность векторного отображения $f$.
Действительно, для любых векторов $u,v \in V$ и чисел $\alpha, \beta \in \mathbb{R}$ имеем
\[ \langle f(\alpha u + \beta v),f(\alpha u + \beta v) \rangle = \langle \alpha u + \beta v, \alpha u + \beta v \rangle. \]
Раскрывая скобки, получаем
\[ \alpha^2 \langle f(u),f(u) \rangle + 2\alpha\beta \langle f(u),f(v) \rangle + \beta^2 \langle f(v),f(v) \rangle = \alpha^2 \langle u,u \rangle + 2\alpha\beta \langle u,v \rangle + \beta^2 \langle v,v \rangle. \]
Так как $f$ сохраняет скалярное произведение, то $\langle f(u),f(u) \rangle = \langle u,u \rangle$ и $\langle f(v),f(v) \rangle = \langle v,v \rangle$.
Поэтому
\[ 2\alpha\beta \langle f(u),f(v) \rangle = 2\alpha\beta \langle u,v \rangle. \]
Таким образом, для любых векторов $u,v \in V$ выполнено $\langle f(\alpha u + \beta v),f(\alpha u + \beta v) \rangle = \langle \alpha u + \beta v, \alpha u + \beta v \rangle$, что
означает, что $f$ сохраняет углы и длины векторов.
Поэтому $f$ является изометрией евклидова пространства $V$, а изометрия линейна.

\addcontentsline{toc}{subsection}{Чётный ранг} \subsection*{Чётный ранг}

Пусть $V$ -- евклидово пространство, на котором задан оператор линейного преобразования $\varphi$, такой что $\varphi(v) \perp v$ для
любого вектора $v \in V$.
Нам нужно доказать, что ранг $\varphi$ является четным числом.

Заметим, что $\varphi(v) \perp v$ означает, что $\varphi(v)$ лежит в ортогональном дополнении к пространству, порожденному вектором $v$.
Таким образом, $\varphi(v)$ ортогонально любому вектору, лежащему в этом пространстве.

Рассмотрим произвольный вектор $v \in V$ и его ортогональное дополнение $W = \{w \in V \mid w \perp v\}$.
Так как $\varphi(v) \in W$ для любого $v \in V$, то $\varphi(W) \subseteq W$.
Более того, $\varphi(W)$ также ортогонально любому вектору из $W$, так как $\varphi(w) \perp w$ для любого $w \in W$.

Рассмотрим теперь ограничение $\varphi$ на $W$.
Так как $\varphi(W) \subseteq W$, то ограничение $\varphi|_W$ является линейным оператором на $W$.
Кроме того, $\varphi(w) \perp w$ для любого $w \in W$, поэтому $\varphi|_W$ является самосопряженным оператором на $W$.

Так как $\varphi|_W$ самосопряжен, то существует ортонормированный базис $e_1, \dots, e_k$ в $W$, состоящий из
собственных векторов $\varphi|_W$, где $k$ -- ранг $\varphi|_W$.
Так как $\varphi(W) \subseteq W$, то любой собственный вектор $\varphi|_W$ также является собственным вектором $\varphi$.

Расширим базис $e_1, \dots, e_k$ до ортонормированного базиса $e_1, \dots, e_n$ в $V$.
Тогда матрица оператора $\varphi$ в этом базисе имеет вид:

\[ [\varphi] =
\begin{pmatrix}
    \lambda_1 I_k & 0 \\
    0             & 0
\end{pmatrix}, \]

где $\lambda_1, \dots, \lambda_k$ -- собственные значения $\varphi|_W$, $I_k$ -- единичная матрица размера $k \times
k$, а $0$ -- нулевая матрица размера $(n-k) \times k$ и $0$ размера $(n-k) \times (n-k)$.

Таким образом, ранг матрицы $[\varphi]$ равен $k$, что является четным числом.
Следовательно, ранг оператора $\varphi$ также является четным числом.

    \addcontentsline{toc}{part}{Математический анализ} \part*{Математический анализ}

    \section*{Первое задание} \addcontentsline{toc}{section}{Первое задание}

    \addcontentsline{toc}{subsection}{Перестановочный ряд} \subsection*{Перестановочный ряд}

    В общем случае нельзя считать ряд, полученный перестановкой условно сходящегося ряда, сходящимся.

    Пусть дан условно сходящийся ряд $\sum_{n=1}^{\infty} a_n$, который сходится к $A$.
    Тогда существует перестановка $\sigma$ натуральных чисел, такая что $\sum_{n=1}^{\infty} a_{\sigma(n)}$ расходится.

    Действительно, по определению условной сходимости, существует такое число $B$, что ряд $\sum_{n=1}^{\infty} |a_n|$
    расходится, но ряд $\sum_{n=1}^{\infty} b_n$ сходится, где $b_n = |a_n|$ для $a_n \geq 0$ и $b_n = -|a_n|$ для $a_n < 0$.
    Рассмотрим частичные суммы ряда $\sum_{n=1}^{\infty} b_{\sigma(n)}$.
    Они могут принимать значения от $-B$ до $B$ (по теореме Римана об упорядочивании условно сходящегося ряда), но
    так как ряд $\sum_{n=1}^{\infty} b_n$ сходится, то существует такое число $C$, что любая частичная сумма ряда
    $\sum_{n=1}^{\infty} b_{\sigma(n)}$ не превосходит $C$.
    Однако, так как ряд $\sum_{n=1}^{\infty} |a_n|$ расходится, то существует такое число $D$, что сумма модулей
    любых $D$ членов ряда $\sum_{n=1}^{\infty} a_n$ больше $B$.
    Тогда найдется такое натуральное число $N$, что $\sum_{n=1}^{N} |a_n| > B$.
    Поскольку $|a_n| \leq b_n$ для всех $n$, то также выполнено $\sum_{n=1}^{N} b_n > B$.

    Теперь рассмотрим ряд $\sum_{n=1}^{\infty} a_{\sigma(n)}$.
    Его частичная сумма $\sum_{n=1}^{N} a_{\sigma(n)}$ равна сумме $N$ членов ряда $\sum_{n=1}^{\infty} a_n$ с
    разными знаками и в произвольном порядке.
    Можно выбрать такое $N$, что $\sum_{n=1}^{N} b_n > B$.
    Тогда найдутся такие индексы $i$ и $j$ ($i < j$), что $\sum_{n=i}^{j} b_n > B$, а следовательно, сумма $j-i+1$
    членов ряда $\sum_{n=1}^{\infty} a_{\sigma(n)}$ также превосходит $B$.
    Это означает, что ряд $\sum_{n=1}^{\infty} a_{\sigma(n)}$ расходится, и мы получили противоречие.

    Таким образом, в общем случае нельзя считать ряд, полученный перестановкой условно сходящегося ряда, сходящимся.
    Однако, в некоторых случаях, например, если ряд $\sum_{n=1}^{\infty} a_n$ абсолютно сходится, то любая
    перестановка его членов также будет сходиться к тому же пределу

\end{document}