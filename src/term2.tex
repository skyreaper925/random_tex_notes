\documentclass[a4paper, 14pt]{article}

%\hypersetup
%{   colorlinks,
%    pdftitle={2.1.6. Journal},
%    pdfauthor={Володин Максим},
%    allcolors=[RGB]{010 090 200}
%}

\usepackage[T2A]{fontenc}
\usepackage[utf8]{inputenc}
\usepackage[english, russian]{babel}
\usepackage[top = 2cm, bottom = 2cm, left = 2cm, right = 2cm]{geometry}
\usepackage{indentfirst}
\usepackage{xcolor}
\usepackage{hyperref}
\usepackage{graphicx}
\usepackage{gensymb}
\usepackage{pgfplots}
\usepackage{amsmath, amsfonts, amssymb, amsthm, mathtools}
\usepackage{physics, multirow, float}
\usepackage{wrapfig, tabularx}
\usepackage{icomma} % Clever comma: 0,2 - number while 0, 2 - two numbers
\usepackage{tikz, standalone}
\usepackage{fancyhdr,fancybox}
\usepackage{lastpage}
\usepackage{booktabs}
\usepackage{listings}
\usepackage{lstmisc}

\graphicspath{{images/}}
\DeclareGraphicsExtensions{.pdf,.png,.jpg}

\restylefloat{table}
\usetikzlibrary{external}

\mathtoolsset{showonlyrefs = true} % Numbers will appear only where \eqref{} in the text LINKED
\pagestyle{fancy}

\fancyhf{}
\fancyhead[R]{Семестровые задания}
\fancyfoot[R]{\thepage /\pageref{LastPage}}
%\fancyhead[L]{2.5.1}

\pgfplotsset{compat=1.18}

\begin{document}

    \tableofcontents

    \addcontentsline{toc}{part}{Линейная алгебра} \part*{Линейная алгебра}

    \addcontentsline{toc}{section}{Первое задание} \section*{Первое задание}

    \input{term2_tasks/первое-задание}

    \section*{Второе задание}

    \input{term2_tasks/второе-задание}

    \addcontentsline{toc}{part}{Математический анализ} \part*{Математический анализ}

    \section*{Первое задание} \addcontentsline{toc}{section}{Первое задание}

    \addcontentsline{toc}{subsection}{Перестановочный ряд} \subsection*{Перестановочный ряд}

    В общем случае нельзя считать ряд, полученный перестановкой условно сходящегося ряда, сходящимся.

    Пусть дан условно сходящийся ряд $\sum_{n=1}^{\infty} a_n$, который сходится к $A$.
    Тогда существует перестановка $\sigma$ натуральных чисел, такая что $\sum_{n=1}^{\infty} a_{\sigma(n)}$ расходится.

    Действительно, по определению условной сходимости, существует такое число $B$, что ряд $\sum_{n=1}^{\infty} |a_n|$
    расходится, но ряд $\sum_{n=1}^{\infty} b_n$ сходится, где $b_n = |a_n|$ для $a_n \geq 0$ и $b_n = -|a_n|$ для $a_n < 0$.
    Рассмотрим частичные суммы ряда $\sum_{n=1}^{\infty} b_{\sigma(n)}$.
    Они могут принимать значения от $-B$ до $B$ (по теореме Римана об упорядочивании условно сходящегося ряда), но
    так как ряд $\sum_{n=1}^{\infty} b_n$ сходится, то существует такое число $C$, что любая частичная сумма ряда
    $\sum_{n=1}^{\infty} b_{\sigma(n)}$ не превосходит $C$.
    Однако, так как ряд $\sum_{n=1}^{\infty} |a_n|$ расходится, то существует такое число $D$, что сумма модулей
    любых $D$ членов ряда $\sum_{n=1}^{\infty} a_n$ больше $B$.
    Тогда найдется такое натуральное число $N$, что $\sum_{n=1}^{N} |a_n| > B$.
    Поскольку $|a_n| \leq b_n$ для всех $n$, то также выполнено $\sum_{n=1}^{N} b_n > B$.

    Теперь рассмотрим ряд $\sum_{n=1}^{\infty} a_{\sigma(n)}$.
    Его частичная сумма $\sum_{n=1}^{N} a_{\sigma(n)}$ равна сумме $N$ членов ряда $\sum_{n=1}^{\infty} a_n$ с
    разными знаками и в произвольном порядке.
    Можно выбрать такое $N$, что $\sum_{n=1}^{N} b_n > B$.
    Тогда найдутся такие индексы $i$ и $j$ ($i < j$), что $\sum_{n=i}^{j} b_n > B$, а следовательно, сумма $j-i+1$
    членов ряда $\sum_{n=1}^{\infty} a_{\sigma(n)}$ также превосходит $B$.
    Это означает, что ряд $\sum_{n=1}^{\infty} a_{\sigma(n)}$ расходится, и мы получили противоречие.

    Таким образом, в общем случае нельзя считать ряд, полученный перестановкой условно сходящегося ряда, сходящимся.
    Однако, в некоторых случаях, например, если ряд $\sum_{n=1}^{\infty} a_n$ абсолютно сходится, то любая
    перестановка его членов также будет сходиться к тому же пределу

\end{document}