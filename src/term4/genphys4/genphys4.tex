%! Author = user
%! Date = 25.05.2024

\documentclass[a4paper, 14pt]{article}
%\documentclass[draft]{article}

\usepackage[T2A]{fontenc}
\usepackage[utf8]{inputenc}
\usepackage[english, russian]{babel}
\usepackage[top = 2cm, bottom = 2cm, left = 2cm, right = 2cm]{geometry}
\usepackage{indentfirst}
\usepackage{xcolor}
\usepackage{hyperref}
\usepackage{gensymb}
\usepackage{pgfplots}
\usepackage{amsmath, amsfonts, amsthm, mathtools}
\usepackage{physics, multirow, float}
\usepackage{wrapfig, tabularx}
\usepackage{icomma} % Clever comma: 0,2 - number while 0, 2 - two numbers
\usepackage{tikz, standalone}
\usepackage{fancyhdr,fancybox}
\usepackage{lastpage}
\usepackage{booktabs}
\usepackage{listings}
\usepackage{stmaryrd}

%\полуторный интервал
\onehalfspacing

\hypersetup
{   colorlinks = false,
    linkcolor = blue,
    pdftitle = {genphys4},
    pdfauthor = {Володин Максим},
    allcolors = [RGB]{010 090 200}
}

%\gravarphicspath{{images/}}
%\DeclareGravarphicsExtensions{.pdf,.png,.jpg}

\restylefloat{table}
\usetikzlibrary{external}

\mathtoolsset{showonlyrefs = true} % Numbers will appear only where \eqref{} in the text LINKED
\pagestyle{fancy}

\fancyhf{}
\fancyhead[L]{Общая физика. Оптика}
\fancyhead[R]{Конспект билетов}
\fancyfoot[L]{}
\fancyfoot[R]{\thepage /\pageref{LastPage}}

\pgfplotsset{compat=1.18}

\begin{document}
{\huge
    \begin{center}
    {\textbf{Конспект билетов}}
        \\
        Общая физика.
        Оптика
    \end{center}
}
    \tableofcontents \newpage
    
    \section{Геометрическая оптика.
    Принцип Ферма, законы преломления и отражения.
    Полное внутреннее отражение}
    
    \subsection{Геометрическая оптика}
    
    \textbf{Опр} \textit{Геометрическая оптика}
    
    \textcolor{blue}{Раздел оптики, изучающий законы распространения света без учёта его волновых свойств}
    
    \subsection{Принцип Ферма, законы преломления и отражения}
    
    По тексту
    
    Вторую формулировку принципа Ферма игнорировать
    
    По тексту
    
    \subsection{Полное внутреннее отражение}
    
    В какой-то момент, при переходе в более оптически плотную среду, преломлённый луч пропадает
    
    По тексту
    
    \section{Центрированные оптические системы.
    Тонкая линза.
    Фокусы и главные плоскости оптической системы.
    Оптические инструменты: лупа, телескоп и микроскоп}
    
    \subsection{Центрированные оптические системы}
    
    \textbf{Опр} \textit{Центрированная оптическая система}
    
    \textcolor{blue}{Совокупность преломляющих и отражающих поверхностей с осью симметрии -- оптической осью}
    
    По тексту
    
    \subsection{Фокусы и главные плоскости оптической системы}
    
    По тексту
    
    \subsection{Тонкая линза}
    
    По тексту.
    Формулы тонкой линзы и обобщённую формулу тонкой линзы не выводить
    
    \subsection{Оптические инструменты: лупа, телескоп и микроскоп}
    
    Чтобы продемонстрировать преимущества лупы, надо нарисовать изображение предмета в тонкой линзе, а затем найти $\Gamma$.
    При рассматривании вплотную к лупе, оптические силы хрусталика и линзы складываются и получаем +1 к $\Gamma$
    
    По тексту
    
    \section{Основы фотометрии.
    Яркость источника, освещённость изображения.
    Теорема о сохранении яркости оптической системой}
    
    \subsection{Основы фотометрии}
    
    По тексту
    
    \subsection{Яркость источника, освещённость изображения}
    
    По тексту
    
    \subsection{Теорема о сохранении яркости оптической системой}
    
    По тексту Назарчук
    
    \section{Волновое уравнение.
    Монохроматические волны.
    Уравнение Гельмгольца.
    Комплексная амплитуда.
    Волновой вектор, фазовая скорость.
    Плоские и сферические волны}
    
    \subsection{Волновое уравнение}
    
    По тексту
    
    \subsection{Монохроматические волны}
    
    \textbf{Опр} \textit{Волна}
    
    \textcolor{blue}{Изменение некоторой совокупности физических величин (характеристик некоторого физического поля или
    материальной среды), которое способно перемещаться, удаляясь от места своего возникновения, или колебаться
    внутри ограниченных областей пространства}
    
    \textbf{Опр} \textit{Монохроматическая волна}
    
    \textcolor{blue}{Строго гармоническая (синусоидальная) волна, в спектре которой наличествует всего одна
    составляющая по частоте (постоянная, как и амплитуда).
    Такая волна на практике не существует, но является удобной физической моделью для теоретического описания
    различных (электромагнитных, акустических и других) явлений волновой природы}
    
    \subsection{Уравнение Гельмгольца}
    
    По тексту
    
    \subsection{Комплексная амплитуда}
    
    По тексту
    
    \subsection{Волновой вектор, фазовая скорость}
    
    \textbf{Опр} \textit{Волнове число}
    
    \textcolor{blue}{Быстрота роста фазы волны $\varphi$ по координате в пространстве: $k = \dfrac{d \varphi}{d x}$}
    
    По тексту
    
    \subsection{Плоские и сферические волны}
    
    По тексту
    
    \section{Поток энергии и импульс электромагнитной волны.
    Давление света}
    
    \subsection{Поток энергии и импульс электромагнитной волны}
    
    При помещении вещества в электрическое поле происходит пространственное перераспределение заряда
    
    \textbf{Опр} \textit{Диэлектрик}
    
    \subsection{Давление света}
    
    \textbf{Опр} \textit{Поляризация}
    
    \section{Электромагнитные волны на границе раздела двух диэлектриков.
    Законы преломления и отражения.
    Зависимость коэффициентов отражения от угла падения.
    Явление Брюстера}
    
    \subsection{Электромагнитные волны на границе раздела двух диэлектриков}
    
    \textbf{Опр} \textit{Поляризация}
    
    \textcolor{blue}{Явление направленного колебания векторов напряжённости электрического поля $E$ или напряжённости
    магнитного поля $H$}
    
    По тексту
    
    \subsection{Законы преломления и отражения}
    
    По тексту, как и в первом билете
    
    \subsection{Зависимость коэффициентов отражения от угла падения}
    
    Игнорируя коэффициенты преломления, выведем коэффициенты отражения.
    
    \begin{enumerate}
        \item Запишем начальные условия и всякие законы сохранения
        \item Преобразуем, используя коэффициенты.
        \item Выразим эти коэффициенты через начальные условия
    \end{enumerate}
    
    \subsection{Явление Брюстера}
    
    По тексту теория, а по тексту Назарчук вывод формулы
    
    \section{Дисперсия волн.
    Волновой пакет, групповая скорость.
    Формула Рэлея}
    
    \subsection{Дисперсия волн}
    
    По тексту
    
    \subsection{Волновой пакет, групповая скорость}
    
    По тексту.
    При обобщении на произвольное количество волн получаем
    
    \textbf{Опр} \textit{Волновой пакет}
    
    \textcolor{blue}{Суперпозиция (наложение) плоских монохроматических волн с близкими значениями частот и волновых
    векторов}
    
    \subsection{Формула Рэлея}
    
    По тексту с привлечением аппарата полной производной
    
    \section{Классическая теория дисперсии света.
    Аномальная дисперсия.
    Поглощение света.
    Дисперсия в плазме и металлах}
    
    \subsection{Классическая теория дисперсии света}
    
    По тексту
    
    \subsection{Аномальная дисперсия}
    
    По тексту с пересказом графика
    
    \subsection{Поглощение света}
    
    По тексту
    
    \subsection{Дисперсия в плазме и металлах}
    
    По тексту, включая части со скоростями и плазменным зеркалом
    
    \section{Интерференция монохроматических волн.
    Интерференция плоских и сферических волн.
    Ширина интерференционных полос.
    Видность полос.
    Примеры схем наблюдения интерференции}
    
    По тексту
    
    \section{Статистическая природа света.
    Модель цугов.
    Функция временной когерентности и её связь с видностью полос.
    Связь временной когерентности со спектральной интенсивностью (теорема Винера– Хинчина).
    Соотношение неопределённостей для ширины спектра и времени когерентности}
    
    \subsection{Статистическая природа света}
    
    \subsection{Модель цугов}
    
    \textbf{Опр} \textit{Цуг}
    
    \textcolor{blue}{Обрывок, компактный гармонический процесс}
    
    По тексту
    
    \subsection{Соотношение неопределённостей для ширины спектра и времени когерентности}
    
    По тексту
    
    \subsection{Функция временной когерентности и её связь с видностью полос}
    
    По тексту
    
    \subsection{Связь временной когерентности со спектральной интенсивностью (теорема Винера– Хинчина)}
    
    Доказывать по видео 3 консультации с 20:50
    
    \section{Временная когерентность.
    Влияние немонохроматичности света на видность интерференционных полос.
    Время и длина когерентности.
    Максимальный порядок интерференции и максимальная разность хода в двулучевых схемах}
    
    \subsection{Временная когерентность}
    
    \textbf{Опр} \textit{Временная когерентность}
    
    \textcolor{blue}{Мера корреляции (скоррелированность, согласованнотсь) фазы волны в разные моменты времени в одной и
    той же точке пространства.
    Она характеризует, насколько монохроматичным является источник света}
    
    \subsection{Время и длина когерентности}
    
    По тексту
    
    \subsection{Влияние немонохроматичности света на видность интерференционных полос}
    
    По тексту
    
    \subsection{Максимальный порядок интерференции и максимальная разность хода в двухлучевых схемах}
    
    По тексту
    
    \section{Пространственная когерентность.
    Апертура интерференционной схемы и влияние размеров источника на видность интерференционной картины.
    Функция пространственной когерентности.
    Радиус пространственной когерентности}
    
    \subsection{Пространственная когерентность}
    
    \textbf{Опр} \textit{Пространственная когерентность}
    
    \textcolor{blue}{Когерентность колебаний, которые совершаются в один и тот же момент времени в разных точках
    плоскости, перпендикулярной направлению распространения волны}
    
    По тексту
    
    \subsection{Апертура интерференционной схемы и влияние размеров источника на видность интерференционной картины}
    
    По тексту на примере опыта Юнга
    
    \subsection{Радиус пространственной когерентности}
    
    По тексту
    
    \subsection{Функция пространственной когерентности}
    
    По тексту Botay под видом теоремы
    
    \section{Принцип Гюйгенса-Френеля.
    Количественная формулировка принципа Гюйгенса-Френеля.
    Общая задача о дифракции на тонком экране.
    Граничные условия.
    Волновой параметр и зависимость характера дифракции от его значения.
    Критерий геометрической оптики}
    
    По тексту
    
    \section{Дифракция Френеля на круглом отверстии.
    Спираль Френеля.
    Пятно Пуассона и условия его наблюдения.
    Дифракция на краю экрана и на щели, спираль Корню (качественно).
    Зонная пластинка Френеля.
    Фокусы зонной пластинки: положение и интенсивность света в них.
    Линза Френеля.
    Идеальная линза с точки зрения дифракции Френеля.
    Оценка размера фокального пятна}
    
    \subsection{Дифракция Френеля на круглом отверстии}
    
    По тексту
    
    \subsection{Спираль Френеля}
    
    По тексту
    
    \subsection{Пятно Пуассона и условия его наблюдения}
    
    По тексту.
    Лишь бы число зон Френеля не было слишком большим, иначе по тексту
    
    \subsection{Дифракция на краю экрана и на щели, спираль Корню (качественно)}
    
    По тексту, иногда обращаясь к Кадыгрову за дополнительными объяснениями.
    Это про край экран; в случае щели можно показать аналогичность по Назарчук
    
    \subsection{Зонная пластинка Френеля}
    
    По тексту
    
    \subsection{Фокусы зонной пластинки: положение и интенсивность света в них}
    
    По тексту без соответствующего раздела и дополнительных фокусов
    
    \subsection{Линза Френеля}
    
    По консультации начиная с 18:05
    
    \subsection{Идеальная линза с точки зрения дифракции Френеля}
    
    По тексту
    
    \subsection{Оценка размера фокального пятна}
    
    По тексту
    
    \section{Дифракция Фраунгофера.
    Связь с преобразованием Фурье.
    Дифракция Фраунгофера на щели (строгий вывод) и круглом отверстии (качественно).
    Поле в фокальной плоскости линзы, размеры фокального пятна.
    Роль дифракции Фраунгофера в оптических приборах.
    Разрешающая способность телескопа и микроскопа.
    Критерий Рэлея.
    Разрешающая способность при когерентном освещении}
    
    \subsection{Дифракция Фраунгофера}
    
    По тексту
    
    \subsection{Связь с преобразованием Фурье}
    
    По тексту
    
    \subsection{Дифракция Фраунгофера на щели (строгий вывод) и круглом отверстии (качественно)}
    
    По тексту
    
    \subsection{Поле в фокальной плоскости линзы, размеры фокального пятна}
    
    По тексту
    
    \subsection{Роль дифракции Фраунгофера в оптических приборах}
    
    Дифракция Фраунгофера позволяет определить разрешающую способность оптических приборов
    
    \subsection{Критерий Рэлея}
    
    По тексту
    
    \subsection{Разрешающая способность телескопа и микроскопа}
    
    По тексту без части после микроскопа
    
    \subsection{Разрешающая способность при когерентном освещении}
    
    Рассматриваем идеальные оптические системы, а конечный рассматриваемый объект как совокупность точечных
    источников, каждый из которых изображается пятном Эйри (с окружающими его дифракционными кольцами).
    Наша задача сводится к рассмотрению двух случаев точечных
    \begin{enumerate}
        \item некогерентных источников --- складываются их интенсивности --- самосветящиеся --- телескоп;
        \item когерентых источнико --- складываются их напряжённости --- освещаемые --- микроскоп.
    \end{enumerate}
    
    Разрешающие способности соответственно:
    \begin{equation*}
        \text{телескоп: } \vartheta_\text{мин} = 1,22 \frac{\lambda}{D}
        \hspace{2 cm}
        \text{микроскоп: } l_\text{мин} = 0.61 \frac{\lambda}{n \sin \alpha}
    \end{equation*}
    где $\alpha$ -- апертурный угол, $l$ -- расстояние между кружками Эйри, $\vartheta$ -- угловой размер наблюдаемого объекта.
    
    \section{Спектральные приборы.
    Общие характеристики спектральных приборов: разрешающая способность, область дисперсии, угловая дисперсия.
    Призма как спектральный прибор, угловая дисперсия и разрешающая способность призмы}
    
    \subsection{Спектральные приборы}
    
    \textbf{Опр} \textit{Спектральный прибор}
    
    \textcolor{blue}{Оптический прибор, в котором осуществляется разложение электромагнитного излучения оптического
    диапазона на монохроматические составляющие}
    
    \subsection{Общие характеристики спектральных приборов: разрешающая способность, область дисперсии, угловая дисперсия}
    
    По тексту, можно без примеров
    
    \subsection{Призма как спектральный прибор, угловая дисперсия и разрешающая способность призмы}
    
    По тексту Кадыгрова разрешающая способность.
    Максимальная разность хода двух лучей с длиной волны $\lambda$ и $\lambda + \delta \lambda$
    \begin{gather*}
        \Delta = a[n(\lambda + \delta \lambda) - n(\lambda)] = a \dfrac{dn}{d \lambda} \delta \lambda\\
        \Delta = h \delta \varphi\\
    \end{gather*}
    Приравнивания разные выражения для одной разности хода получаем
    \[ D = \dfrac{\delta \varphi}{\delta \lambda} = \frac{a}{h} \dfrac{dn}{d \lambda} \]
    
    \section{Дифракция Фраунгофера на амплитудной решётке.
    Положение и интенсивность главных максимумов, их ширина и максимальный порядок.
    Дифракционная решётка как спектральный прибор.
    Разрешающая способность, область дисперсии и угловая дисперсия решётки}
    
    \subsection{Дифракция Фраунгофера на амплитудной решётке}
    
    По тексту
    
    \subsection{Положение и интенсивность главных максимумов, их ширина и максимальный порядок}
    
    Угловое положение максимумов можно найти из их условия (можно воспользоваться малыми углами, но не обязательно).
    Чтобы найти геометрическое, требуется домножить на размер, например, фокус линзы
    
    Интенсивность по тексту
    
    Ширину находим из интенсивности, взяв $\argmax$ числителя и локализовав этот $\argmax$
    
    Максимальный порядок тоже находим из их условия
    
    \subsection{Дифракционная решётка как спектральный прибор}
    
    По тексту
    
    \subsection{Разрешающая способность, область дисперсии и угловая дисперсия решётки}
    
    Если на решётку из $N$ штрихов падает параллельный пучок света перпендикулярно поверхности, то если между ними
    возникает разность хода в $\frac{\lambda}{N}$, то они погасят друг друга (очевидно при $N = 2$).
    Это доказывается с помощью векторных диаграмм.
    Тогда по тексту становится всё ясно
    
    \section{Интерферометр Фабри-Перо как спектральный прибор и как оптический резонатор.
    Разрешающая способность, угловая дисперсия и область дисперсии интерферометра.
    Связь разрешающей способности интерферометра с добротностью резонатора}
    
    \subsection{Интерферометр Фабри-Перо как спектральный прибор и как оптический резонатор}
    
    \textbf{Опр} \textit{Резонатор}
    
    \textcolor{blue}{Колебательная система / устройство, способное накапливать энергию колебаний, поставляемую из
    внешнего источника}
    
    По тексту
    
    \subsection{Разрешающая способность, угловая дисперсия и область дисперсии интерферометра}
    
    Разрешающая способность даётся без вывода по тексту.
    Угловая дисперсия получается из условия $\Delta = 2L \cos(\varphi) \lambda$ и дифференцированием этого равенства.
    
    По тексту
    
    \subsection{Связь разрешающей способности интерферометра с добротностью резонатора}
    
    Они равны в силу приближений по тексту
    
    \section{Принципы Фурье-оптики.
    Пространственное преобразование Фурье, разложение по плоским волнам.
    Метод Рэлея в задачах дифракции.
    Пространственное соотношение неопределённостей}
    
    \subsection{Принципы Фурье-оптики}
    
    Принципы Фурье-оптики предполагают изучение классической оптики с использованием преобразований Фурье.
    В этой области анализируется, как свет распространяется в оптических приборах, учитывая его волновую природу.
    Методы Фурье-оптики используются для численных расчётов (включают одномерные или двумерные интегралы) и
    применимы к различным практическим случаям, когда свет
    распространяется преимущественно в одном направлении.
    
    Фурье-оптика необходима для более глубокого понимания распространения света и находит применение в
    анализе сложных оптических систем, обработке изображений, интерферометрии, оптических пинцетах, атомных ловушках
    и даже квантовых вычислениях.
    Это фундаментальный инструмент в оптической обработке информации, позволяющий осуществлять пространственную
    фильтрацию, оптическую корреляцию и создавать компьютерные голограммы
    
    \subsection{Пространственное преобразование Фурье, разложение по плоским волнам}
    
    \textbf{Опр} \textit{Пространственное преобразование Фурье}
    
    \textcolor{blue}{Представление волнового поля в виде суперпозиции волн, каждой из которых отвечает определённое
    направление волнового вектора}
    
    По тексту
    
    \subsection{Метод Рэлея в задачах дифракции}
    
    По тексту Кадыгрова
    
    \subsection{Пространственное соотношение неопределённостей}
    
    По двухтомнику страница 182 2.4
    
    \section{Теория Аббе формирования оптического изображения.
    Фурье-плоскость оптической системы.
    Разрешающая способность оптической системы с точки зрения Фурье-оптики}
    
    \subsection{Теория Аббе формирования оптического изображения}
    
    По двухтомнику страница 191 2.7.
    Заметим, что рассуждения справедливы только лишь при формировании изображения линзой
    
    \subsection{Фурье-плоскость оптической системы}
    
    По двухтомнику страница 191 2.7 видно, что линза раскладывает пришедшую волну (сигнал) в спектр
    
    \subsection{Разрешающая способность оптической системы с точки зрения Фурье-оптики}
    
    С точки зрения Фурье-оптики, разрешающая способность оптической системы определяется ее способностью передавать
    высокие пространственные частоты.
    Управлять передачей частот позволяет фильтрация в плоскости Фурье
    
    \section{Дифракция на периодических структурах с точки зрения Фурье-оптики.
    Эффект саморепродукции.
    Дифракция на амплитудной и фазовой синусоидальной решетке}
    
    \subsection{Дифракция на периодических структурах с точки зрения Фурье-оптики}
    
    По двухтомнику страница 180 2.3 и до вычисления волнового параметра
    
    \subsection{Эффект саморепродукции}
    
    По тексту
    
    \subsection{Дифракция на амплитудной и фазовой синусоидальной решетке}
    
    По тексту
    
    \section{Принципы пространственной фильтрации.
    Эффект мультиплицирования изображений.
    Методы наблюдения фазовых структур (тёмного поля и фазового контраста)}
    
    \subsection{Принципы пространственной фильтрации}
    
    Особая роль фурье-плоскости обусловлена тем, что именно в этой плоскости возможно избирательное воздействие на
    разные пространственные гармоники: установив в любой точке $x$ фурье-плоскости маленькую пластинку, вносящую
    определённое поглощение и (или) фазовую задержку, мы изменим амплитуду и (или) фазу плоской волны с
    пространственной частотой $u = \frac{kx}{f}$, не изменяя амплитуд и фаз других плоских волн.
    
    Устанавливая в фурье-плоскости различные амплитудно-фазовые маски, можно направленно изменять пространственный
    спектр изображения, влияя таким образом на его характеристики.
    Этим путём можно решать самые разнообразные задачи: улучшение качества изображений, разрешающей способности оптических
    систем, визуализация фазовых объектов, выполнение самых разнообразных преобразований пространственной структуры
    световых полей и т.д., т.е. решать широкий круг задач оптической обработки информации
    
    \subsection{Эффект мультиплицирования изображений}
    
    По двухтомнику со страницы 192 2.8
    
    \subsection{Методы наблюдения фазовых структур (тёмного поля и фазового контраста)}
    
    По консультации начиная с 6:26
    
    \section{Принципы голографии.
    Голограмма точечного источника (голограмма Габора).
    Голограмма с наклонным опорным пучком.
    Разрешающая способность голограммы}
    
    \subsection{Принципы голографии}
    
    \textbf{Опр} \textit{Голография}
    
    \textcolor{blue}{Метод точной записи волновых полей с учётом амплитуды и фазы}
    
    \textbf{Опр} \textit{Голограмма}
    
    \textcolor{blue}{Объёмное изображение на пластинке, полученное с помощью голографии}
    
    По двухтомнику со страницы 205--3
    
    \subsection{Голограмма точечного источника (голограмма Габора)}
    
    По тексту
    
    \subsection{Голограмма с наклонным опорным пучком}
    
    По тексту
    
    \subsection{Разрешающая способность голограммы}
    
    По консультации начиная с 12:45 можно выразить разрешающую способность по определению или по физическому смыслу
    
    \section{Дифракция на объёмных структурах, условие Брэгга–Вульфа.
    Дифракция рентгеновских лучей.
    Принципы записи и восстановления цвётной голограммы}
    
    \subsection{Дифракция на объёмных структурах, условие Брэгга–Вульфа}
    
    По тексту, а условие по тексту Кадыгрова
    
    \subsection{Принципы записи и восстановления цвётной голограммы}
    
    По тексту
    
    \subsection{Дифракция рентгеновских лучей}
    
    По тексту
    
    \section{Поляризация света.
    Линейная, круговая и эллиптическая поляризация.
    Естественно поляризованный свет.
    Степень поляризации.
    Способы получения линейно-поляризованного света.
    Дихроизм, поляроиды, закон Малюса}
    
    По тексту
    
    \section{Электромагнитные волны в одноосных кристаллах.
    Обыкновенная и необыкновенная волны.
    Ориентация векторов $k, E, B, D$ и вектора Пойнтинга $S$ в необыкновенной волне.
    Зависимость показателя преломления необыкновенной волны от угла распространения}
    
    \subsection{Электромагнитные волны в одноосных кристаллах}
    
    По тексту
    
    \subsection{Обыкновенная и необыкновенная волны}
    
    По тексту
    
    \subsection{Ориентация векторов $k, E, B, D$ и вектора Пойнтинга $S$ в необыкновенной волне}
    
    По консультации начиная с 14:56.
    У необыкновенной волны вектор Пойнтинга $S = \frac{c}{4 \pi} [E;H]$ и соответственно групповая скорость не
    коллинеарны в общем случае волновому вектору $k$, то есть направление переноса энергии и направление смещения
    поверхностей равных фаз не совпадают
    
    \subsection{Зависимость показателя преломления необыкновенной волны от угла распространения}
    
    По консультации после с 14:56.
    
    \section{Установившиеся колебания в цепи переменного тока.
    Комплексная форма представления колебаний.
    Векторные диаграммы.
    Комплексное сопротивление (импеданс).
    Правила Кирхгофа для переменных токов.
    Работа и мощность переменного тока}
    
    \section{Спектральное разложение электрических сигналов.
    Спектр одиночного прямоугольного импульса и периодической последовательности импульсов.
    Вынужденные колебания под действием произвольной силы.
    Соотношение неопределённостей}
    
    \subsection{Вынужденные колебания под действием произвольной силы}
    
    Запишем второе правило Кирхгофа и введём линейный оператор \ldots
    
    \textbf{Утв} \textit{Принцип суперпозиции откликов}
    
    \textcolor{blue}{Отклик на суммарное воздействие равен сумме откликов на элементарные}
    
    Таким образом, любую функцию можно представить в виде суперпозиции гармонических, равно как и вычислить суммарное
    воздействие
    
    \subsection{Соотношение неопределённостей}
    
    \begin{enumerate}
        \item Рассмотрим одиночный прямоугольный спектр
        \item График его Фурье-спектра имеет вид колокола
        \item Если мы <сожмём> функцию спектра по $x$ в $A$ раз, то есть перейдём к функции $f_A(x) = f(Ax)$, то её
        спектр растянется во столько же раз: $F_A(k) = const \cdot F\left(\frac{k}{A}\right)$, поскольку частота каждой
        спектральной
        гармоники $e^{ikx}$ этого разложения должны будут, очевидно, умножиться на A
        \item Эта иллюстрация, строго говоря, хоть и носит довольно частный характер, однако она обнажает
        физический смысл иллюстрируемого свойства: когда мы сжимаем сигнал, его частоты во столько же раз увеличиваются.
        Другими словами, невозможно произвольно сконцентрировать как функцию, так и её преобразование Фурье
    \end{enumerate}
    
    \section{Спектральный анализ линейных систем.
    Частотная характеристика и импульсный отклик системы.
    Колебательный контур как спектральный прибор.
    Интегрирующая и дифференцирующая цепочки как высокочастотный и низкочастотный фильтры}
    
    \subsection{Спектральный анализ линейных систем. Колебательный контур как спектральный прибор}
    
    Рассмотрим RLC-контур и запишем для него II правило Кирхгофа и теорему Фурье.
    Если вынуждающая сила $\sim e^{i\omega t}$, то и отклик тоже; подставим в уравнение
    
    \textbf{Опр} \textit{Функция отклика}
    
    Благодаря ей получим
    
    \textbf{Утв} \textit{Связь между компонентами отклика и сигнала}
    
    \textbf{Опр} \textit{Импульсная (дельта) функция}
    
    \subsection{Частотная характеристика и импульсный отклик системы}
    
    Если источник был импульсным, то отклик тоже будет импульсным
    
    \textbf{Опр} \textit{Импульсный отклик системы}
    
    \textbf{Опр} \textit{Частотная характеристика контура}
    
    \subsection{Интегрирующая и дифференцирующая цепочки как высокочастотный и низкочастотный фильтры}
    
    Можно показать, что интегрирующая цепочка может служить фильтром высоких, а дифференцирующая -- низких частот
    
    \section{Модуляция и детектирование сигналов.
    Амплитудная и фазовая модуляции.
    Спектры гармонически модулированных по фазе и амплитуде сигналов.
    Квадратичное детектирование сигналов}
    
    \subsection{Модуляция и детектирование сигналов}
    
    \textbf{Опр} \textit{Модулированные колебания}
    
    Существует три типа модуляции
    
    \subsection{Амплитудная и фазовая модуляции}
    
    \textbf{Опр} \textit{Амплитудно-модулированный сигнал}
    
    \textbf{Опр} \textit{Фазово-модулированный сигнал}
    
    \subsection{Спектры гармонически модулированных по фазе и амплитуде сигналов}
    
    Если найти спектр амплитудно-модулированного сигнала, то он будет состоять из трёх гармоник
    
    Спектр фазово-модулированного сигнала тоже будет состоять из трёх гармоник, но немного других
    
    \subsection{Квадратичное детектирование сигналов}
    
    \textbf{Опр} \textit{Детектирование}
    
    Например, квадратичные детекторы усредняют квадрат функции по времени
    
    Применить детектирование можно как к АМ, так и к ФМ сигналу, как по определению, так и с помощью метода
    комплексных амплитуд
    
    \section{Электрические флуктуации.
    Тепловой шум.
    Тепловые флуктуации в колебательном контуре.
    Интенсивность теплового шума, формула Найквиста}
    
    \subsection{Электрические флуктуации. Тепловой шум}
    
    \textbf{Опр} \textit{Тепловой шум}
    
    Система без активного сопротивления теплового шума не производит
    
    \begin{enumerate}
        \item Пусть есть проводник с концентрацией электронов $n$ в объёме $V$.
        \item Усредним плотность тока, усреднив каждый её элемент, не забыв, что среднее от среднего есть ноль
        \item От плотности тока перейдём к флуктуации полного тока, используя теорию Друде
    \end{enumerate}
    
    \subsection{Тепловые флуктуации в колебательном контуре}
    
    Рассмотрев высокодобротный RLC-контур (с малым затуханием) найдём теряем за период энергию, а также, рассеивающую
    мощность.
    Полученная формула даёт оценку мощности, рассеиваемой в активном сопротивлении, но с другой стороны, точно такая
    же мощность затрачивается на создание флуктуации ЭДС.
    Следовательно, полученная формула даёт выражение для спектра тепловых шумов.
    
    \subsection{Интенсивность теплового шума, формула Найквиста}
    
    Используя формулы для мощности (две), можно получить две формулы Найквиста
    
    \textbf{Опр} \textit{Белый шум}
    
    \textcolor{blue}{Стационарный шум, спектральные составляющие которого равномерно распределены по всему диапазону
    задействованных частот}
    
    \section{Электрические флуктуации. Дробовой шум.
    Интенсивность дробового шума, закон $\sqrt{N}$, формула Шоттки}
    
    \subsection{Электрические флуктуации. Дробовой шум}
    
    \textbf{Опр} \textit{Дробовой шум}
    
    Посчитаем случайный ток от дробового шума
    
    \subsection{Интенсивность дробового шума, закон $\sqrt{N}$, формула Шоттки}
    
    \textbf{Опр} \textit{Интенсивность} \\
    
    Дальнейший рассказ лучше вести по билету Ральфа (страница 64) \\
    
    \textbf{Опр} \textit{формула Шоттки}
    
    \section{Параметрическое возбуждение колебаний.
    Условие параметрического резонанса}
    
    \subsection{Параметрическое возбуждение колебаний}
    
    \textbf{Опр} \textit{Параметрические колебательные системы}
    
    \textbf{Режим} \textit{Вариометра}
    
    \subsection{Условие параметрического резонанса}
    
    Посчитаем закачку и потери энергии за период
    
    \textbf{Условие} \textit{Возбуждения колебаний}
    
    \section{Автоколебания в электрических цепях.
    Положительная обратная связь.
    Условие самовозбуждения}
    
    \subsection{Автоколебания в электрических цепях}
    
    \textbf{Опр} \textit{Автоколебания}
    
    \subsection{Положительная обратная связь}
    
    \textbf{Опр} \textit{Обратная связь}
    
    \textbf{Опр} \textit{Положительная обратная связь}
    
    \textcolor{blue}{Тип обратной связи, при котором изменение выходного сигнала системы приводит к такому изменению
    входного сигнала, которое способствует дальнейшему отклонению выходного сигнала от первоначального значения, то
    есть знак изменения сигнала обратной связи совпадает со знаком изменения входного сигнала}
    
    \textbf{Опр} \textit{Диод}
    
    \textbf{Опр} \textit{Триод}
    
    \textbf{Опр} \textit{Усилитель}
    
    \subsection{Условие самовозбуждения}
    
    \begin{enumerate}
        \item Рассмотрим колебания под действием периодических \(''\)толчков\(''\).
        \item Построим график, изображающие действия толчков.
        \item Соорудим генератор с обратной связью, построим графики заряда, тока и толчков (внешнего напряжения)
        \item Запишем уравнение колебаний и решим его с помощью новых замен
    \end{enumerate}
    
    \textbf{Условие} \textit{Самовозбуждения}
    
    \section{Волновое уравнение как следствие уравнений Максвелла.
    Электромагнитные волны в однородном диэлектрике, их поперечность и скорость распространения}
    
    \subsection{Волновое уравнение как следствие уравнений Максвелла}
    
    Запишем уравнения Максвелла в частном случае.
    Посчитаем ротор ротора двумя способами и получим два волновых уравнения
    
    \subsection{Электромагнитные волны в однородном диэлектрике, их поперечность и скорость распространения}
    
    \textbf{Утв} \textit{$\overrightarrow{k}$}
    
    \textcolor{blue}{Единичный вектор в направлении распространения волны}
    
    Можно показать, что
    
    \textbf{Утв} \textit{Поперечность электромагнитных волн}
    
    \textcolor{blue}{Электромагнитные волны $\bot E, H$}
    
    \textbf{Утв} \textit{Векторы $k, E$ и $H$ образуют правую тройку}
    
    \textbf{Утв} \textit{Поперечность электромагнитных волн}
    
    Исходя из анализа размерности, можно найти скорость распространения волны в пространстве
    
    \textbf{Опр} \textit{Волновое сопротивление}
    
    \section{Монохроматические волны.
    Комплексная амплитуда волны.
    Плоская электромагнитная волна.
    Приближение сферической волны.
    Связь полей $E$ и $B$ в плоской электромагнитной волне.
    Стоячие и бегущие волны.
    Отражение волн от идеального проводника}
    
    \subsection{Монохроматические волны}
    
    \textbf{Опр} \textit{Монохроматическая волна}
    
    \textbf{Опр} \textit{Волновое число}
    
    \subsection{Комплексная амплитуда волны}
    
    Обычную волну можно записать с помощью метода комплексных амплитуд
    
    \textbf{Опр} \textit{Комплексная амплитуда}
    
    \subsection{Плоская электромагнитная волна}
    
    \textbf{Утв} \textit{Уравнение плоской электромагнитной волны}
    
    \textbf{Опр} \textit{Фазовая скорость}
    
    \subsection{Приближение сферической волны}
    
    Если рассмотреть лапласиан сферических координат и записать волновое уравнение с его решением, то можно увидеть,
    что амплитуда убывает обратно пропорционально расстоянию
    
    \subsection{Связь полей $E$ и $B$ в плоской электромагнитной волне}
    
    Через волновое сопротивление можно показать, что $E \bot B$, а также, формульную связь
    
    \subsection{Стоячие и бегущие волны}
    
    До этого мы рассматривали только бегущие волны, но если сложить бегущие в противоположном направлении волны, то
    можно получить уравнение стоячей волны
    
    \subsection{Отражение волн от идеального проводника}
    
    Дальнейший рассказ лучше вести по консультации А.~В.~Гаврикова (билеты 35, 39) \\
    
    Граничные условия для $E$ нам известны и очевидны (внутри проводника поля нет).
    Можно записать граничные условия и для магнитного поля.
    
    \section{Поток энергии в электромагнитной волне.
    Давление излучения.
    Электромагнитный импульс}
    
    \subsection{Поток энергии в электромагнитной волне}
    
    Можно показать, что вектор Пойнтинга равен по модулю и задаёт направление потоку энергии в электромагнитной волне
    
    \textbf{Опр} \textit{Интенсивность излучения}
    
    \subsection{Электромагнитный импульс}
    
    \textbf{Опр} \textit{Электромагнитный импульс}
    
    \textcolor{blue}{Возмущение ЭМ поля, оказывающее влияние на любой материальный объект, находящийся в зоне действия}
    
    Таким импульс обладает и поле в вакууме
    
    \textbf{Опр} \textit{Плотность импульса}
    
    \subsection{Давление излучения}

%    \textbf{Опр} \textit{Объёмная плотность силы}
    
    Посчитать давление излучения, которое оно оказывает на среду при поглощении, можно, используя граничные условия.
    Можно найти давление волны на поверхность как идеального, так и любого другого проводника
    
    \section{Электромагнитные волны на границе раздела двух диэлектриков.
    Формулы Френеля.
    Явление Брюстера.
    Полное внутреннее отражение}
    
    \subsection{Электромагнитные волны на границе раздела двух диэлектриков}
    
    Рассмотрим падающую волну и запишем для неё граничные условия.
    Получим, два уравнения (на $\omega$ и $\overrightarrow{k}$), из которых следует равенства углов падения и
    отражения и закон Снеллиуса
    
    \subsection{Полное внутреннее отражение}
    
    \textbf{Опр} \textit{Угол полного внутреннего отражения}
    
    \subsection{Формулы Френеля}
    
    \textbf{Опр} \textit{Коэффициент отражения}
    
    \textbf{Опр} \textit{Коэффициент преломления}
    
    \textbf{Опр} \textit{$s-$поляризация}
    
    \textbf{Опр} \textit{$p-$поляризация}
    
    Записав граничные условия для каждой поляризации, найдём формулы Френеля
    
    \subsection{Явление Брюстера}
    
    В случае $p-$поляризация возможно явление (эффект) Брюстера, который имеет место для любой среды
    
    \textbf{Опр} \textit{Угол Брюстера}
    
    \section{Излучение электромагнитных волн.
    Зависимость интенсивности дипольного излучения от частоты, диаграмма направленности}
    
    \subsection{Излучение электромагнитных волн}
    
    Дальнейший рассказ можно вести по консультации А.~В.~Гаврикова (билет 40) \\
    
    \textbf{Опр} \textit{Излучение}
    
    Рассмотрим волну, создаваемую точечным диполем в вакууме
    
    \subsection{Зависимость интенсивности дипольного излучения от частоты, диаграмма направленности}
    
    Посчитаем мощность излучения и получим
    
    \textbf{Утв} \textit{Угловое распределение излучения (диаграмма направленности)}
    
    \textbf{Закон} \textit{Рэлея}
    
    Таким образом, поле ускоренно движущегося заряда зависит от промежутка времени до, угла и обратно расстоянию
    
    \section{Линии передачи энергии.
    Двухпроводная линия, коаксиальный кабель.
    Скорость волны, волновое сопротивление.
    Коэффициент стоячей волны.
    Согласованная нагрузка}
    
    \subsection{Линии передачи энергии}
    
    \textbf{Опр} \textit{Линии передачи энергии}
    
    Чаще всего такие линии неквазистационарны
    
    \subsection{Двухпроводная линия, коаксиальный кабель}
    
    \textbf{Опр} \textit{Двухпроводная линия}
    
    \textbf{Опр} \textit{Телеграфные уравнения}
    
    \textbf{Опр} \textit{Фазовая скорость}
    
    Найдём её для двух примеров длинной линии: коаксиального кабеля и двухпроводной линии
    
    \textbf{Опр} \textit{Волновое уравнение сигнала}
    
    У этого уравнения, как и любого другого дифференциального, есть свои граничные условия
    
    \subsection{Скорость волны, волновое сопротивление}
    
    \textbf{Опр} \textit{Скорость волны}
    
    \textcolor{blue}{Скорость распространения возмущения, не может превышать световую и по сути являестя фазовой}
    
    Найдём волновое сопротивление, рассмотрев бегущую волну в двухпроводной линии
    
    \textbf{Опр} \textit{Волновое сопротивление}
    
    \subsection{Согласованная нагрузка}
    
    \textbf{Опр} \textit{Волновая загрузка}
    
    \textcolor{blue}{Такая нагрузка, которая поглощает всю энергию волны (равнозначна отсутствию отражения)}
    
    \subsection{Коэффициент стоячей волны}
    
    \textbf{Опр} \textit{Коэффициент стоячей волны}
    
    Его связь с волновым сопротивлением можно попробовать показать через $E_{fall} + E_{refl}$
    
    \textbf{Опр} \textit{Коэффициент бегущей волны}
    
    \section{Электромагнитные волны в прямоугольном волноводе.
    Простейшие типы волн в волноводе прямоугольного сечения.
    Дисперсионное уравнение, критическая частота, длина волны и фазовая скорость волны в волноводе.
    Объёмные электромагнитные резонаторы}
    
    \subsection{Электромагнитные волны в прямоугольном волноводе}
    
    \textbf{Опр} \textit{Волновод}
    
    Записав упрощённые уравнения Максвелла и граничные условия к ним, получим решение в виде монохроматической волны
    
    \textbf{Уравнение} \textit{Гельмгольца}
    
    \subsection{Простейшие типы волн в волноводе прямоугольного сечения}
    
    Граничным условиям уравнения Гельмгольца не могут быть выполнены одновременно, но могут быть по отдельности.
    Поэтому в волноводе существует несколько типов волн
    
    \textbf{Опр} \textit{$H-$волна}
    
    \textbf{Опр} \textit{$E-$волна}
    
    \textbf{Опр} \textit{Псевдопоперечная волна}
    
    Решение волнового уравнения возможно лишь при некоторых значениях $\gamma_\lambda$
    
    \subsection{Дисперсионное уравнение, критическая частота, длина волны и фазовая скорость волны в волноводе}
    
    \textbf{Опр} \textit{Дисперсионное уравнение}
    
    \textbf{Опр} \textit{Критическая частота}
    
    Можно показать, что, например, в простейшем случае прямоугольного волновода она будет связана с его геометрией
    
    \textbf{Опр} \textit{Фазовая скорость}
    
    \textbf{Опр} \textit{Групповая скорость}
    
    \subsection{Объёмные электромагнитные резонаторы}
    
    \textbf{Опр} \textit{Резонатор}
    
    По сути, это волновод с закрытыми входами и выходами (внутри стоячая волна)
    
    \textbf{Опр} \textit{Объёмный резонатор}
    
    Спектр волн в таком резонатор будет дополнен
    
    \section{Квазистационарное проникновение электрического и магнитного полей в проводящую среду. Скин-эффект.
    Глубина скин-слоя для постоянного и переменного полей}
    
    \subsection{Квазистационарное проникновение электрического и магнитного полей в проводящую среду. Скин-эффект}
    
    \textbf{Опр} \textit{Скин-эффект}
    
    Рассмотрим распространение переменного $E$ в проводнике и получим дифференциальное уравнение.
    
    \subsection{Глубина скин-слоя для постоянного и переменного полей}
    
    Найдём распределение тока по объёму проводника
    
    \textbf{Опр} \textit{Глубина скин-слоя}
    
    В случае постоянного тока ($\omega = 0$) ток течёт по всему сечению проводника
    
    \section{Плазма.
    Дебаевский радиус экранирования.
    Плазменные колебания, плазменная частота}
    
    \subsection{Плазма}
    
    \textbf{Опр} \textit{Плазма}
    
    \textbf{Опр} \textit{Квазинейтральная плазма}
    
    \subsection{Дебаевский радиус экранирования}
    
    Рассмотрим экранирование заряда в плазме
    
    \textbf{Утв} \textit{Потенциал заряда в плазме}
    
    Оценим масштаб, на котором может происходить сильное разделение зарядов
    
    \textbf{Опр} \textit{Дебаевский радиус}
    
    \subsection{Плазменные колебания, плазменная частота}
    
    \textbf{Опр} \textit{Плазменная частота}
    
    \textbf{Опр} \textit{Плазменные колебания}
    
    \section{Диэлектрическая проницаемость холодной плазмы.
    Проникновение электромагнитных волн в плазму}
    
    \subsection{Диэлектрическая проницаемость холодной плазмы}
    
    Рассмотрим движение электрона в холодной плазме
    
    \textbf{Утв} \textit{Диэлектрическая проницаемость холодной плазмы}
    
    \subsection{Проникновение электромагнитных волн в плазму}
    
    Рассмотрим плоскую волну, распространяющую в плазме
    
    \textbf{Утв} \textit{В плазме могут распространяться только волны с частотой больше плазменной}

\end{document}
