%! Author = user
%! Date = 25.05.2024

\documentclass[a4paper, 14pt]{article}
%\documentclass[draft]{article}

\usepackage[T2A]{fontenc}
\usepackage[utf8]{inputenc}
\usepackage[english, russian]{babel}
\usepackage[top = 2cm, bottom = 2cm, left = 2cm, right = 2cm]{geometry}
\usepackage{indentfirst}
\usepackage{xcolor}
\usepackage{hyperref}
\usepackage{gensymb}
\usepackage{pgfplots}
\usepackage{amsmath, amsfonts, amsthm, mathtools}
\usepackage{physics, multirow, float}
\usepackage{wrapfig, tabularx}
\usepackage{icomma} % Clever comma: 0,2 - number while 0, 2 - two numbers
\usepackage{tikz, standalone}
\usepackage{fancyhdr,fancybox}
\usepackage{lastpage}
\usepackage{booktabs}
\usepackage{listings}
\usepackage{stmaryrd}

%\полуторный интервал
\onehalfspacing

\hypersetup
{   colorlinks = false,
    linkcolor = blue,
    pdftitle = {genphys4},
    pdfauthor = {Володин Максим},
    allcolors = [RGB]{010 090 200}
}

%\gravarphicspath{{images/}}
%\DeclareGravarphicsExtensions{.pdf,.png,.jpg}

\restylefloat{table}
\usetikzlibrary{external}

\mathtoolsset{showonlyrefs = true} % Numbers will appear only where \eqref{} in the text LINKED
\pagestyle{fancy}

\fancyhf{}
\fancyhead[L]{Общая физика. Оптика}
\fancyhead[R]{Конспект билетов}
\fancyfoot[L]{}
\fancyfoot[R]{\thepage /\pageref{LastPage}}

\pgfplotsset{compat=1.18}

\begin{document}
{\huge
    \begin{center}
    {\textbf{Конспект билетов}}
        \\
        Общая физика.
        Оптика
    \end{center}
}
    \tableofcontents \newpage
    
    \section{Геометрическая оптика.
    Принцип Ферма, законы преломления и отражения.
    Полное внутреннее отражение}
    
    \subsection{Геометрическая оптика}
    
    \textbf{Опр} \textit{Геометрическая оптика}
    
    \textcolor{blue}{Раздел оптики, изучающий законы распространения света без учёта его волновых свойств}
    
    \subsection{Принцип Ферма, законы преломления и отражения}
    
    По тексту
    
    Вторую формулировку принципа Ферма игнорировать
    
    По тексту
    
    \subsection{Полное внутреннее отражение}
    
    В какой-то момент, при переходе в более оптически плотную среду, преломлённый луч пропадает
    
    По тексту
    
    \section{Центрированные оптические системы.
    Тонкая линза.
    Фокусы и главные плоскости оптической системы.
    Оптические инструменты: лупа, телескоп и микроскоп}
    
    \subsection{Центрированные оптические системы}
    
    \textbf{Опр} \textit{Центрированная оптическая система}
    
    \textcolor{blue}{Совокупность преломляющих и отражающих поверхностей с осью симметрии -- оптической осью}
    
    По тексту
    
    \subsection{Фокусы и главные плоскости оптической системы}
    
    По тексту
    
    \subsection{Тонкая линза}
    
    По тексту.
    Формулы тонкой линзы и обобщённую формулу тонкой линзы не выводить
    
    \subsection{Оптические инструменты: лупа, телескоп и микроскоп}
    
    Чтобы продемонстрировать преимущества лупы, надо нарисовать изображение предмета в тонкой линзе, а затем найти $\Gamma$.
    При рассматривании вплотную к лупе, оптические силы хрусталика и линзы складываются и получаем +1 к $\Gamma$
    
    По тексту
    
    \section{Основы фотометрии.
    Яркость источника, освещённость изображения.
    Теорема о сохранении яркости оптической системой}
    
    \subsection{Основы фотометрии}
    
    По тексту
    
    \subsection{Яркость источника, освещённость изображения}
    
    По тексту
    
    \subsection{Теорема о сохранении яркости оптической системой}
    
    По тексту Назарчук
    
    \section{Волновое уравнение.
    Монохроматические волны.
    Уравнение Гельмгольца.
    Комплексная амплитуда.
    Волновой вектор, фазовая скорость.
    Плоские и сферические волны}
    
    \subsection{Волновое уравнение}
    
    По тексту
    
    \subsection{Монохроматические волны}
    
    \textbf{Опр} \textit{Волна}
    
    \textcolor{blue}{Изменение некоторой совокупности физических величин (характеристик некоторого физического поля или
    материальной среды), которое способно перемещаться, удаляясь от места своего возникновения, или колебаться
    внутри ограниченных областей пространства}
    
    \textbf{Опр} \textit{Монохроматическая волна}
    
    \textcolor{blue}{Строго гармоническая (синусоидальная) волна, в спектре которой наличествует всего одна
    составляющая по частоте (постоянная, как и амплитуда).
    Такая волна на практике не существует, но является удобной физической моделью для теоретического описания
    различных (электромагнитных, акустических и других) явлений волновой природы}
    
    \subsection{Уравнение Гельмгольца}
    
    По тексту
    
    \subsection{Комплексная амплитуда}
    
    По тексту
    
    \subsection{Волновой вектор, фазовая скорость}
    
    \textbf{Опр} \textit{Волнове число}
    
    \textcolor{blue}{Быстрота роста фазы волны $\varphi$ по координате в пространстве: $k = \dfrac{d \varphi}{d x}$}
    
    По тексту
    
    \subsection{Плоские и сферические волны}
    
    По тексту
    
    \section{Поток энергии и импульс электромагнитной волны.
    Давление света}
    
    \subsection{Поток энергии и импульс электромагнитной волны}
    
    При помещении вещества в электрическое поле происходит пространственное перераспределение заряда
    
    \textbf{Опр} \textit{Диэлектрик}
    
    \subsection{Давление света}
    
    \textbf{Опр} \textit{Поляризация}
    
    \section{Электромагнитные волны на границе раздела двух диэлектриков.
    Законы преломления и отражения.
    Зависимость коэффициентов отражения от угла падения.
    Явление Брюстера}
    
    \subsection{Электромагнитные волны на границе раздела двух диэлектриков}
    
    \textbf{Опр} \textit{Поляризация}
    
    \textcolor{blue}{Явление направленного колебания векторов напряжённости электрического поля $E$ или напряжённости
    магнитного поля $H$}
    
    По тексту
    
    \subsection{Законы преломления и отражения}
    
    По тексту, как и в первом билете
    
    \subsection{Зависимость коэффициентов отражения от угла падения}
    
    Игнорируя коэффициенты преломления, выведем коэффициенты отражения.
    
    \begin{enumerate}
        \item Запишем начальные условия и всякие законы сохранения
        \item Преобразуем, используя коэффициенты.
        \item Выразим эти коэффициенты через начальные условия
    \end{enumerate}
    
    \subsection{Явление Брюстера}
    
    По тексту теория, а по тексту Назарчук вывод формулы
    
    \section{Дисперсия волн.
    Волновой пакет, групповая скорость.
    Формула Рэлея}
    
    \subsection{Дисперсия волн}
    
    По тексту
    
    \subsection{Волновой пакет, групповая скорость}
    
    По тексту.
    При обобщении на произвольное количество волн получаем
    
    \textbf{Опр} \textit{Волновой пакет}
    
    \textcolor{blue}{Суперпозиция (наложение) плоских монохроматических волн с близкими значениями частот и волновых
    векторов}
    
    \subsection{Формула Рэлея}
    
    По тексту с привлечением аппарата полной производной
    
    \section{Классическая теория дисперсии света.
    Аномальная дисперсия.
    Поглощение света.
    Дисперсия в плазме и металлах}
    
    \subsection{Классическая теория дисперсии света}
    
    По тексту
    
    \subsection{Аномальная дисперсия}
    
    По тексту с пересказом графика
    
    \subsection{Поглощение света}
    
    По тексту
    
    \subsection{Дисперсия в плазме и металлах}
    
    По тексту, включая части со скоростями и плазменным зеркалом
    
    \section{Интерференция монохроматических волн.
    Интерференция плоских и сферических волн.
    Ширина интерференционных полос.
    Видность полос.
    Примеры схем наблюдения интерференции}
    
    По тексту
    
    \section{Статистическая природа света.
    Модель цугов.
    Функция временной когерентности и её связь с видностью полос.
    Связь временной когерентности со спектральной интенсивностью (теорема Винера– Хинчина).
    Соотношение неопределённостей для ширины спектра и времени когерентности}
    
    \subsection{Статистическая природа света}
    
    \subsection{Модель цугов}
    
    \textbf{Опр} \textit{Цуг}
    
    \textcolor{blue}{Обрывок, компактный гармонический процесс}
    
    По тексту
    
    \subsection{Соотношение неопределённостей для ширины спектра и времени когерентности}
    
    По тексту
    
    \subsection{Функция временной когерентности и её связь с видностью полос}
    
    По тексту
    
    \subsection{Связь временной когерентности со спектральной интенсивностью (теорема Винера– Хинчина)}
    
    Доказывать по видео 3 консультации с 20:50
    
    \section{Временная когерентность.
    Влияние немонохроматичности света на видность интерференционных полос.
    Время и длина когерентности.
    Максимальный порядок интерференции и максимальная разность хода в двухлучевых схемах}
    
    \subsection{Временная когерентность}
    
    \textbf{Опр} \textit{Временная когерентность}
    
    \textcolor{blue}{Мера корреляции (скоррелированность, согласованнотсь) фазы волны в разные моменты времени в одной и
    той же точке пространства.
    Она характеризует, насколько монохроматичным является источник света}
    
    \subsection{Время и длина когерентности}
    
    По тексту
    
    \subsection{Влияние немонохроматичности света на видность интерференционных полос}
    
    По тексту
    
    \subsection{Максимальный порядок интерференции и максимальная разность хода в двухлучевых схемах}
    
    По тексту
    
    \section{Магнитное поле в веществе.
    Магнитная индукция и напряжённость поля.
    Вектор намагниченности.
    Токи проводимости и молекулярные токи.
    Теорема о циркуляции магнитного поля в веществе.
    Граничные условия на границе двух магнетиков.
    Постоянные магниты}
    
    \subsection{Магнитное поле в веществе}
    
    Магнитное поле в веществе создаётся внешним полем и циркулирующими внутри микротоками
    
    \textbf{Опр} \textit{Микрополе}
    
    \subsection{Токи проводимости и молекулярные токи}
    
    \textbf{Опр} \textit{Токи проводимости}
    
    \textbf{Опр} \textit{Молекулярные токи}
    
    \subsection{Вектор намагниченности}
    
    \textbf{Опр} \textit{Вектор намагничивания (намагниченность)}
    
    Она может быть как однородной, так и неоднородной
    
    \textbf{Опр} \textit{Поверхностные токи} \\
    
    Свяжем молекулярные токи с вектором намагничивания.
    
    \begin{enumerate}
        \item Запишем магнитный момент вещества двумя способами
        \item Преобразуем, введём линейную плотность тока.
        \item Выберем в веществе произвольный замкнутый контур и возьмём тор вокруг него
        \item После преобразований и взятия интеграла получим, что
    \end{enumerate}
    
    \textbf{Утв} \textit{Молекулярный ток и вектор намагничивания связаны в интегральной форме}
    
    Применив теорему Стокса, получим, что
    
    \textbf{Утв} \textit{Молекулярный ток и вектор намагничивания связаны в дифференциальной форме}
    
    \subsection{Магнитная индукция и напряжённость поля}
    
    \textbf{Опр} \textit{Напряжённость поля}
    
    \subsection{Теорема о циркуляции магнитного поля в веществе}
    
    Используя связь молекулярного тока и вектор намагничивания в интегральной форме запишем теорему о циркуляции для
    магнитного поля и получим
    
    \textbf{Th} \textit{О циркуляции магнитного поля в веществе в интегральной форме}
    
    Используя дифференциальные формы справа и слева, можно получить
    
    \textbf{Th} \textit{О циркуляции магнитного поля в веществе в дифференциальной форме}
    
    \subsection{Граничные условия на границе двух магнетиков}
    
    Для $B_n$ воспользуемся теоремой Гаусса для магнитного поля \\
    Для $H_\tau$ воспользуемся теоремой о циркуляции \\
    Для $H_n$ представим $\tau$ через векторное произведение и с помощью преобразований получим требуемое
    
    \subsection{Постоянные магниты}
    
    \textbf{Опр} \textit{Постоянный магнит}
    
    \textcolor{blue}{Изделие из материала с высокой остаточной магнитной индукцией, сохраняющее состояние
    намагниченности в течение длительного времени}
    
    Можно посчитать поле на оси постоянного магнита
    
    \begin{enumerate}
        \item Разобъём магнит на колечки с молекулярными токами и, по сути, получим соленоид.
        \item Запишем $dB_x$ колечка через линейную плотность тока, а затем перейдя к параметризации по углу.
        \item Выразим $dx$ через тангенс угла и подставив в формулу получим приятное выражение для $dB_x$
        \item Взяв интеграл, окончательно найдём поле на оси постоянного магнита
        \item При желании можно вернуться к параметризации по координате
    \end{enumerate}
    
    Также при желании можно нарисовать график зависимости векторов $B$ и $H$ от $x$.
    Не лишним будет и показать картину силовых линий векторов внутри и вне магнита
    
    \section{Электромагнитная индукция.
    Поток магнитного поля.
    ЭДС индукции в движущихся и неподвижных проводниках.
    Вихревое электрическое поле.
    Правило Ленца.
    Закон электромагнитной индукции в интегральной и дифференциальной формах.
    Фарадеевская и максвелловская трактовка явления электромагнитной индукции}
    
    \subsection{Поток магнитного поля}
    
    \textbf{Опр} \textit{Поток магнитного поля}
    
    Это определение было известно нам и ранее
    
    \subsection{Электромагнитная индукция}
    
    Рассмотрим проводящую рамку, по которой течёт электрический ток.
    После детального анализа поймём, что там возникает ЭДС индукции, которая создаёт ток в отрицательном направлении
    обхода контура.
    
    \textbf{Опр} \textit{Электромагнитная индукция}
    
    \textcolor{blue}{Явление возникновения электрического тока, электрического поля или электрической поляризации при
    изменении магнитного поля во времени или при движении материальной среды в магнитном поле}
    
    \subsection{Правило Ленца}
    
    \textbf{Правило} \textit{Ленца}
    
    \subsection{ЭДС индукции в движущихся и неподвижных проводниках}
    
    \textbf{Утв} \textit{Правило Ленца верно как в движущихся, так и в неподвижных проводниках}
    
    \subsection{Закон электромагнитной индукции в интегральной и дифференциальной формах}
    
    \textbf{Закон} \textit{Электромагнитной индукции в интегральной форме}
    
    \textcolor{blue}{$\varepsilon_i = \int E_{out} dl$}
    
    Если преобразовать это выражение, то можно получить более частую формулировку этого закона
    
    \textbf{Закон} \textit{Электромагнитной индукции в дифференциальной форме}
    
    \subsection{Вихревое электрическое поле}
    
    Как видно из прошлого закона, $\rot E \neq 0$, поэтому индуцируемое электрическое поле является вихревым
    
    \subsection{Фарадеевская и максвелловская трактовка явления электромагнитной индукции}
    
    \textbf{Трактовка} \textit{Фарадея}
    
    \textbf{Трактовка} \textit{Максвелла}
    
    \section{Коэффициенты само- и взаимоиндукции.
    Теорема взаимности.
    Взаимная индуктивность двух катушек на общем магнитопроводе.
    Взаимная энергия токов.
    Локализация магнитной энергии в пространстве, объёмная плотность магнитной энергии}
    
    \subsection{Коэффициенты само- и взаимоиндукции}
    
    \textbf{Опр} \textit{Коэффициент самоиндукции (индуктивность)}
    
    \textbf{Опр} \textit{Коэффициент взаимоиндукции}
    
    \subsection{Взаимная индуктивность двух катушек на общем магнитопроводе}
    
    Для примера найдём индуктивность идеального соленоида
    
    А также взаимную индуктивность двух катушек на общем магнитопроводе
    
    Рассмотрев произвольный контур, можно найти формулу для
    
    \textbf{Утв} \textit{Магнитная энергия тока}
    
    \subsection{Теорема взаимности}
    
    \textbf{Th} \textit{Взаимности}
    
    \textcolor{blue}{Коэффициенты взаимоиндукции для фиксированной пары витков с током совпадают}
    
    \begin{enumerate}
        \item Посчитаем поток создаваемый на $i$-й виток всеми остальными при фиксированном геометрии системы.
        \item Посчитаем его дифференциал и подставим в дифференциал энергии системы токов.
        \item Возьмём две частные производные второго порядка в разных последовательностях и по теорему Шварца
        получим требуемое
    \end{enumerate}
    
    \subsection{Взаимная энергия токов}
    
    \textbf{Опр} \textit{Взаимная энергия токов}
    
    \textcolor{blue}{Энергия, с которой витки действуют друг на друга (включает коэффициент взаимондукции)}
    
    Если посчитать энергию системы токов, то две группы слагаемых её составляющих сольются в одну красивую сумму
    
    \subsection{Локализация магнитной энергии в пространстве, объёмная плотность магнитной энергии}
    
    \begin{enumerate}
        \item Рассмотрим соленоид и посчитаем его энергию, используя выражения для $H, d \Phi, B$.
        \item Получим выражение, из которого легко отделяется объём, что говорит о локализации магнитной энергии.
        \item Также это позволяет ввести понятие объёмной плотности магнитной энергии и три формулы для неё
        \item В более общем случае можно воспользоваться векторным анализом для вывода объёмной плотности энергии
    \end{enumerate}
    
    \section{Энергетический метод вычисления сил в магнитном поле.
    Вычисление сил при постоянном токе и потоке магнитного поля.
    Магнитные цепи.
    Подъёмная сила электромагнита}
    
    \subsection{Энергетический метод вычисления сил в магнитном поле}
    
    Известно: сила есть частная производная энергии по обобщённой координате
    
    $\delta A_{out} = dW + \delta A_{mech}$
    
    \subsection{Вычисление сил при постоянном токе и потоке магнитного поля}
    
    Для вычисления достаточно выразить $dW$ из выражения выше и продифференцировать
    
    \subsection{Магнитные цепи}
    
    \textbf{Опр} \textit{Магнитная цепь}
    
    \subsection{Подъёмная сила электромагнита}
    
    Если рассмотреть простую магнитную цепь, найти поле в её зазоре, а по нему $dW$ и силу, то можно явно показать
    наличие подъёмной силы у электромагнита
    
    \section{Магнитные свойства вещества.
    Качественные представления о механизме намагничивания пара- и диамагнетиков.
    Качественные представления о ферромагнетиках.
    Ферромагнитный гистерезис}
    
    \subsection{Магнитные свойства вещества}
    
    \textbf{Опр} \textit{Магнитная восприимчивость}
    
    \subsection{Качественные представления о механизме намагничивания пара- и диамагнетиков}
    
    \textbf{Опр} \textit{Парамагнетик}
    
    \textbf{Представление} \textit{Механизм намагничивания парамагнетика}
    
    \textbf{Опр} \textit{Намагниченность насыщения}
    
    \textbf{Закон} \textit{Кюри}
    
    \textbf{Опр} \textit{Постоянная Кюри} \\
    
    \textbf{Опр} \textit{Диамагнетик}
    
    \textbf{Представление} \textit{Механизм намагничивания диамагнетика}
    
    \textbf{Опр} \textit{Ларморовская частота}
    
    \subsection{Качественные представления о ферромагнетиках}
    
    \textbf{Опр} \textit{Ферромагнетик}
    
    \subsection{Ферромагнитный гистерезис}
    
    \textbf{Опр} \textit{Гистерезис}
    
    \textbf{Опр} \textit{Остаточная намагниченность}
    
    \textbf{Опр} \textit{Поле насыщения}
    
    \textbf{Опр} \textit{Коэрцитивная сила}
    
    \textbf{Опр} \textit{Точка Кюри}
    
    \textbf{Опр} \textit{Домен}
    
    \section{Магнитные свойства сверхпроводников I рода, эффект Мейсснера.
    Сверхпроводящий шар в магнитном поле.
    Метод изображений для сверхпроводников}
    
    \subsection{Магнитные свойства сверхпроводников I рода, эффект Мейсснера}
    
    \textbf{Опр} \textit{Сверхпроводимость}
    
    \textbf{Опр} \textit{Критическая температура}
    
    В отличие от сверхпроводников I рода, у сверхпроводников II рода существует смешанное состояние, при котором поле
    частично проникает в объём
    
    Таким образом, сверхпроводники можно назвать идеальными ($B = 0$) диамагнетиками
    
    \textbf{Эффект} \textit{Мейсснера}
    
    \subsection{Сверхпроводящий шар в магнитном поле}
    
    \begin{enumerate}
        \item Пользуясь тем, что поле внутри шара есть ноль, можно записать выражение для $B_n$ на внешней
        поверхности, откуда выразить магнитный момент
        \item Он совпадёт со случаем электростатики
        \item Аналогичный результат можно получить и для $B_\tau$
        \item При желании можно получить граничные условия для сверхпроводника и величину молекулярных токов $i$
    \end{enumerate}
    
    \subsection{Метод изображений для сверхпроводников}
    
    Поле от экранированных сверхпроводящих токов вне сверхпроводника всё равно что поле отражённого диполя (с тем же
    $\mathfrak{M}$)
    
    \section{Относительный характер электрического и магнитного полей.
    Сила Лоренца.
    Преобразование $E$ и $B$ при смене системы отсчёта (при $v \ll c$).
    Поле равномерно движущегося точечного заряда}
    
    \subsection{Относительный характер электрического и магнитного полей}
    
    При переходе из одной СО в другую, сила на частицу не меняется, поэтому происходит преобразование полей.
    Это свидетельствует об их относительности
    
    \subsection{Сила Лоренца}
    
    Таким образом сила Лоренца является инвариантом при переходе между СО
    
    \subsection{Поле равномерно движущегося точечного заряда}
    
    Используя БСЛ и выражение для $E$ можно получить поле равномерно движущегося точечного заряда
    
    \subsection{Преобразование $E$ и $B$ при смене системы отсчёта (при $v \ll c$)}
    
    Из равенства сил можно найти закон преобразования для $E$.
    Используя поле равномерно движущегося точечного заряда, можно найти $B$ в неподвижной СО, как и закон преобразования
    
    \section{Движение заряженных частиц в электрических и магнитных полях.
    Циклотронная частота и ларморовский радиус.
    Дрейф в скрещенных однородных полях}
    
    \subsection{Движение заряженных частиц в электрических и магнитных полях}
    
    Рассмотрим разные случаи.
    В однородном $E$ движение будет равноускоренным
    
    \subsection{Циклотронная частота и ларморовский радиус}
    
    В однородном $B$ разложим скорость по двум направлениям и получим равномерную циклоиду
    
    \textbf{Опр} \textit{Циклотронная частота}
    
    \textbf{Опр} \textit{Ларморовский радиус}
    
    В случае $E \| B$ имеем равноускоренную циклоиду
    
    \subsection{Дрейф в скрещенных однородных полях}
    
    В случае скрещенных полей перейдём в удобную СО (где $E = 0$) и найдём скорость этой СО
    
    \textbf{Опр} \textit{Дрейфовая скорость}
    
    \textcolor{blue}{Средняя скорость движения частиц, приобретаемая в результате воздействия электрического поля}
    
    Получим значения $E$ и $B$ в новой СО и проанализируем движение.
    Получается суперпозиция трёх движений: равномерной циклоиды (2) и дрейфа (+1)
    
    \section{Эффект Холла, влияние магнитного поля на проводящие свойства сред}
    
    Рассмотрим движение носителей заряда в $B_{out}$.
    
    \textbf{Эффект} \textit{Холла}
    
    Выясним связь между электрическим полем $E$ и плотностью тока $j$ в условиях эффекта Холла
    
    \textbf{Опр} \textit{Тензор проводимости}
    
    Определим его компоненты.
    Рассмотрим простейший случай: система содержит носители только одного типа (например, электроны), ток течет вдоль
    $Ox$, а магнитное поле направлено вдоль оси $Oz$.
    Магнитное поле действует на движущиеся заряды с силой $F_y = -q u_x B_z$ по оси $Oy$.
    Ток сможет течь строго вдоль $Ox$, если заряды в среде перераспределятся таким образом, чтобы компенсировать
    магнитную силу
    
    \textbf{Опр} \textit{Холловское электрическое поле}
    
    \textcolor{blue}{\[ E_y = u_x B_z = \frac{j_x}{nq} B_z \]}
    
    \textbf{Опр} \textit{Подвижность носителей тока}
    
    \textcolor{blue}{Коэффициент пропорциональности между дрейфовой скоростью носителей заряда и приложенным внешним
    электрическим полем}
    
    \textbf{Опр} \textit{Обобщённый закон Ома}
    
    Второе слагаемое в этой формуле как раз отвечает эффекту Холла - возникновению поперечного направлению тока $E$.
    Записывая закон покомпонентно, получим
    
    \textbf{Опр} \textit{Тензор удельного сопротивления}
    
    \textbf{Опр} \textit{Тензор проводимости в условиях эффекта Холла}
    
    \section{Магнитное действие переменного электрического поля.
    Ток смещения}
    
    \subsection{Ток смещения}
    
    Запишем теорему о циркуляции для магнитного поля и убедимся, что она работает не всегда
    
    \textbf{Опр} \textit{Ток смещения}
    
    Теперь запишем теоремы о циркуляции и Гаусса в новом виде
    
    \textbf{Опр} \textit{Теорема о циркуляции магнитного поля}
    
    \subsection{Магнитное действие переменного электрического поля}
    
    Таким образом, переменное электрическое поле приводит к возникновению токов смещения, которые в свою очередь
    участвуют в создании магнитного поля
    
    \section{Система уравнений Максвелла в интегральной и дифференциальной форме.
    Граничные условия.
    Материальные уравнения}
    
    \subsection{Система уравнений Максвелла в интегральной и дифференциальной форме}
    
    \textbf{Утв} \textcolor{blue}{Уравнения Максвелла состоят из ранее известных нам теорем}
    
    \textbf{Опр} \textit{Система уравнений Максвелла в дифференциальной форме}
    
    \textbf{Опр} \textit{Система уравнений Максвелла в интегральной форме}
    
    \textbf{Утв} \textcolor{blue}{У уравнений Максвелла есть словесные интерпретации}
    
    \subsection{Граничные условия}
    
    \textbf{Утв} \textcolor{blue}{У уравнений Максвелла есть всего 4 граничных условия (все они нам уже знакомы)}
    
    \subsection{Материальные уравнения}
    
    \textbf{Утв} \textcolor{blue}{Система уравнений Максвелла не полна.
    Но её можно дополнить материальными уравнениями}
    
    \textbf{Утв} \textcolor{blue}{Материальные уравнения}
    
    \section{Энергия переменного электромагнитного поля.
    Поток электромагнитной энергии, теорема Пойнтинга.
    Примеры применения теоремы Пойнтинга}
    
    \subsection{Энергия переменного электромагнитного поля}
    
    Получим ЗСЭ в пространстве с переменными полями
    
    \textbf{Утв} \textcolor{blue}{Энергия переменного электромагнитного поля есть сумма двух энергий (магнитного и
    электрического полей)}
    
    \subsection{Поток электромагнитной энергии, теорема Пойнтинга}
    
    По ходу преобразований получили выражение, которое удобно обозначить одним символом.
    Оно равно потоку электромагнитной энергии (энергия в единицу площади в единицу времени)
    
    \textbf{Опр} \textit{Вектор Пойнтинга}
    
    \textbf{Th} \textit{Пойнтинга в дифференциальной форме}
    
    Используя определение объёмной плотности энергии и теорему Остроградского -- Гаусса, получим
    
    \textbf{Th} \textit{Пойнтинга в интегральной форме}
    
    \subsection{Примеры применения теоремы Пойнтинга}
    
    \textbf{Утв} \textcolor{blue}{Теорему Пойнтинга можно применить для поиска поля внутри конденсатора}
    
    \textbf{Утв} \textcolor{blue}{Теорему Пойнтинга можно применить для поиска потока энергии втекающего в
    длинный провод (равнозначно, поиска джоулевых потерь)}
    
    \section{Квазистационарные электрические цепи, условие квазистационарности.
    Зарядка и разрядка конденсатора.
    Установление тока в катушке индуктивности.
    Интегрирующие и дифференцирующие цепочки}
    
    \subsection{Квазистационарные электрические цепи, условие квазистационарности}
    
    \textbf{Опр} \textit{Квазистационарная электрическая цепь}
    
    По цепи ЭМ сигнал распространяется со скоростью $c \ldots$
    
    \textbf{Утв} \textcolor{blue}{Условие квазистационарности цепи}
    
    \subsection{Зарядка и разрядка конденсатора}
    
    \textbf{Опр} \textit{Зарядка конденсатора}
    
    \textbf{Опр} \textit{Разрядка конденсатора}
    
    \subsection{Установление тока в катушке индуктивности}
    
    Запишем $\int E_L dl$ для неветвлёной цепи со всеми основными элементами.
    После рассмотрения каждого интеграла получим дополненное правило Кирхгофа (нам удалось подвязать к исходному
    катушку индуктивности)
    
    Используя полученное правило, рассмотрим процесс установления тока в катушке и найдём зависимость $I(t)$
    
    \subsection{Интегрирующие и дифференцирующие цепочки}
    
    \textbf{Опр} \textit{Интегрирующая цепочка}
    
    Рассмотрим две интегрирующие цепочки и обоснуем их название
    
    \textbf{Опр} \textit{Дифференцирующая цепочка}
    
    Рассмотрим две дифференцирующие цепочки и обоснуем их название
    
    \section{Свободные колебания в линейных системах. Колебательный RLC-контур.
    Коэффициент затухания, логарифмический декремент и добротность.
    Энергетический смысл добротности}
    
    \subsection{Свободные колебания в линейных системах. Колебательный RLC-контур}
    
    \begin{enumerate}
        \item Рассмотрим колебательный RLC-контур и запишем для него второе правило Кирхгофа+
        \item Пользуясь теорией о решении дифференциальных уравнений, запишем общее решение.
        \item Теперь рассмотрим три частных случая
    \end{enumerate}
    
    \textbf{Случай} \textit{Слабого затухания}
    
    \textbf{Случай} \textit{Сильного затухания в апериодическом режиме}
    
    \textbf{Случай} \textit{Критический режим колебаний}
    
    \subsection{Коэффициент затухания, логарифмический декремент и добротность}
    
    Рассмотрим затухание тока в цепи с течением времени
    
    \textbf{Опр} \textit{Коэффициент затухания}
    
    \textbf{Опр} \textit{Логарифмический декремент затухания}
    
    \textbf{Опр} \textit{Характерное время затухания}
    
    \textbf{Опр} \textit{Добротность}
    
    В RLC-контуре со слабым затуханием добротность имеет простую формулу
    В общем же случае можно показать, что в электрических цепях энергия со временем убывает по закону Джоуля-Ленца
    
    \subsection{Энергетический смысл добротности}
    
    Исходя из определения добротности, можно показать, что она показывает убыль энергии в системе за период колебаний
    
    \section{Вынужденные колебания под действием синусоидальной силы.
    Амплитудная и фазовая характеристики.
    Резонанс.
    Ширина резонанса и ее связь с добротностью.
    Процесс установления вынужденных колебаний, биения}
    
    \subsection{Вынужденные колебания под действием синусоидальной силы}
    
    Рассмотрим вынужденные колебания в RLC-контуре под действием синусоидального ЭДС генератора \ldots \\
    Решение $q(t)$ дифференциального уравнения легче все найти, предварительно перейдя в комплексную плоскость
    
    \subsection{Амплитудная и фазовая характеристики}
    
    \textbf{Опр} \textit{Амплитудная характеристика контура}
    
    \textbf{Опр} \textit{Фазовая характеристика контура}
    
    \subsection{Резонанс}
    
    \textbf{Опр} \textit{Резонанс}
    
    \textcolor{blue}{Резкое увеличении амплитуды колебаний при совпадении частоты внешнего воздействия с
    определённым значениям частоты}
    
    \subsection{Ширина резонанса и ее связь с добротностью}
    
    \begin{enumerate}
        \item Резонанс хорошо виден на графике зависимости $q(\omega)$
        \item Заметим, что $\frac{q_m}{q_0} = Q$
        \item Найдём разность частот (ширину колокола) на высоте $\frac{q_m}{\sqrt{2}}$.
        Для этого надо приравнять $(\Delta \omega)$ к $4 \omega_i \gamma^2$ (тогда из-под знаменателя и вылезет $\sqrt{2}$)
        \item Получим, что $Q = \frac{\omega_0}{\Delta \omega}$
    \end{enumerate}
    
    \subsection{Процесс установления вынужденных колебаний, биения}
    
    Сразу скажем, что в установившемся режиме колебания будут происходить на частоте вынуждающей силы (как в механике)
    
    \begin{enumerate}
        \item Рассмотрим решение $q(t)$ при разных значениях параметров
        \item В случае $\omega \neq \omega_0$ $\gamma = 0$ имеем произведение синусов
        \item Если строить график решения, то медленный синус будет огибающим, а быстрый будет <биться> между ним
        \item В случае малого $\gamma$ зазор каждый период будет только нарастать
        \item В случае большого $\gamma$ практически сразу наступит установившийся режим
        \item При $\omega = \omega_0$ рассмотрим соответсвующий предел и получим актуальное решение $q(t)$
        \item Если бы $\gamma = 0$, то амплитуда была бы бесконечной, но в реальности колебания будут ограничены
        огибающей
    \end{enumerate}
    
    \section{Установившиеся колебания в цепи переменного тока.
    Комплексная форма представления колебаний.
    Векторные диаграммы.
    Комплексное сопротивление (импеданс).
    Правила Кирхгофа для переменных токов.
    Работа и мощность переменного тока}
    
    \subsection{Установившиеся колебания в цепи переменного тока}
    
    В установившемся режиме колебания становятся вынужденными и происходят на частоте внешнего воздействия (возможно с
    некоторым сдвигом фаз)
    
    \subsection{Комплексная форма представления колебаний}
    
    \textbf{Метод} \textit{Комплексных амплитуд}
    
    \textbf{Опр} \textit{Комплексная амплитуда}
    
    \subsection{Векторные диаграммы}
    
    Метод комплексных амплитуд имеет геометрическую интерпретацию, которую проще всего показать с помощью векторных
    диаграмм
    
    \subsection{Комплексное сопротивление (импеданс)}
    
    \textbf{Опр} \textit{Импеданс}
    
    \textbf{Опр} \textit{Активное сопротивление}
    
    \textbf{Опр} \textit{Реактивное сопротивление}
    
    Найдём импеданс каждого элемента в RLC-контуре
    
    \subsection{Правила Кирхгофа для переменных токов}
    
    Запишем известную нам форму правил Кирхгофа.
    Подставим в него импедансы и сократим на экспоненту: получили новую форму записи
    
    \textbf{Правило} \textit{Кирхгофа I}
    
    \textbf{Правило} \textit{Кирхгофа II}
    
    \subsection{Работа и мощность переменного тока}
    
    \begin{enumerate}
        \item Запишем мгновенное значение мощности на резисторе.
        \item Посчитаем среднюю за период мощность (через интеграл)
        \item Другой способ получить то же самое -- через метод комплексных амплитуд (не забыв про сдвиг фаз между
        током и напряжением)
    \end{enumerate}
    
    \section{Спектральное разложение электрических сигналов.
    Спектр одиночного прямоугольного импульса и периодической последовательности импульсов.
    Вынужденные колебания под действием произвольной силы.
    Соотношение неопределённостей}
    
    \subsection{Спектральное разложение электрических сигналов}
    
    \textbf{Опр} \textit{Комплексная форма ряда Фурье периодической функции}
    
    \textbf{Опр} \textit{Коэффициент Фурье}
    
    Ряд Фурье можно обобщить и на случай непериодической функции, определённой на бесконечном временном интервале
    
    \textbf{Опр} \textit{Фурье-спектр}
    
    \subsection{Спектр одиночного прямоугольного импульса и периодической последовательности импульсов}
    
    \begin{enumerate}
        \item Посчитаем периодической последовательности импульсов.
        Для этого найдём $c_k$ коэффициент в разложении
        \item Поймём, какие же границы будут у интеграла от экспоненты и вычислим его
        \item Немного преобразований и коэффициент получен
        \item В случае спектра одиночного прямоугольного импульса ситуация проще: сначала считаем Фурье-спектр, а
        потом и сам интеграл
    \end{enumerate}
    
    \subsection{Вынужденные колебания под действием произвольной силы}
    
    Запишем второе правило Кирхгофа и введём линейный оператор \ldots
    
    \textbf{Утв} \textit{Принцип суперпозиции откликов}
    
    \textcolor{blue}{Отклик на суммарное воздействие равен сумме откликов на элементарные}
    
    Таким образом, любую функцию можно представить в виде суперпозиции гармонических, равно как и вычислить суммарное
    воздействие
    
    \subsection{Соотношение неопределённостей}
    
    \begin{enumerate}
        \item Рассмотрим одиночный прямоугольный спектр
        \item График его Фурье-спектра имеет вид колокола
        \item Если мы <сожмём> функцию спектра по $x$ в $A$ раз, то есть перейдём к функции $f_A(x) = f(Ax)$, то её
        спектр растянется во столько же раз: $F_A(k) = const \cdot F\left(\frac{k}{A}\right)$, поскольку частота каждой
        спектральной
        гармоники $e^{ikx}$ этого разложения должны будут, очевидно, умножиться на A
        \item Эта иллюстрация, строго говоря, хоть и носит довольно частный характер, однако она обнажает
        физический смысл иллюстрируемого свойства: когда мы сжимаем сигнал, его частоты во столько же раз увеличиваются.
        Другими словами, невозможно произвольно сконцентрировать как функцию, так и её преобразование Фурье
    \end{enumerate}
    
    \section{Спектральный анализ линейных систем.
    Частотная характеристика и импульсный отклик системы.
    Колебательный контур как спектральный прибор.
    Интегрирующая и дифференцирующая цепочки как высокочастотный и низкочастотный фильтры}
    
    \subsection{Спектральный анализ линейных систем. Колебательный контур как спектральный прибор}
    
    Рассмотрим RLC-контур и запишем для него II правило Кирхгофа и теорему Фурье.
    Если вынуждающая сила $\sim e^{i\omega t}$, то и отклик тоже; подставим в уравнение
    
    \textbf{Опр} \textit{Функция отклика}
    
    Благодаря ей получим
    
    \textbf{Утв} \textit{Связь между компонентами отклика и сигнала}
    
    \textbf{Опр} \textit{Импульсная (дельта) функция}
    
    \subsection{Частотная характеристика и импульсный отклик системы}
    
    Если источник был импульсным, то отклик тоже будет импульсным
    
    \textbf{Опр} \textit{Импульсный отклик системы}
    
    \textbf{Опр} \textit{Частотная характеристика контура}
    
    \subsection{Интегрирующая и дифференцирующая цепочки как высокочастотный и низкочастотный фильтры}
    
    Можно показать, что интегрирующая цепочка может служить фильтром высоких, а дифференцирующая -- низких частот
    
    \section{Модуляция и детектирование сигналов.
    Амплитудная и фазовая модуляции.
    Спектры гармонически модулированных по фазе и амплитуде сигналов.
    Квадратичное детектирование сигналов}
    
    \subsection{Модуляция и детектирование сигналов}
    
    \textbf{Опр} \textit{Модулированные колебания}
    
    Существует три типа модуляции
    
    \subsection{Амплитудная и фазовая модуляции}
    
    \textbf{Опр} \textit{Амплитудно-модулированный сигнал}
    
    \textbf{Опр} \textit{Фазово-модулированный сигнал}
    
    \subsection{Спектры гармонически модулированных по фазе и амплитуде сигналов}
    
    Если найти спектр амплитудно-модулированного сигнала, то он будет состоять из трёх гармоник
    
    Спектр фазово-модулированного сигнала тоже будет состоять из трёх гармоник, но немного других
    
    \subsection{Квадратичное детектирование сигналов}
    
    \textbf{Опр} \textit{Детектирование}
    
    Например, квадратичные детекторы усредняют квадрат функции по времени
    
    Применить детектирование можно как к АМ, так и к ФМ сигналу, как по определению, так и с помощью метода
    комплексных амплитуд
    
    \section{Электрические флуктуации.
    Тепловой шум.
    Тепловые флуктуации в колебательном контуре.
    Интенсивность теплового шума, формула Найквиста}
    
    \subsection{Электрические флуктуации. Тепловой шум}
    
    \textbf{Опр} \textit{Тепловой шум}
    
    Система без активного сопротивления теплового шума не производит
    
    \begin{enumerate}
        \item Пусть есть проводник с концентрацией электронов $n$ в объёме $V$.
        \item Усредним плотность тока, усреднив каждый её элемент, не забыв, что среднее от среднего есть ноль
        \item От плотности тока перейдём к флуктуации полного тока, используя теорию Друде
    \end{enumerate}
    
    \subsection{Тепловые флуктуации в колебательном контуре}
    
    Рассмотрев высокодобротный RLC-контур (с малым затуханием) найдём теряем за период энергию, а также, рассеивающую
    мощность.
    Полученная формула даёт оценку мощности, рассеиваемой в активном сопротивлении, но с другой стороны, точно такая
    же мощность затрачивается на создание флуктуации ЭДС.
    Следовательно, полученная формула даёт выражение для спектра тепловых шумов.
    
    \subsection{Интенсивность теплового шума, формула Найквиста}
    
    Используя формулы для мощности (две), можно получить две формулы Найквиста
    
    \textbf{Опр} \textit{Белый шум}
    
    \textcolor{blue}{Стационарный шум, спектральные составляющие которого равномерно распределены по всему диапазону
    задействованных частот}
    
    \section{Электрические флуктуации. Дробовой шум.
    Интенсивность дробового шума, закон $\sqrt{N}$, формула Шоттки}
    
    \subsection{Электрические флуктуации. Дробовой шум}
    
    \textbf{Опр} \textit{Дробовой шум}
    
    Посчитаем случайный ток от дробового шума
    
    \subsection{Интенсивность дробового шума, закон $\sqrt{N}$, формула Шоттки}
    
    \textbf{Опр} \textit{Интенсивность} \\
    
    Дальнейший рассказ лучше вести по билету Ральфа (страница 64) \\
    
    \textbf{Опр} \textit{формула Шоттки}
    
    \section{Параметрическое возбуждение колебаний.
    Условие параметрического резонанса}
    
    \subsection{Параметрическое возбуждение колебаний}
    
    \textbf{Опр} \textit{Параметрические колебательные системы}
    
    \textbf{Режим} \textit{Вариометра}
    
    \subsection{Условие параметрического резонанса}
    
    Посчитаем закачку и потери энергии за период
    
    \textbf{Условие} \textit{Возбуждения колебаний}
    
    \section{Автоколебания в электрических цепях.
    Положительная обратная связь.
    Условие самовозбуждения}
    
    \subsection{Автоколебания в электрических цепях}
    
    \textbf{Опр} \textit{Автоколебания}
    
    \subsection{Положительная обратная связь}
    
    \textbf{Опр} \textit{Обратная связь}
    
    \textbf{Опр} \textit{Положительная обратная связь}
    
    \textcolor{blue}{Тип обратной связи, при котором изменение выходного сигнала системы приводит к такому изменению
    входного сигнала, которое способствует дальнейшему отклонению выходного сигнала от первоначального значения, то
    есть знак изменения сигнала обратной связи совпадает со знаком изменения входного сигнала}
    
    \textbf{Опр} \textit{Диод}
    
    \textbf{Опр} \textit{Триод}
    
    \textbf{Опр} \textit{Усилитель}
    
    \subsection{Условие самовозбуждения}
    
    \begin{enumerate}
        \item Рассмотрим колебания под действием периодических \(''\)толчков\(''\).
        \item Построим график, изображающие действия толчков.
        \item Соорудим генератор с обратной связью, построим графики заряда, тока и толчков (внешнего напряжения)
        \item Запишем уравнение колебаний и решим его с помощью новых замен
    \end{enumerate}
    
    \textbf{Условие} \textit{Самовозбуждения}
    
    \section{Волновое уравнение как следствие уравнений Максвелла.
    Электромагнитные волны в однородном диэлектрике, их поперечность и скорость распространения}
    
    \subsection{Волновое уравнение как следствие уравнений Максвелла}
    
    Запишем уравнения Максвелла в частном случае.
    Посчитаем ротор ротора двумя способами и получим два волновых уравнения
    
    \subsection{Электромагнитные волны в однородном диэлектрике, их поперечность и скорость распространения}
    
    \textbf{Утв} \textit{$\overrightarrow{k}$}
    
    \textcolor{blue}{Единичный вектор в направлении распространения волны}
    
    Можно показать, что
    
    \textbf{Утв} \textit{Поперечность электромагнитных волн}
    
    \textcolor{blue}{Электромагнитные волны $\bot E, H$}
    
    \textbf{Утв} \textit{Векторы $k, E$ и $H$ образуют правую тройку}
    
    \textbf{Утв} \textit{Поперечность электромагнитных волн}
    
    Исходя из анализа размерности, можно найти скорость распространения волны в пространстве
    
    \textbf{Опр} \textit{Волновое сопротивление}
    
    \section{Монохроматические волны.
    Комплексная амплитуда волны.
    Плоская электромагнитная волна.
    Приближение сферической волны.
    Связь полей $E$ и $B$ в плоской электромагнитной волне.
    Стоячие и бегущие волны.
    Отражение волн от идеального проводника}
    
    \subsection{Монохроматические волны}
    
    \textbf{Опр} \textit{Монохроматическая волна}
    
    \textbf{Опр} \textit{Волновое число}
    
    \subsection{Комплексная амплитуда волны}
    
    Обычную волну можно записать с помощью метода комплексных амплитуд
    
    \textbf{Опр} \textit{Комплексная амплитуда}
    
    \subsection{Плоская электромагнитная волна}
    
    \textbf{Утв} \textit{Уравнение плоской электромагнитной волны}
    
    \textbf{Опр} \textit{Фазовая скорость}
    
    \subsection{Приближение сферической волны}
    
    Если рассмотреть лапласиан сферических координат и записать волновое уравнение с его решением, то можно увидеть,
    что амплитуда убывает обратно пропорционально расстоянию
    
    \subsection{Связь полей $E$ и $B$ в плоской электромагнитной волне}
    
    Через волновое сопротивление можно показать, что $E \bot B$, а также, формульную связь
    
    \subsection{Стоячие и бегущие волны}
    
    До этого мы рассматривали только бегущие волны, но если сложить бегущие в противоположном направлении волны, то
    можно получить уравнение стоячей волны
    
    \subsection{Отражение волн от идеального проводника}
    
    Дальнейший рассказ лучше вести по консультации А.~В.~Гаврикова (билеты 35, 39) \\
    
    Граничные условия для $E$ нам известны и очевидны (внутри проводника поля нет).
    Можно записать граничные условия и для магнитного поля.
    
    \section{Поток энергии в электромагнитной волне.
    Давление излучения.
    Электромагнитный импульс}
    
    \subsection{Поток энергии в электромагнитной волне}
    
    Можно показать, что вектор Пойнтинга равен по модулю и задаёт направление потоку энергии в электромагнитной волне
    
    \textbf{Опр} \textit{Интенсивность излучения}
    
    \subsection{Электромагнитный импульс}
    
    \textbf{Опр} \textit{Электромагнитный импульс}
    
    \textcolor{blue}{Возмущение ЭМ поля, оказывающее влияние на любой материальный объект, находящийся в зоне действия}
    
    Таким импульс обладает и поле в вакууме
    
    \textbf{Опр} \textit{Плотность импульса}
    
    \subsection{Давление излучения}

%    \textbf{Опр} \textit{Объёмная плотность силы}
    
    Посчитать давление излучения, которое оно оказывает на среду при поглощении, можно, используя граничные условия.
    Можно найти давление волны на поверхность как идеального, так и любого другого проводника
    
    \section{Электромагнитные волны на границе раздела двух диэлектриков.
    Формулы Френеля.
    Явление Брюстера.
    Полное внутреннее отражение}
    
    \subsection{Электромагнитные волны на границе раздела двух диэлектриков}
    
    Рассмотрим падающую волну и запишем для неё граничные условия.
    Получим, два уравнения (на $\omega$ и $\overrightarrow{k}$), из которых следует равенства углов падения и
    отражения и закон Снеллиуса
    
    \subsection{Полное внутреннее отражение}
    
    \textbf{Опр} \textit{Угол полного внутреннего отражения}
    
    \subsection{Формулы Френеля}
    
    \textbf{Опр} \textit{Коэффициент отражения}
    
    \textbf{Опр} \textit{Коэффициент преломления}
    
    \textbf{Опр} \textit{$s-$поляризация}
    
    \textbf{Опр} \textit{$p-$поляризация}
    
    Записав граничные условия для каждой поляризации, найдём формулы Френеля
    
    \subsection{Явление Брюстера}
    
    В случае $p-$поляризация возможно явление (эффект) Брюстера, который имеет место для любой среды
    
    \textbf{Опр} \textit{Угол Брюстера}
    
    \section{Излучение электромагнитных волн.
    Зависимость интенсивности дипольного излучения от частоты, диаграмма направленности}
    
    \subsection{Излучение электромагнитных волн}
    
    Дальнейший рассказ можно вести по консультации А.~В.~Гаврикова (билет 40) \\
    
    \textbf{Опр} \textit{Излучение}
    
    Рассмотрим волну, создаваемую точечным диполем в вакууме
    
    \subsection{Зависимость интенсивности дипольного излучения от частоты, диаграмма направленности}
    
    Посчитаем мощность излучения и получим
    
    \textbf{Утв} \textit{Угловое распределение излучения (диаграмма направленности)}
    
    \textbf{Закон} \textit{Рэлея}
    
    Таким образом, поле ускоренно движущегося заряда зависит от промежутка времени до, угла и обратно расстоянию
    
    \section{Линии передачи энергии.
    Двухпроводная линия, коаксиальный кабель.
    Скорость волны, волновое сопротивление.
    Коэффициент стоячей волны.
    Согласованная нагрузка}
    
    \subsection{Линии передачи энергии}
    
    \textbf{Опр} \textit{Линии передачи энергии}
    
    Чаще всего такие линии неквазистационарны
    
    \subsection{Двухпроводная линия, коаксиальный кабель}
    
    \textbf{Опр} \textit{Двухпроводная линия}
    
    \textbf{Опр} \textit{Телеграфные уравнения}
    
    \textbf{Опр} \textit{Фазовая скорость}
    
    Найдём её для двух примеров длинной линии: коаксиального кабеля и двухпроводной линии
    
    \textbf{Опр} \textit{Волновое уравнение сигнала}
    
    У этого уравнения, как и любого другого дифференциального, есть свои граничные условия
    
    \subsection{Скорость волны, волновое сопротивление}
    
    \textbf{Опр} \textit{Скорость волны}
    
    \textcolor{blue}{Скорость распространения возмущения, не может превышать световую и по сути являестя фазовой}
    
    Найдём волновое сопротивление, рассмотрев бегущую волну в двухпроводной линии
    
    \textbf{Опр} \textit{Волновое сопротивление}
    
    \subsection{Согласованная нагрузка}
    
    \textbf{Опр} \textit{Волновая загрузка}
    
    \textcolor{blue}{Такая нагрузка, которая поглощает всю энергию волны (равнозначна отсутствию отражения)}
    
    \subsection{Коэффициент стоячей волны}
    
    \textbf{Опр} \textit{Коэффициент стоячей волны}
    
    Его связь с волновым сопротивлением можно попробовать показать через $E_{fall} + E_{refl}$
    
    \textbf{Опр} \textit{Коэффициент бегущей волны}
    
    \section{Электромагнитные волны в прямоугольном волноводе.
    Простейшие типы волн в волноводе прямоугольного сечения.
    Дисперсионное уравнение, критическая частота, длина волны и фазовая скорость волны в волноводе.
    Объёмные электромагнитные резонаторы}
    
    \subsection{Электромагнитные волны в прямоугольном волноводе}
    
    \textbf{Опр} \textit{Волновод}
    
    Записав упрощённые уравнения Максвелла и граничные условия к ним, получим решение в виде монохроматической волны
    
    \textbf{Уравнение} \textit{Гельмгольца}
    
    \subsection{Простейшие типы волн в волноводе прямоугольного сечения}
    
    Граничным условиям уравнения Гельмгольца не могут быть выполнены одновременно, но могут быть по отдельности.
    Поэтому в волноводе существует несколько типов волн
    
    \textbf{Опр} \textit{$H-$волна}
    
    \textbf{Опр} \textit{$E-$волна}
    
    \textbf{Опр} \textit{Псевдопоперечная волна}
    
    Решение волнового уравнения возможно лишь при некоторых значениях $\gamma_\lambda$
    
    \subsection{Дисперсионное уравнение, критическая частота, длина волны и фазовая скорость волны в волноводе}
    
    \textbf{Опр} \textit{Дисперсионное уравнение}
    
    \textbf{Опр} \textit{Критическая частота}
    
    Можно показать, что, например, в простейшем случае прямоугольного волновода она будет связана с его геометрией
    
    \textbf{Опр} \textit{Фазовая скорость}
    
    \textbf{Опр} \textit{Групповая скорость}
    
    \subsection{Объёмные электромагнитные резонаторы}
    
    \textbf{Опр} \textit{Резонатор}
    
    По сути, это волновод с закрытыми входами и выходами (внутри стоячая волна)
    
    \textbf{Опр} \textit{Объёмный резонатор}
    
    Спектр волн в таком резонатор будет дополнен
    
    \section{Квазистационарное проникновение электрического и магнитного полей в проводящую среду. Скин-эффект.
    Глубина скин-слоя для постоянного и переменного полей}
    
    \subsection{Квазистационарное проникновение электрического и магнитного полей в проводящую среду. Скин-эффект}
    
    \textbf{Опр} \textit{Скин-эффект}
    
    Рассмотрим распространение переменного $E$ в проводнике и получим дифференциальное уравнение.
    
    \subsection{Глубина скин-слоя для постоянного и переменного полей}
    
    Найдём распределение тока по объёму проводника
    
    \textbf{Опр} \textit{Глубина скин-слоя}
    
    В случае постоянного тока ($\omega = 0$) ток течёт по всему сечению проводника
    
    \section{Плазма.
    Дебаевский радиус экранирования.
    Плазменные колебания, плазменная частота}
    
    \subsection{Плазма}
    
    \textbf{Опр} \textit{Плазма}
    
    \textbf{Опр} \textit{Квазинейтральная плазма}
    
    \subsection{Дебаевский радиус экранирования}
    
    Рассмотрим экранирование заряда в плазме
    
    \textbf{Утв} \textit{Потенциал заряда в плазме}
    
    Оценим масштаб, на котором может происходить сильное разделение зарядов
    
    \textbf{Опр} \textit{Дебаевский радиус}
    
    \subsection{Плазменные колебания, плазменная частота}
    
    \textbf{Опр} \textit{Плазменная частота}
    
    \textbf{Опр} \textit{Плазменные колебания}
    
    \section{Диэлектрическая проницаемость холодной плазмы.
    Проникновение электромагнитных волн в плазму}
    
    \subsection{Диэлектрическая проницаемость холодной плазмы}
    
    Рассмотрим движение электрона в холодной плазме
    
    \textbf{Утв} \textit{Диэлектрическая проницаемость холодной плазмы}
    
    \subsection{Проникновение электромагнитных волн в плазму}
    
    Рассмотрим плоскую волну, распространяющую в плазме
    
    \textbf{Утв} \textit{В плазме могут распространяться только волны с частотой больше плазменной}

\end{document}
