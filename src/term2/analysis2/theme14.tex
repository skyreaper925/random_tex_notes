\subsection{Степенные ряды}

\textbf{Опр} \textit{Предел последовательности комплексных чисел} \textcolor{gray}{Предел модуля разности равен
нулю}

Заметим, что комплексный предел эквивалентен двум вещественным (для действительной и мнимой части)

\textbf{Опр} \textit{Сходящийся комплексный ряд} \textcolor{gray}{Существует конечный предел последовательности
частичных сумм этого ряда}

\textbf{Опр} \textit{Абсолютно сходящийся комплексный ряд} \textcolor{gray}{Сходится вещественный ряд модулей
членов ряда}

И вновь сходимость комплексного ряда эквивалентна сходимости двух вещественных рядов

\textbf{Опр} \textit{Равномерно сходящийся комплекснозначная функциональная последовательность} \textcolor{gray}{
    Вещественнозначная последовательность модулей разности предельной функции и элементов последовательности
    равномерно сходится к нулю на том же множестве}

\textbf{Опр} \textit{Равномерно сходящийся комплексный функциональный ряд} \textcolor{gray}{Последовательность
частичных сумм этого ряда равномерно сходится к сумме этого ряда на том же множестве}

\textbf{Опр} \textit{Степенной ряд} \textcolor{gray}{Если задана последовательность комплексных чисел
и комплексное число, то ...}

Однако удобнее (и мы в дальнейшем будем так делать) работать с рядом без степенной разности, сделав замену
комплексной переменной

\subsection{Формула Коши-Адамара для радиуса сходимости}

\textbf{Опр} \textit{Радиус сходимости степенного ряда} \textcolor{gray}{Неотрицательное число (или бесконечность
    ), определяемое формулой Коши-Адамара}

Притом для этой формулы мы расширили операцию деления

\subsection{Теорема о круге сходимости степенного ряда}

\textbf{Опр} \textit{Круг сходимости степенного ряда} \textcolor{gray}{Круг на комплексной плоскости с центром
в $w_0 (0)$ и радиусом равным радиусу сходимости}

Если радиус сходимости бесконечен, то кругом сходимости считается вся комплексная плоскость

\textbf{Th} \textit{О круге сходимости}

\textcolor{blue}{Степенной ряд абсолютно сходится внутри круга сходимости и расходится вне его}

\begin{enumerate}
    \item Зафиксируем произвольное комплексное число $z_0 \neq 0$, обозначим $q = \frac{z_0}{R}$ и исследуем
    сходимость с помощью обобщённого признака Коши
    \item В тривиально случае $z_0 = 0$ ряд сходится абсолютно
    \item В случае $0 < \abs{z_0} < R$ в силу обобщённого признака Коши ряд сходится абсолютно
    \item В случае $\abs{z_0} > R$ в силу обобщённого признака Коши члены абсолютного ряда не стремятся к нулю,
    как и исходного ряда, а значит, он расходится по отрицанию необходимого условия
\end{enumerate}

\subsection{Первая теорема Абеля}

\textbf{Th} \textit{Первая теорема Абеля}

\textcolor{blue}{Если степенной ряд сходится в точке $z_0$, то он сходится абсолюто в любой точке по модулю
меньшей}

Доказательство следует от противного в силу п.4 теоремы о круге сходимости

\subsection{Теорема о равномерной сходимости степенного ряда}

\textbf{Th} \textit{О равномерной сходимости степенного ряда}

\textcolor{blue}{$\forall r \in (0, R)$ ряд $\sum_{\mathbb{N}}_0 c_k z^k$ сходится равномерно в круге радиуса $r$}

Доказывается через неравенство, применением теоремы о круге сходимости и по признаку Вейерштрасса равномерной
сходимости комплексного ряда

\begin{enumerate}
    \item Зафиксируем произвольное комплексное число $z_0 \neq 0$, обозначим $q = \frac{z_0}{R}$ и исследуем
    сходимость с помощью обобщённого признака Коши
    \item В тривиально случае $z_0 = 0$ ряд сходится абсолютно
    \item В случае $0 < \abs{z_0} < R$ в силу обобщённого признака Коши ряд сходится абсолютно
    \item В случае $\abs{z_0} > R$ в силу обобщённого признака Коши члены абсолютного ряда не стремятся к нулю,
    как и исходного ряда, а значит, он расходится по отрицанию необходимого условия
\end{enumerate}

\subsection{Вторая теорема Абеля}

\textbf{Th} \textit{Вторая теорема Абеля}

\textcolor{blue}{Если степенной ряд сходится в точке $z_0$, то он сходится равномерно на отрезке $[0, z_0]$}

\begin{enumerate}
    \item Разобьём члены ряда на произведение членов произведения с помощью параметра $t \in [0, 1]$
    \item Первый ряд сходится по условию (а значит, по предыдущей теореме, ещё и равномерно)
    \item Второй ряд равномерно ограничен на отрезке и монотонен по индексу
    \item Поэтому два вещественных ряда сходятся равномерно на $[0, 1]$, как и исходный ряд на $[0, z_0]$
\end{enumerate}

\subsection{Сохранение радиуса сходимости при почленном дифференцировании степенного ряда}

\textbf{Th} \textcolor{blue}{Радиусы сходимости степенных рядов, полученные формальным дифференцированием и
интегрированием исходного, совпадают с его радиусом сходимости}

\begin{enumerate}
    \item Радиусы сходимости исходного и продифференцированного рядов совпадают в силу формулы Коши-Адамара
    \item Также они сходятся или расходятся одновременно, потому как при $z = 0$ это очевидно, а в противном
    случае они отличаются на ненулевую константу (как и их пределы)
    \item Так как исходный ряд получается почленным дифференцированием интегрального, то и их радиусы сходимости
    совпадают
\end{enumerate}

\subsection{Теоремы о почленном интегрировании и дифференцировании степенного ряда}

\textbf{Th} \textit{Об интегрировании и дифференцировании степенного ряда}

\textcolor{blue}{Если вещественный степенной ряд имеет ненулевой радиус сходимости, то внутри интервала
сходимости
    \begin{itemize}
        \item справедливы формулы почленного интегрирования
        \item функция ряда имеет производные любого порядка, получаемые почленным дифференцированием ряда
        \item коэффициенты степенного ряда однозначно определяются по обрывку формулы Тейлора
    \end{itemize}   }

\begin{enumerate}
    \item Для почленного интегрирования достаточно ввести новую переменную и воспользоваться теоремами о
    равномерной сходимости степенного ряда и о почленном интегрировании равномерно сходящегося функционального ряда
    \item Для производных достаточно ввести новую переменную и воспользоваться теоремами о сохранении радиуса
    сходимости, о равномерной сходимости степенного ряда и о почленном дифференцировании функционального ряда
    \item Проводя те же рассуждения по индукции, доказываем второе утверждение теоремы
    \item Доказывается аналогично лемме первого семестра перед формулой Тейлора
\end{enumerate}

\subsection{Единственность разложения функции в степенной ряд, ряд Тейлора}

\textbf{Опр} \textit{Бесконечно дифференцируемая функция в точке} \textcolor{gray}{В этой точке существуют
производные функции любого порядка}

\textbf{Опр} \textit{Ряд Тейлора} \textcolor{gray}{Ряд бесконечно дифференцруемой функции в точке с членами ...}

\textbf{Опр} \textit{Регулярная функция в точке $z_0$} \textcolor{gray}{Ряд Тейлора функции в точке $z_0$
    сходится к функции в некоторой окрестности $z_0$}

Из теоремы об интегрировании и дифференцировании степенного ряда следует, что если функция может быть
представлена как сумма степенного ряда $\sum_{\mathbb{N}_0} a_k (z - z_0)^k$ с ненулевым радиусом сходимости, то
этот ряд является рядом Тейлора функции в точке $z_0$.
В этом случае функция является регулярной в точке $z_0$

\textbf{Опр} \textit{Остаточный член формулы Тейлора} \textcolor{gray}{Разность $n$ раз дифференцируемой функции
и формулы Тейлора}

Непосредственно из определений следует, что функция является регулярной в точке
$\Leftrightarrow \lim_{n\to\infty} r_n(x) = 0$.
Притом для доказательства регулярности недостаточно показать ненулевой радиус сходимости функции, надо ещё проверить
её остаток

\subsection{Достаточное условие аналитичности функции}

\textbf{Th} \textit{Достаточное условие регулярности}

\textcolor{blue}{Если $\exists U_\delta (x_0)$, где функция бесконечно дифференцируема и последовательность её
производных равномерно ограничена константой $C > 0$, то функция регулярна в точке и $\forall x \in U_\delta (x_0)$
    раскладывается в ряд Тейлора}

\begin{enumerate}
    \item Применим формулу Тейлора с остаточным членом в форме Лагранжа.
    Тогда остаточный член формулы Тейлора $\leq M \frac{\delta^{n+1}}{(n+1)!}$
    \item Так как факториал растёт быстрее показательной (доказывается через принцип Архимеда, определение
    факториала, цепочку неравенств и предельный переход), то остаточный член стремится к нулю
    \item Поэтому функция регулярна, потому как раскладывается в ряд Тейлора в $x_0$
\end{enumerate}

\subsection{Пример бесконечно дифференцируемой, но неаналитической функции}

\begin{equation}
    f(x) =
    \begin{cases}
        e^{-\frac{1}{x^2}}, x \neq 0; \\
        0, x = 0.
    \end{cases}
\end{equation}

Ряд Тейлора этой бесконечно дифференцируемой в точке $x_0 = 0$ сходится не к функции $f(x)$, а к
некоторой другой функции, не совпадающей с $f(x)$ в сколь угодно малой окрестности точки

\[ \forall k \in \mathbb{N} \lim_{x \to 0} \frac{1}{x^k} e^{-\frac{1}{x^2}} = \lim_{t \to +\infty} t^{\frac{k}{2}} e^{-t} = 0 \]

По индукции легко показать, что если $P_{3n} (t)$ -- многочлен степени $3n$ от $t$, то

\begin{equation}
    f^{(n)}(x) =
    \begin{cases}
        P_{3n} (\frac{1}{x}) e^{-\frac{1}{x^2}}, x \neq 0; \\
        0, x = 0.
    \end{cases}
\end{equation}

Следовательно, все коэффициенты ряда Тейлора функции $f(x)$ в точке $x_0 = 0$ равны нулю.
Поэтому сумма ряда Тейлора функции $f(x)$ в точке $x_0$ равна нулю и не совпадает с функцией $f(x)$ в сколь угодно
малой окрестности точки $x_0$.
Таким образом, хотя функция и бесконечно дифференцируема, она не является регулярной в нуле

\subsection{Представление экспоненты комплексного аргумента степенным рядом}

\textbf{Опр} \textit{Ряд Маклорена} \textcolor{gray}{Ряд Тейлора функции в нуле}

\textbf{Th.1} \textcolor{blue}{Ряды маклорена функций $e^x, \sin(x), \cos(x), \sh(x), \ch(x)$ сходятся к этим
функциям на всей числовой прямой}

\begin{enumerate}
    \item $\forall \delta > 0~\forall x \in U_\delta (0)~e^x < e^\delta$, поэтому выполнено достаточное условие
    регулярности
    \item Аналогично, используя ограниченность последовательности всех производных оставшихся функций доказываем
    их разложения
\end{enumerate}

\textbf{Th.2} \textcolor{blue}{Для комплексной экспоненты её ряд Тейлора не отличается от вещественного}

\begin{enumerate}
    \item В силу предыдущей теоремы радиус сходимости степенного ряда-претендента сходится на всём $\mathbb{C}$,
    поэтому по теореме о круге сходимости он сходится абсолютно для любого $z \in \mathbb{C}$
    \item Зафиксируем произвольное комплексное число в алгебраической форме и воспользуемся определением
    экспоненты комплексного числа, чтобы зафиксировать доказываемое равенство
    \item Покажем, что функция-ряд-претендент обладает свойством экспоненты.
    Для этого воспользуемся теоремой о перемножении абсолютно сходящихся рядов, которая для комплексных рядов
    доказывается точно так же, как и для вещественных (только здесь надо использовать метод \("\)диагоналей\("\))
    \item В результате преобразований получим сумму сумм, которую распределим по этим суммам, и применим формулу
    бинома Ньютона, завершив доказательство свойства
    \item Далее рассмотрим функцию кандидат на чисто мнимом аргументе и путём разложения на чётную и нечётную
    суммы получим выражение для чисто мнимой экспоненты
    \item В итоге, применив свойство экспоненты и убедившись, что функция работает на вещественных аргументах,
    получим разложение комплексной экспоненты в ряд Тейлора в силу единственности
\end{enumerate}

\subsection{Формулы Эйлера}

\textbf{Лемма} \textcolor{blue}{Для любого $z \in \mathbb{C}$ справедливы формулы Эйлера} \textcolor{gray}{Они
используют новопостроенные комплексные функции и подравнивают комплексную тригонометрию к вещественной гиперболике}

\begin{enumerate}
    \item Для доказательства формулы гиперкомплексной экспоненты достаточно разделить сумм на чётную и нечётную,
    а затем воспользоваться $i^2 = -1$
    \item Остальные формулы следуют из первой
\end{enumerate}

\subsection{Формула Тейлора с остаточным членом в интегральной форме}

\textbf{Th} \textit{Формула Тейлора с остаточным членом в интегральной форме}

\textcolor{blue}{Если функция в $U_\delta (x_0)$ имеет непрерывные производные по $n+1$ порядок, то для
остаточного члена формулы Тейлора справедливо представление в интегральной форме: $r_n (x) = \frac{1}{n!} \int_{
    x_0}^x (x - t)^n f^{n+1}(t)dt \forall x \in U_\delta (x_0) $}

\begin{enumerate}
    \item При $n = 0$ теорема справедлива в силу формулы Ньютона -- Лейбница
    \item Пусть теорема справедлива для $n = s - 1$.
    Тогда проинтегрируем $r_{s-1}$ по частям
    \item Затем, расписав $r_s$ по определению, подставим проинтегрированное выражение и получим требуемое равенство
    \item Таким образом, теорема доказана по индукции
\end{enumerate}

\subsection{Представление степенной и логарифмической функций степенными рядами}

\textbf{Th} \textcolor{blue}{Ряд Маклорена степенной функции сходится к этой функции на интервале единичного радиуса}

\begin{enumerate}
    \item Зафиксируем $x \in (-1; 1)$ и учитывая выражение для $f^{n}$ распишем остаточный член в интегральной
    форме, походу дела вынося константы, вводя новые обозначения и переменные интегрирования
    \item Затем воспользуемся ограниченностью $x$ для оценки.
    Осталось показать, что $\lambda_n \rightarrow 0$
    \item В тривиальных случаях $x = 0$ и $\alpha = m \in \mathbb{N}_0, m < n$ утверждение очевидно
    \item В общем случае найдём предел отношения и воспользуемся схожими рассуждениями с доказательством признака
    Даламбера (сравнение с геометрической прогрессией)
\end{enumerate}

Заметим, что при $m \geq n$ ряд Маклорена совпадает с конечной суммой \\

Из доказанного и теоремы о почленном интегрировании степенного ряда при $\abs{x} < 1$ (не забывая про замену
индекса суммирования) получаем ряд Маклорена для логарифма.
Данное разложение справедливо и при $x = 1$.
Действительно, данный ряд будет сходиться по признаку Лейбница.
Следовательно, в силу второй теоремы Абеля этот ряд сходится равномерно на отрезке $[0; 1]$.
Согласно теореме о непрерывности суммы равномерно сходящегося функционального ряда частичные суммы этого ряда будет
непрерывны на отрезке $[0; 1]$.
Поэтому существует требуемый предел