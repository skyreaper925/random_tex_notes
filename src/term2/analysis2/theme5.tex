\subsection{Первообразная и неопределенный интеграл}

\textbf{Опр} \textit{Первообразная} \textcolor{gray}{Функция, производная которой есть наша}

\textbf{Л1} \textcolor{blue}{Производная константы есть ноль}

\begin{enumerate}
    \item Зафиксируем произвольную точку $x_0$ и обозначим $C = f(x_0)$
    \item Применим теорему Лагранжа о среднем и внимательно посмотрим на числитель
\end{enumerate}

\textbf{Th.1} \textit{О структуре множества первообразных} \textcolor{gray}{Критерий первообразной: все
первообразные функции отличаются на константу}

\begin{enumerate}
    \item В одну сторону очевидно (достаточно всего лишь продифференцировать)
    \item В другую запишем разность производных первообразных и применим Л1
\end{enumerate}

\textbf{Опр} \textit{Неопределённый интеграл} \textcolor{gray}{Множество всех первообразных}

\textbf{Th.2} \textit{Обозначение и краткое обозначение неопределённого интеграла} \textcolor{gray}{Крючок равен
множетсву из $F(x) + C$} \\

Важно понимать, что неопределенный интеграл – это не одна функция, а множество функций.
Иначе говоря, константа $C$ , стоящая в правой части последней формулы, – не фиксированная константа, а параметр,
пробегающий $\mathbb{R}$ \\

\textbf{Л2} \textcolor{blue}{Операция взятия дифференциала и операция взятия неопределенного интеграла являются
взаимно обратными} \textcolor{gray}{А также верны два тождества для взаимно-обратных функций}

\begin{enumerate}
    \item Надо воспользоваться определением неопределённого интеграла и подставить его \("\)формулировку\("\)
    \item Надо обозначить $f = F^{'}$ и повторить рассуждения п.1
\end{enumerate}

\subsection{Линейность неопределенного интеграла, замена переменных и интегрирование по частям}

\textbf{Лемма} \textit{Свойство линейности неопределенного интеграла} \textcolor{gray}{Для доказательства достаточно
вспонимть определения первообразной, неопределенного интеграла и воспользоваться линейнойстью производной} \\

\textbf{Th.1} \textit{Замена переменной или метод интегрирования подстановкой} \textcolor{gray}{Для доказательства
достаточно воспользоваться инвариантностью (дифференциалы простой функции и сложной функции могут быть записаны в
одной и той же форме) первого дифференицала} \\

\textbf{Th.2} \textit{Метод интегрирования по частям} \textcolor{gray}{Для доказательства достаточно
воспользоваться формулой производной произведения и Л2}

\subsection{Интегрирование рациональных функций}

Алгоритм интегрирования рациональной дроби

\begin{enumerate}
    \item При необходимости, методом деления многочлена в столбик представить дробь в виде суммы многочлена и
    правильной рациональной дроби
    \item Найти корни знаменателя и разложить знаменатель на элементарные множители, а затем, методом неопределенных коэффициентов разложить правильную
    рациональную дробь в сумму элементарных дробей (это разложение существует и единственно)
    \item Проинтегрировать элементарные дроби и многочлен
\end{enumerate}

Методы интегрирования элементарных дробей

\begin{itemize}
    \item Интегралы вида $\int \frac{Adx}{(x-x_1)^k}$ считается табличным
    \item Интеграл вида $\int \frac{Bx + C}{(x^2 + px + q)^k} dx$ сводится к интегралу $\int \frac{dx}{(x^2 + px
    + q)^k}$ путём представления числителя в виде суммы дифференциала знаменателя и остатка до верного выражения
    \item Интеграл $\int \frac{dx}{(x^2 + px + q)^k} = I_k(t)$ при $k = 1$ вычисляется путём выделения полного
    квадрата в знаменателя и дальнейшего применения табличного интеграла [арктангенса]
    \item Интеграл $I_k(t)$ при $k > 1$ вычисляется рекуррентно, путём интегрирования по частям и с помощью
    умного нуля
\end{itemize}

\subsection{Основные приемы интегрирования иррациональных и трансцендентных функций}

\textbf{Опр} \textit{Одночлен, многочлен} \textcolor{gray}{Функция $n$ переменных и сумма таковых}

\textbf{Опр} \textit{Рациональная функция} \textcolor{gray}{Отношение многочленов}

\begin{itemize}
    \item Интеграл вида $\int R\left(x^{\frac{1}{n}}\right)dx$ сводится к интегралу от рациональной дроби с
    помощью подстановки $t = x^{\frac{1}{n}}$
    \item Интеграл вида $\int R\left(x, \left(\frac{ax+b}{cx+d} = y\right)^{\frac{1}{n}}\right)dx$ сводится к интегралу из п.1 с помощью подстановки $t = y^{\frac{1}{n}}$
    \item Интеграл вида $\int R\left(x, \sqrt{ax^2 + bx + c} \right)dx$ сводится к интегралу от рациональной дроби
    при помощи подстановки $t = \frac{x - x_1}{x - x_2}$ в случае наличия корней.
    В противном случае при $a < 0$ выражение не определено, а при $a > 0$ сводится подстановкой к интегралу от
    рациональной дроби
    \item Интеграл от дифференциального бинома $\int x^m (ax^n + b)^p dx$, где $m, n, p \in \mathbb{Q} $ сводится к
    интегралу от рациональной функции в случаях, если $p \in \mathbb{Z}$ (подстановка $t = x^{\frac{1}{\text{НОЗ}
    ~m~\text{и}~n}}$), $\frac{m+1}{n} \in \mathbb{Z}$ (подстановка $t = (ax^n + b)^{\frac{1}{\text{знаменатель
    дроби}~p}}$) или $\frac{m+1}{n} + p \in \mathbb{Z}$ (подстановка $t = \left(\frac{ax^n + b}{x^n}\right)^{\frac{1}{\text{
        знаменатель дроби}~p}}$). В противном случае через элементарные функции не выражается
    \item Интегралы с гиперболическими и тригонометрическими функциями универсальными подстановками $t = \tg \frac{x}{2}$ и $t = \th \frac{x}{2}$
    сводятся к интегралам от рациональной дроби
\end{itemize}