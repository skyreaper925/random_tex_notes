\subsection{Клеточные множества}

\textbf{Опр} \textit{Клетка} \textcolor{gray}{Декартово произведение ограниченных числовых промежутков}

\textbf{Опр} \textit{Мера клетки} \textcolor{gray}{Произведение длин её числовых промежутков}

Пустое множество будем считать клеткой по определению, а её меру -- равной нулю

\textbf{Опр} \textit{Гиперплоскость в $\mathbb{R}^n$} \textcolor{gray}{Линейная комбинация координат точки
из $\mathbb{R}^n$, равная константе}

\textbf{Опр} \textit{Разрез множества гиперплоскостью} \textcolor{gray}{Подобные гиперплоскости
множества с неравенствами разной строгости (два варианта разреза)}

\textbf{Л1} \textit{О мере клетки}

\textcolor{blue}{Мера клетки, являющейся дизъюнктным объединением клеток, равна сумме мер её составляющих}

\begin{enumerate}
    \item Разрежем клетку на две клетки гиперплоскостью $x_i = c$
    \item Если одна из новых клеток будет иметь нулевую меру, то требуемое равенство будет тривиально выполнено
    \item В случае $c \in w_i = (a_i, b_i)$ старую клетку удобно представить в виде двух клеток заменой $w_i = w^{'}_i + w^{''}_i$.
    Применяя эти рассуждения много раз получим доказываемое равенство
    \item В общем случае у нас может быть неграмотная последовательность разрезов, поэтому придётся дорезать.
    Разрежем клетку по концам числовых промежутков $w_{i_k}$, определяющих клетку $\sqcap_i$
    \item В любом случае получаем, что каждая клетка является дизъюнктным объединением клеток меньшего порядка,
    полученных в результате такого разрезания.
    Поэтому мера исходной клетки есть сумма сумм элементарных
\end{enumerate}

\textbf{Опр} \textit{Клеточное множество} \textcolor{gray}{Конечный дизъюнктный набор клеток}

\textbf{Л2} \textcolor{blue}{Мера клеточного множества не зависит от способа разбиения этого множества на клетки}

\begin{enumerate}
    \item Пусть есть две серии разрезов.
    Тогда рассмотрим клетки $\sqcap_{ij} = \sqcap_{i} \cap \sqcap_{j}$, среди которых могут быть и пустые клетки
    \item Поэтому $m(\sqcap_{i}) = \sum_{j = 1}^{J} m(\sqcap_{ij})$, а $m(\sqcap_{j}) = \sum_{i = 1}^{I} m(\sqcap_{ij})$
    \item Итого мера первых клеток равна сумма сумм элементарных и равна мере вторых клеток
\end{enumerate}

Перечислим свойства клеточных подмножеств $\mathbb{R}^n$

\begin{itemize}
    \item Замкнутость относительно разности множеств.
    Действительно, в случае разности достаточности рассмотреть
    разность клетки и клеточного множества.
    Дополнение клетки до клеточного множества будет дизъюнктным
    объединением конечного числа элементарных клеток, то есть клеточным множеством
    \item Замкнутость относительно пересечения в силу $A \cap B = A~\backslash~(A~\backslash~B)$
    \item Замкнутость относительно объединения, потому что объединение есть дизъюнктное объединение разности клеточных множеств (клеточного
    множества по п.1) и клеточного множества
    \item Из предыдущих пунктов следует, что семейство всех клеточных подмножеств $\mathbb{R}^n$ является кольцом
    \item Аддитивность, то есть $m(A \cup B) = m(A) + m(B) - m(A \cap B)$.
    Действительно, в тривиальном случае нулевого пересечения очевидно.
    В общем случае надо воспользоваться $m(A) = m(A~\backslash~B) + m(A \cap B)$ и $m(A \cup B) = m(A~\backslash~B) + m(B)$ и
    сравнить с доказываемым утверждением
    \item Монотонность, то есть мера надмножества больше подмножества.
    Свойство следует из предыдущего в силу $m(A) = m(A) + m(B~\backslash~A)$
\end{itemize}

\subsection{Верхняя мера Лебега и ее счетная полуаддитивность}

\textbf{Опр} \textit{Верхняя мера множества}

\textcolor{blue}{Инфимум сумм мер клеток по всем счетным наборам клеток, покрывающим множество}

Из этого определения следует монотонность верхних мер (в силу монотонности инфимума).
Также для клеточного множества его верхняя мера равна мере множества

\textbf{Th} \textit{Счётная полуаддитивность верхней меры} \textcolor{gray}{Полуаддитивность свидетельствует о
неравенстве (в одну сторону)}

\textcolor{blue}{Если множестыво покрыто не более чем счётным набором множеств $X_k$, то его мера не превосходит суммы
мер всех множеств из набора}

\begin{enumerate}
    \item В случае $\mu^*(X_k) = +\infty$ утверждение очевидно.
    Иначе будем считать, что все множества $X_k$ конечны
    \item Фиксируем $\forall \varepsilon > 0$.
    Тогда для каждого $X_k$ найдётся счётный набор клеток, такой что из определения инфимума разность меры этих клеток и
    нашего множества будет не превосходить $\frac{\varepsilon}{2^k}$
    \item В итоге, суммируя по всем клеткам и множествам $X_k$ получим разницу между верхними мерами множества и
    набора не более, чем в $\varepsilon$
\end{enumerate}

\subsection{Мера Лебега и ее счетная аддитивность}

\textbf{Опр} \textit{Симметрическая разность} \textcolor{gray}{Первое множество без второго в объединении с наоборот}

\textbf{Опр} \textit{Предел по мере} \textcolor{gray}{При $k \shortrightarrow \infty$ симметрическая разность
множества и $X_k$ стремится к нулю}

\textbf{Опр} \textit{Конечно измеримое множество} \textcolor{gray}{$\exists$ последовательность клеточных множеств, сходящихся к нашему}

\textbf{Опр} \textit{Измеримое по Лебегу множество} \textcolor{gray}{Объединение счётного набора конечно измеримых}

\textbf{Опр} \textit{Мера Лебега} \textcolor{gray}{Для измеримого множества равна его верхней мере}

\textbf{Опр} \textit{Сдвиг множества на вектор} \textcolor{gray}{Сдвиг не меняет меру}

\textbf{Л1} \textit{Об измеримых множествах}

\textcolor{blue}{Объединение, пересечение и разность конечно измеримых множеств измеримо, а также $\mu (X \cup Y) + \mu (X \cap Y) = \mu (X) + \mu (Y)$}

\begin{enumerate}
    \item По определению конечной измеримости найдутся сходящиеся к нашим последовательности клеточных множеств $X_k$ и $Y_k$
    \item Для клеточных множеств утверждение теоремы доказано раннее
    \item Также воспользуемся свойством аддитивности меры клеточных множеств и перейдём к пределу для доказательства
    последнего равенства
\end{enumerate}

Из этой леммы следует, что семейство всех конечно измеримых множеств в $\mathbb{R}^n$ является кольцом

\textbf{Л2} \textit{Об представлении измеримого множества}

\textcolor{blue}{Измеримое множество можно представить в виде дизъюнктного объединения счётного набора конечно измеримых множеств}

\begin{enumerate}
    \item В силу измеримости нашего множеств существует не более чем счётный набор конечно измеримых множеств, покрывающих наше
    \item Из этого набора составим последовательность концентрических вложенных, но дизъюнктных множеств с помощью операций разности и объединения
    \item Тогда условия леммы выполнены (множества нового набора конечно измеримы, а их дизъюнктное объединение по набору покрывает наше)
\end{enumerate}

\textbf{Th} \textit{Счетная аддитивность меры Лебега}

\textcolor{blue}{Если множество является дизъюнктным объединением счетного набора измеримых множеств, то оно
измеримо, а его мера равна сумме мер множеств из набора}

\begin{enumerate}
    \item Рассмотрим случай конечно измеримых множеств $X_k$.
    Тогда наше множество измеримо по определению
    \item С одной стороны его мера не превосходит сумм мер покрытия в силу счётной полуаддитивности верхней меры, а
    с другой, сумма мер любого конечного набора $X_k$ не превосходит меры нашего множества (в силу определения покрытия)
    \item Переходя к пределу по числу $X_k$ в наборе получаем оценку для меры множества снизу.
    Итого, два неравенства дают требуемое равенство
    \item В общем случае надо воспользоваться предыдущей леммой $\forall X_k$ и просуммировать по двум уровням нарезки (по двум индексам)
\end{enumerate}

\subsection{Непрерывность меры Лебега}

\textbf{Th} \textit{Непрерывность меры Лебега}

\textcolor{blue}{Если у нас есть счетный набор измеримых множеств $X_k$ а наше множество покрывается этим набором, то
его мера равна пределу мер $X_k$} \\

Для доказательства применим идею из Л2 предыдущей темы и воспользуемся счётной аддитивностью меры

\subsection{Теорема о том, что семейство измеримых подмножеств $\mathbb{R}^n$ является $\sigma$-кольцом}

\textbf{Опр} \textit{Кольцо множеств} \textcolor{gray}{Система множеств, замкнутая относительно операций
пересечения и разности} \\

\textbf{Л1} \textcolor{blue}{Кольцо множеств замкнуто относительно пересечения} \textcolor{gray}{В
силу $A \cap B = A~\backslash~(A~\backslash~B)$} \\

\textbf{Опр} \textit{$\sigma$-кольцо множеств} \textcolor{gray}{Система множеств, замкнутая относительно операций
счётного пересечения} \\

\textbf{Л2} \textcolor{blue}{$\sigma$-кольцо множеств замкнуто относительно счётного пересечения} \textcolor{gray} {
    Для доказательства достаточно рассмотреть конкретное множество из пересечения, доказать, что его разность с
пересечением лежит в $\sigma$-кольце, и повторить рассуждения Л1} \\

\textbf{Л3} \textcolor{blue}{Пересечение $X$ счётного набора конечо измеримых множеств $X_k$ является конечно измеримым множеством} \textcolor{gray}

\begin{enumerate}
    \item По Л1 предыдущей темы $X_1 \backslash X_k$ измеримо, поэтому и $X_1 \backslash X$ измеримо как счётное
    объединение разностей конечно измеримых множеств
    \item $\mu (X_1 \backslash X) \leq \mu (X_1)$, \undetline{конечно} измеримого множество, то и $X_1 \backslash X$ конечно измеримо (монотонность меры)
    \item Разность множеств конечно измерима, а так как $X = X_1 \backslash (X_1 \backslash X)$, то и $X$ тоже
\end{enumerate}

\textbf{Л4} \textcolor{blue}{Семейство всех измеримых множеств в $\mathbb{R}^n$ является кольцом}

\begin{enumerate}
    \item Пусть $X,Y \subset \mathbb{R}^n$ измеримы
    Тогда они представимы в виде объединений и пересечений счётных наборов конечно измеримых множеств, как и $X \cap Y$
    \item $X_k \backslash Y$ конечно измеримо как пересечение конечно измеримых, а $X \backslash Y$ конечно измеримо как счётное объединение конечно измеримых
    \item Разность множеств конечно измерима, а так как $X = X_1 \backslash (X_1 \backslash X)$, то и $X$ тоже
\end{enumerate}

\textbf{Th} \textcolor{blue}{Семейство всех измеримых множеств в $\mathbb{R}^n$ является $\sigma$-кольцом}

\begin{enumerate}
    \item Каждое множество $X_k$ счётного набора можно представить в виде счётного объединения набора конечно измеримых
    \item Тогда счётное объединение по всему такому набору будет конечно \undetline{измеримым} множеством
    \item В силу Л4 и определения $\sigma$-кольца, получаем требуемое равенство
\end{enumerate}