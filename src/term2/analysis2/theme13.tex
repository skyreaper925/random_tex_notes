\subsection{Связь поточечной и равномерной сходимостей для функциональной последовательности}

\textbf{Опр} \textit{Поточечный предел функциональной последовательности} \textcolor{gray}{Предел в привычном
понимании}

\textbf{Опр} \textit{Равномерный предел функциональной последовательности} \textcolor{gray}{$N \in \mathbb{N}$ не зависит от аргумента}

Из равномерной сходимости следует поточечная, но не наоборот

\textbf{Опр} \textit{Равномерно ограниченная функциональная последовательность} \textcolor{gray}{$N \in \mathbb{N}$
    не зависит от аргумента}

\subsection{Критерий Коши равномерной сходимости функциональной последовательности}

\textbf{Th} \textit{Критерий Коши}

\textcolor{blue}{Последовательность сходится равномерно $\Leftrightarrow$ выполняется условие Коши}

\begin{enumerate}
    \item $\Rightarrow$: дважды применить определение равномерной сходимости и воспользоваться неравенством
    треугольника
    \item $\Leftarrow$: требуется доказать равномерную сходимость из выполнения условия Коши числовой
    последовательности для любого фиксированного $x \in X$.
    В силу КК для числовой последовательности $\lim_{k\to\infty} f_k = f$
    \item Далее надо в силу $\forall p \in \mathbb{N}$ устремить его к $+\infty$ и по теореме о предельном
    переходе в неравенствах получить определение равномерной сходимости
\end{enumerate}

\subsection{Обобщенный признак сравнения для функциональных рядов}

\textbf{Опр} \textit{Поточечный предел функционального ряда} \textcolor{gray}{Сходимость ряда в привычном понимании}

\textbf{Опр} \textit{Равномерный предел функционального ряда} \textcolor{gray}{Если последовательность его
частичных сумм сходится равномерно на том же множестве}

\textbf{Опр} \textit{Остаток поточечно сходящегося функционального ряда} \textcolor{gray}{Разность суммы и
частичной суммы ряда}

\textbf{Th} \textit{Обобщенный признак сравнения}

\textcolor{blue}{Если каждый член нашего ряда по модулю не превосходит члена равномерно сходящегося на том же
множестве ряда, то и наш ряд сходится равномерно}

Доказательство состоит в двукратном применении КК

Из признака следует, что из равномерной абсолютной сходимости ряда следует равномерная сводимость ряда на том же
множестве

\subsection{Признак Вейерштрасса равномерной сходимости функционального ряда}

\textbf{Th} \textit{Признак Вейерштрасса}

\textcolor{blue}{Если каждый член нашего ряда по модулю не превосходит члена сходящегося ряда, то наш ряд
сходится равномерно на том же множестве}

Доказательство состоит в применении обобщенного признака сравнения.
Заметьте, что мы не требуем равномерной сходимости от ряда-мажоранты

\subsection{Признаки Дирихле и Лейбница равномерной сходимости функционального ряда}

Смотреть в рукописном конспекте

\subsection{Признак Абеля равномерной сходимости функционального ряда}

Смотреть в рукописном конспекте

\subsection{Непрерывность равномерного предела, непрерывных функций и суммы равномерно сходящегося функционального ряда с
непрерывными слагаемыми}

\textbf{Th.1} \textit{О непрерывности предельной функции}

\textcolor{blue}{Если последовательность $f_k$ непрерывных на множестве $X$ функций сходится равномерно на
множестве $X$, то $f$ непрерывна на $X$}

\begin{enumerate}
    \item Зафиксируем $\forall \varepsilon > 0$ и $x_0 \in X$
    \item Далее для доказательства достаточно дважды записать определения равномерной сходимости и
    один раз непрерывности функции $f_N (x)$ для нужных долей  $\varepsilon$ и воспользоваться неравенством
    треугольника
\end{enumerate}

\textbf{Th.2} \textit{О непрерывности суммы ряда}

\textcolor{blue}{Если функциональный ряд $u_k$ непрерывных на множестве $X$ функций сходится
равномерно на множестве $X$, то сумма ряда непрерывна на $X$}

Доказательство состоит в применение Th.1 последовательности частичных сумм ряда

\subsection{Почленное интегрирование функциональных последовательностей и рядов}

\textbf{Th.1} \textit{Об интегрировании предельной функции}

\textcolor{blue}{Если последовательность $f_k$ интегрируемых на конечно измеримом множестве $X$ функций сходится
равномерно на множестве $X$ к интегрируемой функции $f$, то интеграл этой функции есть предел интегралов}

\begin{enumerate}
    \item Воспользуемся sup-критерием для $\varepsilon = 1$.
    Тогда из неравенства следует интегрируемость $f$ по признаку сравнения
    \item Расписав супремум для разности интегралов в пределе получим 0, что завершает доказательство
\end{enumerate}

\textbf{Следствие} \textcolor{blue}{Если последовательность непрерывных на компакте $X$ функций $f_k$ сходится
равномерно к функции $f$, то интеграл этой функции есть предел интегралов} \\

Непрерывность $f$ следует из теоремы предыдущей темы, а интегрируемость из достаточного условия интегрируемости,
что позволяет применить предыдущую теорему и доказать утверждение \\

\textbf{Th.2} \textit{Об почленном интегрировании ряда}

\textcolor{blue}{Если функциональный ряд $u_k$ непрерывных на компакте $X$ функций сходится
равномерно, то сумма интеграла есть интеграл суммы} \\

Доказательство состоит в применение следствия из предыдущей теоремы к последовательности частичных сумм ряда с
использованием линейности интеграла

\subsection{Дифференцирование предельной функции и почленное дифференцирование функционального ряда}

\textbf{Th.1} \textit{О дифференцировании предельной функции}

\textcolor{blue}{Если последовательность $f_k$ непрерывно дифференцируемых на отрезке $[a, b]$ функций сходится
хотя бы в одной точке $x_0$, а последовательность производных $f_k^{'}$ сходится равномерно на $[a, b]$, то
последовательность $f_k$ сходится равномерное на $[a, b]$ к некоторой непрерывно дифференицируемой функции $f$,
    притом производная предела есть предел производных}

\begin{enumerate}
    \item Обозначим предельную функцию для $f_k^{'}$ за $\varphi (x)$, непрерывную по теореме, и предел $f_k(x_0)$
    за $A$
    \item Далее определим $f(x) =  A + \int_{x_0}^x \varphi (t) dt$ и $f_k (x) = f_k (x_0) + \int_{x_0}^x f_k^{'}(
    t) dt$
    \item Затем пара хитрых замечаний, работа с супремумом, использование sup-критерия
    \item В итоге получаем равномерную сходимость $f_k$ и требуемое равенство с учётом построения $f(x)$
\end{enumerate}

\textbf{Th.2} \textit{О почленном дифференцировании ряда} \\

\textcolor{blue}{Если функциональный ряд $u_k$ непрерывно дифференцируемых на отрезке $[a, b]$ функций сходится
хотя бы в одной точке $x_0$, а ряд производных $u_k^{'}$ сходится равномерно на $[a, b]$, то
справделива формула почленного дифференицрования ряда, то есть производная суммы ряда есть сумма производных}

Доказательство состоит в применение Th.1 к последовательности частичных сумм ряда