\subsection{Измеримые функции}

\textbf{Опр} \textit{Измеримая функция} \textcolor{gray}{$\forall C \in \mathbb{R}~L_< = \{ x \in X: f(x) < C \}$
    измеримо}

\textbf{Л1} \textit{Об измеримых функциях}

\textcolor{blue}{Если функция измерима, то $L_{\leq}, L_{\geq}, L_>$ измеримы}

\begin{enumerate}
    \item Фиксируем $\forall C \in \mathbb{R}$ и доказываем измеримость $L_{\leq}$, пользуясь определением $\sigma$-кольца
    \item Остальные множества доказываются через измеримость разности
\end{enumerate}

\textbf{Л2} \textcolor{blue}{Открытые и замкнутые множества измеримы}

\begin{enumerate}
    \item Последовательно воспользуемся определением открытости и всюду плотностью $\mathbb{Q}$, чтобы покрыть нашу
    точку рациональной клеткой
    \item Так как мы брали рациональные числа, то набор различных клеток будет не более, чем счётный.
    Таким образом, открытое множество является объединением счетного набора измеримых множеств, то есть измеримым
    множеством
    \item Замкнутое множество является дополнением открытого множества, поэтому измеримо, так как $\mathbb{R}$ есть
    $\sigma$-кольцо
\end{enumerate}

Любая непрерывная функция измерима, потому что её $L_<$ открыто, а значит, измеримо по Лебегу

\subsection{Измеримость суммы и поточечного предела измеримых функций}

\textbf{Л1} \textcolor{blue}{Сумма измеримых функций измерима}

\begin{enumerate}
    \item Расписываем элемент множества $L_<$ для суммы функций, пользуемся всюду плотностью $\mathbb{Q}$
    Мы специально используем $\mathbb{Q}$ для счётности
    \item Затем переходим к пересечению множеств и объединению по $\mathbb{Q}$, что по определению $\sigma$-кольца означает
    измеримость $L_<$, а значит, и суммы функций
\end{enumerate}

Любая линейная комбинация измеримых функций является измеримой функцией.
Это следует из того, что операция сложения сохраняет измеримость функций, а операция умножения на число сохраняет
измеримость согласно определению измеримой функции

\textbf{Л2} \textcolor{blue}{Поточечный предел измеримых функций измерим}

\begin{enumerate}
    \item Меняем $C$ на $C^{'}$ и объявляем $k \geq N, k \in \mathbb{N}$ в силу определения предела, строгого
    неравенства и всюду плотности $\mathbb{Q}$.
    То есть $x \in L_<$ теперь лежит в пересечении по всем $k \geq N$
    \item Затем через операции объединения по $C^{'}$ и $N$ приходим к измеримости нашего множества в силу того, что
    $\mathbb{R}$ есть $\sigma$-кольцо
\end{enumerate}

\subsection{Интеграл Лебега для счетно-ступенчатых и для измеримых функций, линейность интеграла Лебега}

\textbf{Опр} \textit{Счётно- и конечно-ступенчатая функция} \textcolor{gray}{Множество её значений счётно или
конечно}

\textbf{Опр} \textit{Счётное и конечное разбиение множества} \textcolor{gray}{Дизъюнткное покрытие
исходного множества}

\textbf{Опр} \textit{Измеримое разбиение} \textcolor{gray}{Все множества разбияния измеримы по Лебегу}

Функция называется счётно-ступенчатой, еси существует счётное разбиение области определения и соответсвующий
набор значений функций, одинаковых на конкретном множестве из разбиения

Если все множества набора измеримы, то функция тоже будет измерима (из определения измеримости функции)

\textbf{Опр} \textit{Интеграл Лебега для счётно-ступенчатой функции}

\textcolor{blue}{Сумма произведений мер $X_i$ на значении функции $f_i$ на этом $X_i$}

Даже если $f_i = $, $X_i = +\infty$, то их произведение всё равно 0.
А если в сумме содержатся разные по бесконечности слагаемые или не существует конечной / бесконечной суммы, то интеграл Лебега для этой функции не
существует

\textbf{Опр} \textit{Интегрируемая по Лебегу функция}

\textcolor{blue}{Если ряд Лебега для неё сходится абсолютно}

Нам существенна абсолютная сходимость, потому как иначе интегрируемость зависела бы от способа разбиения области
определения.
Это доказывает следующая лемма

\textbf{Л1} \textcolor{blue}{Интеграл Лебега не зависит от измеримого разбиения}

\begin{enumerate}
    \item Пусть есть два различных разбиения.
    Тогда рассмотрим клетки $X_{ij} = X_{i} \cap X_{j}$, среди которых могут быть и пустые клетки
    \item Просуммируем по всем таким множествам, пользуясь счётной аддитивностью меры Лебега.
    Мера $X_{ij}$ не зависит от порядка суммирования (к тому же, ряд сходится абсолютно), поэтому и домножение на $
    f_i$ и $f_j$ (равные на пересечении), тоже не повлияет на сумму ряда
    \item Поэтому можно запросто совершить переход (под знаком равенства) от одной суммы к другой
\end{enumerate}

\textbf{Утв} \textit{Геометрический смысл интеграла Лебега} \textcolor{gray}{Декартово произведение ($x, y$) в
    $\mathbb{R}^{n+1}$}

\textbf{Л2} \textit{Линейность интеграла Лебега для счетно-ступенчатых функций}

Доказывается той же идеей, что и предыдущая лемма, используя по ходу дела соответствующие свойства для абсолютно
сходящихся рядов

\textbf{Опр} \textit{Почти всюду, почти для всех} \textcolor{gray}{За исключением множества нулевой меры}

\textbf{Опр} \textit{Верхний и нижний интегралы Лебега} \textcolor{gray}{Инфимум (супремум) интеграла по
счётно-ступенчатым больше (меньше) нашей}

\textbf{Опр} \textit{Интеграл Лебега} \textcolor{gray}{Равное значение верхнего и нижнего интегралов Лебега}

\textbf{Опр} \textit{Интегрируемая по Лебегу функция} \textcolor{gray}{Измеримая с конечным интегралом Лебега}

Для счётно-ступенчатой функции её интеграл Лебега и общий интеграл Лебега для неё эквивалентны

\textbf{Л1} \textcolor{blue}{Если значения двух функций совпадают для почти всех аргументов, то их интегралы Лебега
существуют или не существуют одновременно, а если существуют, то совпадают}

Это следует из определений верхнего и нижнего интегралов

\textbf{Л2} \textcolor{blue}{Если функция интегрируема, то для почти всех аргументов её значение на них конечно}

Это следует из определений верхнего и нижнего интегралов

\subsection{Теорема о существовании интеграла от неотрицательной измеримой функции}

\textbf{Лемма} \textcolor{blue}{$\forall \varepsilon > 0~\exists$ измеримая и отдельно интегрируемая СС функции такие
    , что первая не больше нашей, а их сумма не меньше нашей, притом вторая бесконечно мала}

Заметим, что от нашей функции, как и от миноранты, мы не требуем быть интегрируемой.
Ведь даже $f(x) = x, x \in [0, +\infty]$ не имеет интегрируемой функции, удовлетворяющей условиям теоремы

\begin{enumerate}
    \item Рассмотрим случай конечной меры области определения $X$ нашей функции.
    Тогда выберем $N \in \mathbb{N}: \frac{\mu(X)}{N} < \varepsilon, \varphi(x) = \frac{1}{N} = const$, а $g(x)$ зададим
    как стандартную СС, что доказывает теорему
    \item В общем случае представим $X$ в виде дизъюнктного объединения счетного набора конечно измеримых
    множеств, для каждого из которых применим предыдущий пункт
    \item В силу бесконечной малости $\varphi_k(x)$, сделаем её меньше $\frac{\varepsilon}{2^k}$ (чтобы при
    интегрировании по всем функциям получить необходимую малость), а затем соберём обе функции в СС-ые, доказав
    общий случай
    \item Так как все члены $\varphi_k(x) \mu (X_k)$ неотрицательны, то ряд сходится абсолютно и, следовательно, функция
    $\varphi_k(x)$ интегрируема по определению
\end{enumerate}

\textbf{Th} \textit{О существовании интеграла от неотрицательной измеримой функции} \textcolor{gray}{
    Неотрицательная измеримая функция имеет (бес)конечный интеграл Лебега}

\begin{enumerate}
    \item В тривиальном случае бесконечного нижнего интеграла получаем бесконечные верхний и интеграл Лебега
    \item В общем случае зафиксируем $\forall \varepsilon > 0$ и применим результат предыдущей леммы, притом так как $g$
    измерима и СС, то будем считать, что область его определения $X$ можно представить в виде счетного
    дизъюнктного объединения конечно-измеримых множеств, то будем считать, что множества $X_k$ конечно-измеримы
    \item Покажем, что $g$ интегрируема от противного.
    Если у нас это удастся сделать, то мы мгновенно докажем теорему в силу равенства крайних интегралов Лебега.
    Предположим, что для $g~\exists\int = +\infty $
    \item Если каждый член суммы конечен, то для любой константы найдётся частичная сумма, большая её.
    Тогда мы можем построить СС (даже конечно-ступенчатую) функцию $g^{'}< g$ на $X$, которая будет интегрируема
    \item В случае наличия бесконечного члена, определим функцию $g^{'}< g$ на $X$ как либо этот конечный член,
    либо 0
    \item В любом случае построена $g^{'}< g$, чей бесконечный интеграл по $X$ не меньше нижнего нашей функции,
    что отрицает её интегрируемость по определению
    \item Итого, $g$ интегрируема, $g + \varphi$ тоже, а их разность может быть сколь угодно малой (в силу малости
    $\int \varphi dx$), что из неравенств, не меньше разности крайних интегралов
    \item В силу произвольности $\varepsilon > 0$ интегралы совпадают, поэтому наша функция интегрируема
\end{enumerate}

\subsection{Связь интегрируемости функции и интегрируемости ее положительной и отрицательной составляющих}

\textbf{Опр} \textit{Положительная и отрицательная составляющие функции} \textcolor{gray}{Соответсвующие максимумы}

Притом сама функция равна сумме этих составляющих

\textbf{Лемма} \textcolor{blue}{Функция измерима $\Leftrightarrow$ обе её составляющие интегрируемы}

\begin{enumerate}
    \item Из интегрируемости $f_+, f_-$ мгновенно следует интегрируемость функции в силу линейности интеграла Лебега
    \item В другую сторону рассмотрим $f_+$, чьё $L_<$ совпадает с $L_<$ функции (в случае $C \geq 0$ очевидно, а в
    случае $C < 0~L_<$ пусто, но всё равно измеримо по определению)
    \item Из интегрируемости $f$ следует, что $\exists h: f \leq h$ почти всюду на $X$, притом $h$ интегрируема.
    Аналогичное
    равенство
    справедливо и для положительных составляющих
    \item Так как $h$ интегрируема, то её положительная составляющая тоже по признаку сравнения для абсолютно
    сходящегося ряда
    \item По теореме предыдущей темы $f_+$ интегрируема, притом в силу теоремы об интегрировании неравенств
    (доказывается через крайние интегралы), интеграл конечен
    \item Произведя аналогичные рассуждения для $f_-$, получим доказательство в другую сторону, то есть полное
    доказательство
\end{enumerate}

\subsection{Связь интегрируемости функции и интегрируемости ее модуля}

\textbf{Th.1} \textit{Признак сравнения}

\textcolor{blue}{Если измеримая функция по модулю не превосходит интегрируемой по Лебегу функции для почти всех
аргументов, то она интегрируема}

\begin{enumerate}
    \item Из измеримости $f$ следует измеримость $f_+, f_-$
    \item Затем достаточно воспользоваться теоремой об интегрировании неравенств дважды и доказать
    интегрируемость $f_+, f_-$
    \item Из предыдущей леммы следует интегрируемость $f$
\end{enumerate}

\textbf{Th.2} \textit{Связь интегрируемость функции и её модуля}

\textcolor{blue}{$f$ интегрируема по Лебегу $\Leftrightarrow$ $f$ измерима и её модуль интегрируем по Лебегу} \\

Из интегрируемости $f$ следует интегрируемость $f_+, f_-$, то есть и их суммы, которая и есть модуль.
Для доказательства в обратную сторону достаточно воспользоваться Th.1

\subsection{Интегральная теорема о среднем}

\textbf{Th} \textit{Интегральная теорема о среднем} \textcolor{gray}{Если $X$ -- линейнос связный компакт
в $\mathbb{R}^n; f, g: X \rightarrow \mathbb{R}$, притом $f$ измерима, а $g$ интегрируема с сохранением знака, то
существует точка из $X$ ...}

\begin{enumerate}
    \item БОО будем полагать $g \geq 0$
    \item В силу теоремы Вейерштрасса, $f$ достигает экстремумов (потому как на компакте), то есть произведение
    функций полуограниченно
    \item В силу признака сравнения, произведение функций интегрируемо, а также, в силу интегрирования неравенств
    и свойства линейности интеграла имеем оценку и для интеграла произведения функций
    \item Если интеграл равен нулю, то теорема справедлива для любого аргумента.
    В противном случае возьмём отношение и посмотрим на неравенство для $C$
    \item По теореме о промежуточном значении, найдётся $\xi \in X: f(\xi) = C$
\end{enumerate}

\subsection{Счетная аддитивность и непрерывность интеграла Лебега по множествам интегрирования}

\textbf{Л1} \textit{Об интегрируемости на подмножестве} \textcolor{gray}{Интегрируемая функция интегрируема на
измеримом подмножестве}

\begin{enumerate}
    \item Из интегрируемости функции следует интегрируемость её модуля, то есть его верхний интеграл Лебега конечен
    \item Тогда существует интегрируемая СС, которая не меньше модуля для почти всех аргументов.
    Она будет интегрируема на подмножестве по определению интеграла от СС функции
    \item В силу признака сравнения и наша функция интегрируема на подмножестве
\end{enumerate}

\textbf{Л2} \textit{Конечная аддитивность интеграла Лебега по множествам} \textcolor{gray}{На непересекающихся
измеримых множествах, интегрируемая на них функция интегрируема на их объединении, притом её интеграл есть сумма
интегралов на множествах}

\begin{enumerate}
    \item Распишем определение верхнего интеграла Лебега и построим новую верхнюю СС для объединения
    \item Осталось применить аддитивность интеграла для СС и получить нижний интеграл для объединения множеств
    \item Аналогично для верхнего интеграла Лебега.
    В итоге в силу конечности суммы конечных слагаемых и определения интеграла Лебега, получаем доказываемое утверждение
\end{enumerate}

\textbf{Th.1} \textit{Непрерывность интеграла по множествам} \textcolor{gray}{Для счётного набора измеримых по
Лебегу вложенных множеств и интегрируемой по Лебегу функции интеграл по счётному объединению равен пределу
интегралов по множествам}

\begin{enumerate}
    \item Рассмотрим случай СС функции и измеримое разбиение области определения.
    Из этого набора составим последовательность концентрических вложенных, но дизъюнктных множеств с помощью
    операций разности
    \item Распишем интеграл функции на множестве, используя счётную аддитивность интеграла от СС функции и конечную
    аддитивность интеграла Лебега
    \item В общем случае интегрируемой по Лебегу функции воспользуемся конечной аддитивностью интеграла Лебега и
    разобьём множество на подмножество и его дополнение.
    Требуется доказать, что интеграл на дополнении есть ноль
    \item Для этого воспользуемся определением интеграла Лебега, теоремой об интегрировании неравенств,
    результатами для СС функций и теоремой о трёх последовательностях
\end{enumerate}

\textbf{Th.2} \textit{Счетная аддитивность интеграла Лебега} \textcolor{gray}{На измеримом разбиении
множества интегрируемая на них функция имеет интеграл, равный счётной сумме по множествам}

\begin{enumerate}
    \item Определим новый набор множеств как частичное объединение старых
    \item Далее используем непрерывность и конечную аддитивность интеграла по множествам
\end{enumerate}