\subsection{Мера декартова произведения двух конечно измеримых множеств}

\textbf{Th} \textcolor{blue}{Если два множества конечно измеримы в своих надмножествах, то их декартово
произведение конечно измеримо в соотвествующем надмножестве с мерой, равной произведению мер}

\begin{enumerate}
    \item В тривиальном случае клеток равенство следует из определения
    \item В случае, если конечно измеримые множества представимы в виде счетного дизъюнктного объединения клеток,
    разобьём их на эти клетки, а потом, в силу теоремы о перемножении абсолютно сходящихся рядов, получим требуемое
    \item Покажем, что для любых конечно измеримых множеств мера их декартового произведения не превосходит
    произведения мер.
    Для этого зафиксируем $\forall \varepsilon > 0$ и счётные покрытия наших множеств клетками (они найдутся по
    определению верхней меры), притом разность мер покрытия и наших множеств не будет
    превосходить $\varepsilon$.
    Тогда распишем неравенство для верхней меры декартова произведения и, устремив $\varepsilon \rightarrow 0$,
    получим требуемое неравенство
    \item Теперь покажем, что если существуют множества, сходящиеся по мере к нашим (с конечной верхней мерой), то
    их декартово произведение также будет сходиться к декартову произведению наших.
    Действительно, для этого надо расписать неравенство для верхней меры симметрической разности, используя
    предыдущий пункт и понять, что она стремится к нулю
    \item В общем случае по определению конечно измеримого множества найдутся последовательности клеточных
    множеств, сходящиеся по мере к нашим.
    Тогда надо последовательно воспользоваться п.4 и п.1, а затем перейти к пределу
\end{enumerate}

Из теоремы следует, тчо декартово произведение множества нулевой меры и произвольного имеет нулевую меру

 \subsection{Выражение меры множества под графиком интегрируемой функции через интеграл}

\textbf{Лемма} \textit{Теорема о трёх последовательностях для конечно измеримых множеств}

\textcolor{blue}{Если задано наше множество и существуют конечно измеримые последовательности
миномажорант для него, которые в пределе имеют одинаковую меру, то наше множество измеримо и имеет ту же меру} \\

Для доказательства нам потребуется перейти от верхней меры (заданной для всех, в том числе для неизвестного нашего
множества) к клеточным множествам, для которых уже есть понятие предела по мере.
Иначе наши рассуждения могли бы быть неприменимы

\begin{enumerate}
    \item Рассмотрим \undetline{верхнюю} меру симметрической разность нашего и последовательности миноранты.
    Из неравенств будет следовать, что она стремится к нулю
    \item Теперь рассмотрим симметрическую разность клеточных множеств $A_{ik}$, покрывающих $A_k$, и саму $A_k$.
    Применив неравенство треугольника, получим, что клеточные множества $A_{ik}$ сходятся по мере к нашему
    \item Аналогичные рассуждения для мажорант доказывают теорему
\end{enumerate}

\textbf{Th} \textit{О геометрическом смысле интеграла}

\textcolor{blue}{Если область определения интегрируемой функции $X$ измерима, то площадь под графиком функции в
соотвествующем надмножестве конечно измерим с мерой равной интегралу лебега этой функции по $X$}

\begin{enumerate}
    \item В тривиальном случае СС функции можно разбить график на дизъюнктное объединение множеств и в силу
    счётной аддитивности интеграла Лебега получить требуемое утверждение
    \item В общем случае обозначим интеграл как $J$ и зафиксируем $\forall \varepsilon > 0$
    \item Воспользуемся определением верхних интегралов и запишем две серии неравенств (для СС-функций и их
    интегралов)
    \item При необходимости заменим значения миномажорант-СС-функций на множестве нулевой мере (чтобы
    доказываемое утверждение было справедливо для всего $X$)
    \item На предыдущем шаге записываем меру площадей графиков функции под миномажорантами и приходим к
    очевидному двойному вложению
    \item Так как в силу произвольности $\varepsilon > 0$ их площади стремятся к $J$ , то в силу леммы, площадь под
    графиком измерима с мерой $J$
\end{enumerate}

\subsection{Площадь круга}

\textbf{Лемма} \textcolor{blue}{Круг измерим с площадь $\pi r^2$}

\begin{enumerate}
    \item Напишем множество верхнего полукруга и после преобразований выразим $y$: $0 \leq y \leq \sqrt{
        r^2 - x^2}$
    \item По предыдущей теореме верхний полукруг измерим с интегралом в половину искомого (интеграл считается
    через замену).
    Аналогично для нижнего полукруга
    \item Так как две части круга имеют нулевую меру пересечения, то по формуле включений-исключений, мера круга
    равна $\pi r^2$
\end{enumerate}

\subsection{Выражение объема тела вращения и длины кривой через интегралы}

\textbf{Опр} \textit{Тело вращения вокруг оси} \textcolor{gray}{Если на отрезке задана неотрицательная функция, то
множество ...}

\textbf{Th.1} \textcolor{blue}{Если неотрицательная функция измерима и ограничена, то тело вращения измеримо...}

\begin{enumerate}
    \item Зафиксируем супремум ограниченной функции, число $N \in \mathbb{N}$, на которое мы разобьём наш отрезок
    множествами $X_k$ и измеримые конечно-ступенчатые функции-миномажаронты
    \item Распишем объём тел вращения для миноранты в терминах декартова произведения площади круга на меру $X_k$ c помощью определения интеграла для СС-функции
    \item Запишем неравенства для полученных объёмов и устремим $N \rightarrow +\infty$
    \item Аналогично распишем для мажоранты
    \item В силу вложенности и стремления по мере в пределе получим объём тела вращения для нашей функции
\end{enumerate}

\textbf{Th.2} \textit{Вычисление длины кривой}

\textcolor{blue}{Если кривая параметризована непрерывно дифференицируемой вектор-функцией, то её длина выражается
формулой ...} \\

Для доказательства достаточно рассмотреть переменную длину дуги, вспомнить теорему о производной переменной длины
дуги и применить формулу Ньютона-Лейбница

\subsection{Связь интегрируемости по Риману и интегрируемости по Лебегу}

\textbf{Опр} \textit{Разбиение отрезка, отрезки разбиения} \textcolor{gray}{Конечный набор точек}

\textbf{Опр} \textit{Выборка} \textcolor{gray}{Набор точек из отрезков разбиения}

\textbf{Опр} \textit{Интегральная сумма Римана} \textcolor{gray}{Сумма конечного числа слагаемых, зависит от
функции, разбиения и выборки}

\textbf{Опр} \textit{Мелкость разбиения} \textcolor{gray}{Максимальный отрезок разбиения}

\textbf{Опр} \textit{Интеграл Римана} \textcolor{gray}{Предел интегральных сумм Римана} \\

Заметим, что этот интеграл всегда конечен в силу работы на компакте (отрезке) \\

\textbf{Опр} \textit{Интегрируемая по Риману функция} \textcolor{gray}{$\exists$ интеграл Римана для этой функции
на этом отрезке}

\textbf{Th.1} \textit{Достаточное условие интегрируемости}

\textcolor{blue}{Если функция непрерывна на компакте, то она интегрируема на нём}

\begin{enumerate}
    \item Так как для любого $C \in \mathbb{R}~L_{\leq}$ замкнуто (а значит, измеримо), то функция измерима на
    компакте
    \item В силу теоремы Вейерштрасса функция ограничена на компакте некоторой константой
    \item Так как константа интегрируема на компакте, то по признаку сравнения функция тоже интегрируема
\end{enumerate}

\textbf{Th.2} \textcolor{blue}{Если функция интегрируема по Риману, то она интегрируема и по Лебегу и интегралы
совпадают}

\begin{enumerate}
    \item Зафиксируем $\forall \varepsilon > 0$ и достаточно мелкое разбиение отрезка
    \item Перепишем предельное неравенство в терминах инфимума и введём новые обозначения, чтобы ввести
    конечно-ступенчатую функцию
    \item Тогда интеграл для минорант будет интегралом Римана функции (записанным в терминах инфимума).
    Поэтому нижний интеграл будет не меньше Риманова
    \item Аналогично верхний интеграл не больше Риманова
    \item Объединив все полученные неравенства в одну строку, получим равенство крайних интегралов и интеграл
    Лебега по определению
\end{enumerate}

\subsection{Интегрируемость по Риману непрерывной на отрезке функции}

\textbf{Th} \textcolor{blue}{Для непрерывной на отрезке функции $f$ интеграл Римана существует и совпадает с
интегралом Лебега}

\begin{enumerate}
    \item Сначала надо воспользоваться теоремой Кантора, определением равномерной непрерывности
    \item Затем зафиксировать разбиение и выборку, определить конечно-ступенчатую функцию
    \item Вспомнить определение интеграла для СС функции и модуля непрерывности
    \item По Th.1 $f$ интегрируема по Лебегу, как и разность $f$ и СС функции в силу линейности интеграла
    \item Переходя к пределу при мелкости разбиения, получаем что интеграл Римана существует по определению,
    притом из рассуждений следует, что он совпадает с интегралом Лебега
\end{enumerate}