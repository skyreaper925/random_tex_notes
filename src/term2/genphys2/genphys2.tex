%! Author = user
%! Date = 06.06.2023

\documentclass[a4paper, 14pt]{article}
%\documentclass[draft]{article}

\usepackage[T2A]{fontenc}
\usepackage[utf8]{inputenc}
\usepackage[english, russian]{babel}
\usepackage[top = 2cm, bottom = 2cm, left = 2cm, right = 2cm]{geometry}
\usepackage{indentfirst}
\usepackage{xcolor}
\usepackage{hyperref}
\usepackage{gensymb}
\usepackage{pgfplots}
\usepackage{amsmath, amsfonts, amsthm, mathtools}
\usepackage{amssymb}
\usepackage{physics, multirow, float}
\usepackage{wrapfig, tabularx}
\usepackage{icomma} % Clever comma: 0,2 - number while 0, 2 - two numbers
\usepackage{tikz, standalone}
\usepackage{fancyhdr,fancybox}
\usepackage{lastpage}
\usepackage{booktabs}
\usepackage{listings}
\usepackage{lstmisc}
\usepackage{stmaryrd}

%\полуторный интервал
\onehalfspacing

\hypersetup
{   colorlinks = false,
    linkcolor = blue,
    pdftitle = {genphys},
    pdfauthor = {Володин Максим},
    allcolors = [RGB]{010 090 200}
}

%\gravarphicspath{{images/}}
%\DeclareGravarphicsExtensions{.pdf,.png,.jpg}

\restylefloat{table}
\usetikzlibrary{external}

\mathtoolsset{showonlyrefs = true} % Numbers will appear only where \eqref{} in the text LINKED
\pagestyle{fancy}

\fancyhf{}
\fancyhead[L]{Общая физика. Термодинамика и молекулярная физика}
\fancyhead[R]{Конспект билетов}
\fancyfoot[L]{}
\fancyfoot[R]{\thepage /\pageref{LastPage}}

\pgfplotsset{compat=1.18}

\begin{document}

    \tableofcontents \newpage

    \section{Термодинамическая система.
    Микроскопические и макроскопические параметры.
    Уравнение состояния (термическое и калорическое).
    Равновесные и неравновесные состояния и процессы}

    \subsection{Термодинамическая система}

    \textbf{Опр} \textit{Система, термодинамическая система} \textcolor{gray}{Совокупность рассматриваемых тел ...}

    \textbf{Опр} \textit{Изолированная, закрытая и открытая термодинамическая система} \textcolor{gray}{Обмен ...}

    \subsection{Микроскопические и макроскопические параметры}

    \textbf{Утв} \textit{Существуют микроскопические и макроскопические состояния} \textcolor{gray}{+ их другие имена}

    \textbf{Опр} \textit{Микроскопическое и макроскопическое состояния} \textcolor{gray}{Состояние системы,}

    \textbf{Опр} \textit{Микроскопические параметры} \textcolor{gray}{Величины, характризующий макросостояние}

    \subsection{Уравнение состояния (термическое и калорическое)}

    \textbf{Опр} \textit{Уравнение состояния} \textcolor{gray}{Состояние, отражающее для конкретного класса величин ...}

    \textbf{Опр} \textit{Термодинамическое, калорическое уравнение состояния} \textcolor{gray}{$f(P, V, T) = 0, j(\dots)$}

    \subsection{Равновесные и неравновесные состояния и процессы}

    \textbf{Опр} \textit{Термодинамическое равновесие} \textcolor{gray}{Три условия}

    \textbf{Закон} \textit{Основное (общее) начало термодинамики}

    \textcolor{blue}{Предоставленная самой себе изолированная система приходит в состояние термодинамического ...}

    \textbf{Закон} \textit{-I начало ТД}

    \textcolor{blue}{Три условия на любую изолированную систему}

    \textbf{Опр} \textit{Неравновесное состояние} \textcolor{gray}{Предоставленная самой себе ...}

    \textbf{Опр} \textit{Релаксация, время релаксации} \textcolor{gray}{Переход из состояния, в котором система ...}

    \textbf{Опр} \textit{Траектория процесса} \textcolor{gray}{Состояния системы и переходы между ними}

    \textbf{Опр} \textit{Равновесный (квазистатический) процесс} \textcolor{gray}{По ходу процесса система ...}

    \textbf{Опр} \textit{Неравновесное состояние} \textcolor{gray}{На траектории процесса встречаются ...}

    \section{Идеальный газ.
    Уравнение состояния идеального газа.
    Идеально-газовое определение температуры.
    Связь давления и температуры идеального газа с кинетической энергией его молекул}

    \subsection{Идеальный газ}

    \textbf{Опр} \textit{Идеальный газ} \textcolor{gray}{Газ, у которого взаимодействием молекул между собой можно ...}

    \subsection{Уравнение состояния идеального газа. Идеально-газовое определение температуры}

    \textbf{Закон} \textit{Бойля -- Мариотта}

    \textcolor{blue}{$PV = const, const$ однозначно определяется количеством газа и степенью его "нагретости"}

    \textbf{Опр} \textit{Газовая постоянная} \textcolor{gray}{Определяется из тройной точки воды. Измеряется в ...}

    \textbf{Опр} \textit{Постоянная Больцмана} \textcolor{gray}{$k = \frac{R}{N_A} = 1,38 \cdot 10^{-23} \frac{joules}{K}$}

    \textbf{Закон} \textit{Уравнение состояния идеального газа Менделеева -- Клапейрона}

    \textcolor{blue}{$PV = \mu RT = NkT = \frac{m}{\mu} RT$}

    Отсюда можно определить температуру по идеально-газовой шкале $T = \frac{PV}{\nu R}$

    \subsection{Связь давления и температуры идеального газа с кинетической энергией его молекул}

    В результате перехода от микрорассмотрения к макро-, получим $P = \frac{1}{3}nvp = \dots \Rightarrow U = N \frac{
        3}{2}kT$

    Из полной кинетической энергии газа в воздухе $E = N \overline{\varepsilon}$ можно получить $\overline{
        \varepsilon} = \frac{3}{2}kT$

    \section{Работа, внутренняя энергия, теплота.
    Первое начало термодинамики.
    Внутренняя энергия идеального газа}

    \subsection{Работа, внутренняя энергия, теплота}

    \textbf{Опр} \textit{Функция состояния} \textcolor{gray}{Величина, принимающая определённое значение в каждом ...}

    \textbf{Опр} \textit{Работа, совершённая системой и над ней} \textcolor{gray}{$PdV, P \in \{P_{in}, P_{out} \}$}

    Для квазистатического процесса $\delta A_{in} = - \delta A_{out}$.

    Заметим, что работа не является функцией состояния

    \textbf{Опр} \textit{Адиабатическая оболочка} \textcolor{gray}{При любых изменениях температуры окружающих ...}

%    Если система заключена в адиабатическую оболочку, то работа внешних сил не зависит от траектории процесса, а
%    определяется только начальным и конечным состояниями системы: $A_{12} = U_2 - U_1, U$ -- внутренняя энергия,
%    функция состояния

    \textbf{Опр} \textit{Количество теплоты} \textcolor{gray}{Если система заключена в жёсткую ...}

    $Q_{in} = U_2 - U_1 = -Q_{out}$

    Первое начало термодинамики

    \subsection{Первое начало термодинамики}

    \textbf{Закон} \textit{Первое начало термодинамики}

    \textcolor{blue}{ЗСЭ, записываемый как $\delta Q_{in} = dU + A_{in} \Rightarrow dU = \delta Q - PdV$}

    \subsection{Внутренняя энергия идеального газа}

    \textbf{Опр} \textit{Внутренняя энергия ИГ} \textcolor{gray}{Функция только температуры, так как определяется ...}

    \[ dU = c_V dT, U = \int c_V dT = \nu N_A \overline{\varepsilon} = \frac{i}{2} \nu RT \]

    \section{Теплоёмкость.
    Теплоёмкости при постоянном объёме и давлении.
    Связь между $c_V$ и $c_P$ для идеального газа (соотношение Майера)}

    \subsection{Теплоёмкость}

    \textbf{Опр} \textit{Теплоёмкость} \textcolor{gray}{$c = \frac{\delta Q_{in}}{dT}$}

    \subsection{Теплоёмкости при постоянном объёме и давлении}

    Для получения указанных формул, достаточно записать I начало ТД и после преобразований записать определение
    теплоёмкости, не забыв, что $H = U + PV$ есть энтальпия

    \subsection{Связь между $c_V$ и $c_P$ для идеального газа (соотношение Майера)}

    \[ c_P dT = dU = d(U + PV) = (c_V + \nu R) dT \Rightarrow c_P = c_V + \nu R\]

    \section{Политропический и адиабатический процессы.
    Уравнение адиабаты и политропы идеального газа.
    Скорость звука в газах}

    \subsection{Политропический и адиабатический процессы}

    \textbf{Опр} \textit{Политропический процесс} \textcolor{gray}{Процесс, в котором теплоёмкость остаётся ...}

    \textbf{Опр} \textit{Адиабатический процесс} \textcolor{gray}{Процесс, происходящий в теплоизолированной ...}

    \subsection{Уравнение адиабаты и политропы идеального газа}

    \textbf{Опр} \textit{Показатель адиабаты, политропы} \textcolor{gray}{$\gamma = \frac{c_P}{c_V}$, $n = \dots$}

    Из уравнения состояния ИГ можно вывести уравнение адиабатического и политропического процессов

    Существует четыре основных политропических процесса: адиабата, изохора, -бара, -терма.
    У каждого из них свои теплоёмкости, показатели политропы $n$ и уравнения

    \subsection{Скорость звука в газах}

    \textbf{Опр} \textit{Скорость звука} \textcolor{gray}{Фазовая скорость продольных волн в бесконечной ...}

    Скорость звука можно запросто вывести из соответствующего уравнения механики

    \textbf{Опр} \textit{Адиабатическая скорость звука} \textcolor{gray}{За время прохождения звука на ...}

    Адиабатическую скорость звука выражается через ту же конечную формулу, с использованием уравнения адиабаты и
    $\rho = \frac{P \mu}{RT}$

    \section{Тепловые машины.
    Цикл Карно.
    КПД машины Карно.
    Теоремы Карно.
    Холодильная машина и тепловой насос.
    Коэффициенты эффективности идеальной холодильной машины и идеального теплового насоса}

    \subsection{Тепловые машины}

    \textbf{Опр} \textit{Тепловая машина} \textcolor{gray}{Устройство, которое преобразует теплоту в работу или ...}

    \subsection{Цикл Карно}

    \textbf{Опр} \textit{Машина Карно} \textcolor{gray}{Тепловая машина, работающая по циклу Карно}

    \textbf{Опр} \textit{Цикл Карно} \textcolor{gray}{Обратимый цикл из двух изотерм и адиабат}

    \subsection{КПД машины Карно}

    \textbf{Опр} \textit{КПД тепловой машины} \textcolor{gray}{Отношение работы, произведённой машиной за один цикл ...}

    \subsection{Теоремы Карно}

    \textbf{Th} \textit{Первая теорема Карно}

    \textcolor{blue}{КПД любой тепловой машины, работающей между между двумя заданными термостатами, не может
    превышать КПД машины Карно, работающей между теми же резервуарами}

    \begin{enumerate}
        \item от противного: пусть у необратимой машины КПД больше.
        Рассмотрим работу этих двух машин в разных направлениях на одних и тех же резервуарах
        \item Подберём $Q_{+1}, Q_{+2}: Q_{+1} = Q_{+2}$.
        Тогда рассмотрим суммарные теплоты и работы за цикл (ведь две тепловые машины всё равно что одна
        многофункциональная)
        \item Итого, получилось что единственным результатом цикла большой машины есть производство $A_{\sum} = Q_+
        (\eta - \eta_k) > 0$ работы за счёт охлаждение холодильника, $w$ со II началом ТД
    \end{enumerate}

    \textbf{Th} \textit{Вторая теорема Карно}

    \textcolor{blue}{КПД любых идеальных машин, работающих по циклу Карно между двумя заданными термостатами, равны и
    не зависят от устройства машин и рабочего тела}

    \begin{enumerate}
        \item Это следствие первой теоремы: надо применить её к двум конкретным машинам Карно и поменять их местами.
        \item Система двух неравенств эквивалентна равенству.
        Независимость от параметров достигнута за счёт рассмотрения общего случая
        \item Чтобы найти точное значение КПД машины Карно, работающей с телами с \underline{температурами} $T_1, T_2$,
        надо рассмотреть идеальный газ как рабочее тело
        \item Затем достаточно вспомнить определение цикла Карно, модифицированное уравнение адиабаты и работу на
        изотерме
    \end{enumerate}

    \subsection{Холодильная машина и тепловой насос}

    \textbf{Опр} \textit{Холодильный цикл} \textcolor{gray}{Имеет в результате потребление работы через отбирание ...}

    \textbf{Опр} \textit{Холодильная машина} \textcolor{gray}{Машина, работающая по холодильному циклу}

    \textbf{Опр} \textit{Тепловой насос} \textcolor{gray}{Машина для передачи тепла к более нагретому телу от ...}

    \subsection{Коэффициенты эффективности идеальной холодильной машины и идеального теплового насоса}

    \textbf{Опр} \textit{Эффективность холодильной машины} \textcolor{gray}{Отношение тепла холодильника к работе ...}

    Чтобы найти эффективность идеальной холодильной машины, надо воспользоваться теоремами Карно.
    Аналогично для идеального теплового насоса

    \section{Второе начало термодинамики.
    Энтропия (термодинамическое определение).
    Неравенство Клаузиуса.
    Энтропия идеального газа}

    \subsection{Второе начало термодинамики}

    \textbf{Опр} \textit{Машина Клаузиуса} \textcolor{gray}{Машина, работающая по круговому циклу, в результате ...}

    \textbf{Закон} \textit{Второе начало термодинамики в формулировке Клаузиуса}

    \textcolor{blue}{Машина Клаузиуса невозможна}

    \textbf{Опр} \textit{Машина Томсона} \textcolor{gray}{Машина, работающая по круговому циклу, в результате ...}

    \textbf{Закон} \textit{Второе начало термодинамики в формулировке Томсона (лорда Кельвина)}

    \textcolor{blue}{Машина Томсона невозможна}

    \textbf{Th} \textcolor{blue}{Формулировки Клаузиуса и Томсона эквивалентны}

    $\Leftarrow:$ производимую мТ работу можно целиком передать нагревателю, создав мК

    $\Rightarrow:$ рассмотрим две машины между заданными термостатами: обыкновенную и мК.
    Результат одновременной работы этих двух машин есть мТ

    \subsection{Энтропия (термодинамическое определение)}

    \textbf{Опр} \textit{Термодинамическая энтропия}

    \begin{enumerate}
        \item Рассматривается произвольный обратимый круговой процесс, проходящий через фиксированные точки, интеграл
        по циклу $\frac{\delta Q}{T}$
        \item Путём преобразований (и, возможно, неравенства Клаузиуса) доказывается, что его величина не зависит от
        пути между точками
        \item Тогда перед нами функция состояния по определению.
        Назовём её энтропией и будем обозначать как $S$
    \end{enumerate}

    \subsection{Неравенство Клаузиуса}

    \textbf{Утв} \textit{Неравенство Клаузиуса}

    \textcolor{blue}{$\oint \frac{\delta Q_i}{T_i} \leq 0$}

    Иногда данное неравенство записывают в дискретной форме.

    \begin{enumerate}
        \item По второму началу термодинамики получим, что данный интеграл в обратимых процессах эквивалентен $-\delta S = 0$ в
        силу того, что обратимыми являются лишь машины, работающие по циклу Карно, а для них данное равенство выполнено
        из выражения для КПД цикла Карно
        \item В случае неравновесного процесса, запишем его КПД и сравним с КПД цикла Карно
        \item Тогда получим, что интеграл Клаузиуса процесса будет меньше, чем интеграл Клаузиуса цикла Карно, то
        есть $\leq 0$
    \end{enumerate}

    \subsection{Энтропия идеального газа}

    Из I начала ТД и определения энтропии, получаем её выражение для ИГ

    \section{Обратимые и необратимые процессы.
    Закон возрастания энтропии.
    Неравновесное расширение газа в пустоту}

    \subsection{Обратимые и необратимые процессы}

    \textbf{Опр} \textit{(Не)обратимые процессы} \textcolor{gray}{Процесс (не)мб проведён в обратном направлении ...}

    \subsection{Закон возрастания энтропии}

    \textbf{Закон} \textit{Неубывания энтропии}

    Для его доказательство достаточно рассмотреть круговой процесс с обратимой и нет частью, воспользоваться
    определением энтропии и неравенством Клаузиуса

    \textbf{Утв} \textit{Постулат Гиббса}

    \textcolor{blue}{Энтропия максимальна в состоянии равновесия}

    \subsection{Неравновесное расширение газа в пустоту}

    $\Delta Q = 0$ в силу теплоизолированности, а $A = 0$, потому что не над чем совершать работу, поэтому и $\Delta
    T = 0$ по I началу ТД.
    Из выражения энтропии для ИГ $\Delta S = \nu R \ln \left(\frac{V_2}{V_1}\right) > 0$.
    По-другому, возрастание энтропии можно объяснить необратимостью процесса в замкнутой системе

    \section{Термодинамические потенциалы: внутренняя энергия, энтальпия, свободная энергия, термодинамический
    потенциал Гиббса.
    Метод получения соотношений Максвелла (соотношений взаимности)}

    \subsection{Термодинамические потенциалы: внутренняя энергия, энтальпия, свободная энергия, термодинамический
    потенциал Гиббса}

    \textbf{Опр} \textit{Термодинамические потенциалы} \textcolor{gray}{Функции определённых наборов ТД параметров, ...}

    Всего есть четыре основных ТД потенциала: внутренняя энергия, энтальпия ($+PV$), свободная энергия ($-TS$) и
    потенциал Гиббса (совокупность двух предыдущих).
    У каждого есть свой набор параметров, полный дифференциал, а также, конкретные частные производные.
    При желании, для их вычисления в случае ИГ можно указать явные формулы через его энтропию

    \subsection{Метод получения соотношений Максвелла (соотношений взаимности)}

    В силу того, что все ТД функции непрерывны, верна теорема Шварца о равенстве смешанных частных производных.
    Это позволяет получить четыре новых равенства, называемых соотношениями взаимности Максвелла

    \section{Фазовые переходы первого рода.
    Уравнение Клапейрона—Клаузиуса.
    Фазовое равновесие «жидкость—пар», зависимость давления насыщенного пара от температуры}

    \subsection{Фазовые переходы первого рода}

    \textbf{Опр} \textit{Фаза} \textcolor{gray}{Физически однородная часть системы, отличающаяся своими физическими ...}

    \textbf{Опр} \textit{Химический потенциал} \textcolor{gray}{Величина, определяющая изменение энергии системы ...}

    Химический потенциал можно приписать каждому из известных до этого ТД потенциалов

    \textbf{Опр} \textit{Экстенсивные и интенсивные параметры} \textcolor{gray}{Величины, пропорциональные и не- ...}

    \textbf{Утв} \textit{Условия равновесия фаз}

    \textcolor{blue}{В состоянии равновесия во всём веществе выполнены условия
        \begin{itemize}
            \item Механического равновесия ($P = const$)
            \item Теплового равновесия ($T = const$)
            \item Равновесия по отношению перехода частиц между различными фазами ($\mu = const$)
        \end{itemize}}

    Притом можно показать, что из выполнения первых двух условий следует третье

    В случае отсутствия равновесия частицы переходят в фазу с меньшим химическим потенциалом

    \subsection{Уравнение Клапейрона—Клаузиуса}

    Фазы могут существовать, если только давление и температура лежать на кривой фазового равновесия

    \textbf{Опр} \textit{Фазовая диаграмма} \textcolor{gray}{Координаты $(P, T)$, на которой нанесены различные ...}

    Из равенства химических потенциалов следует равенство их дифференциалов из потенциала Гиббса.
    Преобразуя равенство и введя теплоту фазового перехода в расчёте на одну частицу (энергозатраты на осуществление
    перехода), получим уравнение Клапейрона -- Клаузиуса

    Также можно показать, что введённая теплота смены фазы есть разница энтальпий

    \subsection{Фазовое равновесие «жидкость -- пар», зависимость давления насыщенного пара от температуры}

    Уравнение кривой равновесия «жидкость -- пар» может быть получено с помощью уравнения ИГ, факта, что $v_2 \gg
    v_1$ и в предположении $q = const$.
    В таком случае зависимость выражается в экспоненциальной форме в двух случаях задания $q$ (удельно или молярно)

    \textbf{Опр} \textit{Насыщенный пар} \textcolor{gray}{Пар, находящийся в равновесии с собственной жидкостью}

    \section{Фазовые диаграммы «твёрдое тело—жидкость—пар».
    Тройная точка, критическая точка}

    \subsection{Фазовые диаграммы «твёрдое тело—жидкость—пар»}

    Запишем уравнения кривых плавления, испарения и возгонки (сублимации).
    Данные уравнения есть равенства соответствующих химический потенциалов, притом только два уравнения независимы.
    То есть точка пересечения двух кривых принадлежит третьей.

    \subsection{Тройная точка, критическая точка}

    \textbf{Опр} \textit{Тройная точка} \textcolor{gray}{(Изолированная) точка пересечения трёх кривых, в которой ...}

    \textbf{Опр} \textit{Критическая точка}

    Кривая фазового равновесия может оборваться при высоких температурах, где исчезает различие между фазами.
    Точка данного события и есть критическая, притом она обязательно существует на кривой «жидкость -- пар»

    Для воды $T_{cr} = 647,3 K, T_3 = 273,16 K$

    Каждая кривая равновесия характеризуется своим значением теплоты фазового перехода.
    Если рассмотреть бесконечно малый цикл вблизи тройной точки, то из равенства Клаузиуса получим $q_{sb} = q_m + q_{v}$

    \section{Поверхностное натяжение.
    Коэффициент поверхностного натяжения, краевой угол.
    Смачивание и несмачивание.
    Формула Лапласа.
    Свободная энергия и внутренняя энергия поверхности}

    \subsection{Поверхностное натяжение}

    Все молекулы жидкости испытывают притяжение со стороны других молекул.
    Вблизи поверхности жидкости у молекулы в сфере молекулярного действия находится меньше молекул, к которым она
    притягивается, поэтому возникает сила, стремящаяся втянуть её с её поверхности внутрь жидкости

    \textbf{Опр} \textit{Поверхностное натяжение} \textcolor{gray}{Работа, необходимая для увеличения поверхности ...}

    \subsection{Коэффициент поверхностного натяжения, краевой угол}

    \begin{enumerate}
        \item Заметим, что в изотермическом процессе работа идёт на изменение свободной энергии: $F = F_V + F_s$, где
        первое слагаемое пропорционально объёму плёнки, а второе -- площади её поверхности
        \item Тогда коэффициент поверхностного натяжения $\sigma = \frac{F_s}{\Pi}$
        \item Другое выражение для $\sigma$ даётся через механическую работу силы $2f$ (двойка, потому как у плёнки
        есть две поверхности -- внешняя и внутренняя)
        \item Получим, что $\sigma$ есть сила, приходящаяся на единицу длины границы поверхности
    \end{enumerate}

    \textbf{Опр} \textit{Краевой угол смачивания} \textcolor{gray}{Угол между касаталеьной, проведённой к ...}

    Если рассмотреть участок на границе трёх сред (газа, плёнки и поверхности), то из равенства сил на этот участок
    получим выражение из определения коэффициента поверхностного натяжения получим выражение для краевого угла
    смачивания

    \subsection{Смачивание и несмачивание}

    Проанализируем выражение для данного угла $\frac{\sigma_{sg} - \sigma_{sl}}{\sigma_{gl}}$:
    \begin{itemize}
        \item $> 1:$ жидкость растекается по поверхности ТТ -- полное смачивание
        \item $<-1:$ жидкость принимает элипсообразную форму капли -- полное несмачивание
        \item $0 < \theta < \frac{\pi}{2}:$ частичное смачивание
        \item $\frac{\pi}{2} < \theta < \pi:$ частичное несмачивание
    \end{itemize}

    \subsection{Формула Лапласа}

    Если рассмотреть небольшую часть сферической поверхности жидкости и использую определения коэффициента
    поверхностного натяжения, радиуса кривизны, площади и воспользоваться малостью угла, то можно получить формулу
    Лапласа, дающей численное выражение разности давления жидкости и газа над поверхностью

    \subsection{Свободная энергия и внутренняя энергия поверхности}

    Запишем выражение для свободной энергии поверхности, воспользовавшись смыслом частных производных и коэффициента
    поверхностного натяжения.
    Получим выражение для поверхностной внутренней энергии

    \section{Зависимость давления насыщенного пара от кривизны поверхности жидкости.
    Роль зародышей в образовании фазы.
    Кипение}

    \subsection{Зависимость давления насыщенного пара от кривизны поверхности жидкости}

    \begin{enumerate}
        \item Рассмотрим хитрый сосуд с плоской частью и капилляром
        \item Используя формулы Торичелли, удельного объёма, Лапласа и барометрическую, получим выражение для
        логарифма отношения давлений через давление.
        \item Данная формула неявная, поэтому чтобы получить аналитичность в случае малой разности, разложим логарифм
        в ряд и получим более простую формулу $P = P_0 + \frac{\nu_l}{\nu_s - \nu_l}\sigma K$
    \end{enumerate}

    Рассмотрим толщу воды при $T \approx T_{lg}$

    \begin{enumerate}
        \item Пусть в толще жидкости образовался пузырёк.
        Тогда запишем его давление через новую формулу и через формулу Лапласа (условие равновесия).
        Равенство достигается в случае критического радиуса пузырька
        \item Если радиус пузыря меньше, то он схлопнется, а если больше -- продолжит расти
    \end{enumerate}

    \subsection{Роль зародышей в образовании фазы}

    \begin{enumerate}
        \item Аналогично выражению для радиуса пузырька, можно получить критический радиус капли в процессе
        конденсации.
        Он будет в $\frac{\nu_l}{\nu_s}$ больше
        \item И вновь, если радиус капли меньше критического, то она схлопнется, а если больше -- то начнётся её рост
        \item Подобные пузырьки и капли могут образовываться около песчинок, взвеси, трещинок и других неровностей
        поверхностей и среды.
    \end{enumerate}

    \subsection{Кипение}

    Если такие неровности достаточно большие, то начнётся кипение

    \textbf{Опр} \textit{Кипение} \textcolor{gray}{Фазовый переход «жидкость -- пар», происходящий с образованием ...}

    \section{Уравнение Ван-дер-Ваальса как модель неидеального газа.
    Изотермы газа Ван-дер-Ваальса.
    Критические параметры.
    Приведённое уравнение Ван-дер-Ваальса, закон соответственных состояний}

    \subsection{Уравнение Ван-дер-Ваальса как модель неидеального газа}

    Модель Ван-дер-Ваальса учитывает две особенности реального газа: это наличие объёма у молекул и их взаимное
    притяжение друг к другу.
    Отсюда следует необходимость введения двух новых параметров-констант $a$ и $b$

    \begin{enumerate}
        \item Учтём запрещённый объём для каждой молекулы введя $b: V^{'} = V - \nu b$
        \item Чтобы учесть притяжение между ними, рассмотрим нейтральную молекулу газа
        \item При сближении с другой нейтральной, они начинают ориентироваться разнонаправленно; между ними возникает
        сила ВдВ
        \item Выразим давление из промежуточного уравнения ВдВ.
        Конечное давление на стенки сосуда будет меньше давления в случае ИГ, потому как часть частиц притягивается и
        сталкивается между собой
        \item Оценим эту разница через череду пропорциональностей: $\Delta P \sim F \sim n \cdot n \sim n^2 \sim \frac{1}{V^2}$
        \item Чтобы записать равенство, введём $a: \Delta P = \frac{a \nu^2}{V^2}$
        \item Приведя промежуточное равенство с учётом поправок к нормальному виду, получим уравнение ВдВ
    \end{enumerate}

    \subsection{Изотермы газа Ван-дер-Ваальса}

    \begin{enumerate}
        \item Начнём изображать изотермы Ван-дер-Ваальса в координатах $P - V$: они имеют вид кубического трёхчлена
        \item Найти координаты точек экстремумов можно из уравнения $(\frac{\partial P}{\partial V})_T = 0$, а если
        приравнять эту производную к давлению, то можно получить уравнение кривой, соединяющей все такие точки --
        спинодали
        \item При увеличении температуры минимум и максимум сольются в одну точку перегиба, а после данной точки
        будут походить на изотермы ИГ (гипербола)
    \end{enumerate}

    \subsection{Критические параметры}

    Чтобы найти эту критическую точку и её параметры ($P, V, T$) мы имеем три уравнения:

    \begin{enumerate}
        \item первая производная равна нулю (минимум и максимум -- экстремум -- слились в ней)
        \item вторая производная равна нулю (точка перегиба)
        \item уравнение Ван-дер-Ваальса
    \end{enumerate}

    Другой способ получения параметров -- записать куб разности через формулу сокращённого умножения

    \subsection{Приведённое уравнение Ван-дер-Ваальса, закон соответственных состояний}

    Если ввести новые переменные -- отношения текущих параметров к критическим и подставить их в уравнение ВдВ, а
    затем подставить выражения для критических параметров и выполнить преобразования, то получим приведённое
    уравнение ВдВ.
    Из него следует

    \textbf{Закон} \textit{Соответственных состояний}

    \textcolor{blue}{Для различных веществ одинаковым значениям $\varphi$ и $\pi$ соотвествует лишь одно (и то же)
        значение $\tau$}

    \section{Метастабильные состояния: переохлаждённый пар, перегретая жидкость (на примере модели Ван-дер-Ваальса).
    Изотермы реального газа, правило Максвелла и правило рычага}

    \subsection{Метастабильные состояния: переохлаждённый пар, перегретая жидкость (на примере модели Ван-дер-Ваальса)}

    \begin{enumerate}
        \item Из уравнения ВдВ и вида изотерм такого газа можно сделать вывод, что одному значению давления могут
        соответствовать разные значения объёма
        \item То есть существуют термодинамически неустойчивые состояния (действительно, мы расширяем газ, а он греется)
        \item Таким образом, на изотермах ВдВ можно различить четыре вида состояний: стабильные, метастабильные,
        термодинамически неустойчивые, а также, устойчивую смесь жидкой и парообразной фазы (прямая бинодаль)
    \end{enumerate}

    \subsection{Изотермы реального газа, правило Максвелла и правило рычага}

    Изотермы реального газа имеют две фазы на участке бинодали и одну выше критической точки

    \textbf{Утв} \textit{Правило Максвелла}

    \textcolor{blue}{Кривая термодинамически нейстойчивого участка пересекает прямую устойчивой смеси так, чтобы
    полученные площади были равны}

    \begin{enumerate}
        \item Рассмотрим квазистатический цикл через эти точки.
        Из равенства Клаузиуса следует, что $\delta Q = 0$ (мы двигались по изотерме, поэтому $T = const$)
        \item Из возврата в ту же точку следует, что $dU = 0 \Rightarrow A = 0$.
        \item Так как в данном цикле мы проходим два круговых участка, притом один по часовой (работа положительна),
        а второй против (отрицательна), то суммарная работа есть ноль только в случае равенства площадей (работа есть
        ориентированная площадь под графиком)
    \end{enumerate}

    Найдём соотношение между количество жидкости и газа для их устойчивой смеси.
    Для этого запишем объём жидкой и газообразной части через значения плотности на концах участках и воспользуемся ЗСМ.
    Получим, что массы жидкости и пара соотносятся по правилу рычага

    \section{Внутренняя энергия и энтропия газа Ван-дер-Ваальса.
    Равновесное и неравновесное расширение газа Ван-дер-Ваальса в теплоизолированном сосуде}

    \subsection{Внутренняя энергия и энтропия газа Ван-дер-Ваальса}

    \begin{enumerate}
        \item Для начала рассмотрим внутреннюю энергию как $U(T, V)$.
        Запишем её полный дифференциал и воспользуемся соотношением Максвелла
        \item В итоге получим термическое уравнение состояния
        \item Теперь получим выражение для внутренней энергии газа ВдВ как $U(T, V)$ через её полный дифференциал
        \item Воспользуемся определениями $c_V$, только термическим уравнением состояния, выражением для
        $(\frac{\partial P}{\partial T})_V$, уравнением ВдВ
        \item Подставляя все выкладки, получим требуемое выражение.
        Заметим, что внутренняя энергия газа ВдВ определена с точностью до константы
        \item Найдём энтропию газа ВдВ как $S(T, V)$ вновь через её полный дифференциал
        \item Распишем каждый частичный дифференциал (частную производную), использую выкладки для внутренней энергии
        и I начало ТД
        \item Вновь получим функцию, определённую с точностью до константы
    \end{enumerate}

    \subsection{Равновесное и неравновесное расширение газа Ван-дер-Ваальса в теплоизолированном сосуде}

    \begin{enumerate}
        \item Рассмотрим свободное расширение газа в вакуум в неравновесном процессе и найдём изменение температуры
        такого газа.
        \item $\Delta Q = 0$ в силу теплоизолированности, а $A = 0$, потому что не над чем совершать работу, поэтому
        и $\Delta U = 0$ по I началу ТД
        \item $\Rightarrow U_i = c_V T_i - \frac{a}{V_i}$.
        Выразим из $U_1 = U_2$ разность температур и получим её отрицательность, то есть газ охладился.
%        \item Это можно объяснить совершением работы против сил притяжения молекул за счёт кинетической энергии молекул
    \end{enumerate}

    Теперь рассмотрим равновесное адиабатическое расширение газа ВдВ, записав изменение его энтропии.
    Получим псевдо ЭДТ

    \section{Течение идеальной жидкости.
    Уравнение Бернулли сжимаемой и несжимаемой жидкости.
    Изоэнтропическое истечение газа из отверстия}

    \subsection{Течение идеальной жидкости. Уравнение Бернулли сжимаемой и несжимаемой жидкости}

    Идеальная жидкость течёт без трения и теплопередач

    \begin{enumerate}
        \item Для каждого сечения $\rho vS = const$ в силу ЗСМ
        \item Записав работу по смещению как разность энергий, получим константную сумму (энтальпия)
        \item Перейдя к удельным величинам и раскрыв состав удельной энергии (кинетическая + потенциальная), получим
        уравнение Бернулли:
        \[ \frac{P}{\rho} + gh + \frac{v^2}{2} + u = const \]
        \item Уравнение для несжимаемой жидкости не будет иметь слагаемого с плотностью (она не изменяется)
    \end{enumerate}

    \subsection{Изоэнтропическое истечение газа из отверстия}

    \begin{enumerate}
        \item Найдём скорость истечения газа из отверстия в условиях малого перепада высот $\iota_i + \frac{v^2}{2} = const$
        \item Считая скорость внутри сосуда пренебрежимой, получим выражение через удельную энтальпию
        \item Вспомнив определение энтальпии и формулу Майера, получим более конкретную формулу; при желании из-под
        корня можно вынести скорость звука
        \item При адиабатическом истечении ИГ можно переписать отношение температур как отношение давлений и получить
        немного другую запись.
        Из неё видно, что скорость газа максимальна при расширении в вакуум $P_2 = 0$, притом $v > c_{sound}$
    \end{enumerate}

    \section{Эффект Джоуля—Томсона.
    Дифференциальный эффект Джоуля–Томсона для газа Ван-дер-Ваальса, температура инверсии}

    \subsection{Эффект Джоуля—Томсона}

    \textbf{Опр} \textit{Дроссель} \textcolor{gray}{Местное препятствие газовому потоку}

    \textbf{Опр} \textit{Дросселирование} \textcolor{gray}{Медленное протекание газа под действием постоянного ...}

    \textbf{Опр} \textit{ЭДТ} \textcolor{gray}{Изменение температуры газа при адиабатическом дросселировании}

    Существует положительный и отрицательный ЭДТ (напомним, разность давлений всегда отрицательна).
    Данный процесс называют изоэнтальпическим (его можно описать с помощью уравнения Бернулли) из-за его медленности.
    Для ИГ он не наблюдается, потому как из постоянства энтальпии следует постоянство температуры.
    Таким образом, ЭДТ позволяет определить степень неидеальности газа

    \subsection{Дифференциальный эффект Джоуля–Томсона для газа Ван-дер-Ваальса, температура инверсии}

    \textbf{Опр} \textit{Дифференциальный ЭДТ} \textcolor{gray}{ЭДТ при малых перепадах давления (т.е. ещё и изобарный)}

    \begin{enumerate}
        \item Представим энтальпию как $I(T, P)$ и выразим $(\frac{\partial T}{\partial P})_I$ через условие $dI = 0$
        \item Числитель выразим через стандартное определение энтальпии и с помощью соотношений Максвелла, а
        знаменатель -- по определению
        \item Далее выразим $(\frac{\partial V}{\partial T})_P$ через формулу $3 = -1$ и получим общий вид ЭДТ
        \item В случае газа ВдВ, подставим значения соответствующих частных производных
        \item Найдём температуру инверсии, приравняв $(\frac{\partial T}{\partial P})_I$ к нулю.
        При меньших температурах имеем охлаждение, а при больших -- нагрев газа при дросселировании
    \end{enumerate}

    \section{Распределение частиц идеального газа по проекциям и модулю скорости (распределение Максвелла).
    Наиболее вероятная, средняя и среднеквадратичная скорости.
    Распределение Максвеллапо энергиям}

    \subsection{Распределение частиц идеального газа по проекциям и модулю скорости (распределение Максвелла)}

    Число молекул в ИГ со средней плотностью $n$, обладающими скоростями в интервале $[v, v + dv]$, определяется
    распределением Максвелла

    Интегрированием по направлениям скорости сводится к замене третьего дифференциала, что приводит к распределению
    по абсолютной величине скорости

    \subsection{Наиболее вероятная, средняя и среднеквадратичная скорости}

    Найдём максимум формулы распределения путём дифференцирования и приравнивания к нулю.
    Получим наиболее вероятную скорость частицы

    Вопрос о среднем значении случайной знаковой величины бессмысленен -- это ноль.
    Если рассматривать лишь положительные значения, то благодаря специальной формуле, получим среднюю скорость.
    Аналогичным образом получается среднеквадратичная скорость

    \subsection{Распределение Максвелла по энергиям}

    Для получения распределения по энергиям, достаточно заменить переменную в распределении Максвелла по скоростям

    \section{Среднее число молекул, сталкивающихся в единицу времени с единичной площадкой.
    Средняя энергия молекул, вылетающих в вакуум через малое отверстие}

    \subsection{Среднее число молекул, сталкивающихся в единицу времени с единичной площадкой}

    \begin{enumerate}
        \item Рассмотрим столкновения газа с неподвижной стенкой и выделим группу молекул со скоростью $v$ плотностью
        $dn(v)$
        \item В соответствующий телесный угол летит лишь доля молекул $\frac{d\Omega}{4\pi}$, а за время $dt$ до
        поверхности долетят лишь молекулы в объёме $v_x Sdt = v \cos \theta Sdt$
        \item Последовательно суммируем по всем углам и по все скоростям, деля промежуточный результат на $Sdt$,
        чтобы получить поток частиц (число в единицу времени и единицу площади)
    \end{enumerate}

    \subsection{Средняя энергия молекул, вылетающих в вакуум через малое отверстией}

    Чтобы найти эту величину, надо посчитать полную уносимую энергию и разделить её на полный поток (отношение
    интегралов)

    \section{Распределение Больцмана в поле внешних сил.
    Барометрическая формула}

    \subsection{Распределение Больцмана в поле внешних сил}

    Поместим газ в потенциальное поле сил $n = n(z)$.
    Если мы захотим посчитать среднее число частиц $dN$ в объёме $dV$ со скоростями в $d^3 v$, то получим
    распределение Максвелла -- Больцмана, где нормировочная константа определяется из условия $\int dN = N$, с $n_0
    = n(v = 0)$

    \subsection{Барометрическая формула}

    \begin{enumerate}
        \item Рассмотрим цилиндрик газа в поле тяжести и запишем для него второй закон Ньютона
        \item Вспомним основную формулу МКТ, разделим переменные и проинтегрируем
        \item В конце перейдём от $n$ к $P$ и получим барометрическую формулу в произвольном потенциальном поле
    \end{enumerate}

    \section{Статистика классических идеальных систем.
    Микро- и макросостояния.
    Статистический вес.
    Распределение Гиббса для идеального газа (без вывода)}

    \subsection{Статистика классических идеальных систем}

    Существует два постулата статистического описания больших систем:

    \begin{enumerate}
        \item Все разрешённые микросостояния равновероятны
        \item Термодинамически равновесным является то микросостояние, которое реализуется наибольшим числом
        микросостояний, то есть является наиболее вероятным состоянием
    \end{enumerate}

    \subsection{Микро- и макросостояния}

    \textbf{Опр} \textit{Микросостояние} \textcolor{gray}{Состояние системы, определяемое одновременным заданием ...}

    \textbf{Опр} \textit{Макросостояние} \textcolor{gray}{Состояние системы, характеризуемое небольшим числом ...}

    Одно макросостояние мб реализовано большим числом микросостояний за счёт перестановки частиц не меняющей
    наблюдаемого состояния

    \subsection{Статистический вес}

    \textbf{Опр} \textit{Фазовое пространство} \textcolor{gray}{Пространство частиц, координат и импульсов системы}

    Если разбить всё фазовое пространство на ячейки такого объёма, что на каждую будет приходиться лишь одно
    микросостояние, то число состояний в объёме окажется равным $dG = \frac{d\Gamma}{\Gamma_0}$.
    Различные ячейки отвечают разным микросостояниям, но могут отвечать одному макросостоянию

    \textbf{Опр} \textit{Статистический вес} \textcolor{gray}{Число микросостояний, реализующих данное макросостояние}

    $G = \frac{N!}{N_1! \dots N_m!}$

    \subsection{Распределение Гиббса для идеального газа (без вывода)}

    Чтобы найти термодинамическое равновесное состояние, нужно найти условия максимума $G$.
    В этом нам поможет распределение Гиббса

    \textbf{Опр} \textit{Явление вырождения} \textcolor{gray}{Случай обладания системой одной и той же энергией в ...}

    \textbf{Опр} \textit{Кратность вырождения} \textcolor{gray}{Число состояний, обладающих одним и тем же ...}

    Выше предполагалось, что энергия частицы (подсистемы) пробегает дискретный набор значений, но результат можно
    обобщить и на непрерывный ряд значений.
    Для этого достаточно приписать частице значение энергии ячейке в фазовом пространстве с объёмом $d\Gamma = dr dp$.
    В этих условиях распределение Гиббса немного изменится

    \section{Статистические определения энтропии и температуры.
    Аддитивность энтропии.
    Закон возрастания энтропии.
    Третье начало термодинамики}

    \subsection{Статистические определения энтропии и температуры}

    \textbf{Утв} \textcolor{blue}{Энтропия определяется формулой Больцмана $S = klnG$}

    Пусть подсистема с энергией $E$ находится в состоянии, близком к равновесному (энтропия максимальна со значением $S$).
    Имеет место связь $S = S(E)$ (если энергия меняется, то меняется и энтропия)

    \textbf{Опр} \textit{Статистическая температура} \textcolor{gray}{$\frac{1}{T} = \frac{dS}{dE}$ в предположении ...}

    Отсюда можно выразить энтропию через сумму, а также, дифференциал энтропии

    \subsection{Аддитивность энтропии}

    \begin{enumerate}
        \item Разобьём систему на две подсистемы.
        Новые микросостояния будут меняться независимо
        \item Статвес системы равен произведению чисел способов, которым могут быть осуществлены состояния каждой
        новой подсистемы: $G = G_1 G_2$
        \item По свойствам логарифма, отсюда следует аддитивность энтропии, то есть энтропия системы есть сумма
        энтропий её подсистем
    \end{enumerate}

    \subsection{Закон возрастания энтропии}

    \textbf{Th} \textit{Закон возрастания энтропии}

    \textcolor{blue}{Среди всех направлений эволюции системы предпочтительным является то, при котором вероятность
    конечного состояния оказывается наибольшей}

    \textbf{Следствие 1} \textcolor{blue}{С наибольшей вероятностью энтропия замкнутой системы растёт (не убывает)}

    \textbf{Следствие 2} \textcolor{blue}{В состоянии термодинамического равновесия энтропия максимальна}

    \subsection{Третье начало термодинамики}

    \textbf{Закон} \textit{III начало ТД}
    \textcolor{blue}{
        \begin{enumerate}
            \item При приближении к абсолютному нулю, энтропия стремится к конечному значению $S_0$
            \item Все процессы при абсолютном нуле, переводящие систему из одного равновесного состояния
            в другое, происходят без изменения энтропии
        \end{enumerate}}

    Энтропию в нуле можно положить нулевой и использовать как предел.
    Третье начало (по-другому -- теорема Нернста) есть экспериментальный факт, верный для всех чистых кристаллических
    веществ, квантовых жидкостей и газов


    \section{Изменение энтропии при смешении газов, парадокс Гиббса}

    \begin{enumerate}
        \item Рассмотрим два разных смешения ИГ, для чего можно использовать привычную формулу для энтропии
        \item Запишем изменение энтропии в случае разных молекул (необратимый процесс смешения).
        Можно даже явно посчитать энтропию и показать её положительность
        \item В случае одинаковых газов ничего как будто и не произошло: $\Delta S = 0$
        \item А если газы будут совсем немного отличаться, то энтропия смешения будет ноль или не ноль?
        Парадокс
    \end{enumerate}

    Парадокс разрешим.
    В реальности данный предельный переход не выполним: либо газы различны, либо нет, поэтому и энтропия либо $>0$,
    либо равна $0$

    \section{Классическая теория теплоёмкостей.
    Закон равномерного распределения энергии теплового движения по степеням свободы.
    Теплоёмкость кристаллов (закон Дюлонга—Пти)}

    \subsection{Классическая теория теплоёмкостей}

    \begin{enumerate}
        \item Как было выяснено с помощью распределения Максвелла, средняя энергия поступательного движения молекул
        есть $\frac{3}{2} kT$
        \item В таком случае в силу наличия трёх поступательных СС получим энергию в $\frac{kT}{2}$ и теплоёмкость
        $\frac{k}{2}$ на одну СС.
        Молярная теплоёмкость будет равна $\frac{3}{2}R$
        \item Теперь рассмотрим молекулу как ТТ с тремя главными моментами инерции и запишем её полную энергию
        вращательного движения
        \item Из распределения Гиббса следует, что $\overline{w^2_i} = \frac{kT}{I_i}$, откуда полная вращательная
        энергия есть $\frac{3}{2} kT$
        \item В случае линейной молекулы ($I_3 = 0$) получим полную энергию $kT$
        \item Для колебательной СС из механики мы знаем, что средняя кинетическая энергия при колебаниях равна
        средней потенциальной, поэтому на каждую такую СС приходится энергия $\frac{kT}{2} + \frac{kT}{2} = kT$
        \item Итого, молекула из $N$ атомов имеет $3N$ СС, из которых по три поступательные и вращательные и $3N - 6$
        колебательных (в случае линейной молекулы $3N - 5$ из-за уменьшения количества колебательных СС)
    \end{enumerate}

    \subsection{Закон равномерного распределения энергии теплового движения по степеням свободы}

    \textbf{Закон} \textit{РРЭСС}

    \textcolor{blue}{Если макроскопическая система подчиняется законам классической механики, то на каждое слагаемое
    в энергии, квадратично зависящее от координат / скоростей молекул, приходится энергия $\frac{kT}{2}$ и
    теплоёмкость $\frac{k}{2}$}

    \subsection{Теплоёмкость кристаллов (закон Дюлонга — Пти)}

    \textbf{Закон} \textit{Дюлонга -- Пти}

    \textcolor{blue}{ТТ (кристалл) представляет собой совокупность атомов, находящихся в окрестности своего п.р.
    В нём эти атомы могут совершать колебания в трёх направлениях, так что каждый обладает средней энергией в $3kT$.
    Отсюда молярная теплоёмкость кристаллов $c_V = 3R$}

    \section{Зависимость теплоёмкости $c_V$ газов от температуры.
    Возбуждение и замораживание степеней свободы, характеристические температуры}

    \subsection{Зависимость теплоёмкости $c_V$ газов от температуры}

    График зависимости выглядит плавно-ступенчато, с разрывами ближе к нулю и последней ступенькой в точке
    $(10^3 K; \frac{7}{2}R)$

    \subsection{Возбуждение и замораживание степеней свободы, характеристические температуры}

    Поступательные СС появляются $T = 0 K$ (при всех положительных температурах).
    Вращательные СС появляются при $T \sim 100 K$.
    Колебательные СС появляются при $T \sim 1000 K$

    Характеристические температуры берутся из квантовой механики и связаны с энергией молекулы, жёсткостью связи и
    приведённой массой

    \section{Флуктуации в термодинамических системах.
    Влияние флуктуаций на чувствительность измерительных приборов (на примере пружинных весов)}

    \subsection{Флуктуации в термодинамических системах}

    \textbf{Опр} \textit{Флуктуация} \textcolor{gray}{Случайное отклонение физической величины от её среднего значения}

    \textbf{Опр} \textit{Среднеквадратичная флуктуация} \textcolor{gray}{Среднее квадрата флуктуации}

    \textbf{Опр} \textit{Дисперсия} \textcolor{gray}{Корень среднеквадратичного отклонения}

    \textbf{Опр} \textit{Относительная среднеквадратичная флуктуация} \textcolor{gray}{Отношение дисперсии к среднему}

    Во многих случаях флуктуации физической величины имеют гауссово распределение

    \subsection{Влияние флуктуаций на чувствительность измерительных приборов (на примере пружинных весов)}

    Рассмотрим потенциальную энергию пружинных весов.
    С одной стороны, она обусловлена энергией пружинки, с другой -- ТРРЭСС, что позволяет нам оценить флуктуацию
    их показаний и минимальную массу для измерения

    \section{Зависимость флуктуаций от числа частиц, составляющих систему.
    Флуктуация числа частиц в выделенном объёме}

    \subsection{Зависимость флуктуаций от числа частиц, составляющих систему}

    \begin{enumerate}
        \item Рассмотрим (экстенсивную) термодинамическую величину.
        Чтобы найти её флуктуацию, распишем среднее квадрата её флуктуации для каждой подсистемы, используя
        статистическую независимость
        \item Получим знакомый результат: квадрат среднего минус среднее в квадрате, то есть среднеквадратичное
        отклонение
        \item Умножим в конце на число элементарных подсистем $N$ и получим искомую флуктуацию
        \item В случае интенсивной величины рассуждения будут похожи, однако среднее (равно как и переход от частей
        к целому) будет считаться по другому.
        Также нам предстоит сразу считать среднеквадратичное отклонение
    \end{enumerate}

    \subsection{Флуктуация числа частиц в выделенном объёме}

    \begin{enumerate}
        \item Рассмотрим участок сосуда малого объёма $\nu$, содержащего $n$ частиц и посчитаем энтропию такой системы
        \item Запишем условие равновесия такой системы (концентрация внутри участка и вне его одинакова)
        \item Тогда можно приближённо вычислить изменение энтропии, разложив её по формуле Тейлора
        \item Отсюда следует, что искомая флуктуации числа частиц есть $\sqrt{n}$
    \end{enumerate}

    \section{Связь вероятности флуктуации и энтропии системы.
    Флуктуации температуры в заданном объёме.
    Флуктуация объёма в изотермическом и адиабатическом процессах}

    \subsection{Связь вероятности флуктуации и энтропии системы}

    Вероятность пропорциональна $\exp \left(\frac{\Delta P \Delta V - \Delta T \Delta S}{2kT}\right)$

    \subsection{Флуктуации температуры в заданном объёме}

    \begin{enumerate}
        \item Пусть система отделена от внешней среды жёсткой ($V = const$) теплопроводящей оболочкой.
        Запишем среднеквадратичное отклонение энергии системы (результат выводится через статсуммы)
        \item Пусть в результате флуктуаций в подсистему поступило количество теплоты $\delta Q$.
        Тогда $\delta T = \frac{\delta Q}{c_V}$, откуда, с учётом определения $c_V$, найдём и среднеквадратичное
        отклонение температуры
    \end{enumerate}

    \subsection{Флуктуация объёма в изотермическом и адиабатическом процессах}

    \begin{enumerate}
        \item Введём силу, действующую на газ под поршнем в изотермическом процессе с одной стороны как $S \Delta P$
        (с использованием частной производной), а с другой подобно пружинке, чью жёсткость выразим через равенство
        потенциальной и температурной энергий
        \item Выразим среднеквадратичное отклонение объёма как $S^2 x^2$
        \item Воспользуемся постоянством температуры, выразим нужную частную производную
        \item Подставляем и находим нужную среднеквадратичную флуктуацию
        \item В адиабатическом процессе отличие будет лишь в виде частной производной (вместо $PV = const$ теперь $PV^\gamma = const$)
    \end{enumerate}

    \section{Столкновения. Эффективное газокинетическое сечение.
    Длина свободного пробега.
    Частота столкновений молекул между собой}

    \subsection{Столкновения. Эффективное газокинетическое сечение}

    \textbf{Опр} \textit{Эффективное газокинетическое сечение молекулы} \textcolor{gray}{Площадь поперечного ...}

    Молекулы провзаимодействуют, если центр какой-либо молекулы попадёт в этот цилиндр.
    Формулы немного отличается для случая, когда молекула движется в среде из молекулы другого размера

    \subsection{Длина свободного пробега}

    В цилиндре, закреплённом за каждой молекулой, находится по $\sigma \lambda n$ частиц.
    Хотя бы одно столкновение произойдёт, если эта величина станет равна единице.
%    (в случае среды с тяжёлыми частицами)
    При желании можно найти время свободного пробега (поделить длину на скорость) и относительно уточнить данную формулу

    \subsection{Частота столкновений молекул между собой}

    Найдём частоту столкновений $f_i$ \underline{одной} молекулы с другими.
    Если в единице объёма находится $n$ молекул, то всего будет произведено $\frac{1}{2}n \frac{1}{\tau}$ столкновений.
    Данное число можно выразить как через микро-, так и через макропараметры

    \section{Диффузия: закон Фика, коэффициент диффузии. Дифференциальное уравнение одномерной диффузии.
    Коэффициент диффузии в газах}

    \subsection{Диффузия: закон Фика, коэффициент диффузии. Дифференциальное уравнение одномерной диффузии}

    \textbf{Опр} \textit{Средняя скорость течения газа} \textcolor{gray}{$\overline{u} = \frac{1}{N} \sum_i v_i$}

    Суммирование производится по всем молекулам в единице объёма

    \textbf{Опр} \textit{Плотность потока} \textcolor{gray}{$\overline{j} = n\overline{u}$}

    \textbf{Опр} \textit{Диффузия} \textcolor{gray}{Неравновесный процесс пространственного перераспределения ...}

    \textbf{Опр} \textit{Относительная концентрация компонентов} \textcolor{gray}{$c_i = \frac{n_i}{n}, n = \sum_i n_i$}

    \textbf{Закон} \textit{Фика}

    В обычном случае применяется коэффициент диффузии, но если $n = const$, то имеем взаимную диффузию и
    соответствующий коэффициент.
    Также существуют поправки в случае ненулевой скорости течения газов: $j_1 + j_2 = n \overline{u}$.
    В трёхмерном случае вводится градиент концентрации

    Вышесказанное позволяет записать дифференциальное уравнение (одномерной диффузии)

    \textbf{Опр} \textit{Самодиффузия} \textcolor{gray}{Диффузия частиц в среде из частиц того же сорта ...}

    Этот процесс можно изучать только если часть частиц как-то помечена

    \subsection{Коэффициент диффузии в газах}

    Найдём значение коэффициента диффузии через рассмотрение переноса молекул вдоль оси.
    Сравнивая полученное выражение с законом Фика, получим $D = \frac{1}{3} \overline{v}\lambda$; при
    желании, его можно расписать по-подробнее

    \section{Теплопроводность: закон Фурье, коэффициент теплопроводности. Дифференциальное уравнение одномерной
    теплопроводности.
    Коэффициент теплопроводности в газах}

    \subsection{Теплопроводность: закон Фурье, коэффициент теплопроводности}

    \textbf{Опр} \textit{Теплопроводность} \textcolor{gray}{Неравновесный процесс, вид передачи тепла от более ...}

    \textbf{Опр} \textit{Плотность потока тепла $q$} \textcolor{gray}{Количество тепловой энергии, пересекающей ...}

    Поток тепла направлен в сторону убывания температуры (это обуславливает знак минус)

    \textbf{Закон} \textit{Фурье}

    В нём используется коэффициент теплопроводности.
    В трёхмерном случае вводится градиент температуры

    Вышесказанное позволяет записать дифференциальное уравнение (одномерной теплопроводности)

    Иногда используют коэффициент температуропроводности $a = \frac{\varkappa}{c_V}$, где $c_V$ -- теплоёмкость
    вещества на единицу объёма

    \subsection{Коэффициент теплопроводности в газах}

    \begin{enumerate}
        \item Найдём значение коэффициента теплопроводности через рассмотрение переноса тепла вдоль оси в условии
        перемещения газа как целого ($N_{up} = N_{down}$)
        \item Для этого надо записать энергию одной молекулы в фиксированной точке (используя $c^1_V$)
        \item Получим непонятное выражение для потока и перейдём к градиенту температур
        \item Сравнивая полученное выражение с законом Фурье, получим $\varkappa = \frac{1}{3} n \overline{v} \lambda
        c_V = n c^m_V D$; при желании, его можно расписать по-подробнее
    \end{enumerate}

    \section{Вязкость: закон Ньютона, коэффициенты динамической и кинематической вязкости.
    Коэффициент вязкости в газах}

    \subsection{Вязкость: закон Ньютона, коэффициенты динамической и кинематической вязкости}

    \textbf{Опр} \textit{Вязкость} \textcolor{gray}{Перенос тангенциальной компоненты импульса в направлении ...}

    \textbf{Закон} \textit{вязкого трения Ньютона}

    В нём используется коэффициент вязкости или динамическая вязкость.
    В трёхмерном случае вводится градиент скорости

    Иногда используют кинематическую вязкость $\nu = \frac{\eta}{\rho}$, где $c_V$ -- теплоёмкость вещества на
    единицу объёма

    \subsection{Коэффициент вязкости в газах}

    \begin{enumerate}
        \item Зафиксируем среднюю скорость упорядоченного движения молекул в условии равенства потоков молекул
        \item Запишем результирующий импульс, приобретаемый верхним слоем за время $\tau$
        \item Делим этот импульс на $\tau$ и сравниваем полученное выражение с законом вязкого трения Ньютона,
        получим $\eta = \frac{1}{3} mn \overline{v} \lambda c_V = \rho D \Rightarrow \nu = \frac{\eta}{\rho} = D$
    \end{enumerate}

    \section{Диффузия как процесс случайных блужданий. Закон смещения частицы при диффузии (закон
    Эйнштейна—Смолуховского).
    Скорость передачи тепла при теплопроводности}

    \subsection{Диффузия как процесс случайных блужданий. Закон смещения частицы при диффузии (закон
    Эйнштейна—Смолуховского)}

    \textbf{Опр} \textit{Подвижность} \textcolor{gray}{$B = \frac{\overline{v}}{F_{fr}}$}

    \textbf{Опр} \textit{Формула Стокса} \textcolor{gray}{$B = \frac{1}{6 \pi R \eta}$}

    \textbf{Закон} \textit{Эйнштейна—Смолуховского}

    \textcolor{blue}{Маленькая частица, движущаяся в среде, испытывает действие двух типов сил: силы торможения за
    счёт вязкого трения и флуктуационной силы со стороны молекул среды, чьё среднее действие есть ноль}

    \begin{enumerate}
        \item Посчитаем полное смещение частицы после $N$ шажков.
        Наивный подход даст нулевой результат
        \item Однако если посмотреть квадрат смещения, то сумма шажков разобьётся на сумму квадратичных и
        перекрёстных слагаемых, притом сумма последних равна нулю в силу их независимости
        \item Получили среднеквадратичное отклонение.
        Заменим число частиц на $\frac{t}{\tau}$ и пристальным взглядом увидим удвоенный коэффициент диффузии
        (удвоенность следует из интегрирования ослабляющегося потока частиц)
        \item Полученная формула и есть ЗЭС, обобщаемый на случай больших размерностей
    \end{enumerate}

    Данный закон показывает, что броуновское движение частиц аналогично процессу диффузии

    \subsection{Скорость передачи тепла при теплопроводности}

    \begin{enumerate}
        \item Рассмотрим процесс передачи тепла в одномерном случае и запишем соответствующие уравнения
        \item После пары преобразований, получим дифференциальное уравнение теплопроводности, в котором можно
        использовать коэффициент температуропроводности
        \item Запишем скорость передачи тепла через производную и элементарные приращения
        \item Получим, что $l \sim \chi t$, то есть оценку на скорость теплопередачи
    \end{enumerate}

    \section{Подвижность макрочастицы.
    Броуновское движение.
    Связь подвижности частицы и коэффициента диффузии облака частиц (соотношение Эйнштейна).
    Закон Эйнштейна—Смолуховского для броуновской частицы}

    \subsection{Подвижность макрочастицы}

    \textbf{Опр} \textit{Подвижность} \textcolor{gray}{$B = \frac{\overline{v}}{F_{fr}}$}

    Видно, что подвижность макрочастицы такая же, как и у макрочастицы

    \subsection{Броуновское движение}

    \textbf{Опр} \textit{Броуновская частица} \textcolor{gray}{Макроскопическая частица размера $\sim 10^{-6} m$}

    \textbf{Опр} \textit{Броуновское движение} \textcolor{gray}{Беспорядочное движение броуновских частиц в жидкосте ...}

    \begin{enumerate}
        \item Рассмотрим движение частицы в среде с помощью 2ЗН, используя запись силы через подвижность
        \item В случае нулевой внешней силы (частица предоставлена сама себе) получим уравнение, проинтегрировав
        которое можно получить формулу скорости такой частицы.
        Такова роль трения в движении частицы
        \item Роль диффузионной силы обуславливает среднеквадратичную скорость, которую можно запросто найти из условия
        равновесия частицы со средой
    \end{enumerate}

    \subsection{Связь подвижности частицы и коэффициента диффузии облака частиц (соотношение Эйнштейна)}

    Теперь опишем движение частицы в координатном пространстве.
    Для этого мы уже не можем пользоваться результатами для микрочастиц.
    Хорошей оценкой является $D \sim \frac{l^2}{\tau} \sim u^2 \tau \sim \dots \sim kTB$, однако она неточна

    Для получения численного коэффициента воспользуемся методом, предложенным Эйнштейном
    \begin{enumerate}
        \item Запишем равенство потоков частиц в поле силы тяжести, обусловленных диффузией и наличие гравитационной
        силы
        \item По ходу дела нам предстоит воспользоваться распределением Больцмана и определение подвижности
        \item Приравниваем потоки и получаем точное значение коэффициента диффузии
    \end{enumerate}

    \subsection{Закон Эйнштейна—Смолуховского для броуновской частицы}

    Для броуновской частицы данный закон будет иметь аналогичный вид, что и для макрочастицы (с учётом размерности
    пространства)

    \section{Явления переноса в разреженных газах: эффузия (эффект Кнудсена), зависимость коэффициента
    теплопроводности газа от давления}

    \textbf{Опр} \textit{Эффузия} \textcolor{gray}{Медленное течение газа через малые отверстия}

    \textbf{Опр} \textit{Число Кнудсена} \textcolor{gray}{$Kn = \frac{\lambda}{L}, L$ -- характерный размер сосуда}

    В случае $Kn \gg 1$ имеем свободное молекулярное течение

    Если аз находится в сосуде, в стенке которого есть малое отверстие площадью $S$, то в состоянии равновесия
    молекулярные потоки с обоих сторон равны ($\frac{1}{4} n_i \overline{v_i}$).
    Тогда можно записать равенство, именуемое эффектом Кнудсена, в чьём общем случае учитываются разные молекулы газа
    с обеих сторон

    Так как $\varkappa \sim \sqrt{\frac{T}{m}}$, то в силу только что полученного соотношения при течении
    разреженных газов имеем $\varkappa \sim \frac{P}{\sqrt{m}}$

    \section{Течение разреженного газа по прямолинейной трубе.
    Формула Кнудсена}

    \subsection{Течение разреженного газа по прямолинейной трубе}

    Когда разреженный газ течёт по прямолинейной трубе, то его $\lambda \sim L$, а так как у стенок трубки находятся
    тоже молекулы, то для описания течения больше подходит процесс диффузии: $N^* = N f\left(\frac{a}{L}\right) \sim
    NC\frac{a}{L}$, притом указанное линейное приближение даёт небольшую погрешность

    В случае если имеется поток с обоих сторон, то можно записать полное число частиц в трубе, и использовав ранее
    известные формулы, преобразовать её.
    Полученное выражение будет сильно отличаться от формулы Пузейля гидродинамического течения.
    При желании можно найти константу $A$ для трубы

    \subsection{Формула Кнудсена}

    В случае течения разреженного газа по трубе, по-новому приблизим $D$ и запишем приращения $\frac{dn}{dx}$ вместо
    дифференциалов.
    В итоге получим формулу Кнудсена

\end{document}
