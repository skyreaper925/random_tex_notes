\subsection{Сумма и пересечение подпространств}

\textbf{Опр} \textit{Пересечение подпространств} \textcolor{gray}{Множество элементов, которые являются их
обычным теоретико-множественным пересечением как подмножеств}

\textbf{Опр} \textit{Сумма подмножеств по Минковскому} \textcolor{gray}{Множество-сумма векторов всех векторов $
a_i$ из каждого $A_i \subset V$, то есть каждый вектор из суммы раскладывается по векторам из всех пространств}

Для суммы Минковского выполняется коммутативность и ассоциативность

\textbf{Утв} \textcolor{blue}{Сумма линейных оболочек есть линейная оболочка объединения}

Каждый элемент суммы есть линейная оболочка какого-то подпространства, поэтому если мы сложим все линейные
оболочки по Минковскому, то получим линейную оболочку объединения (совокупности)

\textbf{Следствие} \textcolor{blue}{Сумма конечного числа подпространств есть подпространство}

Потому как сумма есть минимальное по включению подпространство, содержащее в себе каждое из пространств

\textbf{Утв} \textcolor{blue}{Размерность суммы не превосходит суммы размерностей}

Размерность равна рангу, а для рангов ранг объединения не превосходит суммы рангов (доказывается от противного)

\subsection{Линейно независимые подпространства, прямая сумма подпространств, её характеризации, прямое дополнение
подпространства, проекция на подпространство вдоль прямого дополнения}

\textbf{Опр} \textit{Прямая (внешняя) сумма} \textcolor{gray}{Декартово произвдение $(a_1, \dots, a_n)$}

Сложение и умножение на скаляр определены для внешней суммы покомпонентно

\textbf{Опр} \textit{Прямая (внутренняя) сумма} \textcolor{gray}{Сумма, вектора $
a_i$ в разложении которой из каждого $A_i \subset V$ определены однозначно}

Например, всё пространство разлагается в прямую сумму своих базисных векторов

Через $\overline{U_i}$ обозначим сумму всех рассматриваемых пространств, за исключением $U_i: U_1 + \dots + U_{i-1} +
U_{i+1} + \dots + U_n$

\textbf{Th.1} \textit{Первый критерий прямой суммы}

\textcolor{blue}{Сумма подпространств прямая $\Leftrightarrow U_i \cap \overline{U_i}$}

\begin{itemize}
    \item $\Rightarrow:$ от противного.
    Пусть БОО условие не выполнено для $U_1$.
    Тогда там есть ненулевой вектор, который принадлежит $U_1$ и раскладывается по остальным пространствам.
    Тогда у нас существует два представления нулевого вектора (одно тривиальное, другое новое), $w$ с прямостью суммы
    \item $\Leftarrow:$ от противного.
    Пусть существует два различных разложения $v$ по $a_i$ и $b_i$.
    БОО хотя бы $a_1 \neq b_1$, поэтому если возьмём их разность, то с одной стороны она $\in U_1$, а с другой ---
    $\in \overline{U_i}$, то есть пересечение не тривиально, $w$
\end{itemize}

\textbf{Th.2} \textit{Второй критерий прямой суммы}

\textcolor{blue}{Для конечномерных подпространств следующие условия эквивалентны:
    \begin{enumerate}
        \item Сумма $U = \oplus U_i$ прямая
        \item Система из $\sum_i \dim U_i$ векторов из объединения базисов есть базис в $U$
        \item $\dim U = \sum_i \dim U_i$
    \end{enumerate} }

\begin{itemize}
    \item $2 \Leftrightarrow 3:$ по следствию основной теоремы о рангах $\dim U = rg e = rg \cup_i e^i$, поэтому
    $ \dim U = \sum_i \dim U_i \Leftrightarrow rg e = \sum_i \dim U_i \Leftrightarrow e $ л.н.з система
    \item $1 \Rightarrow 2:$ от противного.
    Пусть $e$ л.з. система.
    Запишем нетривиальную л.к. всей суммы.
    Так как хотя бы одна компонента нетривиальная, то БОО $l_1 \neq 0$.
    Тогда эта комбинация одновременно принадлежит и $U_1$, и $\overline{U_1}$, по той же идее, что и в Th.1, это $w$
    \item $2 \Rightarrow 1:$ от противного.
    Воспользуемся Th.1 получим ненулевое пересечение, распишем вектор оттуда (нетривиальная л.к.) и получим $w$ с л.н
    .з. $e$
\end{itemize}

\textbf{Опр} \textit{Прямое дополнение подпространства} \textcolor{gray}{Недостающий член в прямой сумме до всего
пространства}

В случае двух подпространств, они входят в определение симметрично

\textbf{Утв} \textcolor{blue}{Сумма размерностей подпространства и любоего его прямого дополнения равна
размерности всего пространства $V$}

\textbf{Утв} \textcolor{blue}{У любого подпространства конечномерного простраснтва $V$ существует прямое дополнение}

Для нахождения дополнения достаточно выбрать базис в подпространстве и дополнить его до базиса в пространстве.
Тогда по Th.2 эта система и будет прямым дополнением

Если $a = a_1 + a_2$, то $a_1$ называется проекцией $a$ на $U_1$ вдоль (параллельно) $U_2$

\subsection{Связь размерностей суммы и пересечения подпространств}

Смотреть в рукописном конспекте

\subsection{Понятие факторпространства, его базис и размерность}

\textbf{Опр} \textit{Факторпространство} \textcolor{gray}{Фактор-множество (множество всех классов
эквивалентности для заданного отношения эквивалентности) $a \sim b \Leftrightarrow b - a \in U$}

Элементы факторпространства есть смежные классы вида $a + U$
\begin{gather*}
    (a + U) + (b + U) = (a + b) + U\\
    \lambda (a + U) = \lambda a + U\\
\end{gather*}
Если $W$ - прямое дополнение $U$, то существует естественный изоморфизм $W \rightarrow V/U (a \mapsto a + U)$.
Он является ограничением на $W$ линейного отображения $\pi: V \rightarrow V/U, \pi (v) = v + U$ и называется
канонической проекцией.
Действительно, из определения прямого дополнения следует единственность $u \in U: v = u + w$.
Применим $\pi$ и получим $v + U = w + U$, что влечёт биективность $\pi$

Отсюда следует, что дополнение произвольного базиса в $U$ до базиса в $V$ после применения к ней $\pi$ будет
базисом в $V/U$, притом $\dim V/U = \dim V - \dim U$, что следует из теоремы о сумме размерностей ядра и образа