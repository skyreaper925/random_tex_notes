\subsection{Собственные векторы и собственные значения}

\textbf{Опр} \textit{Собственное значение} \textcolor{gray}{Существует $a \in V:$}

\textbf{Опр} \textit{Собственный вектор} \textcolor{gray}{Ненулевой вектор $a$ преобразования ...}

\textbf{Утв} \textcolor{blue}{Ненулевой вектор $a$ собственный для $\varphi \Leftrightarrow <a>$ инвариантна
относительно $\varphi$}

В силу эквивалентности инвариантности наличию собственного значения

\textbf{Утв} \textcolor{blue}{Ненулевой вектор $a$ собственный для $\varphi$ с собственным
значением $\lambda \Leftrightarrow a \in \ker (\varphi - \lambda)$}

Достаточно вспомнить определение ядра

\subsection{Собственные подпространства}

\textbf{Опр} \textit{Собственное подпространство} \textcolor{gray}{Ядро $\ker (\varphi - \lambda)$, содержащее ...}

\textbf{Утв} \textcolor{blue}{Сумма подпространств $V_{\lambda_i}$ прямая}

\begin{enumerate}
    \item От противного: возьмём $a_1 \in V_{\lambda_1} \cap \sum_2^n V_{\lambda_i}$, то есть $a_1 = \sum_2^n a_i$
    \item Применим к этому равенству преобразование $\sqcap_2^k (\varphi - \lambda_k)$
    \item Справа у нас получится ноль, а слева -- нет, $w$
\end{enumerate}

\subsection{Характеристический многочлен и его инвариантность}

\textbf{Опр} \textit{Характеристический многочлен} \textcolor{gray}{Функция от константы. Не забыть про обозначение}

\textbf{Опр} \textit{Характеристическое уравнение} \textcolor{gray}{Равенства многочлена нулю}

\textbf{Опр} \textit{Характеристические числа} \textcolor{gray}{Корни характерестического многочлена}

Характеристический многочлен можно записать и с учётом алгебраической кратности его корней

\textbf{Утв} \textcolor{blue}{Характерестический многочлен имеет
вид $(-1)^n \lambda^n + (-1)^{n-1} tr A + \dots + \abs{A}$}

Достаточно знать, что определитель есть функция от всех элементов матрицы, затем просто расписать коэффициенты
перед требуемыми степенями

Отсюда, в соответствии с теоремой Виета, сумма всех характеристических чисел равна следу, а произведение есть $\det A$

Стоит учесть, что данное утверждения верно лишь в $\mathbb{C}$.
В $\mathbb{R}$ собственные значения есть только вещественные характеристические числа

\subsection{Определитель и след преобразования}

\textbf{Утв} \textcolor{blue}{Если матрица оператора верхнетреугольна, то характеристические числа совпадают с
диагональными элементами}

Верно в силу того, что определитель верхнетреугольной матрицы равен произведению диагональных элементов

\textbf{Th} \textit{Инвариантность характеристического многочлена}

\textcolor{blue}{Характеристический многочлен не зависит от выбора базиса}

Достаточно записать характеристическое уравнение в двух базисах, перейти от одного к другому с помощью матрицы
перехода и преобразовать выражение

\textbf{Следствие} \textcolor{blue}{Определитель, след, набор характеристических чисел матрицы оператора не
зависят от выбора базиса}

Все вышеперечисленные термины выражаются через коэффициенты характеристического многочлена