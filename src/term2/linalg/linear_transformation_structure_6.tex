\subsection{Корневые подпространства, их размерность}

\textbf{Опр} \textit{Корневое подпространство} \textcolor{gray}{Первое стабильное ядро}

Притом корневое пространство равно и все стабильным ядрам большей размерности

\textbf{Лемма} \textcolor{blue}{$\dim \ker (\varphi - \lambda_i)^{s_i} \geq s_i$}

\begin{enumerate}
    \item Запишем матрицу в верхнетреугольном виде, притом расположим $\lambda_i$ в первых $s_i$ диагональных
    клетках
    \item Применим к матрице преобразование $\varphi - \lambda_i$.
    Получим нильпотентный левый верхний блок
    \item Возведём матрицу в нужную степень с учётом перемножения блочных матриц и получим левый верхний блок нулей
    \item Тогда получим, что первые $s_i$ векторов принадлежат соответствующему ядру.
    А так как такими же могут быть и последующие векторы, то возможно строгое неравенство
\end{enumerate}

\textbf{Следствие} \textcolor{blue}{$\dim V^{\lambda_i} \geq s_i$}

\subsection{Разложение пространства в прямую сумму корневых}

\textbf{Лемма} \textcolor{blue}{Сумма любых степеней ядер $\varphi - \lambda_i$ прямая}

\begin{enumerate}
    \item От противного: пусть $\exists a_1 = V^{\lambda_1} \cap \sum_2^k V^{\lambda_i}$
    \item Тогда этот вектор можно разложить по этим подпространствам
    \item Применим к обеим частям равенства $\psi = \prod_2^k (\varphi - \lambda_i)$
    \item Тогда справа получим ноль, а слева -- нет, $w$
\end{enumerate}

В частности, сумма корневых пространств прямая

\textbf{Th} \textcolor{blue}{
    \begin{enumerate}
        \item $V = \oplus_i V^{\lambda_i}$
        \item $\dim V^\lambda_i = s_i$
        \item $m_i \leq s_i$, то есть стабилизация ядер наступает не позже $s_i$ шага
    \end{enumerate}             }

\begin{enumerate}
    \item В силу $\dim V^\lambda_i \geq s_i$ сложим неравенства по всем $i$.
    Тогда $\sum_i s_i \geq n$, однако у нас $\dim V = n$, поэтому $V = \oplus_i V^{\lambda_1}$
    \item Предыдущий пункт возможен лишь когда во всех неравенствах выполнено равенство
    \item $\dim \ker (\varphi - \lambda_i)^{s_i} \geq s_i = \dim V^\lambda_i$, однако $\ker (\varphi -
    \lambda_i) \subset V^\lambda_i$.
    В противном случае не выполнена формула суммы размерностей ядра и образа
\end{enumerate}

\subsection{Жорданова нормальная форма, её существование и единственность}

\textbf{Опр} \textit{Жорданова клетка} \textcolor{gray}{Верхнетреугольная матрица, в которой на главной диагонали ...}

\textbf{Опр} \textit{Жорданова матрица, ЖНФ} \textcolor{gray}{Блочно-диагональная матрица, каждый блок которой ...}

\textbf{Опр} \textit{Жорданов базис} \textcolor{gray}{Базис, в котором оператор имеет ЖНФ}

\textbf{Опр} \textit{Жорданова цепочка, присоединённый вектор}

\begin{enumerate}
    \item Рассмотрим жорданову клетку запишем её действие в строчном виде
    \item Рассмотрим новый оператор $\psi = \varphi - \lambda_0$.
    Под его действием векторы сваливаются в ядра меньшего по степени оператора
    \item Полученная последовательность называется жордановой цепочкой
    \item Вектор над данным называется присоединённым.
    Ясно, что он может быть и не единственен
\end{enumerate}

\textbf{Th} \textit{Существование ЖНФ}

\textcolor{blue}{Существует базис, в котором матрица оператора жорданова}

\begin{enumerate}
    \item Требуется доказать, что существует базис, являющийся объединением жордановых цепочек, то есть так надо
    сделать в каждом корневом подпространстве
    \item Рассмотрим нильпотентный оператор $\psi: V^\lambda_i \rightarrow V^\lambda_i$, являющийся ограничением
    оператора $\varphi - \lambda_i$ на $V^\lambda_i$
    \item Заметим, что если $a \in \ker \psi^t$, то $a \in \ker \psi^{t-1}$ (непосредственно проверяется)
    \item Теперь рассмотрим вложенную цепочку ядер и определим к последнему, $V^\lambda_i$ прямое (нулевое дополнение
    ) $W_{m_i + 1}$ до следующего ядра (которое будет являться самим $V^\lambda_i$)
    \item Рассмотрим ядро $\psi (W_{t+1})$, для которого выполнено три условия
    \item Из них мы можем определить $W_t$ как прямое дополнение $\ker \psi^{t-1}$ до $\ker \psi^t$ (дополним до базиса)
    \item Таким образом, применяя оператор $\psi (W_{t+1})$ мы спускаемся вниз по цепочке и, дополнив до базиса,
    продолжаем спуск
    \item Жорданов базис есть объединение базисов $W_i$ каждое из которых лежит в $\ker \psi^i$, то есть базисных
    разных $W_i$ пересекаются тривиально (к том же мы пользовались л.н.з. дополнением)
\end{enumerate}

\textbf{Th} \textit{Единственность ЖНФ}

\textcolor{blue}{Для данного базиса ЖНФ единственна с точностью до перестановки жордановых клеток}

\begin{enumerate}
    \item Требуется доказать, что $\forall \lambda_i$ количество жордановых цепочек данной длины в жордановом базисе
    определено однозначно
    \item Рассмотрим все цепочки, соответствующие фиксированному $\lambda_i$.
    Их линейная оболочка находится $\in V^\lambda_i$, $\dim <B_\lambda_i> \leq s_i$
    \item Так как всего в (жордановом) базисе у нас $n$ векторов, а  $<B_\lambda_i>$ и составляют этот базис, то в
    неравенствах выше достигается равенство
    \item Рассмотрим нильпотентный оператор $\psi: V^\lambda_i \rightarrow V^\lambda_i$, являющийся ограничением
    оператора $\varphi - \lambda_i$ на $V^\lambda_i$
    \item Его образ состоит из всех не верхних векторов в цепочках, образ его образа состоит из всех векторов, кроме \ldots
    \item Введём обозначения для количества жордановых цепочек длины $d$, отвечающих $\lambda_i$ за $c_i^d$
    \item Составим уравнения на их суммы, чья система легко решается и однозначно выражается через характеристики
    оператора $\varphi$, то есть инвариантно
\end{enumerate}

Из указанного рассмотрения нетрудно заметить, что степень $\lambda - \lambda_i$ характеристического многочлена $s_i$
равна длине всех цепочек, отвечающих $\forall \lambda_i$, а степень $m_i$ минимального многочлена равна длине
максимальной (иначе оператор обнулится не полностью, не по всем цепочкам-базисным векторам)

\subsection{Минимальный многочлен, критерий диагонализируемости оператора в терминах минимального многочлена}

\textbf{Опр} \textit{Минимальный многочлен} \textcolor{gray}{Многочлен, обгуляющий оператор, минимальной степени}

\textbf{Утв} \textcolor{blue}{Минимальный многочлен является аннулирующим}

\begin{enumerate}
    \item Хотя бы один аннулирующий многочлен существует, ведь если взять степени оператора, которые будет больше
    размерности пространства, то эта система будет л.з., а значит, у соответствующего многочлена будут ненулевые
    коэффициенты
    \item Теперь возьмём произвольный вектор $a \in V = \oplus_i V^{\lambda_i}$
    \item Так как ядра каждого одночлена содержатся в ядре минимального, то каждый член разложения из ядра
    минимального многочлена.
    Тогда и $a$ тоже
    \item Итого, в силу произвольности $a$, минимальный многочлен аннулирующий
\end{enumerate}

\textbf{Утв} \textcolor{blue}{Минимальный многочлен есть $\prod_i (\lambda - \lambda_i)^{m_i}$}

\begin{enumerate}
    \item От противного: пусть хотя бы одна степень тут меньше, то есть БОО $m_1^{'} < m_1$
    \item Тогда $\dim \ker (\varphi - \lambda_1)^{m_1^{'}} < s_1$ по лемме
    \item Если мы сложим все ядра такого вида то получи строгое неравенство.
    То есть существуют вектор пространства $a: \mu_\varphi (\varphi(a)) \neq 0$, то есть новый многочлен не минимальный
\end{enumerate}

\textbf{Th} \textit{Второй критерий диагонализируемости}

\textcolor{blue}{Если $\varphi \in L(V, V)$ имеет попарно различные собтсвенные значения $\lambda_i$
    кратнойстей $s_i$, то следующие условия эквивалентны:
    \begin{enumerate}
        \item $\varphi$ диагонализируем
        \item $V_{\lambda_i} = V^{\lambda_i}$
        \item $\mu_\varphi$ раскладывается на различные линейные множители
    \end{enumerate}
                }

\begin{itemize}
    \item $1 \Leftrightarrow 2:$ в силу того, что $V_\lambda_i \subseteq V^\lambda_i$ и $V = \oplus_i V_{\lambda
    _i}$, достигается равенство множеств
    \item $2 \Rightarrow 3:$ из 2 следует, что $m_i = 1 \forall i$, поэтому все $n$ множителей различны
\end{itemize}